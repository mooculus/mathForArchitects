\documentclass[noauthor,nooutcomes,handout,hints]{ximera}

\graphicspath{  
{./}
{./whoAreYou/}
{./drawingWithTheTurtle/}
{./bisectionMethod/}
{./circles/}
{./anglesAndRightTriangles/}
{./lawOfSines/}
{./lawOfCosines/}
{./plotter/}
{./staircases/}
{./pitch/}
{./qualityControl/}
{./symmetry/}
{./nGonBlock/}
}


%% page layout
\usepackage[cm,headings]{fullpage}
\raggedright
\setlength\headheight{13.6pt}


%% fonts
\usepackage{euler}

\usepackage{FiraMono}
\renewcommand\familydefault{\ttdefault} 
\usepackage[defaultmathsizes]{mathastext}
\usepackage[htt]{hyphenat}

\usepackage[T1]{fontenc}
\usepackage[scaled=1]{FiraSans}

%\usepackage{wedn}
\usepackage{pbsi} %% Answer font


\usepackage{cancel} %% strike through in pitch/pitch.tex


%% \usepackage{ulem} %% 
%% \renewcommand{\ULthickness}{2pt}% changes underline thickness

\tikzset{>=stealth}

\usepackage{adjustbox}

\setcounter{titlenumber}{-1}

%% journal style
\makeatletter
\newcommand\journalstyle{%
  \def\activitystyle{activity-chapter}
  \def\maketitle{%
    \addtocounter{titlenumber}{1}%
                {\flushleft\small\sffamily\bfseries\@pretitle\par\vspace{-1.5em}}%
                {\flushleft\LARGE\sffamily\bfseries\thetitlenumber\hspace{1em}\@title \par }%
                {\vskip .6em\noindent\textit\theabstract\setcounter{question}{0}\setcounter{sectiontitlenumber}{0}}%
                    \par\vspace{2em}
                    \phantomsection\addcontentsline{toc}{section}{\thetitlenumber\hspace{1em}\textbf{\@title}}%
                     }}
\makeatother



%% thm like environments
\let\question\relax
\let\endquestion\relax

\newtheoremstyle{QuestionStyle}{\topsep}{\topsep}%%% space between body and thm
		{}                      %%% Thm body font
		{}                              %%% Indent amount (empty = no indent)
		{\bfseries}            %%% Thm head font
		{)}                              %%% Punctuation after thm head
		{ }                           %%% Space after thm head
		{\thmnumber{#2}\thmnote{ \bfseries(#3)}}%%% Thm head spec
\theoremstyle{QuestionStyle}
\newtheorem{question}{}



\let\freeResponse\relax
\let\endfreeResponse\relax

%% \newtheoremstyle{ResponseStyle}{\topsep}{\topsep}%%% space between body and thm
%% 		{\wedn\bfseries}                      %%% Thm body font
%% 		{}                              %%% Indent amount (empty = no indent)
%% 		{\wedn\bfseries}            %%% Thm head font
%% 		{}                              %%% Punctuation after thm head
%% 		{3ex}                           %%% Space after thm head
%% 		{\underline{\underline{\thmname{#1}}}}%%% Thm head spec
%% \theoremstyle{ResponseStyle}

\usepackage[tikz]{mdframed}
\mdfdefinestyle{ResponseStyle}{leftmargin=1cm,linecolor=black,roundcorner=5pt,
, font=\bsifamily,}%font=\wedn\bfseries\upshape,}


\ifhandout
\NewEnviron{freeResponse}{}
\else
%\newtheorem{freeResponse}{Response:}
\newenvironment{freeResponse}{\begin{mdframed}[style=ResponseStyle]}{\end{mdframed}}
\fi



%% attempting to automate outcomes.

%% \newwrite\outcomefile
%%   \immediate\openout\outcomefile=\jobname.oc
%% \renewcommand{\outcome}[1]{\edef\theoutcomes{\theoutcomes #1~}%
%% \immediate\write\outcomefile{\unexpanded{\outcome}{#1}}}

%% \newcommand{\outcomelist}{\begin{itemize}\theoutcomes\end{itemize}}

%% \NewEnviron{listOutcomes}{\small\sffamily
%% After answering the following questions, students should be able to:
%% \begin{itemize}
%% \BODY
%% \end{itemize}
%% }
\usepackage[tikz]{mdframed}
\mdfdefinestyle{OutcomeStyle}{leftmargin=2cm,rightmargin=2cm,linecolor=black,roundcorner=5pt,
, font=\small\sffamily,}%font=\wedn\bfseries\upshape,}
\newenvironment{listOutcomes}{\begin{mdframed}[style=OutcomeStyle]After answering the following questions, students should be able to:\begin{itemize}}{\end{itemize}\end{mdframed}}



%% my commands

\newcommand{\snap}{{\bfseries\itshape\textsf{Snap!}}}
\newcommand{\flavor}{\link[\snap]{https://snap.berkeley.edu/}}
\newcommand{\mooculus}{\textsf{\textbf{MOOC}\textnormal{\textsf{ULUS}}}}


\usepackage{tkz-euclide}
\tikzstyle geometryDiagrams=[rounded corners=.5pt,ultra thick,color=black]
\colorlet{penColor}{black} % Color of a curve in a plot



\ifhandout\newcommand{\mynewpage}{\newpage}\else\newcommand{\mynewpage}{}\fi


\title{Solar zenith angle}

\author{Bart Snapp}











\begin{document}
\begin{abstract}
  The solar zenith angle measures the height of the Sun.
\end{abstract}
\maketitle

\begin{listOutcomes}
\item Understand what is meant by the solar zenith angle.
\item Connect the solar zenith angle to lines of latitude.
\item Identify North using a analog wrist watch on a clear sunny day.
\end{listOutcomes}

%% \begin{listObjectives}
%%  \item Learn and apply basic geometric formulas,
%% \item Use trigonometry to solve common problems.
%% \end{listObjectives}

\mynewpage

\begin{question}
  Below we have a schematic diagram of the Earth and the Sun's rays
  at summer equinox.
  \begin{center}
      \begin{tikzpicture}[geometryDiagrams,scale=1.5]
        
        \coordinate (O) at (0,0);
        \coordinate (E) at (4,0);
        \coordinate (E') at (5,0);
        \coordinate (P) at (0,4);
        \coordinate (P') at (0,5);
        
        \coordinate (L1) at ({4*cos(18)},{4*sin(18)});
        \coordinate (L2) at ({4*cos(36)},{4*sin(36)});
        \coordinate (L3) at ({4*cos(54)},{4*sin(54)});
        \coordinate (L4) at ({4*cos(72)},{4*sin(72)});

        \coordinate (L1') at ({5*cos(18)},{5*sin(18)});
        \coordinate (L2') at ({5*cos(36)},{5*sin(36)});
        \coordinate (L3') at ({5*cos(54)},{5*sin(54)});
        \coordinate (L4') at ({5*cos(72)},{5*sin(72)});
       
        \coordinate (esun1) at (10,0);
        \coordinate (esun2) at (6,0);

        \coordinate (psun1) at (10,5);
        \coordinate (psun2) at (6,5);
        
        \coordinate (l1sun1) at (10,{5*sin(18)});
        \coordinate (l1sun2) at (6,{5*sin(18)});

        \coordinate (l2sun1) at (10,{5*sin(36)});
        \coordinate (l2sun2) at (6,{5*sin(36)});

        \coordinate (l3sun1) at (10,{5*sin(54)});
        \coordinate (l3sun2) at (6,{5*sin(54)});

        \coordinate (l4sun1) at (10,{5*sin(72)});
        \coordinate (l4sun2) at (6,{5*sin(72)});
        
        \begin{scope}
        \clip (-.5,-.5) rectangle (6,6);
        \tkzDrawCircle[ultra thick,fill=blue!10!white,draw=blue!50!black](O,E)
        %\tkzDrawSegment[dashed](S',S)
        %\tkzDrawSegment[dashed](A',KK)
        %\tkzDrawSegment[line width=1mm, black,line cap=rect](A,H)
        \tkzDrawSegment[line width=1mm, black!20!brown](E,E')
        \tkzDrawSegment[line width=1mm, black!20!brown](P,P')
        \tkzDrawSegment[line width=1mm, black!20!brown](L1,L1')
        \tkzDrawSegment[line width=1mm, black!20!brown](L2,L2')
        \tkzDrawSegment[line width=1mm, black!20!brown](L3,L3')
        \tkzDrawSegment[line width=1mm, black!20!brown](L4,L4')
        \end{scope}
        %% \tkzDrawSegment[line width=2mm, yellow!50!black,->](esun1,esun2)
        %% \tkzDrawSegment[line width=2mm, yellow!50!black,->](psun1,psun2)
        %% \tkzDrawSegment[line width=2mm, yellow!50!black,->](l1sun1,l1sun2)
        %% \tkzDrawSegment[line width=2mm, yellow!50!black,->](l2sun1,l2sun2)
        %% \tkzDrawSegment[line width=2mm, yellow!50!black,->](l3sun1,l3sun2)
        %% \tkzDrawSegment[line width=2mm, yellow!50!black,->](l4sun1,l4sun2)
        
        \tkzDrawSegment[line width=1mm, yellow!50!orange,->](esun1,esun2)
        \tkzDrawSegment[line width=1mm, yellow!50!orange,->](psun1,psun2)
        \tkzDrawSegment[line width=1mm, yellow!50!orange,->](l1sun1,l1sun2)
        \tkzDrawSegment[line width=1mm, yellow!50!orange,->](l2sun1,l2sun2)
        \tkzDrawSegment[line width=1mm, yellow!50!orange,->](l3sun1,l3sun2)
        \tkzDrawSegment[line width=1mm, yellow!50!orange,->](l4sun1,l4sun2)

        \node at (8,2) {Sun's rays};

        
        
        \node[above] at (L1') {$18^\circ$};
        \node[above] at (L2') {$36^\circ$};
        \node[above] at (L3') {$54^\circ$};
        \node[above] at (L4') {$72^\circ$};
        \node[right] at (E') {$0^\circ$};
        \node[left] at (P') {$90^\circ$};


        \tkzDrawPoint(E)
        \tkzDrawPoint(P)
        \tkzDrawPoint(L1)
        \tkzDrawPoint(L2)
        \tkzDrawPoint(L3)
        \tkzDrawPoint(L4)

        %\tkzMarkAngle[size=.5,mark={}](S,O,A)

        %\tkzMarkAngle[size=.3,mark={}](H,K,A)
      \end{tikzpicture}
  \end{center}
  \begin{enumerate}
  \item Why are the Sun's rays parallel?
  \item At each drawn point on the Earth above, we have draw a stick
    pointing ``straight up.'' Moreover, we have labeled this with the
    so-called ``solar zenith angle.'' Identify and mark the geometric angle
    that determines the value of the solar zenith angle.
  \item \textbf{Lines of latitude} are formed by making an angle from
    a desired position, to the center of the Earth, to the point on
    the equator directly North/South of your position. \textbf{Explain}
    the connection between the solar zenith angle and the lines of latitude.
  \end{enumerate}
 
\end{question}
\mynewpage




\begin{question}
  Here are some questions tangentially related to the solar zenith angle.
  \begin{enumerate}
  \item What is special about the area of the Earth between the tropics? Explain using the solar zenith angle.
  \item Why do the seasons happen? Explain using the solar zenith angle.
  \item Columbus Ohio is located at very nearly $40^\circ$
    latitude. What's the \textbf{highest solar zenith angle} in
    Columbus? When does it happen? What's the \textbf{lowest
      solar zenith angle} in Columbus? When does it happen?
  \end{enumerate}
\end{question}

\mynewpage




\begin{question}
  Use the INTERNET to figure out how to use an analog wrist watch to
  deduce which direction is North is based on the Sun. 
  \begin{enumerate}
  \item Suppose it is morning and your hour hand is pointed at the
    Sun.  Demonstrate understanding by showing the direction North in
    the diagram below:
    \begin{center}
      \watch{7}{15}
    \end{center}
     \item Suppose it is afternoon and your hour hand is pointed at
       the Sun. Demonstrate understanding by showing the direction
       North in the diagram below:
    \begin{center}
      \watch{1}{30}
    \end{center}
  \item Suppose it is the afternoon \textit{golden hour} and your hour
    hand is pointed at the Sun. Demonstrate understanding by showing
    the direction North in the diagram below:
    \begin{center}
      \watch{6}{25}
    \end{center}
  \item Explain WHY this trick works. As part of your explanation,
    include how to deal with times before/after noon AND how things
    change if you are in the Southern Hemisphere.
  \end{enumerate}




  
\end{question}




\end{document}
