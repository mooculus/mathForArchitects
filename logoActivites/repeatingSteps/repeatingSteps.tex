\documentclass[noauthor,nooutcomes,12pt]{ximera}

%% page layout
\usepackage[in,headings]{fullpage}
\raggedright
\setlength\headheight{13.6pt}


%% fonts
\usepackage{euler}

\usepackage{FiraMono}
\renewcommand\familydefault{\ttdefault} 
\usepackage{mathastext}
\usepackage[htt]{hyphenat}

\usepackage[T1]{fontenc}
\usepackage[scaled=1]{FiraSans}

\usepackage{wedn}
\usepackage[T1]{fontenc}

%% wrap text around scripts
\usepackage{wrapfig}

\tikzset{>=stealth}
%% snap! scripts
\usepackage{scratch3}

\usepackage{adjustbox}

%% journal style
\makeatletter
\newcommand\journalstyle{%
  \def\activitystyle{activity-chapter}
  \def\maketitle{%
    \addtocounter{titlenumber}{1}%
                {\flushleft\small\sffamily\bfseries\@pretitle\par\vspace{-1.5em}}%
                {\flushleft\LARGE\sffamily\bfseries\thetitlenumber\hspace{1em}\@title \par }%
                {\vskip .6em\noindent\textit\theabstract\setcounter{question}{0}\setcounter{sectiontitlenumber}{0}}%
                    \par\vspace{2em}
                    \phantomsection\addcontentsline{toc}{section}{\thetitlenumber\hspace{1em}\textbf{\@title}}%
                     }}
\makeatother



%% thm like environments
\let\question\relax
\let\endquestion\relax

\newtheoremstyle{QuestionStyle}{\topsep}{\topsep}%%% space between body and thm
		{}                      %%% Thm body font
		{}                              %%% Indent amount (empty = no indent)
		{\bfseries}            %%% Thm head font
		{)}                              %%% Punctuation after thm head
		{ }                           %%% Space after thm head
		{\thmnumber{#2}\thmnote{ \bfseries(#3)}}%%% Thm head spec
\theoremstyle{QuestionStyle}
\newtheorem{question}{}



\let\freeResponse\relax
\let\endfreeResponse\relax

%% \newtheoremstyle{ResponseStyle}{\topsep}{\topsep}%%% space between body and thm
%% 		{\wedn\bfseries}                      %%% Thm body font
%% 		{}                              %%% Indent amount (empty = no indent)
%% 		{\wedn\bfseries}            %%% Thm head font
%% 		{}                              %%% Punctuation after thm head
%% 		{3ex}                           %%% Space after thm head
%% 		{\underline{\underline{\thmname{#1}}}}%%% Thm head spec
%% \theoremstyle{ResponseStyle}

\usepackage[tikz]{mdframed}
\mdfdefinestyle{ResponseStyle}{leftmargin=1cm,linecolor=black,roundcorner=5pt,frametitlefont=\wedn\bfseries,%frametitle={\underline{\underline{Response}}:}
, font=\wedn\bfseries,}%\begin{mdframed}[style=mystyle]foo\end{mdframed}


\ifhandout
\NewEnviron{freeResponse}{}
\else
%\newtheorem{freeResponse}{Response:}
\newenvironment{freeResponse}{\begin{mdframed}[style=ResponseStyle]}{\end{mdframed}}
\fi



%% attempting to automate outcomes.

\newwrite\outcomefile
  \immediate\openout\outcomefile=\jobname.oc
\renewcommand{\outcome}[1]{\edef\theoutcomes{\theoutcomes #1~}%
\immediate\write\outcomefile{\unexpanded{\outcome}{#1}}}

%% \newcommand{\outcomelist}{\begin{itemize}\theoutcomes\end{itemize}}



%% my commands

\newcommand{\snap}{{\bfseries\itshape\textsf{Snap!}}}
\newcommand{\flavor}{\link[\snap]{https://snap.berkeley.edu/}}


\usepackage{tkz-euclide}
\tikzstyle geometryDiagrams=[rounded corners=.5pt,ultra thick,color=black]
\colorlet{penColor}{black} % Color of a curve in a plot

\usepackage{fullpage}
\makeatletter
%% no number for activity
\newcommand\logostyle{%
  \def\activitystyle{activity-chapter}
  \def\maketitle{%
                {\flushleft\small\sffamily\bfseries\@pretitle\par\vspace{-1.5em}}%
                {\flushleft\LARGE\sffamily\bfseries\@title \par }%
                {\vskip .6em\noindent\textit\theabstract\setcounter{problem}{0}\setcounter{sectiontitlenumber}{0}}%
                    \par\vspace{2em}
                    \phantomsection\addcontentsline{toc}{section}{\textbf{\@title}}%
                     \setcounter{titlenumber}{0}}}
\makeatother
\newcommand{\nameblankgen}{\noindent\textbf{Name(s) (please print):}\ \hrulefill \\

\hrulefill}
\logostyle



%% %%% Graph paper background
%% \def\mygraphpaper{%
%%   \begin{tikzpicture}
%%     %\draw[line width=.4pt,draw=black!30] (0,0) grid[step=1mm] (\paperwidth,\paperheight);
%%     \draw[line width=.4pt,draw=blue!50] (0,0) grid[step=.2in] (\paperwidth,\paperheight);
%%   \end{tikzpicture}%
%% }
%% \usepackage{background}
%% \backgroundsetup{
%%   angle=0,
%%   contents=\mygraphpaper,
%%   color=black,
%%   scale=1,
%% }





\pagenumbering{gobble}



\title{Repeating steps}
\author{Bart Snapp}

\begin{document}
\begin{abstract}
  Don't type so much, let the turtle repeat!
\end{abstract}
\maketitle

\nameblankgen

\begin{multicols}{2}
  Drawing picturs in \LOGO\ often means making the turtle do things
  over and over again. The drawing-turtle doesn't mind, but there are
  efficient ways to communicate. One way is the \lc{repeat} command.
\begin{logo}
repeat 4 [
  fd 100
  lt 90
  ]
\end{logo}

Remember, she's a drawing turtle, drawing a path wherever she goes.

To help her turn, use a command like \lc{lt 90} to mean ``turn $90$
degrees to the left.'' So for example, the code below draws a square:

\begin{logo}
fd 100
lt 90
fd 100
lt 90
fd 100
lt 90
fd 100
\end{logo}
\begin{logoout}
\begin{tikzpicture}[turtle/distance=2cm]
  \draw [thick,black,turtle={home,forward,left,forward,left,forward,left,forward}];
  \node at (0,0) {\turtler};
\end{tikzpicture}
\end{logoout}

Note, now she has turned, and is pointing directly right, like
$\turtler$, thus proving the old adage,
\begin{quote}
  Two wrongs don't make a right, but four lefts do.
\end{quote}
\end{multicols}

\newpage

\begin{problem}
  Use the techniques above to help the turtle draw an equilateral
  triangle with side length $250$. Show your code and your picture.
\end{problem}

\newpage

\begin{problem}
  Use the techniques above to help the turtle draw a regular pentagon
  triangle with side length $90$. Show your code and your picture. 
\end{problem}

\newpage


\begin{problem}
  Use the techniques above to help the turtle draw an regular hexagon
  with side length $80$. Show your code and your picture. 
\end{problem}

\newpage

\begin{problem}
  Use the techniques above to help the turtle draw a regular
  five-pointed star of a reasonable size. Show your code and your
  picture.
\end{problem}

\newpage

\begin{problem}
  Use the techniques above to draw the rep-4-tile known as the sphinx,
  that is, produce this picture:
  \begin{logoout}
\begin{tikzpicture}[turtle/distance=2cm]
  \draw [thick,black,turtle={home,right,forward,forward,forward,left=120,forward,left=60,forward,right=60,forward,left=120,forward,forward}];
  %\node at (0,0) {\rotatebox{140}{\turtle}};
\end{tikzpicture}
\end{logoout}

  Show your code and your
  picture.
\end{problem}



\end{document}
