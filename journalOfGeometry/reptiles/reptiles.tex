\documentclass[nooutcomes,noauthor,hints,handout]{ximera}

%% page layout
\usepackage[in,headings]{fullpage}
\raggedright
\setlength\headheight{13.6pt}


%% fonts
\usepackage{euler}

\usepackage{FiraMono}
\renewcommand\familydefault{\ttdefault} 
\usepackage{mathastext}
\usepackage[htt]{hyphenat}

\usepackage[T1]{fontenc}
\usepackage[scaled=1]{FiraSans}

\usepackage{wedn}
\usepackage[T1]{fontenc}

%% wrap text around scripts
\usepackage{wrapfig}

\tikzset{>=stealth}
%% snap! scripts
\usepackage{scratch3}

\usepackage{adjustbox}

%% journal style
\makeatletter
\newcommand\journalstyle{%
  \def\activitystyle{activity-chapter}
  \def\maketitle{%
    \addtocounter{titlenumber}{1}%
                {\flushleft\small\sffamily\bfseries\@pretitle\par\vspace{-1.5em}}%
                {\flushleft\LARGE\sffamily\bfseries\thetitlenumber\hspace{1em}\@title \par }%
                {\vskip .6em\noindent\textit\theabstract\setcounter{question}{0}\setcounter{sectiontitlenumber}{0}}%
                    \par\vspace{2em}
                    \phantomsection\addcontentsline{toc}{section}{\thetitlenumber\hspace{1em}\textbf{\@title}}%
                     }}
\makeatother



%% thm like environments
\let\question\relax
\let\endquestion\relax

\newtheoremstyle{QuestionStyle}{\topsep}{\topsep}%%% space between body and thm
		{}                      %%% Thm body font
		{}                              %%% Indent amount (empty = no indent)
		{\bfseries}            %%% Thm head font
		{)}                              %%% Punctuation after thm head
		{ }                           %%% Space after thm head
		{\thmnumber{#2}\thmnote{ \bfseries(#3)}}%%% Thm head spec
\theoremstyle{QuestionStyle}
\newtheorem{question}{}



\let\freeResponse\relax
\let\endfreeResponse\relax

%% \newtheoremstyle{ResponseStyle}{\topsep}{\topsep}%%% space between body and thm
%% 		{\wedn\bfseries}                      %%% Thm body font
%% 		{}                              %%% Indent amount (empty = no indent)
%% 		{\wedn\bfseries}            %%% Thm head font
%% 		{}                              %%% Punctuation after thm head
%% 		{3ex}                           %%% Space after thm head
%% 		{\underline{\underline{\thmname{#1}}}}%%% Thm head spec
%% \theoremstyle{ResponseStyle}

\usepackage[tikz]{mdframed}
\mdfdefinestyle{ResponseStyle}{leftmargin=1cm,linecolor=black,roundcorner=5pt,frametitlefont=\wedn\bfseries,%frametitle={\underline{\underline{Response}}:}
, font=\wedn\bfseries,}%\begin{mdframed}[style=mystyle]foo\end{mdframed}


\ifhandout
\NewEnviron{freeResponse}{}
\else
%\newtheorem{freeResponse}{Response:}
\newenvironment{freeResponse}{\begin{mdframed}[style=ResponseStyle]}{\end{mdframed}}
\fi



%% attempting to automate outcomes.

\newwrite\outcomefile
  \immediate\openout\outcomefile=\jobname.oc
\renewcommand{\outcome}[1]{\edef\theoutcomes{\theoutcomes #1~}%
\immediate\write\outcomefile{\unexpanded{\outcome}{#1}}}

%% \newcommand{\outcomelist}{\begin{itemize}\theoutcomes\end{itemize}}



%% my commands

\newcommand{\snap}{{\bfseries\itshape\textsf{Snap!}}}
\newcommand{\flavor}{\link[\snap]{https://snap.berkeley.edu/}}


\usepackage{tkz-euclide}
\tikzstyle geometryDiagrams=[rounded corners=.5pt,ultra thick,color=black]
\colorlet{penColor}{black} % Color of a curve in a plot

\title{Rep-tiles}

\author{Bart Snapp}

\begin{document}
\begin{abstract}
  We will investigate rep-tiles.
\end{abstract}
\maketitle



\begin{listOutcomes}
\item Describe geometric properties that ``scaling'' preserves.
\item Explain what a rep-tile is.
\item Give basic examples of rep-$n$-tiles.
\item Critique and dismantle reasonable hypotheses in regard to
  geometry and arithmetic.
\item Witness the fact that the sum of the interior angles of a
  triangle sum to $180^\circ$.
\end{listOutcomes}

When an object is SCALED:
\begin{itemize}
\item Its angles are unchanged.
\item The relative proportions of its linear measurements are
  unchanged.
\end{itemize}
A \textbf{rep-tile}\index{rep-tile} is a polygon where several copies
of a given rep-tile fit together to make a larger, similar (MEANING
SCALED), version of itself. 



If $2$ copies are used, we call it a
\textit{rep-$2$-tile}, if $3$ copies are used, we call it a
\textit{rep-$3$-tile}, and if $n$ copies are used, we call it a
\textit{rep-$n$-tile}.

\mynewpage

\begin{question}
EXPLAIN why using words, pictures, computations, and so on, as
needed/helpful:
\begin{enumerate}
\item A rectangle whose sides have lengths $1$ and $\sqrt{2}$ is a
  rep-$2$-tile.
\item An isosceles right triangle whose sides have lengths $1$,
  $1$, and $\sqrt{2}$ is a rep-$2$-tile.
\item \textit{Geometry Giorgio} suggests that a rectangle whose sides
  have lengths $1$ and $4$ is also a rep-$2$-tile. Is he correct?
\end{enumerate}
In each case, if you want to convince me that you have a rep-tile, you
must EXPLAIN why the angles are unchanged, and why the proportions of
the linear measurements are unchanged.
\begin{freeResponse}
  \begin{enumerate}
  \item Behold:
    \begin{center}
      \begin{tikzpicture}[geometryDiagrams]
        \coordinate (A) at (0,0);
        \coordinate (B) at (1,0);
        \coordinate (C) at (1,1.414);
        \coordinate (D) at (0,1.414);
        \coordinate (E) at (2,0);
        \coordinate (F) at (2,1.414);
        \tkzDrawSegment (A,B)
        \tkzDrawSegment (B,C)
        \tkzDrawSegment (C,D)
        \tkzDrawSegment (D,A)
        \tkzDrawSegment (B,E)
        \tkzDrawSegment (F,E)
        \tkzDrawSegment (F,C)

        \tkzLabelSegment[left](A,D){$\sqrt{2}$}
        \tkzLabelSegment[below](A,B){$1$}
        \tkzLabelSegment[below](B,E){$1$}
      \end{tikzpicture}
    \end{center}
    All the angles are clearly the same, $90^\circ$. To see the two
    small rectangles have the same linear proportions as the large
    rectangle, consider
    \[
    \frac{\text{long side}}{\text{short side}} = \frac{\sqrt{2}}{1} = \frac{2}{\sqrt{2}}.
    \]
  \item Behold:
    \begin{center}
      \begin{tikzpicture}[geometryDiagrams]
        \coordinate (A) at (0,0);
        \coordinate (B) at (2,0);
        \coordinate (C) at (2,2);
        \coordinate (D) at (4,0);
        \tkzDrawSegment (A,B)
        \tkzDrawSegment (C,B)
        \tkzDrawSegment (C,A)
        \tkzDrawSegment (D,B)
        \tkzDrawSegment (D,C)
                
        \tkzMarkRightAngle[thin](A,B,C)
        \tkzMarkRightAngle[thin](C,B,D)

        \tkzLabelSegment[above left](A,C){$\sqrt{2}$}
        \tkzLabelSegment[above right](C,D){$\sqrt{2}$}
        \tkzLabelSegment[below](A,B){$1$}
        \tkzLabelSegment[below](D,B){$1$}
        \tkzLabelSegment[left](C,B){$1$}
        
      \end{tikzpicture}
    \end{center}  
    Here we have two isosceles triangles whose sides have lengths $1$,
    $1$, and $\sqrt{2}$, making a larger isosceles triangle with
    sides, $\sqrt{2}$, $\sqrt{2}$, $2$. Since two $45^\circ$ angles
    form a right angle, the top angle is a right angle, and the angles
    of the smaller two triangles are the same as the angles of the
    larger triangle. On the other hand, the sides have maintained
    their proportions as
    \[
    \frac{\text{long side}}{\text{short side}} = \frac{\sqrt{2}}{1} = \frac{2}{\sqrt{2}}.
    \]
    Thus this is a rep-$2$-tile.
  
  \item NO WAY! While we can put the rectangles together to get a new
    rectangle, the proportions are no longer preserved. If we put the two short sides together we find
    \[
    \frac{\text{long side}}{\text{short side}} = \frac{4}{1} \ne
    \frac{8}{1}.
    \]
    If we put the two long sides together we find
    \[
    \frac{\text{long side}}{\text{short side}} = \frac{4}{1} \ne
    \frac{4}{2}.
    \]
    This is not a rep-$2$-tile.
  \end{enumerate}
\end{freeResponse}
\end{question}
\mynewpage






\begin{question}
EXPLAIN why using words, pictures, and so on, as needed/helpful:
\begin{enumerate}
\item A rectangle whose sides have lengths $1$ and $\sqrt{3}$, is a rep-$3$-tile.
\item A $30$-$60$-$90$ right triangle whose shortest side has length $1$, is a rep-$3$-tile.
\end{enumerate}
In each case, if you want to convince me that you have a rep-tile, you
must EXPLAIN why the angles are unchanged, and why the proportions of
the linear measurements are unchanged.
\begin{freeResponse}
  \begin{enumerate}
  \item Behold:
    \begin{center}
      \begin{tikzpicture}[geometryDiagrams]
        \coordinate (A) at (0,0);
        \coordinate (B) at (1,0);
        \coordinate (C) at (1,1.732);
        \coordinate (D) at (0,1.732);
        \coordinate (E) at (2,0);
        \coordinate (F) at (2,1.732);
        \coordinate (G) at (3,0);
        \coordinate (H) at (3,1.732);
        \tkzDrawSegment (A,B)
        \tkzDrawSegment (B,C)
        \tkzDrawSegment (C,D)
        \tkzDrawSegment (D,A)
        \tkzDrawSegment (B,E)
        \tkzDrawSegment (F,E)
        \tkzDrawSegment (F,C)

        \tkzDrawSegment (G,E)
        \tkzDrawSegment (G,H)
        \tkzDrawSegment (H,F)

        \tkzLabelSegment[left](A,D){$\sqrt{3}$}
        \tkzLabelSegment[below](A,B){$1$}
        \tkzLabelSegment[below](B,E){$1$}
        \tkzLabelSegment[below](G,E){$1$}
      \end{tikzpicture}
    \end{center}
    All the angles are clearly the same, $90^\circ$. To see the three
    small rectangles have the same linear proportions as the large
    rectangle, consider
    \[
    \frac{\text{long side}}{\text{short side}} = \frac{\sqrt{3}}{1} = \frac{3}{\sqrt{3}}.
    \]
  \item Behold:
    \begin{center}
      \begin{tikzpicture}[geometryDiagrams]%,scale=3]
        \coordinate (A) at (0,0);
        \coordinate (B) at (3,0);
        \coordinate (C) at (0,{3*1.732});
        \coordinate (D) at (9,0);
        \tkzDefMidPoint(C,D) \tkzGetPoint{E}
        
        \tkzDrawSegment (A,D)
        \tkzDrawSegment (A,C)
        \tkzDrawSegment (B,C)
        \tkzDrawSegment (C,D)
        \tkzDrawSegment (E,B)
        
        \tkzLabelSegment[left](A,C){$\sqrt{3}$}
        \tkzLabelSegment[below](A,B){$1$}
        \tkzLabelSegment[below](B,D){$2$}
        \tkzLabelSegment[above right](D,E){$\sqrt{3}$}
        \tkzLabelSegment[above right](C,E){$\sqrt{3}$}
        \tkzLabelSegment[above right](C,B){$2$}
        \tkzLabelSegment[below right](E,B){$1$}
        
        \tkzMarkRightAngle[thin](B,A,C)
        \tkzMarkRightAngle[thin](B,E,C)
        \tkzMarkRightAngle[thin](D,E,B)

        \tkzMarkAngle[arc=ll,size=.7cm,thin,mark=none](C,B,A)
        \tkzLabelAngle(C,B,A){$60^\circ$}
        \tkzMarkAngle[arc=ll,size=.5cm,thin,mark=none](D,B,E)
        \tkzLabelAngle(D,B,E){$60^\circ$}
        \tkzMarkAngle[arc=ll,size=.6cm,thin,mark=none](E,B,C)
        \tkzLabelAngle(E,B,C){$60^\circ$}

        \tkzMarkAngle[size=.6cm,thin,mark=none](C,D,A)
        \tkzLabelAngle(C,D,A){$30^\circ$}
        \tkzMarkAngle[size=.8cm,thin,mark=none](A,C,B)
        \tkzLabelAngle[pos=1.2](A,C,B){$30^\circ$}

        \tkzMarkAngle[size=.6cm,thin,mark=none](B,C,D)
        \tkzLabelAngle[pos=1.2](B,C,D){$30^\circ$}
      \end{tikzpicture}
    \end{center}
    We see that all the angles of the larger figure are the same as
    our starting $30$-$60$-$90$ right triangle. Also we see
    \begin{align*}
      \frac{\text{long leg}}{\text{short leg}} &= \frac{\sqrt{3}}{1} = \frac{3}{\sqrt{3}},\\
      \frac{\text{hypotenuse}}{\text{short leg}} &= \frac{2}{1} = \frac{2\sqrt{3}}{\sqrt{3}},\\
      \frac{\text{hypotenuse}}{\text{long leg}} &= \frac{2}{\sqrt{3}} = \frac{2\sqrt{3}}{3}.\\
    \end{align*}
  \end{enumerate}
\end{freeResponse}
\end{question}
\mynewpage



\begin{question} %% give a specific triangle. put them on grid paper.
  Consider the following ``generic'' triangle:
  \begin{center}
    \begin{tikzpicture}[geometryDiagrams]
      \coordinate (A) at (0,0);
      \coordinate (B) at (5,2);
      \coordinate (C) at (7,0);
      \tkzDrawSegment (A,B)
      \tkzDrawSegment (A,C)
      \tkzDrawSegment (C,B)
      \tkzLabelSegment[above left](A,B){$c$}
      \tkzLabelSegment[below](A,C){$b$}
      \tkzLabelSegment[above right](B,C){$a$}  
      
      \tkzMarkAngle[size=1.5cm,thin,mark=](C,A,B)
      \tkzLabelAngle[pos=1.2](C,A,B){$\alpha$}
      
      \tkzMarkAngle[size=0.8cm,thin,mark=](A,B,C)
      \tkzLabelAngle[pos=.5](A,B,C){$\beta$}
      
      \tkzMarkAngle[mark=,size=.9,thin](B,C,A)
      \tkzLabelAngle[pos=.6](B,C,A){$\gamma$}
      
    \end{tikzpicture}
  \end{center}
  \begin{enumerate}
  \item EXPLAIN using words, pictures, computations, and so on, as
    needed/helpful WHY every triangle is a rep-$4$-tile.
  \item Use your work from the part above to explain HOW someone can
    directly WITNESS the fact that the sum of the interior angles of a
    triangle is $180^\circ$.
  \end{enumerate}
    \begin{freeResponse}
  \begin{enumerate}
  \item Behold:
    \begin{center}
      \begin{tikzpicture}[geometryDiagrams]
        \coordinate (A) at (0,0);
        \coordinate (B) at (5,2);
        \coordinate (C) at (7,0);
        \coordinate (BB) at (12,2);
        \coordinate (CC) at (14,0);
        \coordinate (BBB) at (10,4);
        
        \tkzDrawSegment (A,B)
        \tkzDrawSegment (A,C)
        \tkzDrawSegment (C,B)
        \tkzLabelSegment[above left](A,B){$c$}
        \tkzLabelSegment[below](A,C){$b$}
        \tkzLabelSegment[above right](B,C){$a$}  
      
        \tkzMarkAngle[size=1.5cm,thin,mark=](C,A,B)
        \tkzLabelAngle[pos=1.2](C,A,B){$\alpha$}
        
        \tkzMarkAngle[size=0.8cm,thin,mark=](A,B,C)
        \tkzLabelAngle[pos=.5](A,B,C){$\beta$}
        
        \tkzMarkAngle[mark=,size=.9,thin](B,C,A)
        \tkzLabelAngle[pos=.6](B,C,A){$\gamma$}


        \tkzDrawSegment (C,BB)
        \tkzDrawSegment (BB,CC)
        \tkzDrawSegment (CC,C)
        \tkzLabelSegment[above left](C,BB){$c$}
        \tkzLabelSegment[below](C,CC){$b$}
        \tkzLabelSegment[above right](BB,CC){$a$}  

        \tkzMarkAngle[size=1.5cm,thin,mark=](CC,C,BB)
        \tkzLabelAngle[pos=1.2](CC,C,BB){$\alpha$}
        
        \tkzMarkAngle[size=0.8cm,thin,mark=](C,BB,CC)
        \tkzLabelAngle[pos=.5](C,BB,CC){$\beta$}
        
        \tkzMarkAngle[mark=,size=.9,thin](BB,CC,C)
        \tkzLabelAngle[pos=.6](BB,CC,C){$\gamma$}




        \tkzDrawSegment (BBB,B)
        \tkzDrawSegment (BBB,BB)
        \tkzDrawSegment (B,BB)
        \tkzLabelSegment[above left](B,BBB){$c$}
        \tkzLabelSegment[below](B,BB){$b$}
        \tkzLabelSegment[above right](BB,BBB){$a$}  

        \tkzMarkAngle[size=1.5cm,thin,mark=](BB,B,BBB)
        \tkzLabelAngle[pos=1.2](BB,B,BBB){$\alpha$}
        
        \tkzMarkAngle[size=0.8cm,thin,mark=](B,BBB,BB)
        \tkzLabelAngle[pos=.5](B,BBB,BB){$\beta$}
        
        \tkzMarkAngle[mark=,size=.9,thin](BBB,BB,B)
        \tkzLabelAngle[pos=.6](BBB,BB,B){$\gamma$}



        \tkzMarkAngle[size=1.5cm,thin,mark=](B,BB,C)
        \tkzLabelAngle[pos=1.2](B,BB,C){$\alpha$}
        
        \tkzMarkAngle[size=0.8cm,thin,mark=](BB,C,B)
        \tkzLabelAngle[pos=.5](BB,C,B){$\beta$}
        
        \tkzMarkAngle[mark=,size=.9,thin](C,B,BB)
        \tkzLabelAngle[pos=.6](C,B,BB){$\gamma$}
      \end{tikzpicture}
    \end{center}
    We see that all the angles of the larger figure are the same as
    our starting ``generic'' triangle. Also we see
    \begin{align*}
      \frac{a+a}{b+b} = \frac{a}{b},\\
      \frac{b+b}{c+c} = \frac{b}{c},\\
      \frac{c+c}{a+a} = \frac{c}{a}.\\
    \end{align*}
    Thus the ratios of the sides are the same. So we see every
    triangle is a rep-$4$-tile.
  \item Above we see that if we place the angles $\alpha$, $\beta$,
    and $\gamma$ next to each other, they form a straight line. Hence
    we can directly witness that
    \[
    \alpha + \beta + \gamma = 180^\circ.
    \]
  \end{enumerate}
\end{freeResponse}
\end{question}

\end{document}
