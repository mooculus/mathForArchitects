\documentclass[handout,nooutcomes,noauthor,hints]{ximera}

%% page layout
\usepackage[in,headings]{fullpage}
\raggedright
\setlength\headheight{13.6pt}


%% fonts
\usepackage{euler}

\usepackage{FiraMono}
\renewcommand\familydefault{\ttdefault} 
\usepackage{mathastext}
\usepackage[htt]{hyphenat}

\usepackage[T1]{fontenc}
\usepackage[scaled=1]{FiraSans}

\usepackage{wedn}
\usepackage[T1]{fontenc}

%% wrap text around scripts
\usepackage{wrapfig}

\tikzset{>=stealth}
%% snap! scripts
\usepackage{scratch3}

\usepackage{adjustbox}

%% journal style
\makeatletter
\newcommand\journalstyle{%
  \def\activitystyle{activity-chapter}
  \def\maketitle{%
    \addtocounter{titlenumber}{1}%
                {\flushleft\small\sffamily\bfseries\@pretitle\par\vspace{-1.5em}}%
                {\flushleft\LARGE\sffamily\bfseries\thetitlenumber\hspace{1em}\@title \par }%
                {\vskip .6em\noindent\textit\theabstract\setcounter{question}{0}\setcounter{sectiontitlenumber}{0}}%
                    \par\vspace{2em}
                    \phantomsection\addcontentsline{toc}{section}{\thetitlenumber\hspace{1em}\textbf{\@title}}%
                     }}
\makeatother



%% thm like environments
\let\question\relax
\let\endquestion\relax

\newtheoremstyle{QuestionStyle}{\topsep}{\topsep}%%% space between body and thm
		{}                      %%% Thm body font
		{}                              %%% Indent amount (empty = no indent)
		{\bfseries}            %%% Thm head font
		{)}                              %%% Punctuation after thm head
		{ }                           %%% Space after thm head
		{\thmnumber{#2}\thmnote{ \bfseries(#3)}}%%% Thm head spec
\theoremstyle{QuestionStyle}
\newtheorem{question}{}



\let\freeResponse\relax
\let\endfreeResponse\relax

%% \newtheoremstyle{ResponseStyle}{\topsep}{\topsep}%%% space between body and thm
%% 		{\wedn\bfseries}                      %%% Thm body font
%% 		{}                              %%% Indent amount (empty = no indent)
%% 		{\wedn\bfseries}            %%% Thm head font
%% 		{}                              %%% Punctuation after thm head
%% 		{3ex}                           %%% Space after thm head
%% 		{\underline{\underline{\thmname{#1}}}}%%% Thm head spec
%% \theoremstyle{ResponseStyle}

\usepackage[tikz]{mdframed}
\mdfdefinestyle{ResponseStyle}{leftmargin=1cm,linecolor=black,roundcorner=5pt,frametitlefont=\wedn\bfseries,%frametitle={\underline{\underline{Response}}:}
, font=\wedn\bfseries,}%\begin{mdframed}[style=mystyle]foo\end{mdframed}


\ifhandout
\NewEnviron{freeResponse}{}
\else
%\newtheorem{freeResponse}{Response:}
\newenvironment{freeResponse}{\begin{mdframed}[style=ResponseStyle]}{\end{mdframed}}
\fi



%% attempting to automate outcomes.

\newwrite\outcomefile
  \immediate\openout\outcomefile=\jobname.oc
\renewcommand{\outcome}[1]{\edef\theoutcomes{\theoutcomes #1~}%
\immediate\write\outcomefile{\unexpanded{\outcome}{#1}}}

%% \newcommand{\outcomelist}{\begin{itemize}\theoutcomes\end{itemize}}



%% my commands

\newcommand{\snap}{{\bfseries\itshape\textsf{Snap!}}}
\newcommand{\flavor}{\link[\snap]{https://snap.berkeley.edu/}}


\usepackage{tkz-euclide}
\tikzstyle geometryDiagrams=[rounded corners=.5pt,ultra thick,color=black]
\colorlet{penColor}{black} % Color of a curve in a plot

\title{Rep-tiles}

\author{Bart Snapp}

\begin{document}
\begin{abstract}
  We will investigate rep-tiles.
\end{abstract}
\maketitle



\begin{listOutcomes}
\item Explain what a rep-tile is.
\item Give basic examples of rep-n-tiles.
\end{listOutcomes}


A \textbf{rep-tile}\index{rep-tile} is a polygon where several copies
of a given rep-tile fit together to make a larger, similar (MEANING
SCALED), version of itself. If $2$ copies are used, we call it a
\textit{rep-2-tile}, if $3$ copies are used, we call it a
\textit{rep-3-tile}, and if $n$ copies are used, we call it a
\textit{rep-n-tile}.

\mynewpage

\begin{question}
EXPLAIN why using words, pictures, computations, and so on, as
needed/helpful:
\begin{enumerate}
\item An isosceles right triangle whose sides have lengths $1''$,
  $1''$, and $\sqrt{2}''$ is a rep-2-tile.
\item A rectangle whose sides have lengths $1''$ and $\sqrt{2}''$ is a
  rep-2-tile.
\item Geometry Giorgio suggests that a rectangle whose sides have
  lengths $1''$ and $4''$ is also a rep-2-tile. Is he correct?
  \item Search the internet for other examples of rep-2-tiles, you
    might find a surprise. DISPLAY one of the other rep-2-tiles below.
    \begin{hint}
      You might want to include the word ``dragon'' in your
      search.
    \end{hint}
\end{enumerate}
\begin{freeResponse}
  \begin{enumerate}
  \item Behold:
  \item Behold:
  \item NO WAY!
  \item Here is a DRAGON rep-2-tile.
  \end{enumerate}
\end{freeResponse}
\end{question}
\mynewpage


\begin{question}
EXPLAIN why using words, pictures, and so on, as needed/helpful:
\begin{enumerate}
\item A rectangle whose sides have lengths $1''$ and $\sqrt{3}''$, is a rep-3-tile.
\item A 30-60-90 right triangle whose shortest side has length $1''$, is a rep-3-tile.
\end{enumerate}
\begin{freeResponse}
  \begin{enumerate}
  \item Behold:
  \item Behold:
  \end{enumerate}
\end{freeResponse}
\end{question}
\mynewpage



\begin{question}
EXPLAIN why EVERY triangle and EVERY parallelogram is a rep-4-tile.
DISPLAY a representative example of each.
\begin{freeResponse}
  \begin{enumerate}
  \item Behold:
  \item Behold:
  \end{enumerate}
\end{freeResponse}
\end{question}

\end{document}
