\documentclass[noauthor,nooutcomes,hints,handout]{ximera}

%% page layout
\usepackage[in,headings]{fullpage}
\raggedright
\setlength\headheight{13.6pt}


%% fonts
\usepackage{euler}

\usepackage{FiraMono}
\renewcommand\familydefault{\ttdefault} 
\usepackage{mathastext}
\usepackage[htt]{hyphenat}

\usepackage[T1]{fontenc}
\usepackage[scaled=1]{FiraSans}

\usepackage{wedn}
\usepackage[T1]{fontenc}

%% wrap text around scripts
\usepackage{wrapfig}

\tikzset{>=stealth}
%% snap! scripts
\usepackage{scratch3}

\usepackage{adjustbox}

%% journal style
\makeatletter
\newcommand\journalstyle{%
  \def\activitystyle{activity-chapter}
  \def\maketitle{%
    \addtocounter{titlenumber}{1}%
                {\flushleft\small\sffamily\bfseries\@pretitle\par\vspace{-1.5em}}%
                {\flushleft\LARGE\sffamily\bfseries\thetitlenumber\hspace{1em}\@title \par }%
                {\vskip .6em\noindent\textit\theabstract\setcounter{question}{0}\setcounter{sectiontitlenumber}{0}}%
                    \par\vspace{2em}
                    \phantomsection\addcontentsline{toc}{section}{\thetitlenumber\hspace{1em}\textbf{\@title}}%
                     }}
\makeatother



%% thm like environments
\let\question\relax
\let\endquestion\relax

\newtheoremstyle{QuestionStyle}{\topsep}{\topsep}%%% space between body and thm
		{}                      %%% Thm body font
		{}                              %%% Indent amount (empty = no indent)
		{\bfseries}            %%% Thm head font
		{)}                              %%% Punctuation after thm head
		{ }                           %%% Space after thm head
		{\thmnumber{#2}\thmnote{ \bfseries(#3)}}%%% Thm head spec
\theoremstyle{QuestionStyle}
\newtheorem{question}{}



\let\freeResponse\relax
\let\endfreeResponse\relax

%% \newtheoremstyle{ResponseStyle}{\topsep}{\topsep}%%% space between body and thm
%% 		{\wedn\bfseries}                      %%% Thm body font
%% 		{}                              %%% Indent amount (empty = no indent)
%% 		{\wedn\bfseries}            %%% Thm head font
%% 		{}                              %%% Punctuation after thm head
%% 		{3ex}                           %%% Space after thm head
%% 		{\underline{\underline{\thmname{#1}}}}%%% Thm head spec
%% \theoremstyle{ResponseStyle}

\usepackage[tikz]{mdframed}
\mdfdefinestyle{ResponseStyle}{leftmargin=1cm,linecolor=black,roundcorner=5pt,frametitlefont=\wedn\bfseries,%frametitle={\underline{\underline{Response}}:}
, font=\wedn\bfseries,}%\begin{mdframed}[style=mystyle]foo\end{mdframed}


\ifhandout
\NewEnviron{freeResponse}{}
\else
%\newtheorem{freeResponse}{Response:}
\newenvironment{freeResponse}{\begin{mdframed}[style=ResponseStyle]}{\end{mdframed}}
\fi



%% attempting to automate outcomes.

\newwrite\outcomefile
  \immediate\openout\outcomefile=\jobname.oc
\renewcommand{\outcome}[1]{\edef\theoutcomes{\theoutcomes #1~}%
\immediate\write\outcomefile{\unexpanded{\outcome}{#1}}}

%% \newcommand{\outcomelist}{\begin{itemize}\theoutcomes\end{itemize}}



%% my commands

\newcommand{\snap}{{\bfseries\itshape\textsf{Snap!}}}
\newcommand{\flavor}{\link[\snap]{https://snap.berkeley.edu/}}


\usepackage{tkz-euclide}
\tikzstyle geometryDiagrams=[rounded corners=.5pt,ultra thick,color=black]
\colorlet{penColor}{black} % Color of a curve in a plot


\title{Symmetries of the regular triangle}

\author{Bart Snapp}

\begin{document}
\begin{abstract}
  We explore the symmetries of the regular triangle.
\end{abstract}
\maketitle

\begin{listOutcomes}
\item Witness the symmetries of the regular triangle.
\item Describe symmetries of the regular triangle.
\item Think of the symmetries of the regular triangle as functions.
\item Use the algebra of symmetries to understand a composition of symmetries.
\end{listOutcomes}

\begin{listObjectives}
 \item Use mathematical language to describe symmetries.
 \item Increase student confidence in their ability to solve difficult math problems by using previous results, trying different methods, asking questions, and working with others.
\item Improve student’s mathematical communication skills.
\end{listObjectives}

\mynewpage
\begin{question}
  The \textbf{symmetries} of the regular triangle are those that
  \textbf{leave it unchanged}. So for example, here is a symmetry of
  the regular triangle:
  \[
  \raisebox{-.4\height}{\includegraphics[width=2.5in]{blackSymTri.png}}\resizebox{.7in}{!}{$\mapsto$} \raisebox{-.4\height}{\includegraphics[width=2.5in]{blackSymTri.png}}
  \]
  You may think nothing happened above, because the triangle is
  unchanged, but you would be wrong.  To witnness symmetries, we must
  color-in the sides of the triangle. For example once, I color in the
  sides we see that the symmetry above is actually:
  \[
  \raisebox{-.4\height}{\includegraphics[width=2.5in]{eTri.png}}\resizebox{.7in}{!}{$\mapsto$} \raisebox{-.4\height}{\includegraphics[width=2.5in]{rfTri.png}}
  \]
  While the colors change, the actual triangle is \textbf{unchanged}.
  \begin{enumerate}
  \item What physical action would cause the triangle to change the colors the way?
  \item What other physical actions are symmetries of an equilateral triangle?
  \item How many symmetries does the equilateral triangle have?
   \end{enumerate}
  \begin{freeResponse}
    \begin{enumerate}
    \item There are SIX symmetries of the regular triangle.
    \item Here they are:
    \begin{enumerate}
    \item $\raisebox{-.4\height}{\includegraphics[width=2in]{eTri.png}}\resizebox{.5in}{!}{$\mapsto$} \raisebox{-.4\height}{\includegraphics[width=2in]{eTri.png}}$
    \item $\raisebox{-.4\height}{\includegraphics[width=2in]{eTri.png}}\resizebox{.5in}{!}{$\mapsto$} \raisebox{-.4\height}{\includegraphics[width=2in]{rTri.png}}$
    \item $\raisebox{-.4\height}{\includegraphics[width=2in]{eTri.png}}\resizebox{.5in}{!}{$\mapsto$} \raisebox{-.4\height}{\includegraphics[width=2in]{r2Tri.png}}$
    \item $\raisebox{-.4\height}{\includegraphics[width=2in]{eTri.png}}\resizebox{.5in}{!}{$\mapsto$} \raisebox{-.4\height}{\includegraphics[width=2in]{fTri.png}}$
    \item $\raisebox{-.4\height}{\includegraphics[width=2in]{eTri.png}}\resizebox{.5in}{!}{$\mapsto$} \raisebox{-.4\height}{\includegraphics[width=2in]{rfTri.png}}$
    \item $\raisebox{-.4\height}{\includegraphics[width=2in]{eTri.png}}\resizebox{.5in}{!}{$\mapsto$} \raisebox{-.4\height}{\includegraphics[width=2in]{r2fTri.png}}$
    \end{enumerate}
    \end{enumerate}
  \end{freeResponse}
\end{question}
\mynewpage

\begin{question}
  Let $e$ be the do-nothing symmetry, $r$ be a clockwise $120^\circ$
  rotation about the center of the triangle, and $f$ be a flip across
  a vertical line down the middle of the triangle. When $e$ is
  applied, the triangle looks like this:
  \[
  \raisebox{-.4\height}{\includegraphics[width=2.5in]{eTri.png}}\resizebox{.7in}{!}{$\overset{\scriptscriptstyle e}{\mapsto}$} \raisebox{-.4\height}{\includegraphics[width=2.5in]{eTri.png}}
  \]
  Display the result of applying the following functions
  (actions/transformations) to your triangle:
  \begin{enumerate}
  \item $rf$
  \item $fr$
  \item $r^2 f$
  \item $f r^2$
  \end{enumerate}
  In each case, show off your work using colors, numbering, or some other description.
  
  \textbf{WARNING:} Unlike multiplication with numbers, changing the order gives different answers.
  \begin{freeResponse}
    \begin{enumerate}
    \item For $rf$ I have
      \[
      \raisebox{-.4\height}{\includegraphics[width=2.5in]{eTri.png}}\resizebox{.7in}{!}{$\overset{\scriptscriptstyle rf}{\mapsto}$} \raisebox{-.4\height}{\includegraphics[width=2.5in]{rfTri.png}}
      \]
    \item For $fr$ I have
      \[
      \raisebox{-.4\height}{\includegraphics[width=2.5in]{eTri.png}}\resizebox{.7in}{!}{$\overset{\scriptscriptstyle fr}{\mapsto}$} \raisebox{-.4\height}{\includegraphics[width=2.5in]{r2fTri.png}}
      \]
    \item For $r^2f$ I have
      \[
      \raisebox{-.4\height}{\includegraphics[width=2.5in]{eTri.png}}\resizebox{.7in}{!}{$\overset{\scriptscriptstyle r^2f}{\mapsto}$} \raisebox{-.4\height}{\includegraphics[width=2.5in]{r2fTri.png}}
      \]
    \item For $fr^2$ I have
      \[
      \raisebox{-.4\height}{\includegraphics[width=2.5in]{eTri.png}}\resizebox{.7in}{!}{$\overset{\scriptscriptstyle  fr^2}{\mapsto}$} \raisebox{-.4\height}{\includegraphics[width=2.5in]{rfTri.png}}
      \]
    \end{enumerate}
    \end{freeResponse}
\end{question}
\mynewpage


\begin{question}
  Do you remember the multiplication tables? We can make a similar
  table for the symmetries of the regular triangle. Here, I've
  started it for you:
  \[
  \begin{array}{|c||c|c|c|c|c|c|}
    \hline
      & e & r & r^2 & f & rf & r^2f\\ \hline\hline
    e & e & r & r^2 & f & rf & r^2f\\ \hline
    r & r & r^2 & \color{red}r^3 & rf & r^2f & \color{red}r^3f\\ \hline
    r^2 & r^2 & \color{red}r^3 & \color{red}r^4 & r^2f & \color{red}r^3f &\color{red} r^4f\\ \hline
    f  & f & \color{red}fr & \color{red}fr^2 & \color{red}f^2 & \color{red}frf & \color{red}fr^2f\\ \hline
    rf & rf & \color{red}rfr & \color{red}rfr^2 & \color{red}rf^2 & \color{red}rfrf & \color{red}rfr^2f\\ \hline
    r^2f & r^2f & \color{red}r^2fr & \color{red}r^2fr^2 & \color{red}r^2f^2 &\color{red} r^2frf &\color{red} r^2fr^2f\\ \hline
  \end{array}
  \]
  Each of the symmetries in red is actually equal to one of the
  symmetries in top row (or left-most column)
  \[
  e,\quad r,\quad r^2,\quad f,\quad rf,\quad r^2f
  \]
  where $e$ is the ``do-nothing'' symmetry.
  \begin{enumerate}
    \item Simplify each of the red symmetries as one of
      $e,r,r^2,f,rf,r^2f$ and update the table with these values.
    \item Use your table to express
      \[
      f^3r^5f^7r^9
      \]
      as one of $e,r,r^2,f,rf,r^2f$.
  \end{enumerate}
  \begin{freeResponse}
      \begin{enumerate}
      \item
      Here is my updated table:
       \[
  \begin{array}{|c||c|c|c|c|c|c|}
    \hline
    & e & r & r^2 & f & rf & r^2f\\ \hline\hline
    e & e & r & r^2 & f & rf & r^2f\\ \hline
    r & r & r^2 & e & rf & r^2f & f\\ \hline
    r^2 & r^2 & e & r & r^2f & f &rf\\ \hline
    f  & f & r^2f & rf & e & r^2 & r\\ \hline
    rf & rf & f & r^2f & r & e & r^2\\ \hline
    r^2f & r^2f & rf & f & r^2 & r & e\\ \hline
  \end{array}
  \]
\item Using the table I see
  \begin{align*}
    f^3r^5f^7r^9 &= fr^2f\\
    &= rf^2\\
    &= r.
  \end{align*}
    \end{enumerate}
  \end{freeResponse}
\end{question}
\end{document}
