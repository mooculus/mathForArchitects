\documentclass[noauthor,nooutcomes,12pt]{ximera}

%% page layout
\usepackage[in,headings]{fullpage}
\raggedright
\setlength\headheight{13.6pt}


%% fonts
\usepackage{euler}

\usepackage{FiraMono}
\renewcommand\familydefault{\ttdefault} 
\usepackage{mathastext}
\usepackage[htt]{hyphenat}

\usepackage[T1]{fontenc}
\usepackage[scaled=1]{FiraSans}

\usepackage{wedn}
\usepackage[T1]{fontenc}

%% wrap text around scripts
\usepackage{wrapfig}

\tikzset{>=stealth}
%% snap! scripts
\usepackage{scratch3}

\usepackage{adjustbox}

%% journal style
\makeatletter
\newcommand\journalstyle{%
  \def\activitystyle{activity-chapter}
  \def\maketitle{%
    \addtocounter{titlenumber}{1}%
                {\flushleft\small\sffamily\bfseries\@pretitle\par\vspace{-1.5em}}%
                {\flushleft\LARGE\sffamily\bfseries\thetitlenumber\hspace{1em}\@title \par }%
                {\vskip .6em\noindent\textit\theabstract\setcounter{question}{0}\setcounter{sectiontitlenumber}{0}}%
                    \par\vspace{2em}
                    \phantomsection\addcontentsline{toc}{section}{\thetitlenumber\hspace{1em}\textbf{\@title}}%
                     }}
\makeatother



%% thm like environments
\let\question\relax
\let\endquestion\relax

\newtheoremstyle{QuestionStyle}{\topsep}{\topsep}%%% space between body and thm
		{}                      %%% Thm body font
		{}                              %%% Indent amount (empty = no indent)
		{\bfseries}            %%% Thm head font
		{)}                              %%% Punctuation after thm head
		{ }                           %%% Space after thm head
		{\thmnumber{#2}\thmnote{ \bfseries(#3)}}%%% Thm head spec
\theoremstyle{QuestionStyle}
\newtheorem{question}{}



\let\freeResponse\relax
\let\endfreeResponse\relax

%% \newtheoremstyle{ResponseStyle}{\topsep}{\topsep}%%% space between body and thm
%% 		{\wedn\bfseries}                      %%% Thm body font
%% 		{}                              %%% Indent amount (empty = no indent)
%% 		{\wedn\bfseries}            %%% Thm head font
%% 		{}                              %%% Punctuation after thm head
%% 		{3ex}                           %%% Space after thm head
%% 		{\underline{\underline{\thmname{#1}}}}%%% Thm head spec
%% \theoremstyle{ResponseStyle}

\usepackage[tikz]{mdframed}
\mdfdefinestyle{ResponseStyle}{leftmargin=1cm,linecolor=black,roundcorner=5pt,frametitlefont=\wedn\bfseries,%frametitle={\underline{\underline{Response}}:}
, font=\wedn\bfseries,}%\begin{mdframed}[style=mystyle]foo\end{mdframed}


\ifhandout
\NewEnviron{freeResponse}{}
\else
%\newtheorem{freeResponse}{Response:}
\newenvironment{freeResponse}{\begin{mdframed}[style=ResponseStyle]}{\end{mdframed}}
\fi



%% attempting to automate outcomes.

\newwrite\outcomefile
  \immediate\openout\outcomefile=\jobname.oc
\renewcommand{\outcome}[1]{\edef\theoutcomes{\theoutcomes #1~}%
\immediate\write\outcomefile{\unexpanded{\outcome}{#1}}}

%% \newcommand{\outcomelist}{\begin{itemize}\theoutcomes\end{itemize}}



%% my commands

\newcommand{\snap}{{\bfseries\itshape\textsf{Snap!}}}
\newcommand{\flavor}{\link[\snap]{https://snap.berkeley.edu/}}


\usepackage{tkz-euclide}
\tikzstyle geometryDiagrams=[rounded corners=.5pt,ultra thick,color=black]
\colorlet{penColor}{black} % Color of a curve in a plot


\title{Angles and right triangles}
\author{Bart Snapp}

\begin{document}
\begin{abstract}
  We learn about Pythagorean triples.
\end{abstract}
\maketitle

\begin{listOutcomes}
\item Something
\end{listOutcomes}
\mynewpage


\begin{question}
  Use the INTERNET to look up \textbf{Pythagorean triples}. EXPLAIN
  what a Pythagorean triple is as you would like to have them
  explained to you. In particular, as part of this explanation:
  \begin{itemize}
    \item Explain what a \textbf{primitive} Pythagorean triple is.
    \item Give $3$ EXAMPLES of nonprimitive Pythagorean triples and
      $3$ more EXAMPLES of primitive Pythagorean triples.
    \item Someone once said, ``every Pythagorean triple is
      \textbf{similar} to a primitive Pythagorean triple.'' EXPLAIN
      what this means.
  \end{itemize}
  \begin{freeResponse}
    A \underline{Pythagorean triple} is a list of three whole numbers
    $(a,b,c)$ where if $c$ is the largest such number,
    \[
    a^2 + b^2 = c^2.
    \]
    This means, from our previous work, that there is a right triangle
    with these whole number side lengths.
  \end{freeResponse}
\end{question}
\mynewpage

\begin{question}
  Here are $3$ Pythagorean triples:
  \[
  (60,80,100), \qquad (50,120,130), \qquad (120, 209, 241).
  \]
  In each case:
  \begin{enumerate}
        \item Use the bisection method to find the angles of these
          triangles. Display your work with a TABLE as we have done
          before.
        \item Make \snap\ scripts that will draw these
          triangles. Show off your work by giving screenshots of your
          SCRIPTS and STAGES as provided from \snap.
  \end{enumerate}
  \begin{freeResponse}
    \begin{description}
      \item[\underline{(60,80,100)}:] blahs
    \end{description}
  \end{freeResponse}
\end{question}
\mynewpage

\begin{question}
  Give an algorithm for using bisection method to find the angles of a
  right triangle, assuming you are given the side lengths.
  \begin{freeResponse}
    \begin{enumerate}
    \item Find the sum of the two known sides. This is the maximum
      length the third side could be, call it $M$.
    \item Let $0$ be the minimum length the third side could be. Call this number $m$.
    \item\label{I:stop} If $M=m$ or $|M-m|< small-number$ we are done, so STOP.
    \item Now, it must be that $M>m$. Average $M$ and $m$ and call it $a$.
    \item Is $a$ the correct length? If so, then we are done, so STOP.
    \item If $a$ is too long, change $M$ to make it equal to $a$.
    \item If $a$ is too short, change $m$ to make it equal to $a$.
    \item Goto step~\ref{I:stop}.
    \end{enumerate}
  \end{freeResponse}
\end{question}





\end{document}
