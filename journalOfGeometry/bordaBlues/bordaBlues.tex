\documentclass[nooutcomes,noauthor,hints,handout]{ximera}

\graphicspath{  
{./}
{./whoAreYou/}
{./drawingWithTheTurtle/}
{./bisectionMethod/}
{./circles/}
{./anglesAndRightTriangles/}
{./lawOfSines/}
{./lawOfCosines/}
{./plotter/}
{./staircases/}
{./pitch/}
{./qualityControl/}
{./symmetry/}
{./nGonBlock/}
}


%% page layout
\usepackage[cm,headings]{fullpage}
\raggedright
\setlength\headheight{13.6pt}


%% fonts
\usepackage{euler}

\usepackage{FiraMono}
\renewcommand\familydefault{\ttdefault} 
\usepackage[defaultmathsizes]{mathastext}
\usepackage[htt]{hyphenat}

\usepackage[T1]{fontenc}
\usepackage[scaled=1]{FiraSans}

%\usepackage{wedn}
\usepackage{pbsi} %% Answer font


\usepackage{cancel} %% strike through in pitch/pitch.tex


%% \usepackage{ulem} %% 
%% \renewcommand{\ULthickness}{2pt}% changes underline thickness

\tikzset{>=stealth}

\usepackage{adjustbox}

\setcounter{titlenumber}{-1}

%% journal style
\makeatletter
\newcommand\journalstyle{%
  \def\activitystyle{activity-chapter}
  \def\maketitle{%
    \addtocounter{titlenumber}{1}%
                {\flushleft\small\sffamily\bfseries\@pretitle\par\vspace{-1.5em}}%
                {\flushleft\LARGE\sffamily\bfseries\thetitlenumber\hspace{1em}\@title \par }%
                {\vskip .6em\noindent\textit\theabstract\setcounter{question}{0}\setcounter{sectiontitlenumber}{0}}%
                    \par\vspace{2em}
                    \phantomsection\addcontentsline{toc}{section}{\thetitlenumber\hspace{1em}\textbf{\@title}}%
                     }}
\makeatother



%% thm like environments
\let\question\relax
\let\endquestion\relax

\newtheoremstyle{QuestionStyle}{\topsep}{\topsep}%%% space between body and thm
		{}                      %%% Thm body font
		{}                              %%% Indent amount (empty = no indent)
		{\bfseries}            %%% Thm head font
		{)}                              %%% Punctuation after thm head
		{ }                           %%% Space after thm head
		{\thmnumber{#2}\thmnote{ \bfseries(#3)}}%%% Thm head spec
\theoremstyle{QuestionStyle}
\newtheorem{question}{}



\let\freeResponse\relax
\let\endfreeResponse\relax

%% \newtheoremstyle{ResponseStyle}{\topsep}{\topsep}%%% space between body and thm
%% 		{\wedn\bfseries}                      %%% Thm body font
%% 		{}                              %%% Indent amount (empty = no indent)
%% 		{\wedn\bfseries}            %%% Thm head font
%% 		{}                              %%% Punctuation after thm head
%% 		{3ex}                           %%% Space after thm head
%% 		{\underline{\underline{\thmname{#1}}}}%%% Thm head spec
%% \theoremstyle{ResponseStyle}

\usepackage[tikz]{mdframed}
\mdfdefinestyle{ResponseStyle}{leftmargin=1cm,linecolor=black,roundcorner=5pt,
, font=\bsifamily,}%font=\wedn\bfseries\upshape,}


\ifhandout
\NewEnviron{freeResponse}{}
\else
%\newtheorem{freeResponse}{Response:}
\newenvironment{freeResponse}{\begin{mdframed}[style=ResponseStyle]}{\end{mdframed}}
\fi



%% attempting to automate outcomes.

%% \newwrite\outcomefile
%%   \immediate\openout\outcomefile=\jobname.oc
%% \renewcommand{\outcome}[1]{\edef\theoutcomes{\theoutcomes #1~}%
%% \immediate\write\outcomefile{\unexpanded{\outcome}{#1}}}

%% \newcommand{\outcomelist}{\begin{itemize}\theoutcomes\end{itemize}}

%% \NewEnviron{listOutcomes}{\small\sffamily
%% After answering the following questions, students should be able to:
%% \begin{itemize}
%% \BODY
%% \end{itemize}
%% }
\usepackage[tikz]{mdframed}
\mdfdefinestyle{OutcomeStyle}{leftmargin=2cm,rightmargin=2cm,linecolor=black,roundcorner=5pt,
, font=\small\sffamily,}%font=\wedn\bfseries\upshape,}
\newenvironment{listOutcomes}{\begin{mdframed}[style=OutcomeStyle]After answering the following questions, students should be able to:\begin{itemize}}{\end{itemize}\end{mdframed}}



%% my commands

\newcommand{\snap}{{\bfseries\itshape\textsf{Snap!}}}
\newcommand{\flavor}{\link[\snap]{https://snap.berkeley.edu/}}
\newcommand{\mooculus}{\textsf{\textbf{MOOC}\textnormal{\textsf{ULUS}}}}


\usepackage{tkz-euclide}
\tikzstyle geometryDiagrams=[rounded corners=.5pt,ultra thick,color=black]
\colorlet{penColor}{black} % Color of a curve in a plot



\ifhandout\newcommand{\mynewpage}{\newpage}\else\newcommand{\mynewpage}{}\fi

\title{Borda blues and other impossibilities}

\author{Bart Snapp}

\begin{document}
\begin{abstract}
  The subtleties of voting are strange.
\end{abstract}
\maketitle

\begin{listOutcomes}
\item Critically analyze data.
\item Witness subtleties of voting systems through examples.
\item Accommodate new knowledge from a primary source. 
\end{listOutcomes}



\mynewpage

\begin{question}
The Borda Count is pretty cool, but it does have some flaws. One of
which is that adding irrelevant candidates can change the result of an
election. Consider our \mooculus~apartment election. What if we add an additional candidate $X$ where:

\begin{itemize}
\item $16$ residents prefer dogs, better than cats, better than no pets at all, better than $X$.
\item $4$ residents prefer cats, better than dogs, better than no pets at all, better than $X$.
\item $18$ residents prefer no pets at all, better than $X$, better than cats, better
  than dogs.
\end{itemize}
Now answer these questions:
\begin{enumerate}
\item Who wins the Borda Count before candidate $X$ is introduced?
\item Who wins the Borda Count after candidate $X$ is introduced? (note,
  now the options are awarded $3,2,1,0$ points)
\item Did any of the residents change their order of preference in
  regards to your answers in the two questions above?
\item Why is this strange? Demonstrate understanding through your
  explanation.
\end{enumerate}
\end{question}
\mynewpage







\begin{question}
  In the mid 20th century, Kenneth Arrow proved the following theorem in his doctoral thesis.
 \begin{mdframed}[style=OutcomeStyle]
  \begin{quote}
    Suppose that there is a voting system where:
    \begin{itemize}
    \item If we exclude all candidates except for $X$ and $Y$, then
      the groups preference for $X$ and $Y$ should be unchanged when
      all candidates are considered. This is known as \textit{Pareto
        Efficiency}.
    \item If voters prefer $X$ to $Y$, and more candidates are added
      with their rankings interleaved with the previous, the groups
      preference of $X$ to $Y$ is unchanged. This is known as
      \textit{Independence of Irrelevant Alternatives}.
    \end{itemize}
    \textbf{Arrow's theorem} says that in this case,
    there is a single voter who can manipulate the entire result of
    the election, simply by changing their own vote.  That voter is
    referred to as the ``dictator.''
  \end{quote}
 \end{mdframed}
This theorem is a bit difficult to understand, so let's think in terms
of our example:
\begin{itemize}
\item $16$ residents prefer dogs, better than cats, better than no pets at all.
\item $4$ residents prefer cats, better than dogs, better than no pets at all.
\item $18$ residents prefer no pets at all, better than cats, better
  than dogs.
\end{itemize}
In particular, we'll explore why the theorem does NOT apply for the cases we consider. 
\begin{enumerate}
\item Who wins if the only options were cats versus dogs? What about
  cats versus no pets?
\item Explain why Arrow's theorem does \textbf{not apply to Plurality Voting} and
  why Arrow's theorem does \textbf{not apply to Instant Run-Off Voting}.
\item Now explain why Arrow's theorem does \textbf{not apply to the Borda Count}.
\end{enumerate}

\end{question}
\mynewpage



\begin{question}
  Arrow's theorem is often called an ``impossibility'' theorem because it
  can be stated as:
  \begin{quote}
    It's impossible to have a voting scheme that is Pareto Efficient,
    Independent of Irrelevant Alternatives, with no dictators.
  \end{quote}
  Not to be outdone, Professor Mixion of The Ohio State University
  coauthored the following paper: \link[\textit{An Impossibility
      Theorem for
      Gerrymandering}]{https://www.tandfonline.com/doi/full/10.1080/00029890.2018.1517571}.


  Explain the statement of this so-called impossibility theorem for
  gerrymandering in an accurate, but easy to understand way, in your
  own words. To do this correctly you'll need to be specific and
  explain quite a bit. However, all the answers are in Mixion and
  Alexeev's paper, and you are more than equipped to find them!
  \begin{hint}
    As part of your explanation, you will need to use/define/explain
    the terms: Districting system; one person, one vote; Polsby-Popper
    compactness; isoperimetric; bounded efficiency gap.
  \end{hint}
\end{question}





\end{document}
