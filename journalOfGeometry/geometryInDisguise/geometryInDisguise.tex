\documentclass[noauthor,nooutcomes]{ximera}

%% page layout
\usepackage[in,headings]{fullpage}
\raggedright
\setlength\headheight{13.6pt}


%% fonts
\usepackage{euler}

\usepackage{FiraMono}
\renewcommand\familydefault{\ttdefault} 
\usepackage{mathastext}
\usepackage[htt]{hyphenat}

\usepackage[T1]{fontenc}
\usepackage[scaled=1]{FiraSans}

\usepackage{wedn}
\usepackage[T1]{fontenc}

%% wrap text around scripts
\usepackage{wrapfig}

\tikzset{>=stealth}
%% snap! scripts
\usepackage{scratch3}

\usepackage{adjustbox}

%% journal style
\makeatletter
\newcommand\journalstyle{%
  \def\activitystyle{activity-chapter}
  \def\maketitle{%
    \addtocounter{titlenumber}{1}%
                {\flushleft\small\sffamily\bfseries\@pretitle\par\vspace{-1.5em}}%
                {\flushleft\LARGE\sffamily\bfseries\thetitlenumber\hspace{1em}\@title \par }%
                {\vskip .6em\noindent\textit\theabstract\setcounter{question}{0}\setcounter{sectiontitlenumber}{0}}%
                    \par\vspace{2em}
                    \phantomsection\addcontentsline{toc}{section}{\thetitlenumber\hspace{1em}\textbf{\@title}}%
                     }}
\makeatother



%% thm like environments
\let\question\relax
\let\endquestion\relax

\newtheoremstyle{QuestionStyle}{\topsep}{\topsep}%%% space between body and thm
		{}                      %%% Thm body font
		{}                              %%% Indent amount (empty = no indent)
		{\bfseries}            %%% Thm head font
		{)}                              %%% Punctuation after thm head
		{ }                           %%% Space after thm head
		{\thmnumber{#2}\thmnote{ \bfseries(#3)}}%%% Thm head spec
\theoremstyle{QuestionStyle}
\newtheorem{question}{}



\let\freeResponse\relax
\let\endfreeResponse\relax

%% \newtheoremstyle{ResponseStyle}{\topsep}{\topsep}%%% space between body and thm
%% 		{\wedn\bfseries}                      %%% Thm body font
%% 		{}                              %%% Indent amount (empty = no indent)
%% 		{\wedn\bfseries}            %%% Thm head font
%% 		{}                              %%% Punctuation after thm head
%% 		{3ex}                           %%% Space after thm head
%% 		{\underline{\underline{\thmname{#1}}}}%%% Thm head spec
%% \theoremstyle{ResponseStyle}

\usepackage[tikz]{mdframed}
\mdfdefinestyle{ResponseStyle}{leftmargin=1cm,linecolor=black,roundcorner=5pt,frametitlefont=\wedn\bfseries,%frametitle={\underline{\underline{Response}}:}
, font=\wedn\bfseries,}%\begin{mdframed}[style=mystyle]foo\end{mdframed}


\ifhandout
\NewEnviron{freeResponse}{}
\else
%\newtheorem{freeResponse}{Response:}
\newenvironment{freeResponse}{\begin{mdframed}[style=ResponseStyle]}{\end{mdframed}}
\fi



%% attempting to automate outcomes.

\newwrite\outcomefile
  \immediate\openout\outcomefile=\jobname.oc
\renewcommand{\outcome}[1]{\edef\theoutcomes{\theoutcomes #1~}%
\immediate\write\outcomefile{\unexpanded{\outcome}{#1}}}

%% \newcommand{\outcomelist}{\begin{itemize}\theoutcomes\end{itemize}}



%% my commands

\newcommand{\snap}{{\bfseries\itshape\textsf{Snap!}}}
\newcommand{\flavor}{\link[\snap]{https://snap.berkeley.edu/}}


\usepackage{tkz-euclide}
\tikzstyle geometryDiagrams=[rounded corners=.5pt,ultra thick,color=black]
\colorlet{penColor}{black} % Color of a curve in a plot


\title{Geometry in disguise}
\author{Bart Snapp}

\begin{document}
\begin{abstract}
  We'll solve an algebra problem using geometry.
\end{abstract}
\maketitle

\begin{listOutcomes}
\item Apply the Pythagorean theorem to solve problems.
\item Apply the distance formula to solve problems.
\item Extend ideas from two-dimensions to three-dimensions.
\item Transform an algebra problem into a geometry problem.
\end{listOutcomes}

I was browsing \textsl{Facebook} when I saw:
\begin{center}
  \includegraphics[width=.6\textwidth]{fbQuestion.png}
\end{center}


Two ``replies'' that caught my eye:
\[
a = 2\quad b=4\quad c=0
\]
and
\begin{quote}
  It is unsolvable, so the math is a waste of time, besides letting
  me use my brain.
\end{quote}
\textbf{Neither of these replies is correct.} Let's crack this nut!


\mynewpage
  
\begin{question}
  Let $(a,b,c)$ be a point in three-dimensional space.
  \begin{enumerate}
  \item In three dimensions, what is the distance of the point
    $(a,b,c)$ from the origin $(0,0,0)$?
  \item EXPLAIN, using words, pictures, and so on, how one could find
    the distance from the point $(a,b,c)$ to the origin using the
    Pythagorean theorem.
  \end{enumerate}
  \begin{freeResponse}
    \begin{enumerate}
    \item The distance from $(a,b,c)$ to the origin is
      \[
      \sqrt{a^2+b^2+c^2}.
      \]
    \item To see why this is true, draw a picture!
    \begin{center}
      \begin{tikzpicture}
        \begin{axis}[
            axis lines=none,
            clip=false,
            width=5in,
            height=2.5in,
          ]          
          \addplot [->,textColor] plot coordinates {(0,0) (-2,-4)}; %% x axis
          \addplot [->,textColor] plot coordinates {(0,0) (4,0)}; %% y axis
          \addplot [->,textColor] plot coordinates {(0,0) (0,6)}; %% z axis
                   
          
          \addplot [dashed] plot coordinates {(-1.3,-2.6) (2.6,-2.6)}; %%bottom dashed line
          \addplot [dashed] plot coordinates {(2.6,-2.6) (3.5,0)}; %% bottom right dashed
          \addplot [dashed] plot coordinates {(3.5,0) (3.5,5.5)};  %% back right vert dashed
          \addplot [dashed] plot coordinates {(2.6,-2.6) (2.6,2.9)}; %% front right vert dashed
          \addplot [dashed] plot coordinates {(3.5,5.5) (2.6,2.9)}; %% top right dashed
          \addplot [dashed] plot coordinates {(0,5.5) (-1.3,2.9)}; %% top left dashed
          \addplot [dashed] plot coordinates {(-1.3,-2.6) (-1.3,2.9)}; %% left vert front
          \addplot [dashed] plot coordinates {(-1.3,2.9) (2.6,2.9)}; %% top horz
          \addplot [dashed] plot coordinates {(0,5.5) (3.5,5.5)}; %% top horz rear
          
          \node at (axis cs:-1.3,-2.6) [anchor=east] {$a$};          
          \node at (axis cs:3.5,0) [anchor=south west] {$b$};
          \node at (axis cs:0,5.5) [anchor=south east] {$c$};
         
          \addplot[black,only marks,mark=*] coordinates{(2.6,2.9)};  %% closed hole

         
          \node[black,above right]  at (axis cs: 2.5,2.9) {$(a,b,c)$};

          
        \end{axis}
      \end{tikzpicture}
    \end{center}
    Ignoring the vertical components of the points, we can make a
    right-triangle whose legs have lengths $a$ and $b$.
    \begin{center}
      \begin{tikzpicture}
        \begin{axis}[
            axis lines=none,
            clip=false,
            width=5in,
            height=2.5in,
          ]          
          \addplot [->,black] plot coordinates {(0,0) (-2,-4)}; %% x axis
          \addplot [->,black] plot coordinates {(0,0) (4,0)}; %% y axis
          \addplot [->,black] plot coordinates {(0,0) (0,6)}; %% z axis
                   
          %% \addplot [dashed] plot coordinates {(-.7,-1.4) (1.4,-1.4)}; %%bottom dashed line
          %% \addplot [dashed] plot coordinates {(1.4,-1.4) (2.1,0)}; %% bottom right dashed
          %% \addplot [dashed] plot coordinates {(2.1,0) (2.1,4.1)};  %% back right vert dashed
          %% \addplot [dashed] plot coordinates {(1.4,-1.4) (1.4,2.7)}; %% front right vert dashed
          %% \addplot [dashed] plot coordinates {(2.1,4.1) (1.4,2.7)}; %% top right dashed
          %% \addplot [dashed] plot coordinates {(0,4.1) (-.7,2.7)}; %% top left dashed
          %% \addplot [dashed] plot coordinates {(-.7,-1.4) (-.7,2.7)}; %% left vert front
          %% \addplot [dashed] plot coordinates {(-.7,2.7) (1.4,2.7)}; %% top horz
          %% \addplot [dashed] plot coordinates {(0,4.1) (2.1,4.1)}; %% top horz rear

          \addplot [dashed] plot coordinates {(-1.3,-2.6) (2.6,-2.6)}; %%bottom dashed line
          \addplot [dashed] plot coordinates {(2.6,-2.6) (3.5,0)}; %% bottom right dashed
          \addplot [dashed] plot coordinates {(3.5,0) (3.5,5.5)};  %% back right vert dashed
          \addplot [dashed] plot coordinates {(2.6,-2.6) (2.6,2.9)}; %% front right vert dashed
          \addplot [dashed] plot coordinates {(3.5,5.5) (2.6,2.9)}; %% top right dashed
          \addplot [dashed] plot coordinates {(0,5.5) (-1.3,2.9)}; %% top left dashed
          \addplot [dashed] plot coordinates {(-1.3,-2.6) (-1.3,2.9)}; %% left vert front
          \addplot [dashed] plot coordinates {(-1.3,2.9) (2.6,2.9)}; %% top horz
          \addplot [dashed] plot coordinates {(0,5.5) (3.5,5.5)}; %% top horz rear

          \addplot [ultra thick] plot coordinates {(-1.3,-2.6) (0,0)}; %% left side bottom tri
          \addplot [ultra thick] plot coordinates {(-1.3,-2.6) (2.6,-2.6)}; %% bottom side bottom tri
          \addplot [ultra thick] plot coordinates {(0,0) (2.6,-2.6)}; %% hyp bottom tri
          %\addplot [thick] plot coordinates {(0.8,-2.6) (1,-2.2) (1.4,-2.2) (1.2,-2.6)}; %% 90 deg box
          \addplot [thick] plot coordinates {(-1.3,-2.6) (-1.1,-2.2) (-.7,-2.2) (-.9,-2.6)}; %% 90 deg box


          \node at (axis cs:-1.3,-2.6) [anchor=east] {$a$};          
          \node at (axis cs:3.5,0) [anchor=south west] {$b$};
          \node at (axis cs:0,5.5) [anchor=south east] {$c$};

          
          
          
          \addplot[black,only marks,mark=*] coordinates{(2.6,2.9)};  %% closed hole
          \node[black,above right]  at (axis cs: 2.5,2.9) {$(a,b,c)$};
          
        \end{axis}
      \end{tikzpicture}
    \end{center}
    Thus by the Pythagorean Theorem, the length of the hypotenuse of
    the right-triangle above is:
    \[
    \sqrt{a^2+b^2}
    \]
    Now, moving this segment up, we can form another right-triangle
    where the legs have lengths $\sqrt{a^2+b^2}$ and $c$:
    \begin{center}
      \begin{tikzpicture}
        \begin{axis}[
            axis lines=none,
            clip=false,
            width=5in,
            height=2.5in,
          ]          
          \addplot [->,black] plot coordinates {(0,0) (-2,-4)}; %% x axis
          \addplot [->,black] plot coordinates {(0,0) (4,0)}; %% y axis
          \addplot [->,black] plot coordinates {(0,0) (0,6)}; %% z axis
                   

          \addplot [dashed] plot coordinates {(-1.3,-2.6) (2.6,-2.6)}; %%bottom dashed line
          \addplot [dashed] plot coordinates {(2.6,-2.6) (3.5,0)}; %% bottom right dashed
          \addplot [dashed] plot coordinates {(3.5,0) (3.5,5.5)};  %% back right vert dashed
          \addplot [dashed] plot coordinates {(2.6,-2.6) (2.6,2.9)}; %% front right vert dashed
          \addplot [dashed] plot coordinates {(3.5,5.5) (2.6,2.9)}; %% top right dashed
          \addplot [dashed] plot coordinates {(0,5.5) (-1.3,2.9)}; %% top left dashed
          \addplot [dashed] plot coordinates {(-1.3,-2.6) (-1.3,2.9)}; %% left vert front
          \addplot [dashed] plot coordinates {(-1.3,2.9) (2.6,2.9)}; %% top horz
          \addplot [dashed] plot coordinates {(0,5.5) (3.5,5.5)}; %% top horz rear

          \addplot [ultra thick,gray] plot coordinates {(-1.3,-2.6) (0,0)}; %% left side bottom tri
          \addplot [ultra thick,gray] plot coordinates {(-1.3,-2.6) (2.6,-2.6)}; %% bottom side bottom tri
          \addplot [ultra thick,gray] plot coordinates {(0,0) (2.6,-2.6)}; %% hyp bottom tri
          \addplot [thick,gray] plot coordinates {(-1.3,-2.6) (-1.1,-2.2) (-.7,-2.2) (-.9,-2.6)}; %% 90 deg box

          \addplot [ultra thick] plot coordinates {(0,0) (2.6,-2.6) (2.6,2.9) (0,0)}; %% New tri
          %\addplot [thick] plot coordinates {(2.4,1.7) (2.4,2.2) (2.6,2)}; %% 90 deg box
          \addplot [thick] plot coordinates {(2.4,-2.3) (2.4,-1.8) (2.6,-2)}; %% 90 deg box
          
          \node at (axis cs:-1.3,-2.6) [anchor=east] {$a$};          
          \node at (axis cs:3.5,0) [anchor=south west] {$b$};
          \node at (axis cs:0,5.5) [anchor=south east] {$c$};

          \addplot[black,only marks,mark=*] coordinates{(2.6,2.9)};  %% closed hole


          \node[black,above right]  at (axis cs: 2.5,2.9) {$(a,b,c)$};
        \end{axis}
      \end{tikzpicture}
    \end{center}
    The length of the hypotenuse of this new triangle is the distance
    between $(a,b,c)$ and the origin,
    \begin{align*}
      \text{distance}^2&= \left(\sqrt{a^2+b^2}\right)^2 + ^2\\
       \text{distance}&=  \sqrt{a^2+b^2 + c^2}.
    \end{align*}
    \end{enumerate}
  \end{freeResponse}
\end{question}
\mynewpage


\begin{question}
  Consider all points $(x,y,z)$ such that
  \[
  x^2 + y^2 + z^2 = 18.
  \]
  Describe what this set of points looks like in three-dimensions.
  \begin{freeResponse}
    This is the set of points of distance $\sqrt{18}$ from the
    origin. It is a SPHERE.
  \end{freeResponse}
\end{question}
\mynewpage


\begin{question}
  Use GEOMETRY to solve the problem from \textsl{Facebook}.
  \begin{freeResponse}
    Drawing a picture, we see that this is the intersection of the sphere
      \[
      x^2 + y^2 + z^2 = 18.
      \]
      and the line $x+y=6$. Since the radius of the sphere is
      $\sqrt{18}$, we see that this is the only point on the line.

      IN THE FUTURE YOU SHOULD ADD A PICTURE.
  \end{freeResponse}
\end{question}





\end{document}
