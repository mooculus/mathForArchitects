\documentclass[noauthor,nooutcomes,12pt,handout,hints]{ximera}

\graphicspath{  
{./}
{./whoAreYou/}
{./drawingWithTheTurtle/}
{./bisectionMethod/}
{./circles/}
{./anglesAndRightTriangles/}
{./lawOfSines/}
{./lawOfCosines/}
{./plotter/}
{./staircases/}
{./pitch/}
{./qualityControl/}
{./symmetry/}
{./nGonBlock/}
}


%% page layout
\usepackage[cm,headings]{fullpage}
\raggedright
\setlength\headheight{13.6pt}


%% fonts
\usepackage{euler}

\usepackage{FiraMono}
\renewcommand\familydefault{\ttdefault} 
\usepackage[defaultmathsizes]{mathastext}
\usepackage[htt]{hyphenat}

\usepackage[T1]{fontenc}
\usepackage[scaled=1]{FiraSans}

%\usepackage{wedn}
\usepackage{pbsi} %% Answer font


\usepackage{cancel} %% strike through in pitch/pitch.tex


%% \usepackage{ulem} %% 
%% \renewcommand{\ULthickness}{2pt}% changes underline thickness

\tikzset{>=stealth}

\usepackage{adjustbox}

\setcounter{titlenumber}{-1}

%% journal style
\makeatletter
\newcommand\journalstyle{%
  \def\activitystyle{activity-chapter}
  \def\maketitle{%
    \addtocounter{titlenumber}{1}%
                {\flushleft\small\sffamily\bfseries\@pretitle\par\vspace{-1.5em}}%
                {\flushleft\LARGE\sffamily\bfseries\thetitlenumber\hspace{1em}\@title \par }%
                {\vskip .6em\noindent\textit\theabstract\setcounter{question}{0}\setcounter{sectiontitlenumber}{0}}%
                    \par\vspace{2em}
                    \phantomsection\addcontentsline{toc}{section}{\thetitlenumber\hspace{1em}\textbf{\@title}}%
                     }}
\makeatother



%% thm like environments
\let\question\relax
\let\endquestion\relax

\newtheoremstyle{QuestionStyle}{\topsep}{\topsep}%%% space between body and thm
		{}                      %%% Thm body font
		{}                              %%% Indent amount (empty = no indent)
		{\bfseries}            %%% Thm head font
		{)}                              %%% Punctuation after thm head
		{ }                           %%% Space after thm head
		{\thmnumber{#2}\thmnote{ \bfseries(#3)}}%%% Thm head spec
\theoremstyle{QuestionStyle}
\newtheorem{question}{}



\let\freeResponse\relax
\let\endfreeResponse\relax

%% \newtheoremstyle{ResponseStyle}{\topsep}{\topsep}%%% space between body and thm
%% 		{\wedn\bfseries}                      %%% Thm body font
%% 		{}                              %%% Indent amount (empty = no indent)
%% 		{\wedn\bfseries}            %%% Thm head font
%% 		{}                              %%% Punctuation after thm head
%% 		{3ex}                           %%% Space after thm head
%% 		{\underline{\underline{\thmname{#1}}}}%%% Thm head spec
%% \theoremstyle{ResponseStyle}

\usepackage[tikz]{mdframed}
\mdfdefinestyle{ResponseStyle}{leftmargin=1cm,linecolor=black,roundcorner=5pt,
, font=\bsifamily,}%font=\wedn\bfseries\upshape,}


\ifhandout
\NewEnviron{freeResponse}{}
\else
%\newtheorem{freeResponse}{Response:}
\newenvironment{freeResponse}{\begin{mdframed}[style=ResponseStyle]}{\end{mdframed}}
\fi



%% attempting to automate outcomes.

%% \newwrite\outcomefile
%%   \immediate\openout\outcomefile=\jobname.oc
%% \renewcommand{\outcome}[1]{\edef\theoutcomes{\theoutcomes #1~}%
%% \immediate\write\outcomefile{\unexpanded{\outcome}{#1}}}

%% \newcommand{\outcomelist}{\begin{itemize}\theoutcomes\end{itemize}}

%% \NewEnviron{listOutcomes}{\small\sffamily
%% After answering the following questions, students should be able to:
%% \begin{itemize}
%% \BODY
%% \end{itemize}
%% }
\usepackage[tikz]{mdframed}
\mdfdefinestyle{OutcomeStyle}{leftmargin=2cm,rightmargin=2cm,linecolor=black,roundcorner=5pt,
, font=\small\sffamily,}%font=\wedn\bfseries\upshape,}
\newenvironment{listOutcomes}{\begin{mdframed}[style=OutcomeStyle]After answering the following questions, students should be able to:\begin{itemize}}{\end{itemize}\end{mdframed}}



%% my commands

\newcommand{\snap}{{\bfseries\itshape\textsf{Snap!}}}
\newcommand{\flavor}{\link[\snap]{https://snap.berkeley.edu/}}
\newcommand{\mooculus}{\textsf{\textbf{MOOC}\textnormal{\textsf{ULUS}}}}


\usepackage{tkz-euclide}
\tikzstyle geometryDiagrams=[rounded corners=.5pt,ultra thick,color=black]
\colorlet{penColor}{black} % Color of a curve in a plot



\ifhandout\newcommand{\mynewpage}{\newpage}\else\newcommand{\mynewpage}{}\fi


\title{Arccosine} \author{Bart Snapp}

\begin{document}
\begin{abstract}
  Arccosine tells us the arc covered by an angle based on the value of
  cosine.
\end{abstract}
\maketitle

\begin{listOutcomes}
\item Understand that arccosine accepts a ratio of sides as input, and
  outputs the measure of an angle.
\item Understand that arccosine is the inverse function of cosine for
  angles between $0$ and $180$ degrees.
\item Recognize when arccosine and cosine undo each other, and also
  when they do not undo each other.
\end{listOutcomes}
\mynewpage




\begin{question}
  Here is a table of cosine and arccosine values:
  \[
  \begin{array}{|l|l|} \hline
    \cos(\theta)  = x     & \arccos(x) = \theta \\ \hline\hline
    \cos(0)  = 1     & \arccos(1) = 0 \\ \hline
    \cos(30) = 0.866 & \arccos(0.866) = 30\\ \hline
    \cos(60) = 0.5 & \arccos(0.5) = 60\\ \hline
    \cos(90) = 0 & \arccos(0) = 90\\ \hline
    \cos(120) = -0.5 & \arccos(-0.5) = 120\\ \hline
    \cos(150) = -0.866 & \arccos(-0.866) = 150\\ \hline
    \cos(180) =-1 & \arccos(-1) = 180  \\\hline
  \end{array}
  \]
  Explain what arccosine does when $0^\circ \le \theta \le 180^\circ$
  and $-1\le x\le 1$, as based on the table above.
  \begin{freeResponse}
    If you have an angle between $0^\circ$ and $180^\circ$ and $-1\le
    x\le 1$, then arccosine ``undoes'' cosine and vice versa. So if
    $0^\circ \le \theta \le 180^\circ$ and $-1\le x\le 1$,
    \[
    \cos(\arccos(x)) = x\qquad\text{and}\qquad \arccos(\cos(\theta)) =
    \theta.
    \]
  \end{freeResponse}
\end{question}
\mynewpage



\begin{question}
  Here are more values of cosine and arccosine:
  \[
  \begin{array}{|l|l|}
    \hline
    \cos(\theta) =x      & \arccos(x) = \theta\\ \hline\hline
    \cos(180) =-1      & \arccos(-1) = 180\\ \hline
    \cos(210) = -0.866 & \arccos(-0.866) = 150\\ \hline
    \cos(240) = -0.5   & \arccos(-0.5) = 120\\ \hline
    \cos(270) = 0      & \arccos(0) = 90\\ \hline
    \cos(300) = 0.5     & \arccos(0.5) = 60\\ \hline
    \cos(330) = 0.866   & \arccos(0.866) = 30\\ \hline
    \cos(360)  = 1     & \arccos(1) = 0 \\    \hline
  \end{array}
  \]
  Use the information from problem 1 and 2 to enhance your answer from
  problem 1. That is EXPLAIN what arccosine does when $180^\circ \le
  \theta \le 360^\circ$ and $-1\le x\le 1$, as based on the table
  above.
  \begin{freeResponse}
    If you have an angle $\theta$ between $180^\circ$ and $360^\circ$
    and $-1\le x\le 1$, then cosine still undoes arccosine,
    \[
    \cos(\arccos(x)) = x
    \]
    BUT arccosine no longer undoes ``cosine.'' Instead in this case
    \[
    \arccos(\cos(\theta)) = 360-\theta.
    \]
  \end{freeResponse}
\end{question}
\mynewpage



\begin{question}
  One of the following statements is TRUE, the other is FALSE,
  \[
  \cos(\arccos(x)) = x \qquad \text{or} \qquad \arccos(\cos(\theta))
  =\theta.
  \]
  EXPLAIN:
  \begin{enumerate}
  \item WHY we use two different letters, $x$ and $\theta$. In
    particular, explain what $x$ and $\theta$ represent.
  \item WHY the true statement is true.
  \item WHY the false statement is false.
  \end{enumerate}
  Use words, pictures, examples, and so on, as needed/helpful in your
  explanations.
  \begin{freeResponse}
    \begin{enumerate}
    \item We use two different letters because they represent
      different things.
      \begin{align*}
        \theta &= \text{an angle},\\
        x &=\text{a number between $-1$ and $1$}.
      \end{align*}
    \item The TRUE statement is:
      \[
      \cos(\arccos(x)) = x
      \]
      This is always true, because:
      \begin{center}
        \begin{tikzpicture} 
          \node[anchor=east,rounded corners=5pt,draw] at (0,0) {$-1\le x\le 1$};
          \draw [->,thick] (.2,.5) to [out=30,in=150] (2.8,.5);
          \node[anchor=west,rounded corners=5pt,draw] at (3,0) {$0^\circ\le \theta\le 180^\circ$};
          \draw [->,thick] (2.8,-.5) to [out=-150,in=-30] (.2,-.5);
          \node at (1.5,1.2) {$\arccos$};
          \node at (1.5,-1.2) {$\cos$};
        \end{tikzpicture}
      \end{center}
      Arccosine turns ratios between $-1$ and $1$ into angles, then
      cosine sends these angles back to the ratios.
    \item The FALSE statement is:
      \[
      \arccos(\cos(\theta)) = \theta
      \]
      This is false if $\theta$ is not between $0^\circ$ and
      $180^\circ$, because:
      \begin{center}
        \begin{tikzpicture} 
          \node[anchor=east,rounded corners=5pt,draw] at (0,0) {$180^\circ\le \theta\le 360^\circ$};
          \draw [->,thick] (.2,0) to  (2.8,0);
          \node[anchor=west,rounded corners=5pt,draw] at (3,0) {$-1\le x\le 1$};
          \draw [->,thick] (5.65,0) to  (8.25,0);
          \node[anchor=west,rounded corners=5pt,draw] at (8.45,0) {$0^\circ\le \theta\le 180^\circ$};
          \node at (1.5,.5) {$\cos$};
          \node at (6.95,.5) {$\arccos$};
        \end{tikzpicture}
      \end{center}
      Cosine turns angles (even those outside the range of $0^\circ$
      to $180^\circ$) into ratios between $-1$ and $1$. Arccosine
      turns these into angles between $0^\circ$ and $180^\circ$. If
      you start with an angle $\theta$ outside of this range, $\arccos(\cos(\theta)) \neq \theta$.
      
    \end{enumerate}
  \end{freeResponse}
\end{question}



\end{document}
