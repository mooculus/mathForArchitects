\documentclass[noauthor,nooutcomes,hints,handout]{ximera}

%% page layout
\usepackage[in,headings]{fullpage}
\raggedright
\setlength\headheight{13.6pt}


%% fonts
\usepackage{euler}

\usepackage{FiraMono}
\renewcommand\familydefault{\ttdefault} 
\usepackage{mathastext}
\usepackage[htt]{hyphenat}

\usepackage[T1]{fontenc}
\usepackage[scaled=1]{FiraSans}

\usepackage{wedn}
\usepackage[T1]{fontenc}

%% wrap text around scripts
\usepackage{wrapfig}

\tikzset{>=stealth}
%% snap! scripts
\usepackage{scratch3}

\usepackage{adjustbox}

%% journal style
\makeatletter
\newcommand\journalstyle{%
  \def\activitystyle{activity-chapter}
  \def\maketitle{%
    \addtocounter{titlenumber}{1}%
                {\flushleft\small\sffamily\bfseries\@pretitle\par\vspace{-1.5em}}%
                {\flushleft\LARGE\sffamily\bfseries\thetitlenumber\hspace{1em}\@title \par }%
                {\vskip .6em\noindent\textit\theabstract\setcounter{question}{0}\setcounter{sectiontitlenumber}{0}}%
                    \par\vspace{2em}
                    \phantomsection\addcontentsline{toc}{section}{\thetitlenumber\hspace{1em}\textbf{\@title}}%
                     }}
\makeatother



%% thm like environments
\let\question\relax
\let\endquestion\relax

\newtheoremstyle{QuestionStyle}{\topsep}{\topsep}%%% space between body and thm
		{}                      %%% Thm body font
		{}                              %%% Indent amount (empty = no indent)
		{\bfseries}            %%% Thm head font
		{)}                              %%% Punctuation after thm head
		{ }                           %%% Space after thm head
		{\thmnumber{#2}\thmnote{ \bfseries(#3)}}%%% Thm head spec
\theoremstyle{QuestionStyle}
\newtheorem{question}{}



\let\freeResponse\relax
\let\endfreeResponse\relax

%% \newtheoremstyle{ResponseStyle}{\topsep}{\topsep}%%% space between body and thm
%% 		{\wedn\bfseries}                      %%% Thm body font
%% 		{}                              %%% Indent amount (empty = no indent)
%% 		{\wedn\bfseries}            %%% Thm head font
%% 		{}                              %%% Punctuation after thm head
%% 		{3ex}                           %%% Space after thm head
%% 		{\underline{\underline{\thmname{#1}}}}%%% Thm head spec
%% \theoremstyle{ResponseStyle}

\usepackage[tikz]{mdframed}
\mdfdefinestyle{ResponseStyle}{leftmargin=1cm,linecolor=black,roundcorner=5pt,frametitlefont=\wedn\bfseries,%frametitle={\underline{\underline{Response}}:}
, font=\wedn\bfseries,}%\begin{mdframed}[style=mystyle]foo\end{mdframed}


\ifhandout
\NewEnviron{freeResponse}{}
\else
%\newtheorem{freeResponse}{Response:}
\newenvironment{freeResponse}{\begin{mdframed}[style=ResponseStyle]}{\end{mdframed}}
\fi



%% attempting to automate outcomes.

\newwrite\outcomefile
  \immediate\openout\outcomefile=\jobname.oc
\renewcommand{\outcome}[1]{\edef\theoutcomes{\theoutcomes #1~}%
\immediate\write\outcomefile{\unexpanded{\outcome}{#1}}}

%% \newcommand{\outcomelist}{\begin{itemize}\theoutcomes\end{itemize}}



%% my commands

\newcommand{\snap}{{\bfseries\itshape\textsf{Snap!}}}
\newcommand{\flavor}{\link[\snap]{https://snap.berkeley.edu/}}


\usepackage{tkz-euclide}
\tikzstyle geometryDiagrams=[rounded corners=.5pt,ultra thick,color=black]
\colorlet{penColor}{black} % Color of a curve in a plot


\title{Symmetries of the regular hexagon}
\author{Bart Snapp}

\begin{document}
\begin{abstract}
  We explore the symmetries of the regular hexagon.
\end{abstract}
\maketitle

\begin{listOutcomes}
\item Describe symmetries of the regular hexagon without finding them
  all via pictures.
\item Think of the symmetries of the regular hexagon as functions.
\item Compose symmetries and use \snap\ to show the result.
\item Use the algebra of symmetries to understand a composition of
  symmetries.
\end{listOutcomes}
\mynewpage


\begin{question}
  The symmetries of the regular hexagon are those that leave it
  unchanged. Let $e$ be the do-nothing symmetry, $r$ be a clockwise
  $60^\circ$ rotation about the center of the hexagon, and $f$ be a
  flip across a vertical line down the middle of the hexagon.

  
  How many symmetries are there of the regular hexagon? Just
  EXPLAIN with words and AS FEW PICTURES AS NECESSARY.
  \begin{freeResponse}
    To start, there's the do-nothing symmetry, $e$. Then there are
    $5$ clockwise rotations, $r$, $r^2$, $r^3$, $r^4$, and $r^5$.

    There are flips across a line perpendicular to every face of the
    hexagon. There are $3$ of these.

    There are flips across lines through opposite vertices. There are
    $3$ of these.

    Hence there are $12$ symmetries of the regular hexagon.
  \end{freeResponse}
\end{question}
\mynewpage

\begin{question}
  Let $e$ be the do-nothing symmetry, $r$ be a clockwise $60^\circ$
  rotation about the center of the hexagon, and $f$ be a flip across a
  vertical line down the middle of the hexagon. Display the result of
  applying the following functions (actions/transformations) to your
  hexagon:
  \begin{enumerate}
  \item $e$
  \item $rf$
  \item $r^2 f$
  \item $r^3 f$
  \item $r^4 f$
  \item $r^5 f$
  \end{enumerate}
  In each case, show off your work by displaying your SCRIPT and
  STAGE.
  \begin{freeResponse}
    \begin{enumerate}
    \item For $e$ we have:
      \begin{center}
        \includegraphics[width=.3\textwidth]{eHexSCRIPT.png}   \qquad \fbox{\includegraphics[width=.3\textwidth]{eHex.png}}
      \end{center}
    \item For $rf$ we have:
      \begin{center}
        \includegraphics[width=.3\textwidth]{rfHexSCRIPT.png}   \qquad \fbox{\includegraphics[width=.3\textwidth]{rfHex.png}}
      \end{center}
    \item For $r^2f$ we have:
      \begin{center}
        \includegraphics[width=.3\textwidth]{r2fHexSCRIPT.png}   \qquad \fbox{\includegraphics[width=.3\textwidth]{r2fHex.png}}
      \end{center}
    \item For $r^3f$ we have:
      \begin{center}
        \includegraphics[width=.3\textwidth]{r3fHexSCRIPT.png}   \qquad \fbox{\includegraphics[width=.3\textwidth]{r3fHex.png}}
      \end{center}
    \item For $r^4f$ we have:
      \begin{center}
        \includegraphics[width=.3\textwidth]{r4fHexSCRIPT.png}   \qquad \fbox{\includegraphics[width=.3\textwidth]{r4fHex.png}}
      \end{center}
      \item For $r^5f$ we have:
      \begin{center}
        \includegraphics[width=.3\textwidth]{r5fHexSCRIPT.png}   \qquad \fbox{\includegraphics[width=.3\textwidth]{r5fHex.png}}
      \end{center}
    \end{enumerate}
    \end{freeResponse}
\end{question}
\mynewpage


\begin{question}
 EXPRESS each of the symmetries below as one of:
 \[
 e,r,r^2,r^3,r^4,r^5,f,rf,r^2f,r^3f, r^4f,r^5f.
 \]

 \begin{enumerate}
 \item $fr$
 \item $fr^2$
 \item $fr^3$
 \item $fr^4$
 \item $fr^5$
 \item $fr^7f^2r^9f^5r^4f^3$
 \end{enumerate}
 \begin{freeResponse}
   \begin{enumerate}
   \item $fr = r^5f$
   \item $fr^2 = r^4f$
   \item $fr^3 = r^3f$
   \item $fr^4 = r^2f$
   \item $fr^5 = rf$
   \item And finally,
     \begin{align*}
       fr^7f^2r^9f^5r^4f^3 &= fr^4fr^4f \\
       &= r^2f r^2\\
       &= f.
     \end{align*}
   \end{enumerate}
 \end{freeResponse}
\end{question}
\end{document}
