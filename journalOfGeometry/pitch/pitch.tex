\documentclass[12pt,noauthor,nooutcomes,handout,hints]{ximera}

%% page layout
\usepackage[in,headings]{fullpage}
\raggedright
\setlength\headheight{13.6pt}


%% fonts
\usepackage{euler}

\usepackage{FiraMono}
\renewcommand\familydefault{\ttdefault} 
\usepackage{mathastext}
\usepackage[htt]{hyphenat}

\usepackage[T1]{fontenc}
\usepackage[scaled=1]{FiraSans}

\usepackage{wedn}
\usepackage[T1]{fontenc}

%% wrap text around scripts
\usepackage{wrapfig}

\tikzset{>=stealth}
%% snap! scripts
\usepackage{scratch3}

\usepackage{adjustbox}

%% journal style
\makeatletter
\newcommand\journalstyle{%
  \def\activitystyle{activity-chapter}
  \def\maketitle{%
    \addtocounter{titlenumber}{1}%
                {\flushleft\small\sffamily\bfseries\@pretitle\par\vspace{-1.5em}}%
                {\flushleft\LARGE\sffamily\bfseries\thetitlenumber\hspace{1em}\@title \par }%
                {\vskip .6em\noindent\textit\theabstract\setcounter{question}{0}\setcounter{sectiontitlenumber}{0}}%
                    \par\vspace{2em}
                    \phantomsection\addcontentsline{toc}{section}{\thetitlenumber\hspace{1em}\textbf{\@title}}%
                     }}
\makeatother



%% thm like environments
\let\question\relax
\let\endquestion\relax

\newtheoremstyle{QuestionStyle}{\topsep}{\topsep}%%% space between body and thm
		{}                      %%% Thm body font
		{}                              %%% Indent amount (empty = no indent)
		{\bfseries}            %%% Thm head font
		{)}                              %%% Punctuation after thm head
		{ }                           %%% Space after thm head
		{\thmnumber{#2}\thmnote{ \bfseries(#3)}}%%% Thm head spec
\theoremstyle{QuestionStyle}
\newtheorem{question}{}



\let\freeResponse\relax
\let\endfreeResponse\relax

%% \newtheoremstyle{ResponseStyle}{\topsep}{\topsep}%%% space between body and thm
%% 		{\wedn\bfseries}                      %%% Thm body font
%% 		{}                              %%% Indent amount (empty = no indent)
%% 		{\wedn\bfseries}            %%% Thm head font
%% 		{}                              %%% Punctuation after thm head
%% 		{3ex}                           %%% Space after thm head
%% 		{\underline{\underline{\thmname{#1}}}}%%% Thm head spec
%% \theoremstyle{ResponseStyle}

\usepackage[tikz]{mdframed}
\mdfdefinestyle{ResponseStyle}{leftmargin=1cm,linecolor=black,roundcorner=5pt,frametitlefont=\wedn\bfseries,%frametitle={\underline{\underline{Response}}:}
, font=\wedn\bfseries,}%\begin{mdframed}[style=mystyle]foo\end{mdframed}


\ifhandout
\NewEnviron{freeResponse}{}
\else
%\newtheorem{freeResponse}{Response:}
\newenvironment{freeResponse}{\begin{mdframed}[style=ResponseStyle]}{\end{mdframed}}
\fi



%% attempting to automate outcomes.

\newwrite\outcomefile
  \immediate\openout\outcomefile=\jobname.oc
\renewcommand{\outcome}[1]{\edef\theoutcomes{\theoutcomes #1~}%
\immediate\write\outcomefile{\unexpanded{\outcome}{#1}}}

%% \newcommand{\outcomelist}{\begin{itemize}\theoutcomes\end{itemize}}



%% my commands

\newcommand{\snap}{{\bfseries\itshape\textsf{Snap!}}}
\newcommand{\flavor}{\link[\snap]{https://snap.berkeley.edu/}}


\usepackage{tkz-euclide}
\tikzstyle geometryDiagrams=[rounded corners=.5pt,ultra thick,color=black]
\colorlet{penColor}{black} % Color of a curve in a plot


\title{Pitch}
\author{Bart Snapp}

\begin{document}
\begin{abstract}
  We think about pitch and slope.
\end{abstract}
\maketitle

\begin{listOutcomes}
\item Give the definitions of pitch and slope.
\item Distinguish between pitch and slope.
\item Solve a problem where slope is given.
\item Deduce the slope in a given context.
\item Use tangent to transform angles into slope and pitch.
\end{listOutcomes}



The steepness of a roof is called its \textit{slope} or
\textit{pitch}. This is a fraction where the numerator is the
\textit{rise} of the roof and the denominator is the \textit{run}. The
denominator is usually $12$. Moreover, when one talks of ``pitch''
they usually omit the denominator of $12$. Hence
\begin{center}
\fbox{a slope of $\frac{7}{12}$}\quad is the same as\quad \fbox{a pitch of $7$}.
\end{center}
This slope (or pitch) says that the roof goes up $7''$ for every
$12''$. Note there are several other notations for pitch. The one we
are presenting here is not only rather common, it is also
mathematically correct!

\mynewpage


\begin{question}
Here is a diagram of a roof that is pointed in the middle:
\begin{center}
  \begin{tikzpicture}[x=1.5cm,y=1.5cm]
    \draw[ultra thick] (0,0) rectangle (4.8,3);
    \draw[dashed] (.5,.5) rectangle (4.3,2.5);
    \draw[ultra thick] (0,1.5) -- (4.8,1.5);
    \draw[decoration={brace,raise=.2cm},decorate,thin] (0,0)--(0,3);
    \draw[decoration={brace,raise=.2cm,mirror},decorate,thin] (0,0)--(4.8,0);
    \node at (-.4,1.5) {$30'$};
    \node at (2.4,-.44) {$48'$};
    \node[above] at (2.4,1.5) {slope of $6/12$};
  \end{tikzpicture}
\end{center}
Find the area of the roof.  EXPLAIN the work in your solution.

\begin{freeResponse}
  If we look at the roof from the side, it looks like:
  \begin{center}
    \begin{tikzpicture}[x=1.5cm,y=1.5cm]
    \draw[ultra thick] (0,0) -- (1.5,0)  -- (1.5, .75) -- (0,0) ;
    \draw[decoration={brace,raise=.2cm,mirror},decorate,thin] (0,0)--(1.5,0);
    \draw[decoration={brace,raise=.2cm,mirror},decorate,thin] (1.5,0)--(1.5,.75);
    \node at (.75,-.4) {$15'$};
    \node at (2.4,.375) {$15'\cdot 6/12$};
    \end{tikzpicture}
    
  \end{center}
    By the Pythagorean theorem, the length of the hypotenuse is
    \[
    \sqrt{15^2+ 7.5^2} \approx 16.77~\text{feet}.
    \]
    Thus, the area of the roof is
    \[
    2\cdot 16.77\cdot 48 \approx 1610~\text{square feet}.
    \]
\end{freeResponse}
\end{question}
\mynewpage


\begin{question}
Here is a diagram of with different slopes for each side:
\begin{center}
  \begin{tikzpicture}[x=1.5cm,y=1.5cm]
    \draw[ultra thick] (0,0) rectangle (4.8,2.4);
    \draw[dashed] (.3,.3) rectangle (4.5,2.1);
    \draw[ultra thick] (0,1.6) -- (4.8,1.6);
    \draw[decoration={brace,raise=.2cm},decorate,thin] (0,0)--(0,1.6);
    \draw[decoration={brace,raise=.2cm,mirror},decorate,thin] (4.8,1.6)--(4.8,2.4);
    \draw[decoration={brace,raise=.2cm,mirror},decorate,thin] (0,0)--(4.8,0);
    \node at (-.5,.8) {$16'$};
    \node at (5.2,2) {$8'$};
    \node at (2.4,-.44) {$48'$};
    \node at (2.4,1) {slope of $6/12$};
  \end{tikzpicture}
\end{center}
First find the missing slope, then find the area of the roof. EXPLAIN
the work in your solution.

\begin{freeResponse}
    If we look at the roof from the side, it looks like:
  \begin{center}
    \begin{tikzpicture}[x=1.5cm,y=1.5cm]
    \draw[ultra thick] (0,0) -- (1.6,0)  -- (1.6, .8) -- (0,0) ;
    \draw[decoration={brace,raise=.2cm,mirror},decorate,thin] (0,0)--(1.6,0);
    \draw[decoration={brace,raise=.2cm,mirror},decorate,thin] (1.6,0)--(1.6,.8);
    \node at (.8,-.4) {$16'$};
    \node at (2.5,.4) {$16'\cdot 6/12$};
    \end{tikzpicture}
  \end{center}
    By the Pythagorean theorem, the length of the hypotenuse is
    \[
    \sqrt{16^2+ 8^2} \approx 17.89~\text{feet}.
    \]
    Thus, the area of that side of the roof is
    \[
    17.89\cdot 48 \approx 859~\text{square feet}.
    \]
    For the other side, consider:
    \begin{center}
    \begin{tikzpicture}[x=2cm,y=2cm]
    \draw[ultra thick] (0,0) -- (.8,0)  -- (.8, .8) -- (0,0) ;
    \draw[decoration={brace,raise=.2cm,mirror},decorate,thin] (0,0)--(.8,0);
    \draw[decoration={brace,raise=.2cm,mirror},decorate,thin] (.8,0)--(.8,.8);
    \node at (.4,-.3) {$8'$};
    \node at (1.7,.4) {$8'=16'\cdot 6/12$};
    \end{tikzpicture}
    \end{center}
     By the Pythagorean theorem, the length of the hypotenuse is
    \[
    \sqrt{8^2 + 8^2} \approx 11.3~\text{feet}.
    \]
    So the area of this side of the roof is
    \[
    11.3\cdot 48 \approx 542~\text{square feet}.
    \]
    Thus the total area of the roof is approximately $1401$ square
    feet.
\end{freeResponse}
\end{question}
\mynewpage


\begin{question}
In Europe, it is more common to measure the steepness of a roof using
the angle of the roof from horizontal. This is nice if you want to
``think'' in angles \textbf{(easy to visualize)}, but give directions
in terms of slope \textbf{(easier to measure)}. In this case,
\textbf{tangent} comes to the rescue:
\[
\tan(\theta) = \frac{\sin(\theta)}{\cos(\theta)} = \frac{\frac{opposite}{\cancel{hypotenuse}}}{\frac{adjacent}{\cancel{hypotenuse}}}=\frac{opposite}{adjacent}.
\]



\begin{enumerate}
\item If we have a line in the $(x,y)$-plane that makes an angle of
  $\theta$ with the $x$-axis,
  \begin{center}
    \begin{tikzpicture}
      \begin{axis}[
          xmin=-2.2,
          xmax=2.2,
          ymin=-2.2,
          ymax=2.2,
          axis lines=center,
          xlabel=$x$,
          ticks=none,
          ylabel=$y$,
          every axis y label/.style={at=(current axis.above origin),anchor=south},
          every axis x label/.style={at=(current axis.right of origin),anchor=west},disabledatascaling
        ]
        \addplot [ultra thick, black, smooth] {x+1};
        \draw (-.4,0) arc (0:28:1);
        \node at (-.6,.2) {$\theta$};
        \node[right] at (.5,1.3) {$y=m\cdot x+b$};
      \end{axis}
    \end{tikzpicture}
  \end{center}
  Then $\tan(\theta) = m$. EXPLAIN using words, pictures, and so on,
  WHY this is true.
\item Find a whole number numerator $n$ such that a slope of
  $\frac{n}{12}$ best corresponds to a $30^\circ$ angle. Show your
  work.
\item Find a pitch $p$ (a whole number) such that a pitch of $p$ best
  corresponds to a $40^\circ$ angle. Show your work.
\end{enumerate}
\begin{freeResponse}
  \begin{enumerate}
  \item If the slope of the line is
    \[
    m = \frac{rise}{run}
    \]
    then we can draw:
    \begin{center}
      \begin{tikzpicture}
        \begin{axis}[
            xmin=-2.2,
            xmax=2.2,
            ymin=-2.2,
            ymax=2.2,
            axis lines=center,
            xlabel=$x$,
            ticks=none,
            ylabel=$y$,
            every axis y label/.style={at=(current axis.above origin),anchor=south},
            every axis x label/.style={at=(current axis.right of origin),anchor=west},disabledatascaling
          ]
          \addplot [ultra thick, black, smooth] {x+1};
          \draw[line width=3,->] (.3,0)--(.3,1.3);          
          \draw[line width=3,->] (-1,0)--(.3,0);
          \draw (-.4,0) arc (0:28:1);
          \node at (-.6,.2) {$\theta$};
          \node[right] at (.3,.65) {$rise$};
          \node[below] at (-.35,0) {$run$};
        \end{axis}
      \end{tikzpicture}
    \end{center}
    Thus
    \[
    \tan(\theta) =\frac{opposite}{adjacent} = \frac{rise}{run} = m.
    \]
  \item So $12\cdot \tan(30^\circ) \approx 6.93$. This means a slope
    of $7/12$ would work best.

  \item So $12\cdot \tan(40^\circ)\approx 10$. This means a pitch of
    $10$ would be best.
  \end{enumerate}
  \end{freeResponse}
\end{question}

\end{document}
