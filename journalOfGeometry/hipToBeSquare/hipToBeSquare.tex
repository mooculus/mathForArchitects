\documentclass[handout,nooutcomes,noauthor,hints]{ximera}

%% page layout
\usepackage[in,headings]{fullpage}
\raggedright
\setlength\headheight{13.6pt}


%% fonts
\usepackage{euler}

\usepackage{FiraMono}
\renewcommand\familydefault{\ttdefault} 
\usepackage{mathastext}
\usepackage[htt]{hyphenat}

\usepackage[T1]{fontenc}
\usepackage[scaled=1]{FiraSans}

\usepackage{wedn}
\usepackage[T1]{fontenc}

%% wrap text around scripts
\usepackage{wrapfig}

\tikzset{>=stealth}
%% snap! scripts
\usepackage{scratch3}

\usepackage{adjustbox}

%% journal style
\makeatletter
\newcommand\journalstyle{%
  \def\activitystyle{activity-chapter}
  \def\maketitle{%
    \addtocounter{titlenumber}{1}%
                {\flushleft\small\sffamily\bfseries\@pretitle\par\vspace{-1.5em}}%
                {\flushleft\LARGE\sffamily\bfseries\thetitlenumber\hspace{1em}\@title \par }%
                {\vskip .6em\noindent\textit\theabstract\setcounter{question}{0}\setcounter{sectiontitlenumber}{0}}%
                    \par\vspace{2em}
                    \phantomsection\addcontentsline{toc}{section}{\thetitlenumber\hspace{1em}\textbf{\@title}}%
                     }}
\makeatother



%% thm like environments
\let\question\relax
\let\endquestion\relax

\newtheoremstyle{QuestionStyle}{\topsep}{\topsep}%%% space between body and thm
		{}                      %%% Thm body font
		{}                              %%% Indent amount (empty = no indent)
		{\bfseries}            %%% Thm head font
		{)}                              %%% Punctuation after thm head
		{ }                           %%% Space after thm head
		{\thmnumber{#2}\thmnote{ \bfseries(#3)}}%%% Thm head spec
\theoremstyle{QuestionStyle}
\newtheorem{question}{}



\let\freeResponse\relax
\let\endfreeResponse\relax

%% \newtheoremstyle{ResponseStyle}{\topsep}{\topsep}%%% space between body and thm
%% 		{\wedn\bfseries}                      %%% Thm body font
%% 		{}                              %%% Indent amount (empty = no indent)
%% 		{\wedn\bfseries}            %%% Thm head font
%% 		{}                              %%% Punctuation after thm head
%% 		{3ex}                           %%% Space after thm head
%% 		{\underline{\underline{\thmname{#1}}}}%%% Thm head spec
%% \theoremstyle{ResponseStyle}

\usepackage[tikz]{mdframed}
\mdfdefinestyle{ResponseStyle}{leftmargin=1cm,linecolor=black,roundcorner=5pt,frametitlefont=\wedn\bfseries,%frametitle={\underline{\underline{Response}}:}
, font=\wedn\bfseries,}%\begin{mdframed}[style=mystyle]foo\end{mdframed}


\ifhandout
\NewEnviron{freeResponse}{}
\else
%\newtheorem{freeResponse}{Response:}
\newenvironment{freeResponse}{\begin{mdframed}[style=ResponseStyle]}{\end{mdframed}}
\fi



%% attempting to automate outcomes.

\newwrite\outcomefile
  \immediate\openout\outcomefile=\jobname.oc
\renewcommand{\outcome}[1]{\edef\theoutcomes{\theoutcomes #1~}%
\immediate\write\outcomefile{\unexpanded{\outcome}{#1}}}

%% \newcommand{\outcomelist}{\begin{itemize}\theoutcomes\end{itemize}}



%% my commands

\newcommand{\snap}{{\bfseries\itshape\textsf{Snap!}}}
\newcommand{\flavor}{\link[\snap]{https://snap.berkeley.edu/}}


\usepackage{tkz-euclide}
\tikzstyle geometryDiagrams=[rounded corners=.5pt,ultra thick,color=black]
\colorlet{penColor}{black} % Color of a curve in a plot

\title{Hip to be square}

\author{Bart Snapp and Claire Merriman}

\begin{document}
\begin{abstract}
  We extend scaling ideas from squares and cubes to arbitrary shapes.
\end{abstract}
\maketitle

\begin{listOutcomes}
\item Understand how scaling a square changes its area,
\item Understand how scaling a cube changes its surface area,
\item Understand how scaling a cube changes its volume,
\item Understand how scaling an irregular 2D object changes its area.
\end{listOutcomes}

%% \begin{listObjectives}
%%  \item Explain the relationship between scaling lengths and scaling areas and volumes,
%% \item Learn and apply basic geometric formulas.
%% \end{listObjectives}


\begin{definition}
 The \emph{scale factor} of a model/object is how much each linear dimension (ie, length, width, height) is scaled. This is often written as a ratio, where $1:x$ would mean that every length is scaled (ie, multiplied) by $x$. \end{definition}

\mynewpage

\begin{question}
  Let's think about scaling squares and cubes. Let's start with a
  square with dimensions $s\times s$ and cube with dimensions $s\times
  s\times s$.
  \begin{enumerate}
  \item Scale a square by a scale factor of $3$. What's the \emph{area} of the new square? 
  \item Scale a cube by a scale factor of $3$. What's the \emph{volume} of the new cube?
  \item Scale a cube by a scale factor of $3$. What's the \emph{area of one face} of the new cube?
  \item Scale a cube by a scale factor of $3$. What's the \emph{surface area} of the new cube?
  \end{enumerate}
  In each case above, \emph{draw pictures} to help you explain why
  your answer is correct.
\end{question}
\mynewpage

\begin{question}
  Let's think about scaling squares and cubes. Let's start with a
  square with dimensions $s\times s$ and cube with dimensions $s\times
  s\times s$.
  \begin{enumerate}
  \item Scale a square by a scale factor of $x$. What's the \emph{area} of the new square?
  \item Scale a cube by a scale factor of $x$. What's the \emph{volume} of the new cube?
  \item Scale a cube by a scale factor of $x$. What's the \emph{area of one face} of the new cube?
  \item Scale a cube by a scale factor of $x$. What's the \emph{surface area} of the new cube?
  \end{enumerate}
  In each case above, \emph{draw pictures} to help you explain why your
  answer is correct.
\end{question}
\mynewpage

\begin{question}
  We know how squares and cubes scale, but what about weird shapes?
  Louie Llama can help us out here.  Consider this picture:
  \begin{center}
    \includegraphics{llamaScaled.pdf}
  \end{center}
  Let\index{Louie Llama}
  \begin{align*}
    L &= \text{Area of Louie Llama}\\
    S &= \text{Area of Square}\\
    B &= \text{Area of bigger llama, Blouie Llama}.
  \end{align*}
  \begin{enumerate}
  \item If $x$ is the scale-factor, explain why:
    \[
    \frac{L}{S} = \frac{B}{x^2 S}
    \]
  \item Solve for $B$ above, and explain how this demonstrates how
    scaling linear dimensions changes the area of \emph{any} 2D object.
  \end{enumerate}
  
\end{question}



\end{document}
