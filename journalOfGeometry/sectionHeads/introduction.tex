\documentclass[handout,nooutcomes,noauthor]{ximera}
\graphicspath{  
{./}
{./whoAreYou/}
{./drawingWithTheTurtle/}
{./bisectionMethod/}
{./circles/}
{./anglesAndRightTriangles/}
{./lawOfSines/}
{./lawOfCosines/}
{./plotter/}
{./staircases/}
{./pitch/}
{./qualityControl/}
{./symmetry/}
{./nGonBlock/}
}


%% page layout
\usepackage[cm,headings]{fullpage}
\raggedright
\setlength\headheight{13.6pt}


%% fonts
\usepackage{euler}

\usepackage{FiraMono}
\renewcommand\familydefault{\ttdefault} 
\usepackage[defaultmathsizes]{mathastext}
\usepackage[htt]{hyphenat}

\usepackage[T1]{fontenc}
\usepackage[scaled=1]{FiraSans}

%\usepackage{wedn}
\usepackage{pbsi} %% Answer font


\usepackage{cancel} %% strike through in pitch/pitch.tex


%% \usepackage{ulem} %% 
%% \renewcommand{\ULthickness}{2pt}% changes underline thickness

\tikzset{>=stealth}

\usepackage{adjustbox}

\setcounter{titlenumber}{-1}

%% journal style
\makeatletter
\newcommand\journalstyle{%
  \def\activitystyle{activity-chapter}
  \def\maketitle{%
    \addtocounter{titlenumber}{1}%
                {\flushleft\small\sffamily\bfseries\@pretitle\par\vspace{-1.5em}}%
                {\flushleft\LARGE\sffamily\bfseries\thetitlenumber\hspace{1em}\@title \par }%
                {\vskip .6em\noindent\textit\theabstract\setcounter{question}{0}\setcounter{sectiontitlenumber}{0}}%
                    \par\vspace{2em}
                    \phantomsection\addcontentsline{toc}{section}{\thetitlenumber\hspace{1em}\textbf{\@title}}%
                     }}
\makeatother



%% thm like environments
\let\question\relax
\let\endquestion\relax

\newtheoremstyle{QuestionStyle}{\topsep}{\topsep}%%% space between body and thm
		{}                      %%% Thm body font
		{}                              %%% Indent amount (empty = no indent)
		{\bfseries}            %%% Thm head font
		{)}                              %%% Punctuation after thm head
		{ }                           %%% Space after thm head
		{\thmnumber{#2}\thmnote{ \bfseries(#3)}}%%% Thm head spec
\theoremstyle{QuestionStyle}
\newtheorem{question}{}



\let\freeResponse\relax
\let\endfreeResponse\relax

%% \newtheoremstyle{ResponseStyle}{\topsep}{\topsep}%%% space between body and thm
%% 		{\wedn\bfseries}                      %%% Thm body font
%% 		{}                              %%% Indent amount (empty = no indent)
%% 		{\wedn\bfseries}            %%% Thm head font
%% 		{}                              %%% Punctuation after thm head
%% 		{3ex}                           %%% Space after thm head
%% 		{\underline{\underline{\thmname{#1}}}}%%% Thm head spec
%% \theoremstyle{ResponseStyle}

\usepackage[tikz]{mdframed}
\mdfdefinestyle{ResponseStyle}{leftmargin=1cm,linecolor=black,roundcorner=5pt,
, font=\bsifamily,}%font=\wedn\bfseries\upshape,}


\ifhandout
\NewEnviron{freeResponse}{}
\else
%\newtheorem{freeResponse}{Response:}
\newenvironment{freeResponse}{\begin{mdframed}[style=ResponseStyle]}{\end{mdframed}}
\fi



%% attempting to automate outcomes.

%% \newwrite\outcomefile
%%   \immediate\openout\outcomefile=\jobname.oc
%% \renewcommand{\outcome}[1]{\edef\theoutcomes{\theoutcomes #1~}%
%% \immediate\write\outcomefile{\unexpanded{\outcome}{#1}}}

%% \newcommand{\outcomelist}{\begin{itemize}\theoutcomes\end{itemize}}

%% \NewEnviron{listOutcomes}{\small\sffamily
%% After answering the following questions, students should be able to:
%% \begin{itemize}
%% \BODY
%% \end{itemize}
%% }
\usepackage[tikz]{mdframed}
\mdfdefinestyle{OutcomeStyle}{leftmargin=2cm,rightmargin=2cm,linecolor=black,roundcorner=5pt,
, font=\small\sffamily,}%font=\wedn\bfseries\upshape,}
\newenvironment{listOutcomes}{\begin{mdframed}[style=OutcomeStyle]After answering the following questions, students should be able to:\begin{itemize}}{\end{itemize}\end{mdframed}}



%% my commands

\newcommand{\snap}{{\bfseries\itshape\textsf{Snap!}}}
\newcommand{\flavor}{\link[\snap]{https://snap.berkeley.edu/}}
\newcommand{\mooculus}{\textsf{\textbf{MOOC}\textnormal{\textsf{ULUS}}}}


\usepackage{tkz-euclide}
\tikzstyle geometryDiagrams=[rounded corners=.5pt,ultra thick,color=black]
\colorlet{penColor}{black} % Color of a curve in a plot



\ifhandout\newcommand{\mynewpage}{\newpage}\else\newcommand{\mynewpage}{}\fi

\title{Introduction}

\author{Bart Snapp}

\begin{document}
\begin{abstract}
  Let's get started.
\end{abstract}
\maketitle


Reinforce foundational mathematics
Support needed mathematics
Help students learn to make fewer mistakes
Help students to learn how to discover mathematics on their own
Have fun




\begin{listSectionOutcomes}
\item Translate classroom mathematics into real world mathematics. %% ACCOMPLISHED THROUGH SNAP  -- not any more!
\item Learn and apply basic geometric formulas.
\item Explain why concepts and formulas are true.
\item Critique and dismantle reasonable hypotheses in regard to
  geometry and arithmetic.
\end{listSectionOutcomes}


%%%%% In this course, we accept the maxim:
%%%%% \begin{quote}
%%%%%   It is best to learn by doing.
%%%%% \end{quote}
%%%%% To this end, 


\textbf{Everyone uses math everyday.}  Typically our application of
math is \textbf{invisible} to us; this is because the math we use in
our daily life is tied to a \textbf{specific context.} When working in
a specific context, we cannot see the math we use, because it is
overshadowed by the context itself.  This raises the following
question:
\begin{quote}
  Why do we need to learn math if we already use it in our own daily
  contexts?
\end{quote}
The answer is that \textbf{change is the essence of life}; and to be
successful in our changing life, 
\begin{quote}
  \textbf{we must be able to apply the lessons learned from one
    context to another context.}
\end{quote}
Part of the study of mathematics is taking a \textbf{solution to a
  problem in a given context,} removing all extraneous detail, and
\textbf{applying that same solution to another problem.} With that
said, I take it as a definition that to ``understand a mathematical
concept'' means that we can apply that concept to multiple contexts.
In these \textit{Journal Entries} we focus on
\begin{enumerate}
\item Translate classroom mathematics into real world mathematics.
\item Translate classroom mathematics 
\end{enumerate}





\section*{For students}




By completing these notes, we hope to literally enlighten our view of
our own knowledge.  Often in this course you will be required to
EXPLAIN a mathematical concept.

\subsection*{Standard of explanation}
\begin{itemize}
  \item The audience for your explanation should be a ``student in our
    class.'' That is, you are writing so that a student in our class
    can understand. 
  \item It is your job to \textbf{convince me} that you \text{understand what you are
    talking about.} You do this by being specific, and not pushing the
    intellectual load onto the reader.
\end{itemize}


When you finish a calculation, take a break. Then come back to your
work.  Re-read the problem. Did you answer everything that was asked?
Re-read your solution, be critical, do the numbers make sense? Is
there a "quick approximation" you can do to see if your work is close?
Is there another way that you can check your work?

I understand that you might not be used to doing this, but this
process of reflection, is sort of the key idea. (Right? I ask you to
do this almost every activity) Throughout the course, I attempt to get
student to think about answers, which are reasonable, which are not.

And perhaps, you do notice an answer is wrong, but are unsure what to
do.

In this case, you should ask for help. You should say, I computed this
this way, and I got this answer, and this makes sense to me, BUT I
know it cannot be right because X,Y,Z.

Chances are, you are making a reasonable mistake, one that others will
make too, and identifying this mistake, understanding it, really is a
valuable learning experience for all of us.




Novice mathematicians are overconfident of their knowledge of theorems
and formulas. It is common for them to confuse the formula for the
area of a triangle with the Pythagorean theorem. Moreover, even
stating the Pythagorean theorem is problematic for novices. Novice
mathematicians are quick to make mistakes that ``simplify'' their
work. Finally, novice mathematicians seldom \textbf{reflect} on their
work when it is completed. This results in them often having answers
that are \textbf{obviously wrong}.

\subsection*{List of common mistakes}
\begin{enumerate}
\item Novice mathematicians have trouble making ``general''
  pictures. For example, if asked to draw a triangle, most novice
  mathematicians will draw an \textbf{equilateral} triangle. This is a
  very \textbf{special triangle} and using this as a tool for
  reasoning about triangles in general is difficult.
\item Novice mathematicians have trouble stating the Pythagorean
  theorem. Usually when asked to state the theorem, they reply
  \[
  a^2+b^2=c^2.
  \]
  \textbf{This is wrong.} The Pythagorean theorem is a theorem
  about \textbf{triangles}. 
\item Novice mathematicians are quick to find a ``formula'' to solve
  the problem, yet often unable to use the formula correctly. Consider
  the formula for the surface area of a pyramid:
  \[
  here
  \]
  STILL HARD
\item When presented with
  \[
  c^2 = a^2 + b^2
  \]
  and asked to simplify, novice mathematicians will often write
  \[
  c = a +b
  \]
  \textbf{This is wrong.} If it were correct, then the
  conclusion of the Pythagorean theorem would be silly!
\item  When presented with
  \[
  \frac{a + b}{a}
  \]
  and asked to simplify, novice mathematicians will often write
  \[
  1+b
  \]
  \textbf{This is wrong.}
\item Novice mathematicians do not reflect on their work. For example
  novice mathematicians have written
  \[
  \frac{10}{2\pi} \approx  15.7\qquad\textbf{this is wrong!}
  \]
  Moreover, it is \textbf{obviously wrong} since $2\pi\approx 6$ and
  $1<\frac{10}{6} <2$. I'll leave it as an exercise for the reader to
  think about HOW such an error was made.

\item Novice mathematicians do not use counterfactual arguments. For example, if BLANK were true, then OTHER-BLANK would be true. But OTHER-BLANK is not true, so therefor BLANK is false.
\end{enumerate}


\subsection*{How to avoid the common mistakes}




\section*{Structure of the course}





Identify KEY journal entries, and identifying the supporting ones. 


These are the goals: 

Quick questions

Let's apply ourselves

Not so quick questions

Seven types of frieze





\begin{tikzpicture}
\tikzset{% This is the style settings for nodes
    dep/.style={circle,minimum size=1cm,fill=orange!20,draw=orange,
                general shadow={fill=gray!60,shadow xshift=1pt,shadow yshift=-1pt}},
    cli/.style={circle,minimum size=1cm,fill=white,draw,
                general shadow={fill=gray!60,shadow xshift=1pt,shadow yshift=-1pt}},
    spl/.style={cli,append after command={
                  node[circle,draw,dotted,
                       minimum size=1.5cm] at (\tikzlastnode.center) {}}},
    c1/.style={-stealth,very thick,red!80!black},
    v2/.style={-stealth,very thick,yellow!65!black},
    v4/.style={-stealth,very thick,purple!70!black}}
\node[dep] (0) at (0,0) {0};
\node[cli] (7) at (-1,-2) {7};
\node[spl] (8) at (2,-2) {8};
\draw[c1] (0) to[bend right] (7);
\draw[v2] (7) -- (8);
\draw[v4] (8) -- (2,-1) -- (0);
\end{tikzpicture}



WE COULD HAVE A STYLE FOR ESSENTIAL ACTIVITIES, and  A COLOR FOR REAL WORLD EG

NODES IN A SEQUENCE SHOULD BE DIRECTLY ABOVE EACH OTHER.


2,3,4,5 deal with understanding formulas for area and volume, with an
emphasis on estimation, and the relationships between perimeter/area
and surface-area/volume.


6 we use formulas and or pictures to solve a real world problem, and
start to understand scaling

7-8 we give model for scaling through rep-tiles. 



\begin{tikzpicture}[x=1cm,y=2cm]
  \tikzset{% This is the style settings for nodes
    every node/.style={font=\bfseries},
    every path/.style={-stealth,very thick},
    %essential/.style={circle,minimum size=.5cm,fill=orange!20,draw=orange,
         %       general shadow={fill=gray!60,shadow xshift=1pt,shadow yshift=-1pt}},
    essential/.style={circle,minimum size=.5cm,fill=white,draw,
      general shadow={fill=gray!60,shadow xshift=1pt,shadow yshift=-1pt}},
    enrich/.style={circle,minimum size=.5cm,fill=white,draw,dashed,
      general shadow={fill=gray!60,shadow xshift=1pt,shadow yshift=-1pt}},
    goal/.style={essential,append after command={
                  node[circle,draw,
                       minimum size=1cm] at (\tikzlastnode.center) {}}},
    c1/.style={-stealth,very thick,red!80!black},
    v2/.style={-stealth,very thick,yellow!65!black},
    v4/.style={-stealth,very thick,purple!70!black}}
\node[essential] (0) at (0,0) {0};
\node[essential] (1) at (1,0) {1};
\node[essential] (2) at (2,1) {2};
\node[enrich] (3) at (1,1.5) {3};
\node[essential] (4) at (2,2) {4};
\node[essential] (5) at (3,3) {5};
\node[goal] (6) at (3,4) {6};
\node[essential] (7) at (4,3) {7};
\node[essential] (8) at (4,4) {8};
\node[essential] (9) at (5,5) {9};
\node[goal] (10) at (5,6) {10};
\node[enrich] (11) at (6,6) {11};
\node[enrich] (12) at (7,6) {12};

\draw (1) -- (2);
\draw (2) -- (4);
\draw (4) -- (5);
\draw (5) -- (6);
\draw (6) -- (7);
\draw (7) -- (8);
\draw (8) -- (9);
\draw (9) -- (10);


\node[essential] (13) at (6,0) {13};
\node[enrich] (14) at (5,.5) {14};
\node[essential] (15) at (6,1) {15};
\node[enrich] (16) at (6,2) {16};
\node[essential] (17) at (6,3) {17};


\end{tikzpicture}






This ``geometry journal'' was created by Dr.\ Bart Snapp and
Dr.\ Jenny Sheldon at The Ohio State University. Additional content
was developed by developed by Dr.\ Betsy McNeal, Dr.\ Vic Ferdinand,
Dr.\ Bradford Findell, and Dr.\ Herbert Clemens.  And of course, there
is \textbf{one more content creator} in this Journal, \textbf{YOU!}

\end{document}
