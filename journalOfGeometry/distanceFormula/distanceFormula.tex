\documentclass[nooutcomes,noauthor]{ximera}

\graphicspath{  
{./}
{./whoAreYou/}
{./drawingWithTheTurtle/}
{./bisectionMethod/}
{./circles/}
{./anglesAndRightTriangles/}
{./lawOfSines/}
{./lawOfCosines/}
{./plotter/}
{./staircases/}
{./pitch/}
{./qualityControl/}
{./symmetry/}
{./nGonBlock/}
}


%% page layout
\usepackage[cm,headings]{fullpage}
\raggedright
\setlength\headheight{13.6pt}


%% fonts
\usepackage{euler}

\usepackage{FiraMono}
\renewcommand\familydefault{\ttdefault} 
\usepackage[defaultmathsizes]{mathastext}
\usepackage[htt]{hyphenat}

\usepackage[T1]{fontenc}
\usepackage[scaled=1]{FiraSans}

%\usepackage{wedn}
\usepackage{pbsi} %% Answer font


\usepackage{cancel} %% strike through in pitch/pitch.tex


%% \usepackage{ulem} %% 
%% \renewcommand{\ULthickness}{2pt}% changes underline thickness

\tikzset{>=stealth}

\usepackage{adjustbox}

\setcounter{titlenumber}{-1}

%% journal style
\makeatletter
\newcommand\journalstyle{%
  \def\activitystyle{activity-chapter}
  \def\maketitle{%
    \addtocounter{titlenumber}{1}%
                {\flushleft\small\sffamily\bfseries\@pretitle\par\vspace{-1.5em}}%
                {\flushleft\LARGE\sffamily\bfseries\thetitlenumber\hspace{1em}\@title \par }%
                {\vskip .6em\noindent\textit\theabstract\setcounter{question}{0}\setcounter{sectiontitlenumber}{0}}%
                    \par\vspace{2em}
                    \phantomsection\addcontentsline{toc}{section}{\thetitlenumber\hspace{1em}\textbf{\@title}}%
                     }}
\makeatother



%% thm like environments
\let\question\relax
\let\endquestion\relax

\newtheoremstyle{QuestionStyle}{\topsep}{\topsep}%%% space between body and thm
		{}                      %%% Thm body font
		{}                              %%% Indent amount (empty = no indent)
		{\bfseries}            %%% Thm head font
		{)}                              %%% Punctuation after thm head
		{ }                           %%% Space after thm head
		{\thmnumber{#2}\thmnote{ \bfseries(#3)}}%%% Thm head spec
\theoremstyle{QuestionStyle}
\newtheorem{question}{}



\let\freeResponse\relax
\let\endfreeResponse\relax

%% \newtheoremstyle{ResponseStyle}{\topsep}{\topsep}%%% space between body and thm
%% 		{\wedn\bfseries}                      %%% Thm body font
%% 		{}                              %%% Indent amount (empty = no indent)
%% 		{\wedn\bfseries}            %%% Thm head font
%% 		{}                              %%% Punctuation after thm head
%% 		{3ex}                           %%% Space after thm head
%% 		{\underline{\underline{\thmname{#1}}}}%%% Thm head spec
%% \theoremstyle{ResponseStyle}

\usepackage[tikz]{mdframed}
\mdfdefinestyle{ResponseStyle}{leftmargin=1cm,linecolor=black,roundcorner=5pt,
, font=\bsifamily,}%font=\wedn\bfseries\upshape,}


\ifhandout
\NewEnviron{freeResponse}{}
\else
%\newtheorem{freeResponse}{Response:}
\newenvironment{freeResponse}{\begin{mdframed}[style=ResponseStyle]}{\end{mdframed}}
\fi



%% attempting to automate outcomes.

%% \newwrite\outcomefile
%%   \immediate\openout\outcomefile=\jobname.oc
%% \renewcommand{\outcome}[1]{\edef\theoutcomes{\theoutcomes #1~}%
%% \immediate\write\outcomefile{\unexpanded{\outcome}{#1}}}

%% \newcommand{\outcomelist}{\begin{itemize}\theoutcomes\end{itemize}}

%% \NewEnviron{listOutcomes}{\small\sffamily
%% After answering the following questions, students should be able to:
%% \begin{itemize}
%% \BODY
%% \end{itemize}
%% }
\usepackage[tikz]{mdframed}
\mdfdefinestyle{OutcomeStyle}{leftmargin=2cm,rightmargin=2cm,linecolor=black,roundcorner=5pt,
, font=\small\sffamily,}%font=\wedn\bfseries\upshape,}
\newenvironment{listOutcomes}{\begin{mdframed}[style=OutcomeStyle]After answering the following questions, students should be able to:\begin{itemize}}{\end{itemize}\end{mdframed}}



%% my commands

\newcommand{\snap}{{\bfseries\itshape\textsf{Snap!}}}
\newcommand{\flavor}{\link[\snap]{https://snap.berkeley.edu/}}
\newcommand{\mooculus}{\textsf{\textbf{MOOC}\textnormal{\textsf{ULUS}}}}


\usepackage{tkz-euclide}
\tikzstyle geometryDiagrams=[rounded corners=.5pt,ultra thick,color=black]
\colorlet{penColor}{black} % Color of a curve in a plot



\ifhandout\newcommand{\mynewpage}{\newpage}\else\newcommand{\mynewpage}{}\fi

\title{The distance formula}

\author{Bart Snapp}

\begin{document}
\begin{abstract}
  Let's think of consequences of the Pythagorean theorem.
\end{abstract}
\maketitle


\begin{listOutcomes}
\item Find various loci of points.
\item Use facts about the foci of ellipses to solve problems.
\item State the definition of a circle.
\item State the definition of an ellipse.
\item{Critique and dismantle reasonable hypotheses in regard to geometry and arithmetic.}
\item State the definition of the circumcenter of a triangle.
\item State the definition of the circumcircle of a triangle.
\item Recognize that circles are determined by three points.
\end{listOutcomes}


\mynewpage



\begin{question}
  Use the INTERNET to look up the distance formula. 
  \begin{enumerate}
    \item STATE the distance formula. 
    \item EXPLAIN (perhaps with the help of a picture) how the
      distance formula follows from the Pythagorean theorem.
    \item A \textbf{circle} is the set of points equidistant from a
      given point. DERIVE (you may use the INTERNET to help you) the
      \textbf{standard equation of a circle} of radius $r$ centered at
      the origin.  
  \end{enumerate}
  \begin{freeResponse}
    \begin{enumerate}
    \item The \underline{distance formula} says that the distance between two points $(x,y)$ and $(a,b)$ is
      \[
      \text{distance} = \sqrt{(x-a)^2 + (y-b)^2}.
      \]
    \item This is basically the Pythagorean theorem. Plot the points
      $(a,b)$ and $(x,y)$:
    \begin{center}
      \begin{tikzpicture}[geometryDiagrams]
	\begin{axis}[
            xmin=0, xmax=5,ymin=0,ymax=5,
            axis lines =left, 
            every axis y label/.style={at=(current axis.above origin),anchor=south},
            every axis x label/.style={at=(current axis.right of origin),anchor=west},
            xtick={1,3}, xticklabels={$a$,$x$},
            ytick={2,3}, yticklabels={$b$,$y$},
            axis on top,
          ]       
          \addplot[,dashed] plot coordinates {(1,0) (1,2)};
          \addplot[,dashed] plot coordinates {(0,2) (1,2)};
          \addplot[,dashed] plot coordinates {(3,0) (3,3)};
          \addplot[,dashed] plot coordinates {(0,3) (3,3)};
          \addplot[,only marks,mark=*] coordinates{(1,2)};  %% closed hole          
          \addplot[,only marks,mark=*] coordinates{(3,3)};  %% closed hole          
        \end{axis}
      \end{tikzpicture}
    \end{center}
    We may now construct a right triangle with horizontal side length
    $(x-a)$ and vertical side length
    $(y-b)$ whose hypotenuse is the shortest path
    between the two points:
    \begin{center}
      \begin{tikzpicture}[geometryDiagrams]
        \begin{axis}[
            xmin=0, xmax=5,ymin=0,ymax=5,
            axis lines =left, 
            every axis y label/.style={at=(current axis.above origin),anchor=south},
            every axis x label/.style={at=(current axis.right of origin),anchor=west},
            xtick={1,3}, xticklabels={$a$,$x$},
            ytick={2,3}, yticklabels={$b$,$y$},
            axis on top,
          ]       
          \node at (axis cs:2,1.75) {$(x-a)$}; 
          \node at (axis cs:3.5,2.5) {$(y-b)$}; 
          \addplot[dashed] plot coordinates {(1,2) (3,3)};
          \addplot[] plot coordinates {(1,2) (3,2)};
          \addplot[] plot coordinates {(3,2) (3,3)};
          \addplot[] plot coordinates {(2.83,2) (2.83,2.2) (3,2.2)};
          \addplot[,only marks,mark=*] coordinates{(1,2)};  %% closed hole          
          \addplot[only marks,mark=*] coordinates{(3,3)};  %% closed hole          
        \end{axis}
      \end{tikzpicture}
    \end{center}
    By the Pythagorean theorem, the length of this path is given by 
    \[
    \text{distance}=\sqrt{(x-a)^{2}+(y-b)^{2}}.
    \]
  \item Well, write
      \begin{align*}
        r &= \sqrt{x^2 + y^2}\\
        r^2 &= x^2 + y^2.
      \end{align*}
      This is the standard equaion of a circle of radius $r$ centered
      at the origin.
    \end{enumerate}
  \end{freeResponse}
\end{question}
\mynewpage













\begin{question}
  Ellipses are also directly related to the distance formula.  Given
  two points
  \[
  (-c,0) \qquad\text{and} \qquad (c,0)
  \]
  that we will call \textbf{foci}, an \textbf{ellipse} is the set
  of points such that the sum of the distances from any point on
  the ellipse to the foci is constant.
  \begin{enumerate}      
  \item Consider an ellipse with foci:
    \[
    (-3,0) \qquad \text{and}\qquad (3,0)
    \]%\frac{x^{2}}{5^{2}}+\frac{y^{2}}{4^{2}}=1
    Which of the following points belong on the ellipse where the sum
    of the distances from any point on the ellipse to the foci is
    $10$?
    \begin{enumerate}
    \item $(3,3.2)$ % yes = 10
    \item $(-4,2.8)$ % no = 10.5124...
    \item $(4.8,0.7)$ % no = 9.76267,,
    \item $(2,-3.6666\dots)$ % yes = 10.
    \end{enumerate}
    \item Consider this drawing of an ellipse:
      \begin{center}
        \begin{tikzpicture}[geometryDiagrams,scale=.8]
          \draw (0,0) ellipse (2in and 1in);
          \draw[->,thin] (0in,-1.5in) -- (0in,1.5in);
          \draw[->,thin] (-2.5in,0in) -- (2.5in,0in);
          \filldraw[black] (2in,0) circle (2pt) node[above right] {(a,0)};
          \filldraw[black] (0,1in) circle (2pt) node[above right] {(0,b)};

          \filldraw[black] (1.73in,0in) circle (2pt) node[below left] {(c,0)};
          \filldraw[black] (-1.73in,0in) circle (2pt) node[below right] {(-c,0)};

          \draw[dashed,thin] (-1.73in,0in) -- (.9in,.89in);
          \draw[dashed,thin] (1.73in,0in) -- (.9in,.89in);

          \node at (-.75in,.5in) {$X$};
          \node at (1.2in,.35in) {$Y$};
          
          \filldraw[black] (.9in,.89in) circle (2pt);
        \end{tikzpicture}
      \end{center}
      Express $X+Y$ (as simply as you can!) in terms of $a$ and/or $b$
      and/or $c$.
    \item \textit{Geometry Giorgio} suggests that the foci of this ellipse
      \begin{center}
        \begin{tikzpicture}[geometryDiagrams,scale=.8]
          \draw (0,0) ellipse (2in and 1in);
          \draw[->,thin] (0in,-1.5in) -- (0in,1.5in);
          \draw[->,thin] (-2.5in,0in) -- (2.5in,0in);

          
          \filldraw[black] (2in,0) circle (2pt) node[above right] {(2,0)};
          \filldraw[black] (0,1in) circle (2pt) node[above right] {(0,1)};

          \filldraw[black] (1in,0in) circle (2pt) node[below] {(1,0)};
          \filldraw[black] (-1in,0in) circle (2pt) node[below] {(-1,0)};
        \end{tikzpicture}
      \end{center}
      are at $(-1,0)$ and $(1,0)$. Prove him RIGHT or prove him WRONG.
  \end{enumerate}
  \begin{freeResponse}
    \begin{enumerate}
    \item Let's see, of these
      \begin{itemize}
      \item $\sqrt{(3-3)^2+3.2^2} + \sqrt{(3+3)^2 + 3.2^2} = 10$, so this point is on the ellipse.
      \item $\sqrt{(-4-3)^2+2.8^2} + \sqrt{(-4+3)^2 + 2.8^2} \approx 10.5$, so this point is NOT on the ellipse.
      \item $\sqrt{(4.8-3)^2+0.7^2} + \sqrt{(4.8+3)^2 + 0.7^2} \approx 9.8$, so this point is NOT on the ellipse.
      \item $\sqrt{(2-3)^2+(-3.\bar{6})^2} + \sqrt{(2+3)^2 + (-3.\bar{6})^2} = 10$, so this point is on the ellipse.
      \end{itemize}
    \item Well $X+Y=2a$, as if we look at the distance from the points to $(a,0)$ we find
      \[
      2c + (a-c) + (a-c) = 2a.
      \]
    \item NO Geometry Giorgio is WRONG. The sum of the distances from
      the foci to the point $(0,1)$ should be $4$, but instead they
      are
      \[
      \sqrt{2} + \sqrt{2} = 2\sqrt{2} \ne 4.
      \]
    \end{enumerate}
  \end{freeResponse}
\end{question}
\mynewpage












\begin{question}
  Now we're going to touch on something that will come up later.
  \begin{enumerate}
  \item Suppose you have two points. SKETCH the set of points that are
    each simultaneously EQUIDISTANT from the two given points.
  \item Now suppose you have THREE points, not in a line. How many
    points are EQUIDISTANT to each of your three points?
  \item Explain why the locus of points equidistant from the point
    found in the previous part is the center of a circle containing
    those three points from the previous part.
  \item Connect your three points with lines. Now imagine a
    triangle. If you want a point equidistant from the vertices of the
    triangle, should you:
    \begin{itemize}
    \item Intersect the angle bisectors?
    \item Intersect the perpendicular bisectors of the sides?
    \item Something else entirely?
    \end{itemize}
    Explain your reasoning.
  \item Cool folks say that ``A circle is determined by three points.'' What does this mean?
  \end{enumerate}
  \begin{freeResponse}
    \begin{enumerate}
    \item Given two points, the set of points that is simultaneously
      equidistant from both is a line:
      \begin{center}
        \begin{tikzpicture}[geometryDiagrams]
          \filldraw[black] (.5in,.5in) circle (2pt) ;
          \filldraw[black] (-.5in,-.5in) circle (2pt) ;
          \draw[<->,dashed] (-2,2) -- (2,-2);
        \end{tikzpicture}
      \end{center}
    \item This is going to be a single point!
    \item
    \end{enumerate}
  \end{freeResponse}
\end{question}

\end{document}
