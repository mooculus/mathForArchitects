\documentclass[nooutcomes,noauthor,handout]{ximera}

\graphicspath{  
{./}
{./whoAreYou/}
{./drawingWithTheTurtle/}
{./bisectionMethod/}
{./circles/}
{./anglesAndRightTriangles/}
{./lawOfSines/}
{./lawOfCosines/}
{./plotter/}
{./staircases/}
{./pitch/}
{./qualityControl/}
{./symmetry/}
{./nGonBlock/}
}


%% page layout
\usepackage[cm,headings]{fullpage}
\raggedright
\setlength\headheight{13.6pt}


%% fonts
\usepackage{euler}

\usepackage{FiraMono}
\renewcommand\familydefault{\ttdefault} 
\usepackage[defaultmathsizes]{mathastext}
\usepackage[htt]{hyphenat}

\usepackage[T1]{fontenc}
\usepackage[scaled=1]{FiraSans}

%\usepackage{wedn}
\usepackage{pbsi} %% Answer font


\usepackage{cancel} %% strike through in pitch/pitch.tex


%% \usepackage{ulem} %% 
%% \renewcommand{\ULthickness}{2pt}% changes underline thickness

\tikzset{>=stealth}

\usepackage{adjustbox}

\setcounter{titlenumber}{-1}

%% journal style
\makeatletter
\newcommand\journalstyle{%
  \def\activitystyle{activity-chapter}
  \def\maketitle{%
    \addtocounter{titlenumber}{1}%
                {\flushleft\small\sffamily\bfseries\@pretitle\par\vspace{-1.5em}}%
                {\flushleft\LARGE\sffamily\bfseries\thetitlenumber\hspace{1em}\@title \par }%
                {\vskip .6em\noindent\textit\theabstract\setcounter{question}{0}\setcounter{sectiontitlenumber}{0}}%
                    \par\vspace{2em}
                    \phantomsection\addcontentsline{toc}{section}{\thetitlenumber\hspace{1em}\textbf{\@title}}%
                     }}
\makeatother



%% thm like environments
\let\question\relax
\let\endquestion\relax

\newtheoremstyle{QuestionStyle}{\topsep}{\topsep}%%% space between body and thm
		{}                      %%% Thm body font
		{}                              %%% Indent amount (empty = no indent)
		{\bfseries}            %%% Thm head font
		{)}                              %%% Punctuation after thm head
		{ }                           %%% Space after thm head
		{\thmnumber{#2}\thmnote{ \bfseries(#3)}}%%% Thm head spec
\theoremstyle{QuestionStyle}
\newtheorem{question}{}



\let\freeResponse\relax
\let\endfreeResponse\relax

%% \newtheoremstyle{ResponseStyle}{\topsep}{\topsep}%%% space between body and thm
%% 		{\wedn\bfseries}                      %%% Thm body font
%% 		{}                              %%% Indent amount (empty = no indent)
%% 		{\wedn\bfseries}            %%% Thm head font
%% 		{}                              %%% Punctuation after thm head
%% 		{3ex}                           %%% Space after thm head
%% 		{\underline{\underline{\thmname{#1}}}}%%% Thm head spec
%% \theoremstyle{ResponseStyle}

\usepackage[tikz]{mdframed}
\mdfdefinestyle{ResponseStyle}{leftmargin=1cm,linecolor=black,roundcorner=5pt,
, font=\bsifamily,}%font=\wedn\bfseries\upshape,}


\ifhandout
\NewEnviron{freeResponse}{}
\else
%\newtheorem{freeResponse}{Response:}
\newenvironment{freeResponse}{\begin{mdframed}[style=ResponseStyle]}{\end{mdframed}}
\fi



%% attempting to automate outcomes.

%% \newwrite\outcomefile
%%   \immediate\openout\outcomefile=\jobname.oc
%% \renewcommand{\outcome}[1]{\edef\theoutcomes{\theoutcomes #1~}%
%% \immediate\write\outcomefile{\unexpanded{\outcome}{#1}}}

%% \newcommand{\outcomelist}{\begin{itemize}\theoutcomes\end{itemize}}

%% \NewEnviron{listOutcomes}{\small\sffamily
%% After answering the following questions, students should be able to:
%% \begin{itemize}
%% \BODY
%% \end{itemize}
%% }
\usepackage[tikz]{mdframed}
\mdfdefinestyle{OutcomeStyle}{leftmargin=2cm,rightmargin=2cm,linecolor=black,roundcorner=5pt,
, font=\small\sffamily,}%font=\wedn\bfseries\upshape,}
\newenvironment{listOutcomes}{\begin{mdframed}[style=OutcomeStyle]After answering the following questions, students should be able to:\begin{itemize}}{\end{itemize}\end{mdframed}}



%% my commands

\newcommand{\snap}{{\bfseries\itshape\textsf{Snap!}}}
\newcommand{\flavor}{\link[\snap]{https://snap.berkeley.edu/}}
\newcommand{\mooculus}{\textsf{\textbf{MOOC}\textnormal{\textsf{ULUS}}}}


\usepackage{tkz-euclide}
\tikzstyle geometryDiagrams=[rounded corners=.5pt,ultra thick,color=black]
\colorlet{penColor}{black} % Color of a curve in a plot



\ifhandout\newcommand{\mynewpage}{\newpage}\else\newcommand{\mynewpage}{}\fi

\title{The distance formula}

\begin{document}
\begin{abstract}
  Let's think of consequences of the Pythagorean theorem.
\end{abstract}
\maketitle


\begin{listOutcomes}
\item Find various loci of points.
\item Use facts about the foci of ellipses to solve problems.
\item State the definition of a circle.
\item State the definition of an ellipse.
\item{Critique and dismantle reasonable hypotheses in regard to geometry and arithmetic.}
\item State the definition of the circumcenter of a triangle.
\item State the definition of the circumcircle of a triangle.
\item Recognize that circles are determined by three points.
\end{listOutcomes}


\mynewpage



\begin{question}
  Use the INTERNET to look up the distance formula. 
  \begin{enumerate}
    \item STATE the distance formula. 
    \item EXPLAIN how the distance formula follows from the
      Pythagorean theorem.
    \item Now suppose you have two points. USE the distance formula to
      find the set of points that are EQUIDISTANT from the two given
      points.
    \item A \textbf{circle} is the set of points equidistant from a
      given point. DERIVE (you may use the INTERNET to help you) the
      \textbf{standard equation of a circle} of radius $r$ centered at
      the origin.  
  \end{enumerate}
  \begin{freeResponse}
  \end{freeResponse}
\end{question}
\mynewpage


\begin{question}
  Ellipses are also directly related to the distance formula.  Given
  two points
  \[
  (-c,0) \qquad\text{and} \qquad (c,0)
  \]
  that we will call \textbf{foci}, an \textbf{ellipse} is the set
  of points such that the sum of the distances from any point on
  the ellipse to the foci is constant.
  \begin{enumerate}      
  \item Consider an ellipse with foci:
    \[
    (-3,0) \qquad \text{and}\qquad (3,0)
    \]%\frac{x^{2}}{5^{2}}+\frac{y^{2}}{4^{2}}=1
    Which of the following points belong on the ellipse where the sum
    of the distances from any point on the ellipse to the foci is
    $10$?
    \begin{enumerate}
    \item $(3,3.2)$
    \item $(-4,2.8)$
    \item $(4.8,0.7)$
    \item $(2,-3.6666\dots)$
    \end{enumerate}
    \item Consider this drawing of an ellipse:
      \begin{center}
        \begin{tikzpicture}[geometryDiagrams,scale=.8]
          \draw (0,0) ellipse (2in and 1in);
          \draw[->,thin] (0in,-1.5in) -- (0in,1.5in);
          \draw[->,thin] (-2.5in,0in) -- (2.5in,0in);
          \filldraw[black] (2in,0) circle (2pt) node[above right] {(a,0)};
          \filldraw[black] (0,1in) circle (2pt) node[above right] {(0,b)};

          \filldraw[black] (1.73in,0in) circle (2pt) node[below left] {(c,0)};
          \filldraw[black] (-1.73in,0in) circle (2pt) node[below right] {(-c,0)};

          \draw[dashed,thin] (-1.73in,0in) -- (.9in,.89in);
          \draw[dashed,thin] (1.73in,0in) -- (.9in,.89in);

          \node at (-.75in,.5in) {$X$};
          \node at (1.2in,.35in) {$Y$};
          
          \filldraw[black] (.9in,.89in) circle (2pt);
        \end{tikzpicture}
      \end{center}
      Express $X+Y$ (as simply as you can!) in terms of $a$ and/or $b$
      and/or $c$.
    \item \textit{Geometry Giorgio} suggests that the foci of this ellipse
      \begin{center}
        \begin{tikzpicture}[geometryDiagrams,scale=.8]
          \draw (0,0) ellipse (2in and 1in);
          \draw[->,thin] (0in,-1.5in) -- (0in,1.5in);
          \draw[->,thin] (-2.5in,0in) -- (2.5in,0in);

          
          \filldraw[black] (2in,0) circle (2pt) node[above right] {(2,0)};
          \filldraw[black] (0,1in) circle (2pt) node[above right] {(0,1)};

          \filldraw[black] (1in,0in) circle (2pt) node[below] {(1,0)};
          \filldraw[black] (-1in,0in) circle (2pt) node[below] {(-1,0)};
        \end{tikzpicture}
      \end{center}
      are at $(-1,0)$ and $(1,0)$. Prove him RIGHT or prove him WRONG.
  \end{enumerate}
\end{question}
\mynewpage


\begin{question}
  Imagine you have three points on a plane and the points are NOT in a
  line.
  \begin{enumerate}
  \item How many points are EQUIDISTANT to each of your three points?
  \item Explain why your locus is the center of a circle containing those three points.
  \item Connect your three points with lines. Now imagine a
    triangle. If you want a point equidistant from the vertices of the
    triangle, should you:
    \begin{itemize}
    \item Intersect the angle bisectors?
    \item Intersect the perpendicular bisectors of the sides?
    \item Something else entirely?
    \end{itemize}
    Explain your reasoning.
  \item Cool folks say that ``A circle is determined by three points.'' What does this mean?
  \end{enumerate}
\end{question}

\end{document}
