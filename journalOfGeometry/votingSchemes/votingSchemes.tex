\documentclass[nooutcomes,noauthor,hints,handout]{ximera}

\graphicspath{  
{./}
{./whoAreYou/}
{./drawingWithTheTurtle/}
{./bisectionMethod/}
{./circles/}
{./anglesAndRightTriangles/}
{./lawOfSines/}
{./lawOfCosines/}
{./plotter/}
{./staircases/}
{./pitch/}
{./qualityControl/}
{./symmetry/}
{./nGonBlock/}
}


%% page layout
\usepackage[cm,headings]{fullpage}
\raggedright
\setlength\headheight{13.6pt}


%% fonts
\usepackage{euler}

\usepackage{FiraMono}
\renewcommand\familydefault{\ttdefault} 
\usepackage[defaultmathsizes]{mathastext}
\usepackage[htt]{hyphenat}

\usepackage[T1]{fontenc}
\usepackage[scaled=1]{FiraSans}

%\usepackage{wedn}
\usepackage{pbsi} %% Answer font


\usepackage{cancel} %% strike through in pitch/pitch.tex


%% \usepackage{ulem} %% 
%% \renewcommand{\ULthickness}{2pt}% changes underline thickness

\tikzset{>=stealth}

\usepackage{adjustbox}

\setcounter{titlenumber}{-1}

%% journal style
\makeatletter
\newcommand\journalstyle{%
  \def\activitystyle{activity-chapter}
  \def\maketitle{%
    \addtocounter{titlenumber}{1}%
                {\flushleft\small\sffamily\bfseries\@pretitle\par\vspace{-1.5em}}%
                {\flushleft\LARGE\sffamily\bfseries\thetitlenumber\hspace{1em}\@title \par }%
                {\vskip .6em\noindent\textit\theabstract\setcounter{question}{0}\setcounter{sectiontitlenumber}{0}}%
                    \par\vspace{2em}
                    \phantomsection\addcontentsline{toc}{section}{\thetitlenumber\hspace{1em}\textbf{\@title}}%
                     }}
\makeatother



%% thm like environments
\let\question\relax
\let\endquestion\relax

\newtheoremstyle{QuestionStyle}{\topsep}{\topsep}%%% space between body and thm
		{}                      %%% Thm body font
		{}                              %%% Indent amount (empty = no indent)
		{\bfseries}            %%% Thm head font
		{)}                              %%% Punctuation after thm head
		{ }                           %%% Space after thm head
		{\thmnumber{#2}\thmnote{ \bfseries(#3)}}%%% Thm head spec
\theoremstyle{QuestionStyle}
\newtheorem{question}{}



\let\freeResponse\relax
\let\endfreeResponse\relax

%% \newtheoremstyle{ResponseStyle}{\topsep}{\topsep}%%% space between body and thm
%% 		{\wedn\bfseries}                      %%% Thm body font
%% 		{}                              %%% Indent amount (empty = no indent)
%% 		{\wedn\bfseries}            %%% Thm head font
%% 		{}                              %%% Punctuation after thm head
%% 		{3ex}                           %%% Space after thm head
%% 		{\underline{\underline{\thmname{#1}}}}%%% Thm head spec
%% \theoremstyle{ResponseStyle}

\usepackage[tikz]{mdframed}
\mdfdefinestyle{ResponseStyle}{leftmargin=1cm,linecolor=black,roundcorner=5pt,
, font=\bsifamily,}%font=\wedn\bfseries\upshape,}


\ifhandout
\NewEnviron{freeResponse}{}
\else
%\newtheorem{freeResponse}{Response:}
\newenvironment{freeResponse}{\begin{mdframed}[style=ResponseStyle]}{\end{mdframed}}
\fi



%% attempting to automate outcomes.

%% \newwrite\outcomefile
%%   \immediate\openout\outcomefile=\jobname.oc
%% \renewcommand{\outcome}[1]{\edef\theoutcomes{\theoutcomes #1~}%
%% \immediate\write\outcomefile{\unexpanded{\outcome}{#1}}}

%% \newcommand{\outcomelist}{\begin{itemize}\theoutcomes\end{itemize}}

%% \NewEnviron{listOutcomes}{\small\sffamily
%% After answering the following questions, students should be able to:
%% \begin{itemize}
%% \BODY
%% \end{itemize}
%% }
\usepackage[tikz]{mdframed}
\mdfdefinestyle{OutcomeStyle}{leftmargin=2cm,rightmargin=2cm,linecolor=black,roundcorner=5pt,
, font=\small\sffamily,}%font=\wedn\bfseries\upshape,}
\newenvironment{listOutcomes}{\begin{mdframed}[style=OutcomeStyle]After answering the following questions, students should be able to:\begin{itemize}}{\end{itemize}\end{mdframed}}



%% my commands

\newcommand{\snap}{{\bfseries\itshape\textsf{Snap!}}}
\newcommand{\flavor}{\link[\snap]{https://snap.berkeley.edu/}}
\newcommand{\mooculus}{\textsf{\textbf{MOOC}\textnormal{\textsf{ULUS}}}}


\usepackage{tkz-euclide}
\tikzstyle geometryDiagrams=[rounded corners=.5pt,ultra thick,color=black]
\colorlet{penColor}{black} % Color of a curve in a plot



\ifhandout\newcommand{\mynewpage}{\newpage}\else\newcommand{\mynewpage}{}\fi

\title{Voting schemes}

\author{Bart Snapp}

\begin{document}
\begin{abstract}
  We investigate three popular voting schemes for when there are three
  or more options.
\end{abstract}
\maketitle

\begin{listOutcomes}
\item Organize and accommodate data.
\item Critically analyze data.
\item Translate classroom mathematics into real world mathematics.
\end{listOutcomes}


The \mooculus~apartment building is holding an `election' among
their $38$ residents to see if they should allow dogs, or cats, or no
pets at all. Of the residents:
\begin{itemize}
\item $16$ residents prefer dogs, better than cats, better than no pets at all.
\item $4$ residents prefer cats, better than dogs, better than no pets at all.
\item $18$ residents prefer no pets at all, better than cats, better
  than dogs.
\end{itemize}



If there are three or more choices, then we have options as to how
exactly we vote. 
\begin{description}
\item[Plurality Voting:] The option with the most
  votes wins.
\item[Borda Count:] Voters enumerate their preferences as $2,1,0$
  with $2$ meaning `most preferred' and $0$ meaning `least preferred.'
  The option with the largest sum from this ranked-vote wins.
\item[Instant Run-Off Voting:] Voters list their preferences from `most
  preferred' to `least preferred.' A plurality vote is taken among the
  most preferred. All votes for losers, are transferred to remaining
  options based on the order. This process is repeated until there is
  one winner. This is also known as \textbf{Ranked Choice Voting}.
\end{description}







\mynewpage



%% MAybe do all 3 then analyze?
%1) count
%2) analyze
%3) Borda blues?

\begin{question}
  Compute the outcomes of the \mooculus~apartment pet `election'
  using
  \begin{enumerate}
  \item \textbf{Plurality Voting},
  \item \textbf{Borda Count}, and
  \item \textbf{Instant Run-off Voting}.
  \end{enumerate}
  Explain your reasoning and show all work.
  \begin{freeResponse}
    Using \underline{Plurality Voting}, we see that ``no pets'' wins, as that
    option got $18$ votes, compared to the other option's $16$ and
    $4$.


    
    Using \underline{Borda Count}, we have
    \begin{description}
      \item[Dogs:] $16\cdot 2 + 4 = 36$.
      \item[Cats:] $16 +4\cdot 2+18 =42$.
      \item[No Pets:] $18\cdot 2 = 36$.
    \end{description}
  
  We see that ``cats'' wins.

  

  Using \underline{Instant Run-Off Voting}, we have that ``dogs'' and
  ``no pets'' win the first round, but the $4$ votes for ``cats'' go
  to ``dogs'' next, and ``dogs'' win.
  \end{freeResponse}
\end{question}
\mynewpage



\begin{question}
  Analyze the results of each of your `elections' from above. \textbf{For each
  method of counting the vote,} answer
  \begin{enumerate}
  \item How many people got their most preferred option?
  \item How many people got their medium option? 
  \item How many people got their least preferred option?
  \end{enumerate}
  by filling in the table below
  \begin{center}\renewcommand{\arraystretch}{1.5}
    \begin{tabular}{|l||c|c|c|}\hline
      Voting Scheme & \# most preferred & \# medium & \# least preferred \\ \hline\hline
      Plurality  Voting &  \rule[7mm]{10mm}{0mm}& & \\ \hline
      Borda Count &  \rule[7mm]{10mm}{0mm}&  & \\ \hline
      Instant Run-Off Voting & \rule[7mm]{10mm}{0mm}&  &\\\hline
    \end{tabular}
  \end{center}
  
  After you have evaluated each voting system, which \textbf{voting system} do
  \textbf{you think} reflects the ``will of the people'' \textbf{best?}
\end{question}
\mynewpage








\begin{question}
  Find an example of Instant Run-Off Voting (Rank Choice Voting) and
  example of Borda Count being used in the real world. Briefly
  describe the situation where each has been used below.
    
\end{question}

\end{document}
