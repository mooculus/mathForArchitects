\documentclass[handout,nooutcomes,noauthor]{ximera}

%% page layout
\usepackage[in,headings]{fullpage}
\raggedright
\setlength\headheight{13.6pt}


%% fonts
\usepackage{euler}

\usepackage{FiraMono}
\renewcommand\familydefault{\ttdefault} 
\usepackage{mathastext}
\usepackage[htt]{hyphenat}

\usepackage[T1]{fontenc}
\usepackage[scaled=1]{FiraSans}

\usepackage{wedn}
\usepackage[T1]{fontenc}

%% wrap text around scripts
\usepackage{wrapfig}

\tikzset{>=stealth}
%% snap! scripts
\usepackage{scratch3}

\usepackage{adjustbox}

%% journal style
\makeatletter
\newcommand\journalstyle{%
  \def\activitystyle{activity-chapter}
  \def\maketitle{%
    \addtocounter{titlenumber}{1}%
                {\flushleft\small\sffamily\bfseries\@pretitle\par\vspace{-1.5em}}%
                {\flushleft\LARGE\sffamily\bfseries\thetitlenumber\hspace{1em}\@title \par }%
                {\vskip .6em\noindent\textit\theabstract\setcounter{question}{0}\setcounter{sectiontitlenumber}{0}}%
                    \par\vspace{2em}
                    \phantomsection\addcontentsline{toc}{section}{\thetitlenumber\hspace{1em}\textbf{\@title}}%
                     }}
\makeatother



%% thm like environments
\let\question\relax
\let\endquestion\relax

\newtheoremstyle{QuestionStyle}{\topsep}{\topsep}%%% space between body and thm
		{}                      %%% Thm body font
		{}                              %%% Indent amount (empty = no indent)
		{\bfseries}            %%% Thm head font
		{)}                              %%% Punctuation after thm head
		{ }                           %%% Space after thm head
		{\thmnumber{#2}\thmnote{ \bfseries(#3)}}%%% Thm head spec
\theoremstyle{QuestionStyle}
\newtheorem{question}{}



\let\freeResponse\relax
\let\endfreeResponse\relax

%% \newtheoremstyle{ResponseStyle}{\topsep}{\topsep}%%% space between body and thm
%% 		{\wedn\bfseries}                      %%% Thm body font
%% 		{}                              %%% Indent amount (empty = no indent)
%% 		{\wedn\bfseries}            %%% Thm head font
%% 		{}                              %%% Punctuation after thm head
%% 		{3ex}                           %%% Space after thm head
%% 		{\underline{\underline{\thmname{#1}}}}%%% Thm head spec
%% \theoremstyle{ResponseStyle}

\usepackage[tikz]{mdframed}
\mdfdefinestyle{ResponseStyle}{leftmargin=1cm,linecolor=black,roundcorner=5pt,frametitlefont=\wedn\bfseries,%frametitle={\underline{\underline{Response}}:}
, font=\wedn\bfseries,}%\begin{mdframed}[style=mystyle]foo\end{mdframed}


\ifhandout
\NewEnviron{freeResponse}{}
\else
%\newtheorem{freeResponse}{Response:}
\newenvironment{freeResponse}{\begin{mdframed}[style=ResponseStyle]}{\end{mdframed}}
\fi



%% attempting to automate outcomes.

\newwrite\outcomefile
  \immediate\openout\outcomefile=\jobname.oc
\renewcommand{\outcome}[1]{\edef\theoutcomes{\theoutcomes #1~}%
\immediate\write\outcomefile{\unexpanded{\outcome}{#1}}}

%% \newcommand{\outcomelist}{\begin{itemize}\theoutcomes\end{itemize}}



%% my commands

\newcommand{\snap}{{\bfseries\itshape\textsf{Snap!}}}
\newcommand{\flavor}{\link[\snap]{https://snap.berkeley.edu/}}


\usepackage{tkz-euclide}
\tikzstyle geometryDiagrams=[rounded corners=.5pt,ultra thick,color=black]
\colorlet{penColor}{black} % Color of a curve in a plot

\title{Not so quick questions}


\author{Bart Snapp}

\begin{document}
\begin{abstract}
  Let's apply our knowledge.
\end{abstract}
\maketitle


\begin{listOutcomes}
\item 
\end{listOutcomes}


Imagine a house where the roof where two slopes meet. This will either
form a ``hip'' or a ``valley''.
\begin{center}
  \includegraphics[width=.8\textwidth]{house.jpg}
\end{center}
There are two basic questions we'd like to answer about shapes like these:
\begin{itemize}
\item Given the dimensions (as seen from the top) of the roof, What is the AREA of the roof?
\item Given the SLOPE of the roofs that meet, What the slope (or
  pitch) of the hip or valley?
\end{itemize}


\mynewpage

%% IDENTIFY OBIVIOUSLY CORRECT AND INCORRECT ANSWERS
%% AGAIN WITH THE AREA/VOL
%% solve a complex diagram
%% Give several quick questions and ask how they were supposed to figure them out.


%https://www2.strongtie.com/webapps/SlopeSkew/?source=app

%% EASY COMBINED PITCH
%% HARDER COMBINED PITCH
%% computer combined pitch
%% QUIZ: COULD IT JUST BE IS THIS A VALLEY OR A RIDGE?

\begin{question}
\begin{enumerate}
\item Consider the \link[\textit{Luxor Las
    Vegas}]{https://en.wikipedia.org/wiki/Luxor_Las_Vegas}:
  \begin{center}
    \includegraphics[width=.4\textwidth]{pyramid.jpg} 
  \end{center}
 Assuming that the (square) base of this building has a side length of
 $646$ feet and that the pyramid is $350$ feet tall, compute the
 SURFACE AREA in square feet, NOT including the bottom. 
 
\item Here are plans for a $1$-car garage that I got from \link[\textit{Garage Plans by Behm Design}]{https://behmdesign.com/shop/}:
   \begin{center}
     \includegraphics[width=.6\textwidth]{oneCarGarage.jpg}
   \end{center}
   Assuming that the hip-end has a slope of $\frac{6}{12}$, and that
   the roof has an overhang of $2$ feet on all sides, find the AREA of
   the roof of the garage in square feet.
 \item Considering the $1$-car garage from the previous part, now
   suppose that the hip-end has a slope of $\frac{8}{12}$ with an
   overhange of $3$ feet. Again find the AREA of the roof of the
   garage in square feet.
\end{enumerate}
In ALL cases above, SHOW and EXPLAIN your work.
\begin{freeResponse}
  \begin{enumerate}
    \item We must compute the height of one face of the pyramid. By
      the Pythagorean theorem, we have
      \[
      323^2 + 350^2 = h^2
      \]
      so
      \[
      h \approx 476.
      \]
      This means the top of the pyramid has a surface area of
      \[
      \text{Area} = 4\cdot \frac{646\cdot 476}{2} = 614992~\text{square feet}.
      \]
    \item Since there is an overhang of $2$ feet per side, 
    
  \end{enumerate}
\end{freeResponse}


\end{question}
\mynewpage















\begin{question}
  Now let's think about the SLOPE of a hip or valley. Below we see a
  sneaky way to compute these numbers.
\begin{quote}
  \textbf{Algorithm for computing the slope of a hip or valley:}

  Suppose the slopes are $\frac{s}{12}$ and $\frac{\l}{12}$ with
$s\le \l$.

\begin{enumerate}
\item Draw a point $O$ sitting at the intersection of a horizontal
  and vertical line.
\item Draw a vertical line segment $\bar{OS}$ where $S$ is directly
  above $O$ and $|OS| = s$.
  \item Starting from $S$, draw a line with the larger slope
    $\frac{\l}{12}$ and call the intersection of this line with the
    horizontal line point $X$.
  \item Starting at point $O$, extend the line segment $\bar{OS}$ by
    drawing downward $12$ units, and call the bottom of this segment
    $B$.
     \begin{center}
        \begin{tikzpicture}[geometryDiagrams,scale=.2]
          %% \filldraw[black] (1in,1in) circle (2pt) ;
          %% \filldraw[black] (0in,0in) circle (2pt) ;
          %% \filldraw[black] (0in,1in) circle (2pt) ;
          %% \draw[<->,dashed] (1.5in,-.5in) -- (-.5in,1.5in);

          \tkzDefPoint(0,0){O}
          \tkzDefPoint(-13,0){I}
          \tkzDefPoint(1,0){nI}
          \tkzDefPoint(0,7){II}
          \tkzDefPoint(0,-13){nII}
          
          \tkzDefPoint(0,6){S}
          \tkzDefPoint(-9,0){X}
          \tkzDefPoint(-12,-2){BS}
          \tkzDefPoint(0,-12){B}

          \tkzDrawPoint[](O)
          \tkzDrawPoint[](S)
          \tkzDrawPoint[](X)
          \tkzDrawPoint[](B)
          \tkzDrawLine[ultra thin](II,nII)
          \tkzDrawLine[ultra thin](I,nI)
          \tkzDrawLine(S,BS)
          \tkzDrawSegment(O,B)
          \tkzDrawSegment[ultra thick](B,X)
          \tkzDrawSegment(O,S)

          \tkzLabelPoint(O){$O$}
          \tkzLabelPoint[](S){$S$}
          \tkzLabelPoint[above left](X){$X$}
          \tkzLabelPoint(B){$B$}

          \draw[decoration={brace,raise=.2cm,mirror},decorate,thin] (S)--(X);
          
          \node[anchor=south east] at (-5,4) {slope of $\frac{\l}{12}$};
        \end{tikzpicture}
      \end{center}
  \item The slope you seek is:
    \[
    \frac{|OS|}{|BX|}\qquad \text{with a pitch of} \qquad \frac{12\cdot |OS|}{|BX|}
    \]  
\end{enumerate}
\end{quote}

  Let's see if we can USE the algorithm above.
  \begin{enumerate}
  \item Someone used this drawing to compute the slope of a hip.
    \begin{center}
        \begin{tikzpicture}[geometryDiagrams,scale=.2]
          %% \filldraw[black] (1in,1in) circle (2pt) ;
          %% \filldraw[black] (0in,0in) circle (2pt) ;
          %% \filldraw[black] (0in,1in) circle (2pt) ;
          %% \draw[<->,dashed] (1.5in,-.5in) -- (-.5in,1.5in);

          \tkzDefPoint(0,0){O}
          \tkzDefPoint(-13,0){I}
          \tkzDefPoint(1,0){nI}
          \tkzDefPoint(0,7){II}
          \tkzDefPoint(0,-13){nII}
          
          \tkzDefPoint(0,6){S}
          \tkzDefPoint(-9,0){X}
          \tkzDefPoint(-12,-2){BS}
          \tkzDefPoint(0,-12){B}

          \tkzDrawPoint[](O)
          \tkzDrawPoint[](S)
          \tkzDrawPoint[](X)
          \tkzDrawPoint[](B)
          \tkzDrawLine[ultra thin](II,nII)
          \tkzDrawLine[ultra thin](I,nI)
          \tkzDrawLine(S,BS)
          \tkzDrawSegment(O,B)
          \tkzDrawSegment[ultra thick](B,X)
          \tkzDrawSegment(O,S)


          \draw[decoration={brace,raise=.2cm,mirror},decorate,thin] (O)--(S);
          \node[anchor=west] at (1.5,3) {$6$};


          \draw[decoration={brace,raise=.2cm,mirror},decorate,thin] (B)--(O);
          \node[anchor=west] at (1.5,-6) {$12$};

          \draw[decoration={brace,raise=.2cm,mirror},decorate,thin] (S)--(X);
          \node[anchor=south east] at (-5,4) {slope of $\frac{8}{12}$};

          \draw[decoration={brace,raise=.2cm,mirror},decorate,thin] (X)--(B);
          \node[anchor=north east] at (-5,-6.5) {$15$};
        \end{tikzpicture}
    \end{center}
    Find the slopes of the main-roof, the hip-end, and the hip. Show
    all work and explain your reasoning.
  \item Suppose you have a main-roof with a slope of $\frac{4}{12}$
    and a hip-end with a slope of $\frac{7}{12}$. \textbf{Find the
      slope of the hip.}  Show all work, explain your reasoning, and
    check your work with an \link[online calculator]{https://www2.strongtie.com/webapps/SlopeSkew/?source=app}.
  \item If the slope of the main roof is the same as the slope of the
    hip-end, say they are both $\frac{s}{12}$, what is the slope of
    the hip? Show all work and explain your reasoning.
  \end{enumerate}
\end{question}




\mynewpage


\begin{question}
  Now you should EXPLAIN why the method given above works to find the
  slope of a hip or a valley. 
\end{question}

\end{document}
