\documentclass[noauthor,nooutcomes,12pt]{ximera}

\graphicspath{  
{./}
{./whoAreYou/}
{./drawingWithTheTurtle/}
{./bisectionMethod/}
{./circles/}
{./anglesAndRightTriangles/}
{./lawOfSines/}
{./lawOfCosines/}
{./plotter/}
{./staircases/}
{./pitch/}
{./qualityControl/}
{./symmetry/}
{./nGonBlock/}
}


%% page layout
\usepackage[cm,headings]{fullpage}
\raggedright
\setlength\headheight{13.6pt}


%% fonts
\usepackage{euler}

\usepackage{FiraMono}
\renewcommand\familydefault{\ttdefault} 
\usepackage[defaultmathsizes]{mathastext}
\usepackage[htt]{hyphenat}

\usepackage[T1]{fontenc}
\usepackage[scaled=1]{FiraSans}

%\usepackage{wedn}
\usepackage{pbsi} %% Answer font


\usepackage{cancel} %% strike through in pitch/pitch.tex


%% \usepackage{ulem} %% 
%% \renewcommand{\ULthickness}{2pt}% changes underline thickness

\tikzset{>=stealth}

\usepackage{adjustbox}

\setcounter{titlenumber}{-1}

%% journal style
\makeatletter
\newcommand\journalstyle{%
  \def\activitystyle{activity-chapter}
  \def\maketitle{%
    \addtocounter{titlenumber}{1}%
                {\flushleft\small\sffamily\bfseries\@pretitle\par\vspace{-1.5em}}%
                {\flushleft\LARGE\sffamily\bfseries\thetitlenumber\hspace{1em}\@title \par }%
                {\vskip .6em\noindent\textit\theabstract\setcounter{question}{0}\setcounter{sectiontitlenumber}{0}}%
                    \par\vspace{2em}
                    \phantomsection\addcontentsline{toc}{section}{\thetitlenumber\hspace{1em}\textbf{\@title}}%
                     }}
\makeatother



%% thm like environments
\let\question\relax
\let\endquestion\relax

\newtheoremstyle{QuestionStyle}{\topsep}{\topsep}%%% space between body and thm
		{}                      %%% Thm body font
		{}                              %%% Indent amount (empty = no indent)
		{\bfseries}            %%% Thm head font
		{)}                              %%% Punctuation after thm head
		{ }                           %%% Space after thm head
		{\thmnumber{#2}\thmnote{ \bfseries(#3)}}%%% Thm head spec
\theoremstyle{QuestionStyle}
\newtheorem{question}{}



\let\freeResponse\relax
\let\endfreeResponse\relax

%% \newtheoremstyle{ResponseStyle}{\topsep}{\topsep}%%% space between body and thm
%% 		{\wedn\bfseries}                      %%% Thm body font
%% 		{}                              %%% Indent amount (empty = no indent)
%% 		{\wedn\bfseries}            %%% Thm head font
%% 		{}                              %%% Punctuation after thm head
%% 		{3ex}                           %%% Space after thm head
%% 		{\underline{\underline{\thmname{#1}}}}%%% Thm head spec
%% \theoremstyle{ResponseStyle}

\usepackage[tikz]{mdframed}
\mdfdefinestyle{ResponseStyle}{leftmargin=1cm,linecolor=black,roundcorner=5pt,
, font=\bsifamily,}%font=\wedn\bfseries\upshape,}


\ifhandout
\NewEnviron{freeResponse}{}
\else
%\newtheorem{freeResponse}{Response:}
\newenvironment{freeResponse}{\begin{mdframed}[style=ResponseStyle]}{\end{mdframed}}
\fi



%% attempting to automate outcomes.

%% \newwrite\outcomefile
%%   \immediate\openout\outcomefile=\jobname.oc
%% \renewcommand{\outcome}[1]{\edef\theoutcomes{\theoutcomes #1~}%
%% \immediate\write\outcomefile{\unexpanded{\outcome}{#1}}}

%% \newcommand{\outcomelist}{\begin{itemize}\theoutcomes\end{itemize}}

%% \NewEnviron{listOutcomes}{\small\sffamily
%% After answering the following questions, students should be able to:
%% \begin{itemize}
%% \BODY
%% \end{itemize}
%% }
\usepackage[tikz]{mdframed}
\mdfdefinestyle{OutcomeStyle}{leftmargin=2cm,rightmargin=2cm,linecolor=black,roundcorner=5pt,
, font=\small\sffamily,}%font=\wedn\bfseries\upshape,}
\newenvironment{listOutcomes}{\begin{mdframed}[style=OutcomeStyle]After answering the following questions, students should be able to:\begin{itemize}}{\end{itemize}\end{mdframed}}



%% my commands

\newcommand{\snap}{{\bfseries\itshape\textsf{Snap!}}}
\newcommand{\flavor}{\link[\snap]{https://snap.berkeley.edu/}}
\newcommand{\mooculus}{\textsf{\textbf{MOOC}\textnormal{\textsf{ULUS}}}}


\usepackage{tkz-euclide}
\tikzstyle geometryDiagrams=[rounded corners=.5pt,ultra thick,color=black]
\colorlet{penColor}{black} % Color of a curve in a plot



\ifhandout\newcommand{\mynewpage}{\newpage}\else\newcommand{\mynewpage}{}\fi


\title{Arcsine and arccosine}
\author{Bart Snapp}

\begin{document}
\begin{abstract}
  Arccosin tells us the arc covered by an angle based on the value of
  cosine.
\end{abstract}
\maketitle

\begin{listOutcomes}
\item Understand that arccosine accepts a ratio of sides as input, and
  outputs the measure of an angle.
\item Understand that arccosine is the inverse function of cosine for
  angles between $0$ and $180$ degrees.
\item Reason about elementary properties of cosine and arccosine.
\end{listOutcomes}
\mynewpage




\begin{question}
  Here is a table of cosine and arccosine values:
  \[
  \begin{array}{|l|l|} \hline
    \cos(0)  = 1     & \arccos(1) = 0 \\ \hline
    \cos(30) = 0.866 & \arccos(0.866) = 30\\ \hline
    \cos(60) = 0.5 & \arccos(0.5) = 60\\ \hline
    \cos(90) = 0 & \arccos(0) = 90\\ \hline
    \cos(120) = -0.5 & \arccos(-0.5) = 120\\ \hline
    \cos(150) = -0.866 & \arccos(-0.866) = 150\\ \hline
    \cos(180) =-1 & \arccos(-1) = 180  \\\hline
  \end{array}
  \]
  Explain what arccosine does as based on the table.
  
  
\end{question}
\mynewpage



\begin{question}
  Here is more table of cosine and arccosine values:
  \[
  \begin{array}{|l|l|}
    \hline
    \cos(180) =-1      & \arccos(-1) = 180\\ \hline
    \cos(210) = -0.866 & \arccos(-0.866) = 150\\ \hline
    \cos(240) = -0.5   & \arccos(-0.5) = 120\\ \hline
    \cos(270) = 0      & \arccos(0) = 90\\ \hline
    \cos(300) = 0.5     & \arccos(0.5) = 60\\ \hline
    \cos(330) = 0.866   & \arccos(.866) = 30\\ \hline
    \cos(360)  = 1     & \arccos(1) = 0 \\    \hline
  \end{array}
  \]
  Use the information from problem 1 and 2 to enhance your answer from
  problem 1.
\end{question}
\mynewpage



\begin{question}
  One of the following statements is true, the other is false:
  \[
  \cos(\arccos(x)) = x \qquad \text{or} \qquad \arccos(\cos(x)) =x
  \]
  Explain WHY the true statement is true and WHY the false statement
  is false.  Use words, pictures, examples, and so on, as
  needed/helpful in your explanations.
\end{question}



\end{document}
