\documentclass[handout,noauthor,nooutcomes,12pt,hints]{ximera}

\graphicspath{  
{./}
{./whoAreYou/}
{./drawingWithTheTurtle/}
{./bisectionMethod/}
{./circles/}
{./anglesAndRightTriangles/}
{./lawOfSines/}
{./lawOfCosines/}
{./plotter/}
{./staircases/}
{./pitch/}
{./qualityControl/}
{./symmetry/}
{./nGonBlock/}
}


%% page layout
\usepackage[cm,headings]{fullpage}
\raggedright
\setlength\headheight{13.6pt}


%% fonts
\usepackage{euler}

\usepackage{FiraMono}
\renewcommand\familydefault{\ttdefault} 
\usepackage[defaultmathsizes]{mathastext}
\usepackage[htt]{hyphenat}

\usepackage[T1]{fontenc}
\usepackage[scaled=1]{FiraSans}

%\usepackage{wedn}
\usepackage{pbsi} %% Answer font


\usepackage{cancel} %% strike through in pitch/pitch.tex


%% \usepackage{ulem} %% 
%% \renewcommand{\ULthickness}{2pt}% changes underline thickness

\tikzset{>=stealth}

\usepackage{adjustbox}

\setcounter{titlenumber}{-1}

%% journal style
\makeatletter
\newcommand\journalstyle{%
  \def\activitystyle{activity-chapter}
  \def\maketitle{%
    \addtocounter{titlenumber}{1}%
                {\flushleft\small\sffamily\bfseries\@pretitle\par\vspace{-1.5em}}%
                {\flushleft\LARGE\sffamily\bfseries\thetitlenumber\hspace{1em}\@title \par }%
                {\vskip .6em\noindent\textit\theabstract\setcounter{question}{0}\setcounter{sectiontitlenumber}{0}}%
                    \par\vspace{2em}
                    \phantomsection\addcontentsline{toc}{section}{\thetitlenumber\hspace{1em}\textbf{\@title}}%
                     }}
\makeatother



%% thm like environments
\let\question\relax
\let\endquestion\relax

\newtheoremstyle{QuestionStyle}{\topsep}{\topsep}%%% space between body and thm
		{}                      %%% Thm body font
		{}                              %%% Indent amount (empty = no indent)
		{\bfseries}            %%% Thm head font
		{)}                              %%% Punctuation after thm head
		{ }                           %%% Space after thm head
		{\thmnumber{#2}\thmnote{ \bfseries(#3)}}%%% Thm head spec
\theoremstyle{QuestionStyle}
\newtheorem{question}{}



\let\freeResponse\relax
\let\endfreeResponse\relax

%% \newtheoremstyle{ResponseStyle}{\topsep}{\topsep}%%% space between body and thm
%% 		{\wedn\bfseries}                      %%% Thm body font
%% 		{}                              %%% Indent amount (empty = no indent)
%% 		{\wedn\bfseries}            %%% Thm head font
%% 		{}                              %%% Punctuation after thm head
%% 		{3ex}                           %%% Space after thm head
%% 		{\underline{\underline{\thmname{#1}}}}%%% Thm head spec
%% \theoremstyle{ResponseStyle}

\usepackage[tikz]{mdframed}
\mdfdefinestyle{ResponseStyle}{leftmargin=1cm,linecolor=black,roundcorner=5pt,
, font=\bsifamily,}%font=\wedn\bfseries\upshape,}


\ifhandout
\NewEnviron{freeResponse}{}
\else
%\newtheorem{freeResponse}{Response:}
\newenvironment{freeResponse}{\begin{mdframed}[style=ResponseStyle]}{\end{mdframed}}
\fi



%% attempting to automate outcomes.

%% \newwrite\outcomefile
%%   \immediate\openout\outcomefile=\jobname.oc
%% \renewcommand{\outcome}[1]{\edef\theoutcomes{\theoutcomes #1~}%
%% \immediate\write\outcomefile{\unexpanded{\outcome}{#1}}}

%% \newcommand{\outcomelist}{\begin{itemize}\theoutcomes\end{itemize}}

%% \NewEnviron{listOutcomes}{\small\sffamily
%% After answering the following questions, students should be able to:
%% \begin{itemize}
%% \BODY
%% \end{itemize}
%% }
\usepackage[tikz]{mdframed}
\mdfdefinestyle{OutcomeStyle}{leftmargin=2cm,rightmargin=2cm,linecolor=black,roundcorner=5pt,
, font=\small\sffamily,}%font=\wedn\bfseries\upshape,}
\newenvironment{listOutcomes}{\begin{mdframed}[style=OutcomeStyle]After answering the following questions, students should be able to:\begin{itemize}}{\end{itemize}\end{mdframed}}



%% my commands

\newcommand{\snap}{{\bfseries\itshape\textsf{Snap!}}}
\newcommand{\flavor}{\link[\snap]{https://snap.berkeley.edu/}}
\newcommand{\mooculus}{\textsf{\textbf{MOOC}\textnormal{\textsf{ULUS}}}}


\usepackage{tkz-euclide}
\tikzstyle geometryDiagrams=[rounded corners=.5pt,ultra thick,color=black]
\colorlet{penColor}{black} % Color of a curve in a plot



\ifhandout\newcommand{\mynewpage}{\newpage}\else\newcommand{\mynewpage}{}\fi


\title{Sine and cosine}
\author{Bart Snapp}

\begin{document}
\begin{abstract}
  Sine and cosine encode information about similar right triangles.
\end{abstract}
\maketitle

\begin{listOutcomes}
\item Recall the basic definitions of sine and cosine.
\item View sine and cosine as encoding information about similar right
  triangles.
\item Explain basic facts about sine and cosine based on their
  definitions.
\item Extend the understanding of sine and cosine from triangles to
  the unit circle.
\item Interpret sine and cosine when the angle is larger than $180^\circ$.
\end{listOutcomes}
\mynewpage




\begin{question}
  Consider the following right triangle:
  \begin{center}
      \begin{tikzpicture}[geometryDiagrams]
        \coordinate (A) at (0,0);
        \coordinate (B) at (0,3);
        \coordinate (C) at (7,0);
        \tkzDrawSegment (A,B)
        \tkzDrawSegment (A,C)
        \tkzDrawSegment (C,B)
        \tkzLabelSegment[left](A,B){$a$}
        \tkzLabelSegment[below](A,C){$b$}
        \tkzLabelSegment[above right](B,C){$c$}  

        \tkzMarkRightAngle[thin](C,A,B)
        %\tkzLabelAngle[pos=1.2](C,A,B){$?$}

        \tkzMarkAngle[size=0.8cm,thin,mark=](A,B,C)
        \tkzLabelAngle[pos=.5](A,B,C){$\beta$}

        \tkzMarkAngle[mark=,size=1.2cm,thin](B,C,A)
        \tkzLabelAngle[pos=.9](B,C,A){$\alpha$}
      \end{tikzpicture}
    \end{center}
  Use the INTERNET to look up the definitions of \textbf{sine} and
  \textbf{cosine} of both $\alpha$ and $\beta$ in terms of this right
  triangle's hypotenuse, opposite leg, and adjacent leg. STATE THEM
  HERE as you would like them stated to you.


  In particular, as PART of your discussion:
  \begin{enumerate}
    \item EXPLAIN WHY $\sin(\alpha) = \cos(90^\circ-\alpha)$ and WHY
      $\cos(\alpha) = \sin(90^\circ-\alpha)$.
    \item If I have a triangle whose sides are scalar multiples of
      the one above, say,
      \begin{center}
      \begin{tikzpicture}[geometryDiagrams]
        \coordinate (A) at (0,0);
        \coordinate (B) at (0,3);
        \coordinate (C) at (7,0);
        \tkzDrawSegment (A,B)
        \tkzDrawSegment (A,C)
        \tkzDrawSegment (C,B)
        \tkzLabelSegment[left](A,B){$s\cdot a$}
        \tkzLabelSegment[below](A,C){$s\cdot b$}
        \tkzLabelSegment[above right](B,C){$s\cdot c$}  

        \tkzMarkRightAngle[thin](C,A,B)
        %\tkzLabelAngle[pos=1.2](C,A,B){$?$}

        \tkzMarkAngle[size=1cm,thin,mark=](A,B,C)
        \tkzLabelAngle[pos=.6](A,B,C){$\beta'$}

        \tkzMarkAngle[mark=,size=1.5cm,thin](B,C,A)
        \tkzLabelAngle[pos=1.2](B,C,A){$\alpha'$}
      \end{tikzpicture}
    \end{center}
      EXPLAIN WHY $\sin(\alpha) = \sin(\alpha')$ and $\cos(\alpha) =
      \cos(\alpha')$. Note, you MAY NOT assume that $\alpha =
      \alpha'$.
  \end{enumerate}
  Use words, pictures, and so on, as needed/helpful in your
  explanations.
  \begin{freeResponse}
    Given a right triangle,
    \begin{center}
      \begin{tikzpicture}[geometryDiagrams]
        \coordinate (A) at (0,0);
        \coordinate (B) at (0,3);
        \coordinate (C) at (7,0);
        \tkzDrawSegment (A,B)
        \tkzDrawSegment (A,C)
        \tkzDrawSegment (C,B)
        \tkzLabelSegment[left](A,B){$a$}
        \tkzLabelSegment[below](A,C){$b$}
        \tkzLabelSegment[above right](B,C){$c$}  

        \tkzMarkRightAngle[thin](C,A,B)
        %\tkzLabelAngle[pos=1.2](C,A,B){$?$}

        \tkzMarkAngle[size=0.8cm,thin,mark=](A,B,C)
        \tkzLabelAngle[pos=.5](A,B,C){$\beta$}

        \tkzMarkAngle[mark=,size=1.2cm,thin](B,C,A)
        \tkzLabelAngle[pos=.9](B,C,A){$\alpha$}
      \end{tikzpicture}
    \end{center}
    if we are working with the angle of measure $\alpha$,
    \[
    \sin(\alpha) = \frac{\text{opposite}}{\text{hypotenuse}} = \frac{a}{c},
    \]
    and
    \[
    \cos(\alpha) = \frac{\text{adjacent}}{\text{hypotenuse}} = \frac{b}{c}.
    \]
    If we are working with the angle of measure $\beta$,
    \[
    \sin(\beta) = \frac{\text{opposite}}{\text{hypotenuse}} = \frac{b}{c},
    \]
    and
    \[
    \cos(\beta) = \frac{\text{adjacent}}{\text{hypotenuse}} = \frac{a}{c}.
    \]
    From this we immediately see that
    \[
    \sin(\alpha) = \cos(\beta),
    \]
    and since $\beta = 180-90-\alpha$, we see
    \[
    \sin(\alpha) = \cos(90-\alpha).
    \]
    In an entirely similar way, we have
    \[
    \cos(\alpha) = \sin(90-\alpha).
    \]

    Finally, note, if a triangle with side lengths $(a,b,c)$ is
    similar to another right triangle, then this triangle has side
    lengths:
          \begin{center}
      \begin{tikzpicture}[geometryDiagrams]
        \coordinate (A) at (0,0);
        \coordinate (B) at (0,3);
        \coordinate (C) at (7,0);
        \tkzDrawSegment (A,B)
        \tkzDrawSegment (A,C)
        \tkzDrawSegment (C,B)
        \tkzLabelSegment[left](A,B){$s\cdot a$}
        \tkzLabelSegment[below](A,C){$s\cdot b$}
        \tkzLabelSegment[above right](B,C){$s\cdot c$}  

        \tkzMarkRightAngle[thin](C,A,B)
        %\tkzLabelAngle[pos=1.2](C,A,B){$?$}

        \tkzMarkAngle[size=1cm,thin,mark=](A,B,C)
        \tkzLabelAngle[pos=.6](A,B,C){$\beta'$}

        \tkzMarkAngle[mark=,size=1.5cm,thin](B,C,A)
        \tkzLabelAngle[pos=1.2](B,C,A){$\alpha'$}
      \end{tikzpicture}
    \end{center}
    for some scale factor $s$. Now with our new triangle, we see
    \[
    \sin(\alpha') = \frac{s\cdot a}{s\cdot c} = \frac{a}{c} =
    \sin(\alpha),
    \]
    and
    \[
    \cos(\alpha') = \frac{s\cdot b}{s\cdot c} = \frac{b}{c} =
    \cos(\alpha),
    \]
    hence sine and cosine encode information about \underline{similar} right
    triangles.
  \end{freeResponse}
\end{question}
\mynewpage

\begin{question}
  Now imagine a right triangle with a hypotenuse of length $1$. 
  \begin{enumerate}
  \item For a given nonright angle of the triangle, call it $\theta$,
    EXPLAIN why one leg of the triangle has length $\cos(\theta)$ and
    the other has length $\sin(\theta)$.  Use words, pictures, and so
    on, as needed/helpful in your explanation.
  \item EXPLAIN why
    \[
    \cos^2(\theta) + \sin^2(\theta) = 1.
    \]
  \end{enumerate}
  \begin{freeResponse}
    \begin{enumerate}
    \item Given a right triangle
    \begin{center}
      \begin{tikzpicture}[geometryDiagrams]
        \coordinate (A) at (0,0);
        \coordinate (B) at (0,3);
        \coordinate (C) at (7,0);
        \tkzDrawSegment (A,B)
        \tkzDrawSegment (A,C)
        \tkzDrawSegment (C,B)
        \tkzLabelSegment[left](A,B){$a$}
        \tkzLabelSegment[below](A,C){$b$}
        \tkzLabelSegment[above right](B,C){$1$}  

        \tkzMarkRightAngle[thin](C,A,B)
        %\tkzLabelAngle[pos=1.2](C,A,B){$?$}

        %\tkzMarkAngle[size=0.8cm,thin,mark=](A,B,C)
        %\tkzLabelAngle[pos=.5](A,B,C){$\beta$}

        \tkzMarkAngle[mark=,size=1.2cm,thin](B,C,A)
        \tkzLabelAngle[pos=.9](B,C,A){$\theta$}
      \end{tikzpicture}
    \end{center}
    We see
    \[
    \sin(\theta) = \frac{a}{1} = a\qquad \text{and}\qquad \cos(\theta) = \frac{b}{1} = b.
    \]
  \item With the same right triangle above, we also see via the Pythagorean theorem:
    \[
    \cos^2(\theta) + \sin^2(\theta) = 1.
    \]
    \end{enumerate}
  \end{freeResponse}
\end{question}
\mynewpage


\begin{question}
  EXPLAIN in terms of a right triangle, why:
  \begin{enumerate}
  \item If $\theta$ is near $0^\circ$, then $\sin(\theta)$ is near $0$.
  \item If $\theta$ is near $0^\circ$, then $\cos(\theta)$ is near $1$.
  \item If $\theta$ is near $90^\circ$, then $\sin(\theta)$ is near $1$.
  \item If $\theta$ is near $90^\circ$, then $\cos(\theta)$ is near $0$.
  \item $\sin(180-\theta) = \sin(\theta)$.
  \item The outputs of $\sin(\theta)$ and $\cos(\theta)$ are always
    between $-1$ and $1$.
  \end{enumerate}
  Use words, pictures, and so on, as needed/helpful in your
  explanations.
  \begin{hint}
    The last two are a bit tricky. You have to interpret what a
    negative sign means in this case. As a gesture of friendship, I
    suggest you think of our right triangle as being on the unit circle:
    \begin{center}
      \begin{tikzpicture}[geometryDiagrams]
        \coordinate (A) at (0,0);
        \coordinate (B) at (4.33,0);
        \coordinate (C) at (4.33,2.5);
        \tkzDrawSegment[line width=2pt](A,B)
        \tkzDrawSegment[line width=2pt](A,C)
        \tkzDrawSegment[line width=2pt](C,B)
        \tkzLabelSegment[above left](A,C){$1$}
        
        \coordinate (xmin) at (-6,0);
        \coordinate (xmax) at (6,0);
        \coordinate (ymin) at (0,-6);
        \coordinate (ymax) at (0,6);
        
        \tkzDrawSegment[->](xmin,xmax)
        \tkzDrawSegment[->](ymin,ymax)
        
        \tkzMarkRightAngle[thin](A,B,C)
        \tkzDrawCircle[dashed](A,C)
        \tkzMarkAngle[mark=,size=1.2cm,thin](B,A,C)
        \tkzLabelAngle[pos=.9](B,A,C){$\theta$};
      \end{tikzpicture}
    \end{center}
  \end{hint}
  \begin{freeResponse}
    \begin{enumerate}
    \item First we need to explain why if $\theta$ is near $0^\circ$,
      then $\sin(\theta)$ is near $0$.  Draw this on the unit circle:
      \begin{center}
      \begin{tikzpicture}[geometryDiagrams] %% multiply by 5
        \coordinate (A) at (0,0);
        \coordinate (B) at (4.98,0);
        \coordinate (C) at (4.98,.44);
        \tkzDrawSegment[line width=2pt](A,B)
        \tkzDrawSegment[line width=2pt](A,C)
        \tkzDrawSegment[line width=2pt](C,B)
        \tkzLabelSegment[above left](A,C){$1$}
        
        %% \coordinate (xmin) at (-6,0);
        %% \coordinate (xmax) at (6,0);
        %% \coordinate (ymin) at (0,-6);
        %% \coordinate (ymax) at (0,6);
        
        %% \tkzDrawSegment[->](xmin,xmax)
        %% \tkzDrawSegment[->](ymin,ymax)
        
        %% \tkzMarkRightAngle[thin](A,B,C)
        %% \tkzDrawCircle[dashed](A,C)
      \end{tikzpicture}
      \end{center}
      Since the angle is near $0^\circ$,
      \[
      \sin(\theta) = \frac{\text{something near $0$}}{1} \approx 0.
      \]
    \item Using the same picture from before, since the angle is near $0^\circ$,
      \[
      \cos(\theta) = \frac{\text{something near $1$}}{1} \approx 1.
      \]
    \item If the angle is near $90^\circ$, then draw this picture:
      \begin{center}
      \begin{tikzpicture}[geometryDiagrams] %% multiply by 5
        \coordinate (A) at (0,0);
        \coordinate (B) at ({5*cos(85)},0);
        \coordinate (C) at ({5*cos(85)},{5*sin(85)});
        \tkzDrawSegment[line width=2pt](A,B)
        \tkzDrawSegment[line width=2pt](A,C)
        \tkzDrawSegment[line width=2pt](C,B)
        \tkzLabelSegment[above left](A,C){$1$}
        
        %% \coordinate (xmin) at (-6,0);
        %% \coordinate (xmax) at (6,0);
        %% \coordinate (ymin) at (0,-6);
        %% \coordinate (ymax) at (0,6);
        
        %% \tkzDrawSegment[->](xmin,xmax)
        %% \tkzDrawSegment[->](ymin,ymax)
        
        \tkzMarkRightAngle[thin](A,B,C)
        %\tkzDrawCircle[dashed](A,C)
        
      \end{tikzpicture}
      \end{center}
      Since the angle is near $90^\circ$,
      \[
      \sin(\theta) = \frac{\text{something near $1$}}{1} \approx 1.
      \]
     \item Using the same picture from before, since the angle is near $0^\circ$,
      \[
      \cos(\theta) = \frac{\text{something near $0$}}{1} \approx 0.
      \]
    \item To see $\sin(180-\theta)=\sin(\theta)$ check this out:
      \begin{center}
      \begin{tikzpicture}[geometryDiagrams]
        \coordinate (xmin) at (-6,0);
        \coordinate (xmax) at (6,0);
        \coordinate (ymin) at (0,-6);
        \coordinate (ymax) at (0,6);
        
        \tkzDrawSegment[->](xmin,xmax)
        \tkzDrawSegment[->](ymin,ymax)

        \coordinate (A) at (0,0);
        \coordinate (B) at (-4.33,0);
        \coordinate (C) at (-4.33,2.5);

        
        \coordinate (AA) at (0,0);
        \coordinate (BB) at (4.33,0);
        \coordinate (CC) at (4.33,2.5);


        \tkzDrawSegment[red,line width=2pt](A,B)
        \tkzDrawSegment[red,line width=2pt](A,C)
        \tkzDrawSegment[red,line width=2pt](C,B)
        \tkzLabelSegment[red,above right](A,C){$1$}
        
        \tkzMarkRightAngle[red,thin](A,B,C)
        \tkzDrawCircle[dashed](A,C)
        \tkzMarkAngle[red,mark=,size=1.6cm,thin](BB,AA,C)
        \tkzLabelAngle[red,pos=.9](BB,A,C){$180-\theta$};

        \tkzDrawSegment[blue,line width=2pt](AA,BB)
        \tkzDrawSegment[blue,line width=2pt](AA,CC)
        \tkzDrawSegment[blue,line width=2pt](CC,BB)
        \tkzLabelSegment[blue,above left](AA,CC){$1$}

        \tkzLabelSegment[blue,right](BB,CC){$\sin(\theta)$}
        \tkzLabelSegment[red,left](B,C){$\sin(180-\theta)$}
        
        \tkzMarkRightAngle[blue,thin](AA,BB,CC)
        \tkzMarkAngle[blue,mark=,size=1.2cm,thin](BB,AA,CC)
        \tkzLabelAngle[blue,pos=.9](BB,AA,CC){$\theta$};
      \end{tikzpicture}
      \end{center}
      And so we see $\sin(180-\theta)=\sin(\theta)$.
    \item Sine and cosine are literally the lengths of the legs of
      the triangle embedded in the unit circle. Their values must be
      between $-1$ and $1$.
    \end{enumerate}
  \end{freeResponse}
\end{question}



\end{document}
