\documentclass[noauthor,nooutcomes,hints,handout]{ximera}

\graphicspath{  
{./}
{./whoAreYou/}
{./drawingWithTheTurtle/}
{./bisectionMethod/}
{./circles/}
{./anglesAndRightTriangles/}
{./lawOfSines/}
{./lawOfCosines/}
{./plotter/}
{./staircases/}
{./pitch/}
{./qualityControl/}
{./symmetry/}
{./nGonBlock/}
}


%% page layout
\usepackage[cm,headings]{fullpage}
\raggedright
\setlength\headheight{13.6pt}


%% fonts
\usepackage{euler}

\usepackage{FiraMono}
\renewcommand\familydefault{\ttdefault} 
\usepackage[defaultmathsizes]{mathastext}
\usepackage[htt]{hyphenat}

\usepackage[T1]{fontenc}
\usepackage[scaled=1]{FiraSans}

%\usepackage{wedn}
\usepackage{pbsi} %% Answer font


\usepackage{cancel} %% strike through in pitch/pitch.tex


%% \usepackage{ulem} %% 
%% \renewcommand{\ULthickness}{2pt}% changes underline thickness

\tikzset{>=stealth}

\usepackage{adjustbox}

\setcounter{titlenumber}{-1}

%% journal style
\makeatletter
\newcommand\journalstyle{%
  \def\activitystyle{activity-chapter}
  \def\maketitle{%
    \addtocounter{titlenumber}{1}%
                {\flushleft\small\sffamily\bfseries\@pretitle\par\vspace{-1.5em}}%
                {\flushleft\LARGE\sffamily\bfseries\thetitlenumber\hspace{1em}\@title \par }%
                {\vskip .6em\noindent\textit\theabstract\setcounter{question}{0}\setcounter{sectiontitlenumber}{0}}%
                    \par\vspace{2em}
                    \phantomsection\addcontentsline{toc}{section}{\thetitlenumber\hspace{1em}\textbf{\@title}}%
                     }}
\makeatother



%% thm like environments
\let\question\relax
\let\endquestion\relax

\newtheoremstyle{QuestionStyle}{\topsep}{\topsep}%%% space between body and thm
		{}                      %%% Thm body font
		{}                              %%% Indent amount (empty = no indent)
		{\bfseries}            %%% Thm head font
		{)}                              %%% Punctuation after thm head
		{ }                           %%% Space after thm head
		{\thmnumber{#2}\thmnote{ \bfseries(#3)}}%%% Thm head spec
\theoremstyle{QuestionStyle}
\newtheorem{question}{}



\let\freeResponse\relax
\let\endfreeResponse\relax

%% \newtheoremstyle{ResponseStyle}{\topsep}{\topsep}%%% space between body and thm
%% 		{\wedn\bfseries}                      %%% Thm body font
%% 		{}                              %%% Indent amount (empty = no indent)
%% 		{\wedn\bfseries}            %%% Thm head font
%% 		{}                              %%% Punctuation after thm head
%% 		{3ex}                           %%% Space after thm head
%% 		{\underline{\underline{\thmname{#1}}}}%%% Thm head spec
%% \theoremstyle{ResponseStyle}

\usepackage[tikz]{mdframed}
\mdfdefinestyle{ResponseStyle}{leftmargin=1cm,linecolor=black,roundcorner=5pt,
, font=\bsifamily,}%font=\wedn\bfseries\upshape,}


\ifhandout
\NewEnviron{freeResponse}{}
\else
%\newtheorem{freeResponse}{Response:}
\newenvironment{freeResponse}{\begin{mdframed}[style=ResponseStyle]}{\end{mdframed}}
\fi



%% attempting to automate outcomes.

%% \newwrite\outcomefile
%%   \immediate\openout\outcomefile=\jobname.oc
%% \renewcommand{\outcome}[1]{\edef\theoutcomes{\theoutcomes #1~}%
%% \immediate\write\outcomefile{\unexpanded{\outcome}{#1}}}

%% \newcommand{\outcomelist}{\begin{itemize}\theoutcomes\end{itemize}}

%% \NewEnviron{listOutcomes}{\small\sffamily
%% After answering the following questions, students should be able to:
%% \begin{itemize}
%% \BODY
%% \end{itemize}
%% }
\usepackage[tikz]{mdframed}
\mdfdefinestyle{OutcomeStyle}{leftmargin=2cm,rightmargin=2cm,linecolor=black,roundcorner=5pt,
, font=\small\sffamily,}%font=\wedn\bfseries\upshape,}
\newenvironment{listOutcomes}{\begin{mdframed}[style=OutcomeStyle]After answering the following questions, students should be able to:\begin{itemize}}{\end{itemize}\end{mdframed}}



%% my commands

\newcommand{\snap}{{\bfseries\itshape\textsf{Snap!}}}
\newcommand{\flavor}{\link[\snap]{https://snap.berkeley.edu/}}
\newcommand{\mooculus}{\textsf{\textbf{MOOC}\textnormal{\textsf{ULUS}}}}


\usepackage{tkz-euclide}
\tikzstyle geometryDiagrams=[rounded corners=.5pt,ultra thick,color=black]
\colorlet{penColor}{black} % Color of a curve in a plot



\ifhandout\newcommand{\mynewpage}{\newpage}\else\newcommand{\mynewpage}{}\fi


\title{Dihedral symmetry}
\author{Bart Snapp}

\begin{document}
\begin{abstract}
  We explore the symmetries of two dimensional shapes.
\end{abstract}
\maketitle

\begin{listOutcomes}
\item Explain what is meant by a ``dihedral'' symmetry.
\item Describe the dihedral symmetries of the regular $n$-gon.
\item Describe the dihedral symmetries of a nonregular shape.
\item Explain how two different shapes have the same dihedral
  symmetry.
\end{listOutcomes}
\begin{mdframed}[style=OutcomeStyle]
\begin{quote}
  $\textbf{di}\bullet\textbf{he}\bullet\textbf{dral}$ (d$\bar{\text{\i}}$-h$\bar{\text{e}}'$dr{\rotatebox[origin=c]{180}{e}}l)
  \\
  
  \textit{adjective}\\

  
\quad contained by two planes. 
\end{quote}
\end{mdframed}

We've been studying the symmetries of two-dimensional shapes. These
types of symmetries are called \textbf{dihedral} symmetries. Let's use
our enormous brains to think about dihedral symmetry a little bit
more.




\mynewpage






\begin{question}
  Let's generalize our understanding of the dihedral symmetries of the
  regular $n$-gon.
  \begin{enumerate}
  \item Concerning an arbitrary regular $n$-gon, how many dihedral symmetries
    are there?
  \item Can you always describe the symmetries of the regular $n$-gon
    in the form $r^xf$ for some number $x$? EXPLAIN using words,
    pictures, and so on, why or why not.
  \end{enumerate}
  \begin{freeResponse}
    \begin{enumerate}
    \item A regular $n$-gon has $2\cdot n$ symmetries:
      \begin{itemize}
      \item A do-nothing symmetry, $e$.
      \item $n-1$ honest rotations of representing counterclockwise rotations incremented by $360/n$.
      \item $n$ Flips across lines perpendicular to each side AND
        lines through opposite vertices (when $n$ is even).
      \end{itemize}
    \item Of course! Ask yourself, is did the symmetry change the
      orientation? If so do the flip! Now if you're not where you
      want to be, just rotate until you are! Done!
    \end{enumerate}
  \end{freeResponse}
\end{question}
\mynewpage

\begin{question}
  Consider this rhombus:
  \[ 
  \begin{tikzpicture}
    \draw[ultra thick,rounded corners=.05mm] (0,0) -- (2,-1) -- (4,0) -- (2,1)-- cycle;
    %% \draw[fill] (0,0) circle (1mm) node[below left] {$1$};
    %% \draw[fill] (2,-1) circle (1mm) node[below left] {$2$};
    %% \draw[fill] (4,0) circle (1mm) node[below right] {$3$};
    %% \draw[fill] (2,1) circle (1mm) node[above left] {$4$};
  \end{tikzpicture}
  \]
  \begin{enumerate}
  \item List ALL the symmetries in of this rhombus in terms of:
    \begin{itemize}
    \item $e$: the do-nothing symmetry.
    \item $r$: the rotate $180^\circ$ symmetry.
    \item $f$: the flip across a vertical line through the center of the rhombus.
    \end{itemize}
  \item Make a MULTIPLICATION table (as we did before) for the
    symmetries of this rhombus.
  \end{enumerate}
  \begin{freeResponse}
    \begin{enumerate}
    \item $e, r,f,rf$.
    \item Here is my table:
             \[
  \begin{array}{|c||c|c|c|c|c|c|}
    \hline
       & e  & r  & f  & rf \\ \hline\hline
    e  & e  & r  & f  & rf \\ \hline
    r  & r  & e  & rf & f \\ \hline
    f  & f  & rf & e  & r \\ \hline
    rf & rf & f  & r  & e \\ \hline
  \end{array}
  \]
    \end{enumerate}
  \end{freeResponse}
\end{question}

\mynewpage


\begin{question}
  Consider this nonsquare rectangle:
  \[
  \begin{tikzpicture}
    \draw[ultra thick,rounded corners=.05mm] (0,0) -- (4,0) -- (4,2) -- (0,2)-- cycle;
    %% \draw[fill] (0,0) circle (1mm) node[below left] {$1$};
    %% \draw[fill] (4,0) circle (1mm) node[below right] {$2$};
    %% \draw[fill] (4,2) circle (1mm) node[above right] {$3$};
    %% \draw[fill] (0,2) circle (1mm) node[above left] {$4$};
  \end{tikzpicture}
  \]
  \begin{enumerate}
  \item List ALL the symmetries in of this rectangle in terms of:
    \begin{itemize}
    \item $e$: the do-nothing symmetry.
    \item $r$: the rotate $180^\circ$ symmetry.
    \item $f$: the flip across a vertical line through the center of the rectangle.
    \end{itemize}
  \item Make a MULTIPLICATION table (as we did before) for the
    symmetries of this rectangle.
  \item This multiplication table for this rectangle is the SAME as
    the one for the rhombus. Explain WHY this happens.
  \end{enumerate}
  \begin{freeResponse}
    \begin{enumerate}
    \item $e, r,f,rf$.
    \item Here is my table:
      \[
      \begin{array}{|c||c|c|c|c|c|c|}
    \hline
    & e  & r  & f  & rf \\ \hline\hline
    e  & e  & r  & f  & rf \\ \hline
    r  & r  & e  & rf & f \\ \hline
    f  & f  & rf & e  & r \\ \hline
    rf & rf & f  & r  & e \\ \hline
      \end{array}
      \]
      \item The tables are the same because you can draw both the
        pictures on top of each other, and the symmetries are
        preserved. We can see this with the following picture:
     \[
  \begin{tikzpicture}
    \draw[ultra thick,rounded corners=.05mm] (0,0) -- (4,0) -- (4,2) -- (0,2)-- cycle; %% rect
    \draw[ultra thick,rounded corners=.05mm] (0,1) -- (2,0) -- (4,1) -- (2,2)-- cycle; %% rhomb   
  \end{tikzpicture}
  \]    
    \end{enumerate}
  \end{freeResponse}
\end{question}
\end{document}
