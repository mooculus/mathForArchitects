\documentclass[noauthor,nooutcomes,handout]{../ximera}

\graphicspath{  
{./}
{./whoAreYou/}
{./drawingWithTheTurtle/}
{./bisectionMethod/}
{./circles/}
{./anglesAndRightTriangles/}
{./lawOfSines/}
{./lawOfCosines/}
{./plotter/}
{./staircases/}
{./pitch/}
{./qualityControl/}
{./symmetry/}
{./nGonBlock/}
}


%% page layout
\usepackage[cm,headings]{fullpage}
\raggedright
\setlength\headheight{13.6pt}


%% fonts
\usepackage{euler}

\usepackage{FiraMono}
\renewcommand\familydefault{\ttdefault} 
\usepackage[defaultmathsizes]{mathastext}
\usepackage[htt]{hyphenat}

\usepackage[T1]{fontenc}
\usepackage[scaled=1]{FiraSans}

%\usepackage{wedn}
\usepackage{pbsi} %% Answer font


\usepackage{cancel} %% strike through in pitch/pitch.tex


%% \usepackage{ulem} %% 
%% \renewcommand{\ULthickness}{2pt}% changes underline thickness

\tikzset{>=stealth}

\usepackage{adjustbox}

\setcounter{titlenumber}{-1}

%% journal style
\makeatletter
\newcommand\journalstyle{%
  \def\activitystyle{activity-chapter}
  \def\maketitle{%
    \addtocounter{titlenumber}{1}%
                {\flushleft\small\sffamily\bfseries\@pretitle\par\vspace{-1.5em}}%
                {\flushleft\LARGE\sffamily\bfseries\thetitlenumber\hspace{1em}\@title \par }%
                {\vskip .6em\noindent\textit\theabstract\setcounter{question}{0}\setcounter{sectiontitlenumber}{0}}%
                    \par\vspace{2em}
                    \phantomsection\addcontentsline{toc}{section}{\thetitlenumber\hspace{1em}\textbf{\@title}}%
                     }}
\makeatother



%% thm like environments
\let\question\relax
\let\endquestion\relax

\newtheoremstyle{QuestionStyle}{\topsep}{\topsep}%%% space between body and thm
		{}                      %%% Thm body font
		{}                              %%% Indent amount (empty = no indent)
		{\bfseries}            %%% Thm head font
		{)}                              %%% Punctuation after thm head
		{ }                           %%% Space after thm head
		{\thmnumber{#2}\thmnote{ \bfseries(#3)}}%%% Thm head spec
\theoremstyle{QuestionStyle}
\newtheorem{question}{}



\let\freeResponse\relax
\let\endfreeResponse\relax

%% \newtheoremstyle{ResponseStyle}{\topsep}{\topsep}%%% space between body and thm
%% 		{\wedn\bfseries}                      %%% Thm body font
%% 		{}                              %%% Indent amount (empty = no indent)
%% 		{\wedn\bfseries}            %%% Thm head font
%% 		{}                              %%% Punctuation after thm head
%% 		{3ex}                           %%% Space after thm head
%% 		{\underline{\underline{\thmname{#1}}}}%%% Thm head spec
%% \theoremstyle{ResponseStyle}

\usepackage[tikz]{mdframed}
\mdfdefinestyle{ResponseStyle}{leftmargin=1cm,linecolor=black,roundcorner=5pt,
, font=\bsifamily,}%font=\wedn\bfseries\upshape,}


\ifhandout
\NewEnviron{freeResponse}{}
\else
%\newtheorem{freeResponse}{Response:}
\newenvironment{freeResponse}{\begin{mdframed}[style=ResponseStyle]}{\end{mdframed}}
\fi



%% attempting to automate outcomes.

%% \newwrite\outcomefile
%%   \immediate\openout\outcomefile=\jobname.oc
%% \renewcommand{\outcome}[1]{\edef\theoutcomes{\theoutcomes #1~}%
%% \immediate\write\outcomefile{\unexpanded{\outcome}{#1}}}

%% \newcommand{\outcomelist}{\begin{itemize}\theoutcomes\end{itemize}}

%% \NewEnviron{listOutcomes}{\small\sffamily
%% After answering the following questions, students should be able to:
%% \begin{itemize}
%% \BODY
%% \end{itemize}
%% }
\usepackage[tikz]{mdframed}
\mdfdefinestyle{OutcomeStyle}{leftmargin=2cm,rightmargin=2cm,linecolor=black,roundcorner=5pt,
, font=\small\sffamily,}%font=\wedn\bfseries\upshape,}
\newenvironment{listOutcomes}{\begin{mdframed}[style=OutcomeStyle]After answering the following questions, students should be able to:\begin{itemize}}{\end{itemize}\end{mdframed}}



%% my commands

\newcommand{\snap}{{\bfseries\itshape\textsf{Snap!}}}
\newcommand{\flavor}{\link[\snap]{https://snap.berkeley.edu/}}
\newcommand{\mooculus}{\textsf{\textbf{MOOC}\textnormal{\textsf{ULUS}}}}


\usepackage{tkz-euclide}
\tikzstyle geometryDiagrams=[rounded corners=.5pt,ultra thick,color=black]
\colorlet{penColor}{black} % Color of a curve in a plot



\ifhandout\newcommand{\mynewpage}{\newpage}\else\newcommand{\mynewpage}{}\fi


\outcome{State a physical property of triangles.}
\outcome{Understand what it means for triangles to be congruent.}
\outcome{State the triangle congruence theorems.}
\outcome{Identify situations where congruence theorems apply.}
\outcome{Use a congruence theorem to explain why a triangle is a rigid shape.}

\title{Why triangles?}
\author{Bart Snapp}

\begin{document}
\begin{abstract}
  We seek to understand the importance of triangles and the triangle
  congruence theorems.
\end{abstract}
\maketitle


\begin{listOutcomes}
\item{State a physical property of triangles.}
\item{Understand what it means for triangles to be congruent.}
\item{State the triangle congruence theorems.}
\item{Identify situations where congruence theorems apply.}
\item{Use a congruence theorem to explain why a triangle is a rigid shape.}
\end{listOutcomes}

\mynewpage


\begin{question}
  Someone once told me,
  \begin{quote}
    \textit{a triangle is the strongest shape.}
  \end{quote}
  Explain what they meant by this.
  \begin{freeResponse}
    They mean if you want to build a rigid structure, with a frame,
    then using triangles in your structure will make your structure
    strong. This strength will remain even if there is some play
    or wiggle in the joints.
  \end{freeResponse}
\end{question}
\mynewpage

\begin{question} This question is about congruent triangles.
  \begin{enumerate}
  \item Use the INTERNET to look up what it means for triangles to be
    \textbf{congruent}. Explain what it means for triangles to be
    congruent in your own words.
  \item Use the INTERNET to look up the \textbf{triangle congruence
    theorems: SSS, SAS, ASA, SAA.} EXPLAIN the statements of these
    theorems as you would like to have them explained to you. Use
    words, pictures, and so on, as needed/helpful in your explanation.
  \item Finally, EXPLAIN using words, pictures, and so on, WHY neither AAA
  nor SSA are congruence theorems.
  \end{enumerate}

  \begin{freeResponse}
    \begin{enumerate}
    \item Two different triangles are \underline{congruent} if we can pick one up,
      and place it on the other with each of the sides and angles
      matching.

    \item Let's list and explain the triangle congruence theorems:
    \begin{description}
    \item[\bsifamily\underline{SSS}:] If I know three side lengths
      of a triangle,
      \begin{center}
      \begin{tikzpicture}[geometryDiagrams]
        \coordinate (A) at (0,0);
        \coordinate (B) at (5,2);
        \coordinate (C) at (7,0);
        \tkzDrawSegment (A,B)
        \tkzDrawSegment (A,C)
        \tkzDrawSegment (C,B)
        \tkzLabelSegment[above left](A,B){$c$}
        \tkzLabelSegment[below](A,C){$b$}
        \tkzLabelSegment[above right](B,C){$a$}  

        \tkzMarkAngle[size=1.5cm,thin,mark=](C,A,B)
        \tkzLabelAngle[pos=1.2](C,A,B){$?$}

        \tkzMarkAngle[size=0.8cm,thin,mark=](A,B,C)
        \tkzLabelAngle[pos=.5](A,B,C){$?$}

        \tkzMarkAngle[mark=,size=.9,thin](B,C,A)
        \tkzLabelAngle[pos=.6](B,C,A){$?$}
        
      \end{tikzpicture}
      \end{center}
      then I can find all the angles of the triangle.
    \item[\bsifamily\underline{SAS}:] If I know two sides lengths
      of a triangle and I know the angle between these two sides,
      \begin{center}
      \begin{tikzpicture}[geometryDiagrams]
        \coordinate (A) at (0,0);
        \coordinate (B) at (5,2);
        \coordinate (C) at (7,0);
        \tkzDrawSegment (A,B)
        \tkzDrawSegment (A,C)
        \tkzDrawSegment (C,B)
        \tkzLabelSegment[above left](A,B){$c$}
        \tkzLabelSegment[below](A,C){$b$}
        \tkzLabelSegment[above right](B,C){$?$}  

        \tkzMarkAngle[size=1.5cm,thin,mark=](C,A,B)
        \tkzLabelAngle[pos=1.2](C,A,B){$\alpha$}

        \tkzMarkAngle[size=0.8cm,thin,mark=](A,B,C)
        \tkzLabelAngle[pos=.5](A,B,C){$?$}

        \tkzMarkAngle[mark=,size=.9,thin](B,C,A)
        \tkzLabelAngle[pos=.6](B,C,A){$?$}
        
      \end{tikzpicture}
      \end{center}
      then I can find the third side of the triangle, and the other
      two angles.
    \item[\bsifamily\underline{ASA}:] If I know two angles of a
      triangle and the length of the side between these two angles,
      \begin{center}
        \begin{tikzpicture}[geometryDiagrams]
        \coordinate (A) at (0,0);
        \coordinate (B) at (5,2);
        \coordinate (C) at (7,0);
        \tkzDrawSegment (A,B)
        \tkzDrawSegment (A,C)
        \tkzDrawSegment (C,B)
        \tkzLabelSegment[above left](A,B){$?$}
        \tkzLabelSegment[below](A,C){$b$}
        \tkzLabelSegment[above right](B,C){$?$}  

        \tkzMarkAngle[size=1.5cm,thin,mark=](C,A,B)
        \tkzLabelAngle[pos=1.2](C,A,B){$\alpha$}

        \tkzMarkAngle[size=0.8cm,thin,mark=](A,B,C)
        \tkzLabelAngle[pos=.5](A,B,C){$?$}

        \tkzMarkAngle[mark=,size=.9,thin](B,C,A)
        \tkzLabelAngle[pos=.6](B,C,A){$\gamma$}
        
      \end{tikzpicture}
      \end{center}
      then I can find the other two sides of the triangle, and the
      third angle.
    \item[\bsifamily\underline{SAA}:] If I know two angles of a
      triangle along with a side not between the two angles,
      \begin{center}
        \begin{tikzpicture}[geometryDiagrams]
        \coordinate (A) at (0,0);
        \coordinate (B) at (5,2);
        \coordinate (C) at (7,0);
        \tkzDrawSegment (A,B)
        \tkzDrawSegment (A,C)
        \tkzDrawSegment (C,B)
        \tkzLabelSegment[above left](A,B){$?$}
        \tkzLabelSegment[below](A,C){$?$}
        \tkzLabelSegment[above right](B,C){$a$}  

        \tkzMarkAngle[size=1.5cm,thin,mark=](C,A,B)
        \tkzLabelAngle[pos=1.2](C,A,B){$\alpha$}

        \tkzMarkAngle[size=0.8cm,thin,mark=](A,B,C)
        \tkzLabelAngle[pos=.5](A,B,C){$?$}

        \tkzMarkAngle[mark=,size=.9,thin](B,C,A)
        \tkzLabelAngle[pos=.6](B,C,A){$\gamma$}
        
      \end{tikzpicture}
      \end{center} then I can find the other two sides, and the third angle.
    \end{description}



    \item AAA is not a congruence theorem. Behold:
\begin{center}
      \begin{tikzpicture}[geometryDiagrams]
        \coordinate (A) at (0,0);
        \coordinate (B) at (4,2);
        \coordinate (C) at (5,0);
        \tkzDrawSegment (A,B)
        \tkzDrawSegment (A,C)
        \tkzDrawSegment (C,B)
        \tkzLabelSegment[above left](A,B){$c$}
        \tkzLabelSegment[below](A,C){$b$}
        \tkzLabelSegment[above right](B,C){$a$}  

        \tkzMarkAngle[size=1.5cm,thin,mark=](C,A,B)
        \tkzLabelAngle[pos=1.2](C,A,B){$\alpha$}

        \tkzMarkAngle[size=0.8cm,thin,mark=](A,B,C)
        \tkzLabelAngle[pos=.5](A,B,C){$\beta$}

        \tkzMarkAngle[mark=,size=.9,thin](B,C,A)
        \tkzLabelAngle[pos=.6](B,C,A){$\gamma$}


        \coordinate (A') at (6,0);
        \coordinate (B') at (8,1);
        \coordinate (C') at (8.5,0);
        \tkzDrawSegment (A',B')
        \tkzDrawSegment (A',C')
        \tkzDrawSegment (C',B')
        \tkzLabelSegment[above left](A',B'){$c/2$}
        \tkzLabelSegment[below](A',C'){$b/2$}
        \tkzLabelSegment[right](B',C'){$a/2$}  

        \tkzMarkAngle[size=.8cm,thin,mark=](C',A',B')
        \tkzLabelAngle[pos=.6](C',A',B'){\small$\alpha$}

        \tkzMarkAngle[size=0.5cm,thin,mark=](A',B',C')
        \tkzLabelAngle[pos=.3](A',B',C'){\small$\beta$}

        \tkzMarkAngle[mark=,size=.5,thin](B',C',A')
        \tkzLabelAngle[pos=.3](B',C',A'){\small$\gamma$}
      \end{tikzpicture}
\end{center}
Two noncongruent triangles with the same angles.


Also SSA is not a congruence theorem.  Behold:
    \begin{center}
      \begin{tikzpicture}[geometryDiagrams]
        \coordinate (A) at (0,0);
        \coordinate (B) at (4,2);
        \coordinate (C) at (5,0);
        \tkzDrawSegment (A,B)
        \tkzDrawSegment (A,C)
        \tkzDrawSegment (C,B)
        \tkzLabelSegment[above left](A,B){$c$}
        \tkzLabelSegment[below](A,C){$?$}
        \tkzLabelSegment[above right](B,C){$a$}  

        \tkzMarkAngle[size=1.5cm,thin,mark=](C,A,B)
        \tkzLabelAngle[pos=1.2](C,A,B){$\alpha$}

        \tkzMarkAngle[size=0.8cm,thin,mark=](A,B,C)
        \tkzLabelAngle[pos=.5](A,B,C){$?$}

        \tkzMarkAngle[mark=,size=.9,thin](B,C,A)
        \tkzLabelAngle[pos=.6](B,C,A){$?$}


        \coordinate (A') at (6,0);
        \coordinate (B') at (10,2);
        \coordinate (C') at (9,0);
        \tkzDrawSegment (A',B')
        \tkzDrawSegment (A',C')
        \tkzDrawSegment (C',B')
        \tkzLabelSegment[above left](A',B'){$c$}
        \tkzLabelSegment[below](A',C'){$?$}
        \tkzLabelSegment[right](B',C'){$a$}  

        \tkzMarkAngle[size=1.5cm,thin,mark=](C',A',B')
        \tkzLabelAngle[pos=1.2](C',A',B'){$\alpha$}

        \tkzMarkAngle[size=0.9cm,thin,mark=](A',B',C')
        \tkzLabelAngle[pos=.6](A',B',C'){$?$}

        \tkzMarkAngle[mark=,size=.65,thin](B',C',A')
        \tkzLabelAngle[pos=.35](B',C',A'){$?$}
      \end{tikzpicture}
    \end{center}
    Two different triangles with the same side, side, and angle. The
    drawings are precisely made, you can measure the features
    yourself.
    \end{enumerate}
  \end{freeResponse}
\end{question}
\mynewpage

\begin{question}
  Use the SSS congruence theorem to explain WHY ``triangles are the
  strongest shape.''  As PART of this explanation, EXPLAIN why
  quadrilaterals do NOT make ``strong shapes.''
  \begin{freeResponse}
    Given three side lengths, SSS tells us there is a
    \underline{unique} triangle with these properties. Thus even if
    the ``joints'' are lose and can turn, the triangle
    \underline{cannot} collapse unless a side breaks or attached
    sides disconnect.

    
    On the other hand, there are many quadrilaterals with the same
    side lengths, but different angles. There is no SSSS congruence
    theorem for quadrilaterals, as both

    \begin{center}
      \begin{tikzpicture}
        [geometryDiagrams]
        \coordinate (A) at (0,0);
        \coordinate (B) at (3,0);
        \coordinate (C) at (3,3);
        \coordinate (D) at (0,3);
        \tkzDrawSegment (A,B)
        \tkzDrawSegment (B,C)
        \tkzDrawSegment (C,D)
        \tkzDrawSegment (D,A)
        
        \coordinate (A') at (6,0);
        \coordinate (B') at (9,0);
        \coordinate (C') at (11.12,2.12);
        \coordinate (D') at (8.12,2.12);
        \tkzDrawSegment (A',B')
        \tkzDrawSegment (B',C')
        \tkzDrawSegment (C',D')
        \tkzDrawSegment (D',A')
      \end{tikzpicture}
    \end{center}
    have the same side lengths, but are different shapes. Hence a
    quadrilateral \underline{can} collapse without sides breaking or
    disconnecting sides.

    This makes the quadrilateral weaker than the triangle.
  \end{freeResponse}
\end{question}



\end{document}
