\documentclass[noauthor,nooutcomes,hints,handout]{../ximera}

%% page layout
\usepackage[in,headings]{fullpage}
\raggedright
\setlength\headheight{13.6pt}


%% fonts
\usepackage{euler}

\usepackage{FiraMono}
\renewcommand\familydefault{\ttdefault} 
\usepackage{mathastext}
\usepackage[htt]{hyphenat}

\usepackage[T1]{fontenc}
\usepackage[scaled=1]{FiraSans}

\usepackage{wedn}
\usepackage[T1]{fontenc}

%% wrap text around scripts
\usepackage{wrapfig}

\tikzset{>=stealth}
%% snap! scripts
\usepackage{scratch3}

\usepackage{adjustbox}

%% journal style
\makeatletter
\newcommand\journalstyle{%
  \def\activitystyle{activity-chapter}
  \def\maketitle{%
    \addtocounter{titlenumber}{1}%
                {\flushleft\small\sffamily\bfseries\@pretitle\par\vspace{-1.5em}}%
                {\flushleft\LARGE\sffamily\bfseries\thetitlenumber\hspace{1em}\@title \par }%
                {\vskip .6em\noindent\textit\theabstract\setcounter{question}{0}\setcounter{sectiontitlenumber}{0}}%
                    \par\vspace{2em}
                    \phantomsection\addcontentsline{toc}{section}{\thetitlenumber\hspace{1em}\textbf{\@title}}%
                     }}
\makeatother



%% thm like environments
\let\question\relax
\let\endquestion\relax

\newtheoremstyle{QuestionStyle}{\topsep}{\topsep}%%% space between body and thm
		{}                      %%% Thm body font
		{}                              %%% Indent amount (empty = no indent)
		{\bfseries}            %%% Thm head font
		{)}                              %%% Punctuation after thm head
		{ }                           %%% Space after thm head
		{\thmnumber{#2}\thmnote{ \bfseries(#3)}}%%% Thm head spec
\theoremstyle{QuestionStyle}
\newtheorem{question}{}



\let\freeResponse\relax
\let\endfreeResponse\relax

%% \newtheoremstyle{ResponseStyle}{\topsep}{\topsep}%%% space between body and thm
%% 		{\wedn\bfseries}                      %%% Thm body font
%% 		{}                              %%% Indent amount (empty = no indent)
%% 		{\wedn\bfseries}            %%% Thm head font
%% 		{}                              %%% Punctuation after thm head
%% 		{3ex}                           %%% Space after thm head
%% 		{\underline{\underline{\thmname{#1}}}}%%% Thm head spec
%% \theoremstyle{ResponseStyle}

\usepackage[tikz]{mdframed}
\mdfdefinestyle{ResponseStyle}{leftmargin=1cm,linecolor=black,roundcorner=5pt,frametitlefont=\wedn\bfseries,%frametitle={\underline{\underline{Response}}:}
, font=\wedn\bfseries,}%\begin{mdframed}[style=mystyle]foo\end{mdframed}


\ifhandout
\NewEnviron{freeResponse}{}
\else
%\newtheorem{freeResponse}{Response:}
\newenvironment{freeResponse}{\begin{mdframed}[style=ResponseStyle]}{\end{mdframed}}
\fi



%% attempting to automate outcomes.

\newwrite\outcomefile
  \immediate\openout\outcomefile=\jobname.oc
\renewcommand{\outcome}[1]{\edef\theoutcomes{\theoutcomes #1~}%
\immediate\write\outcomefile{\unexpanded{\outcome}{#1}}}

%% \newcommand{\outcomelist}{\begin{itemize}\theoutcomes\end{itemize}}



%% my commands

\newcommand{\snap}{{\bfseries\itshape\textsf{Snap!}}}
\newcommand{\flavor}{\link[\snap]{https://snap.berkeley.edu/}}


\usepackage{tkz-euclide}
\tikzstyle geometryDiagrams=[rounded corners=.5pt,ultra thick,color=black]
\colorlet{penColor}{black} % Color of a curve in a plot


\author{Bart Snapp}

\title{Five types of symmetry}

\begin{document}
\begin{abstract}
  There are only five types of symmetry for frieze patterns.
\end{abstract}
\maketitle

\begin{listOutcomes}
\item Analyze types of symmetries for frieze patterns.
\item Describe the possible symmetries of frieze patterns.
\end{listOutcomes}

Continuing to think about types of symmetries, let:
\begin{itemize}
  \item $T$ be a generic translation (perhaps by $0$ units of
    distance).
  \item $G$ be a generic glide-reflection (perhaps by $0$ units of
    distance) $G$.
  \item $R$ be a generic $180^\circ$ rotation around some point.
  \item $F_v$ be a generic vertical-reflection.
  \item $F_h$ be a horizontal-reflection. 
\end{itemize}
\begin{center}
  Are these ALL possible symmetries? \\
  OR do we get new types of
symmetries when we combine these?
\end{center}
In our work below, we seek the answer to this question.




\mynewpage


\begin{question}
  Fill in the blanks (not the X) in the type-multiplication table for
  frieze patterns:
  \[\renewcommand{\arraystretch}{2}
  \begin{array}{|c||c|c|c|c|c|}      
    \hline
        & T    & R    & F_h   & F_v & G     \\ \hline\hline
    T   & \phantom{XXX}   & X    & X    &  \phantom{XXX}   &   \phantom{XXX}    \\ \hline
    R   & X    & \phantom{XXX}     & X    &     &       \\ \hline
    F_h & X    & X    &    X   &     &     \\ \hline
    F_v &      &      &       &     &     \\ \hline
    G   &      &      &  \phantom{XXX}     &     &     \\ \hline
\end{array}
  \]
  Each of those blanks has been solved already in ``Combinations of
  symmetry.'' For each relation \textbf{identify} if it came from the
  introduction, question 1, question 2, or question 3.
  \begin{freeResponse}
    From the introduction of ``Combinations of symmetry,'' we find
    that:
    \begin{itemize}
    \item $T\circ T = T$
    \item $T\circ G = G \circ T = G$
    \item $T\circ F_v = F_v \circ T = G$
    \item $G\circ G = T$
    \item $R\circ R = T$
    \item $F_v\circ F_v = T$
    \item $F_v\circ G = G \circ F_v = T$
    \end{itemize}
    From question 1, we have:
    \begin{itemize}
    \item $R\circ F_v = F_v \circ R = F_h$
    \item $R\circ G = G \circ R = F_h$
    \end{itemize}
    From question 3, we have:
    \begin{itemize}
    \item $F_h \circ F_v = F_v\circ F_h = R$
    \item $F_h\circ G = G \circ F_h = R$
    \end{itemize}
    Behold:
    \[\renewcommand{\arraystretch}{2}
  \begin{array}{|c||c|c|c|c|c|}      
    \hline
        & T    & R    & F_h   & F_v & G     \\ \hline\hline
    T   & T    & X    & X    &  G   &  G   \\ \hline
    R   & X    & T     & X    & F_h    & F_h      \\ \hline
    F_h & X    & X    &   X    &   R  &  R   \\ \hline
    F_v & G    &  F_h    &   R    &  T   & T    \\ \hline
    G   & G     & F_h     &  R    &  T  & T    \\ \hline
\end{array}
  \]
  \end{freeResponse}
\end{question}
\mynewpage




\begin{question}
  Fill in the blanks (not the X) in the type-multiplication table for
  frieze patterns:
  \[\renewcommand{\arraystretch}{2}
  \begin{array}{|c||c|c|c|}      
    \hline
        & T    & R    & F_h\\ \hline   
    T   & X   & \phantom{XXX}    & \phantom{XXX}    \\ \hline
    R   & \phantom{XXX}    & X     & X  \\ \hline
    F_h & \phantom{XXX}    & X    &   X  \\ \hline
  \end{array}
  \]
  draw pictures and/or explain your reasoning to demonstrate that you are
  correct.
  \begin{freeResponse}
    First note that a rotation combined with a translation (or vice
    versa) is just a rotation, just \underline{around a different point}. To see
    this it is helpful to imagine a frieze with minimal symmetry. If
    we translate it, it looks the same, then if we rotate it, we get a
    rotation.

    Second note that a horizontal-reflection combined with a
    translation (or vice versa) is just a horizontal-reflection, just
    \underline{across a different vertical line}. To see this it is
    helpful to imagine a frieze with minimal symmetry. If we
    translate it, it looks the same, then if we flip it across a
    vertical line, we get some horizontal-reflection.
  \end{freeResponse}
\end{question}
\mynewpage

\begin{question}
  Finally, \textbf{make a complete type-multiplication table} for frieze
  patterns and use this to explain WHY the types of symmetries for frieze patterns
  are exactly:
  \begin{itemize}
  \item translations,
  \item glide-reflections,
  \item vertical-reflections,
  \item $180^\circ$ rotations, and
  \item horizontal-reflections.
  \end{itemize}
  
  
  \begin{freeResponse}
Well, the only symmetries of regular $n$-gons are reflections and
rotations. The only reflections that a frieze pattern could have
symmetry through are vertical-reflections and
horizontal-reflections. Moreover, the only rotations we could
possibly use are $180^\circ$ rotations, because a frieze is an
``infinite'' strip. We can view the combinations of these symmetries in the table below:
 \[\renewcommand{\arraystretch}{2}
\begin{array}{|c||c|c|c|c|c|}
    \hline
        & T    & R    & F_h   & F_v & G     \\ \hline\hline
    T   & T    & R   & F_h    & G   & G     \\ \hline
    R   & R   & T    & G    & F_h & F_h   \\ \hline
    F_h & F_h   & G   & T     & R   & R   \\ \hline
    F_v & G    & F_h  & R     & T   & T   \\ \hline
    G   & G    & F_h  & R     & T   & T   \\ \hline
\end{array}
\]
From our table, we see that no matter how we combine the symmetries,
we just find one of the five below:
\begin{itemize}
\item translations,
\item glide-reflections,
\item vertical-reflections,
\item $180^\circ$ rotations, and
\item horizontal-reflections.
\end{itemize}
Thus the symmetries above are the only possible symmetries for a frieze patterns.
\end{freeResponse}  
\end{question}
\end{document}
