\documentclass[noauthor,nooutcomes,handout]{ximera}

\graphicspath{  
{./}
{./whoAreYou/}
{./drawingWithTheTurtle/}
{./bisectionMethod/}
{./circles/}
{./anglesAndRightTriangles/}
{./lawOfSines/}
{./lawOfCosines/}
{./plotter/}
{./staircases/}
{./pitch/}
{./qualityControl/}
{./symmetry/}
{./nGonBlock/}
}


%% page layout
\usepackage[cm,headings]{fullpage}
\raggedright
\setlength\headheight{13.6pt}


%% fonts
\usepackage{euler}

\usepackage{FiraMono}
\renewcommand\familydefault{\ttdefault} 
\usepackage[defaultmathsizes]{mathastext}
\usepackage[htt]{hyphenat}

\usepackage[T1]{fontenc}
\usepackage[scaled=1]{FiraSans}

%\usepackage{wedn}
\usepackage{pbsi} %% Answer font


\usepackage{cancel} %% strike through in pitch/pitch.tex


%% \usepackage{ulem} %% 
%% \renewcommand{\ULthickness}{2pt}% changes underline thickness

\tikzset{>=stealth}

\usepackage{adjustbox}

\setcounter{titlenumber}{-1}

%% journal style
\makeatletter
\newcommand\journalstyle{%
  \def\activitystyle{activity-chapter}
  \def\maketitle{%
    \addtocounter{titlenumber}{1}%
                {\flushleft\small\sffamily\bfseries\@pretitle\par\vspace{-1.5em}}%
                {\flushleft\LARGE\sffamily\bfseries\thetitlenumber\hspace{1em}\@title \par }%
                {\vskip .6em\noindent\textit\theabstract\setcounter{question}{0}\setcounter{sectiontitlenumber}{0}}%
                    \par\vspace{2em}
                    \phantomsection\addcontentsline{toc}{section}{\thetitlenumber\hspace{1em}\textbf{\@title}}%
                     }}
\makeatother



%% thm like environments
\let\question\relax
\let\endquestion\relax

\newtheoremstyle{QuestionStyle}{\topsep}{\topsep}%%% space between body and thm
		{}                      %%% Thm body font
		{}                              %%% Indent amount (empty = no indent)
		{\bfseries}            %%% Thm head font
		{)}                              %%% Punctuation after thm head
		{ }                           %%% Space after thm head
		{\thmnumber{#2}\thmnote{ \bfseries(#3)}}%%% Thm head spec
\theoremstyle{QuestionStyle}
\newtheorem{question}{}



\let\freeResponse\relax
\let\endfreeResponse\relax

%% \newtheoremstyle{ResponseStyle}{\topsep}{\topsep}%%% space between body and thm
%% 		{\wedn\bfseries}                      %%% Thm body font
%% 		{}                              %%% Indent amount (empty = no indent)
%% 		{\wedn\bfseries}            %%% Thm head font
%% 		{}                              %%% Punctuation after thm head
%% 		{3ex}                           %%% Space after thm head
%% 		{\underline{\underline{\thmname{#1}}}}%%% Thm head spec
%% \theoremstyle{ResponseStyle}

\usepackage[tikz]{mdframed}
\mdfdefinestyle{ResponseStyle}{leftmargin=1cm,linecolor=black,roundcorner=5pt,
, font=\bsifamily,}%font=\wedn\bfseries\upshape,}


\ifhandout
\NewEnviron{freeResponse}{}
\else
%\newtheorem{freeResponse}{Response:}
\newenvironment{freeResponse}{\begin{mdframed}[style=ResponseStyle]}{\end{mdframed}}
\fi



%% attempting to automate outcomes.

%% \newwrite\outcomefile
%%   \immediate\openout\outcomefile=\jobname.oc
%% \renewcommand{\outcome}[1]{\edef\theoutcomes{\theoutcomes #1~}%
%% \immediate\write\outcomefile{\unexpanded{\outcome}{#1}}}

%% \newcommand{\outcomelist}{\begin{itemize}\theoutcomes\end{itemize}}

%% \NewEnviron{listOutcomes}{\small\sffamily
%% After answering the following questions, students should be able to:
%% \begin{itemize}
%% \BODY
%% \end{itemize}
%% }
\usepackage[tikz]{mdframed}
\mdfdefinestyle{OutcomeStyle}{leftmargin=2cm,rightmargin=2cm,linecolor=black,roundcorner=5pt,
, font=\small\sffamily,}%font=\wedn\bfseries\upshape,}
\newenvironment{listOutcomes}{\begin{mdframed}[style=OutcomeStyle]After answering the following questions, students should be able to:\begin{itemize}}{\end{itemize}\end{mdframed}}



%% my commands

\newcommand{\snap}{{\bfseries\itshape\textsf{Snap!}}}
\newcommand{\flavor}{\link[\snap]{https://snap.berkeley.edu/}}
\newcommand{\mooculus}{\textsf{\textbf{MOOC}\textnormal{\textsf{ULUS}}}}


\usepackage{tkz-euclide}
\tikzstyle geometryDiagrams=[rounded corners=.5pt,ultra thick,color=black]
\colorlet{penColor}{black} % Color of a curve in a plot



\ifhandout\newcommand{\mynewpage}{\newpage}\else\newcommand{\mynewpage}{}\fi


\title{A few more (tri)angles}
\author{Claire Merriman}

%% ONLY REQUIRES WHAT's YOUR ANGLE...



\begin{document}
\begin{abstract}
  We'll explore some more properties of angles and triangles.
\end{abstract}
\maketitle

\begin{listOutcomes}
\item Explain the relationship between vertical and supplementary angles.
\item Use triangle congruence theorems.
\item Use radians to find the length of a circle segment.
\item Use formulas to solve word problems.
\end{listOutcomes}

%% \begin{listObjectives}
%% \item Learn and apply basic geometric formulas,
%% \item Explain why presented concepts and formulas are true,
%% \end{listObjectives}





In the following diagram, the lines $\bar{AB}$ and $\bar{CF}$ are parallel and perpendicular to $\bar{AC}$. Suppose that $\left\vert AE\right\vert =\left\vert
EB\right\vert $ and $\left\vert CE\right\vert =\left\vert
ED\right\vert $.

This diagram is used for two questions.
\begin{center}
\begin{tikzpicture}[rotate=15]
  \coordinate (A) at (-2,0);
  \coordinate (B) at (6,0);
  \coordinate (C) at (-2,-4);
  \coordinate (D) at (6,4);
  \coordinate (E) at (2,0);
  \coordinate (F) at (8,-4);
  \coordinate (G) at (-4,0);
  \coordinate (H) at (8,0);
  \coordinate (I) at (-4,-4);
  \draw (G)--(H);
  \draw (I)--(F);
  \draw (B)--(D)--(C)--(A);
  \tkzLabelPoints[above](D,A,E)
  \tkzLabelPoints[below](B,C,F)
  \tkzMarkSegments[mark=|](A,E E,B)
  \tkzMarkSegments[mark=||](C,E E,D)
  \tkzMarkAngle[size=0.75cm,mark=](B,E,D)
  \tkzMarkAngle[size=0.5cm,mark=](D,E,A)
  \tkzMarkAngle[size=0.75cm,mark=](A,E,C)
  \tkzMarkAngle[size=0.5cm,mark=](C,E,B)
  \tkzMarkAngle[size=0.75,mark=](F,C,D)
  \tkzLabelAngle[pos = -1](A,E,C){$\alpha$}
  \tkzLabelAngle[pos = .7](C,E,B){$\beta$}
  \tkzLabelAngle[pos = .7](D,E,A){$\gamma$}
  \tkzLabelAngle[pos = -1](C,E,A){$\delta$}
  \tkzLabelAngle[pos = 1](F,C,D){$\epsilon$}
\end{tikzpicture}
\end{center}


\mynewpage



\begin{question}
\begin{enumerate}
    \item Find $\beta,\gamma,\delta$ and $\epsilon$ in terms of $\alpha$. Explain your reasoning. 
    \item Look up triangle congruence theorems and cite your sources. Explain why $\left\vert AC\right\vert =\left\vert
BD\right\vert $.
    \item Explain why $\left\vert AC\right\vert =\left\vert
BD\right\vert $ does not require $\bar{AB}$ and $\bar{AC}$ to be perpendicular.
\end{enumerate}
\end{question}
\mynewpage




\begin{question}
\textit{Radians} are another way to measure angles. $1 \ radian$ is the angle made when we take the radius
and wrap it round the circle. On a circle with radius $1\ unit$, the length of the blue arc is also $1\ unit$, so the marked angle has measure $1 \ radian$.


\begin{center}
\begin{tikzpicture}[geometryDiagrams]
  \tkzDefPoint(0,0){O}
  \tkzDefPoint(4,0){A}
  \tkzDefPoint(4*cos(1),4*sin(1)){B}
  \tkzDrawArc[blue,thick](O,A)(B)
  \tkzDrawArc[thick](O,B)(A)
  \tkzDrawLines[add = 0 and .5](O,A O,B)
 % \tkzLablSegments[mark=|](O,A)
  \tkzDrawPoints(O,A,B)
  \tkzMarkAngle[size=0.75cm,mark=](A,O,B)
   \tkzLabelAngle[pos = .8,right](A,O,B){$\theta=1\ rad$}
 \end{tikzpicture}
 
\end{center}

The ever insightful Louie Llama is here to help!

\begin{enumerate}
    \item Louie Llama walks all the way around the  the circle. When he gets back to where he started, he has turn $2\pi \ rad$. Explain why in your own words. 
    \item What is the length of blue segment if $\theta=\pi \ rad$? Why?
    \item What is the length of the blue segment if $\theta=1 \ rad$ and the radius is $r$? Explain your reasoning.
\end{enumerate}
\end{question}
\mynewpage

\begin{question}
  Here a problem that is at least $300$ years old, so let's call it an  ``oldie, but a goodie!''
  \begin{quote}
    Imagine a chain, pulled tight around the Earth's equator. Suppose
   you add an extra $10'$ to the chain. The chain is then raised up
    uniformly, all around the Earth, as high as possible. How high is    that?
  \end{quote}
  Supposing that
  \begin{itemize}
  \item There are $5280$ feet in a mile, and
  \item the radius of the Earth is $4000$ miles,
  \end{itemize}
  it is now time for you to try you hand at this classic.
  \begin{enumerate}
  \item Give a solution in feet. Show all work.
  \item Is the answer surprising? Explain why you say yes or no.
  \end{enumerate}
\end{question}

\end{document}
