\documentclass{ximera}

\graphicspath{  
{./}
{./whoAreYou/}
{./drawingWithTheTurtle/}
{./bisectionMethod/}
{./circles/}
{./anglesAndRightTriangles/}
{./lawOfSines/}
{./lawOfCosines/}
{./plotter/}
{./staircases/}
{./pitch/}
{./qualityControl/}
{./symmetry/}
{./nGonBlock/}
}


%% page layout
\usepackage[cm,headings]{fullpage}
\raggedright
\setlength\headheight{13.6pt}


%% fonts
\usepackage{euler}

\usepackage{FiraMono}
\renewcommand\familydefault{\ttdefault} 
\usepackage[defaultmathsizes]{mathastext}
\usepackage[htt]{hyphenat}

\usepackage[T1]{fontenc}
\usepackage[scaled=1]{FiraSans}

%\usepackage{wedn}
\usepackage{pbsi} %% Answer font


\usepackage{cancel} %% strike through in pitch/pitch.tex


%% \usepackage{ulem} %% 
%% \renewcommand{\ULthickness}{2pt}% changes underline thickness

\tikzset{>=stealth}

\usepackage{adjustbox}

\setcounter{titlenumber}{-1}

%% journal style
\makeatletter
\newcommand\journalstyle{%
  \def\activitystyle{activity-chapter}
  \def\maketitle{%
    \addtocounter{titlenumber}{1}%
                {\flushleft\small\sffamily\bfseries\@pretitle\par\vspace{-1.5em}}%
                {\flushleft\LARGE\sffamily\bfseries\thetitlenumber\hspace{1em}\@title \par }%
                {\vskip .6em\noindent\textit\theabstract\setcounter{question}{0}\setcounter{sectiontitlenumber}{0}}%
                    \par\vspace{2em}
                    \phantomsection\addcontentsline{toc}{section}{\thetitlenumber\hspace{1em}\textbf{\@title}}%
                     }}
\makeatother



%% thm like environments
\let\question\relax
\let\endquestion\relax

\newtheoremstyle{QuestionStyle}{\topsep}{\topsep}%%% space between body and thm
		{}                      %%% Thm body font
		{}                              %%% Indent amount (empty = no indent)
		{\bfseries}            %%% Thm head font
		{)}                              %%% Punctuation after thm head
		{ }                           %%% Space after thm head
		{\thmnumber{#2}\thmnote{ \bfseries(#3)}}%%% Thm head spec
\theoremstyle{QuestionStyle}
\newtheorem{question}{}



\let\freeResponse\relax
\let\endfreeResponse\relax

%% \newtheoremstyle{ResponseStyle}{\topsep}{\topsep}%%% space between body and thm
%% 		{\wedn\bfseries}                      %%% Thm body font
%% 		{}                              %%% Indent amount (empty = no indent)
%% 		{\wedn\bfseries}            %%% Thm head font
%% 		{}                              %%% Punctuation after thm head
%% 		{3ex}                           %%% Space after thm head
%% 		{\underline{\underline{\thmname{#1}}}}%%% Thm head spec
%% \theoremstyle{ResponseStyle}

\usepackage[tikz]{mdframed}
\mdfdefinestyle{ResponseStyle}{leftmargin=1cm,linecolor=black,roundcorner=5pt,
, font=\bsifamily,}%font=\wedn\bfseries\upshape,}


\ifhandout
\NewEnviron{freeResponse}{}
\else
%\newtheorem{freeResponse}{Response:}
\newenvironment{freeResponse}{\begin{mdframed}[style=ResponseStyle]}{\end{mdframed}}
\fi



%% attempting to automate outcomes.

%% \newwrite\outcomefile
%%   \immediate\openout\outcomefile=\jobname.oc
%% \renewcommand{\outcome}[1]{\edef\theoutcomes{\theoutcomes #1~}%
%% \immediate\write\outcomefile{\unexpanded{\outcome}{#1}}}

%% \newcommand{\outcomelist}{\begin{itemize}\theoutcomes\end{itemize}}

%% \NewEnviron{listOutcomes}{\small\sffamily
%% After answering the following questions, students should be able to:
%% \begin{itemize}
%% \BODY
%% \end{itemize}
%% }
\usepackage[tikz]{mdframed}
\mdfdefinestyle{OutcomeStyle}{leftmargin=2cm,rightmargin=2cm,linecolor=black,roundcorner=5pt,
, font=\small\sffamily,}%font=\wedn\bfseries\upshape,}
\newenvironment{listOutcomes}{\begin{mdframed}[style=OutcomeStyle]After answering the following questions, students should be able to:\begin{itemize}}{\end{itemize}\end{mdframed}}



%% my commands

\newcommand{\snap}{{\bfseries\itshape\textsf{Snap!}}}
\newcommand{\flavor}{\link[\snap]{https://snap.berkeley.edu/}}
\newcommand{\mooculus}{\textsf{\textbf{MOOC}\textnormal{\textsf{ULUS}}}}


\usepackage{tkz-euclide}
\tikzstyle geometryDiagrams=[rounded corners=.5pt,ultra thick,color=black]
\colorlet{penColor}{black} % Color of a curve in a plot



\ifhandout\newcommand{\mynewpage}{\newpage}\else\newcommand{\mynewpage}{}\fi


\title{The bisection method}
\author{Bart Snapp}

\begin{document}
\begin{abstract}
  We will use the bisection method to solve triangles.
\end{abstract}
\maketitle

\begin{question}
  Use the INTERNET or a LIBRARY to look up the \textit{bisection
    method}.

  EXPLAIN the bisection method as you wuold like to have it
  explained to you.  Use words, pictures, and so on, as needed/helpful
  in your explanation.
  \begin{freeResponse}
    The \underline{bisection method} helps you find where a curve
    crosses the $x$-axis. The idea is this, if you have a curve and
    you know two $x$-coordinates,
    \begin{itemize}
      \item one where the curve is above the
        $x$-axis, and
      \item one where the curve is below the $x$-axis,
    \end{itemize}
    \begin{center}
      \begin{tikzpicture}
        \begin{axis}[
            xmin=-1.2,
            xmax=5.2,
            ymin=-.6,
            ymax=.6,
            axis lines=center,
            xlabel=$x$,
            ylabel=$y$,
            ticks=none,
            every axis y label/.style={at=(current axis.above origin),anchor=south},
            every axis x label/.style={at=(current axis.right of origin),anchor=west},
            %clip=true,
          ]
          \addplot [ultra thick, penColor, smooth] {x^2/20-.5};

          \node[circle,fill,inner sep=2pt] at (axis cs:4,0) {};
          \node[anchor=south east,] at (axis cs:4,.3) {above $x$-axis};

          \draw[dashed] (axis cs: 4,0) -- (axis cs: 4,.3);

          \draw[dashed] (axis cs: 2,0) -- (axis cs: 2,-.3);
          
          \node[circle,fill,inner sep=2pt] at (axis cs:2,0) {};
          \node[anchor=north west,] at (axis cs:2,-.3) {below $x$-axis};
        \end{axis}
      \end{tikzpicture}
    \end{center}
    then you average those $x$-values. This new $x$-value will be
    closer to where the curve crosses the $x$-axis.
    \begin{center}
      \begin{tikzpicture}
        \begin{axis}[
            xmin=-1.2,
            xmax=5.2,
            ymin=-.6,
            ymax=.6,
            axis lines=center,
            xlabel=$x$,
            ylabel=$y$,
            ticks=none,
            every axis y label/.style={at=(current axis.above origin),anchor=south},
            every axis x label/.style={at=(current axis.right of origin),anchor=west},
            %clip=true,
          ]
          \addplot [ultra thick, penColor, smooth] {x^2/20-.5};

          \node[circle,fill,inner sep=2pt] at (axis cs:4,0) {};
          %\node[anchor=south east,] at (axis cs:4,.3) {above $x$-axis};

          \draw[dashed] (axis cs: 4,0) -- (axis cs: 4,.3);

          \draw[dashed] (axis cs: 3,0) -- (axis cs:3,-.05);


          \node[circle,fill,inner sep=2pt,black] at (axis cs:3,0) {};
          \node[] at (axis cs:3,.4) {new point};
          \draw[->] (axis cs: 3,.3) -- (axis cs:3,.1);
          
          \node[circle,fill,inner sep=2pt,black!50!white] at (axis cs:2,0) {};
          \node[] at (axis cs:1,-.1) {old point};

          \draw[->] (axis cs: 3.7,-.4) -- (axis cs:3.95,-.05);

          \draw[->] (axis cs: 3.3,-.4) -- (axis cs:3.05,-.05);
          
          \node[] at (axis cs:3.5,-.5) {average these};
        \end{axis}
      \end{tikzpicture}
    \end{center}


    Repeat this
    process of averaging the $x$-values that are on opposides of the
    $x$-axis. Your $x$-coordinates will get closer and closer to where
    the curve crosses the axis.
  \end{freeResponse}
\end{question}


\begin{question}
  Here is a \snap\ script:
  
  \adjustbox{valign=t}{
    \begin{tabular}{lp{.7\textwidth}}
      \begin{scratch}[num blocks,scale=.8]
        \blockinit{when \greenflag clicked}
        \blocklook{hide}
        \blockpen{pen down}
        \blockmove{move \ovalnum{100} steps}
        \blockmove{turn \turnright{} \ovalnum{163.74} degrees}
        \blockmove{move \ovalnum{75} steps}
        \blockmove{turn \turnright{} \ovalnum{53.13} degrees}
        \blockmove{move \ovalnum{??} steps}
      \end{scratch}
      & Clearly EXPLAIN how you can use the bisection method to find
      the final length and thus complete the \snap\ script for the
      triangle.

      
      Display your work with a table:
          \[
          \begin{array}{|c|c|c|c|}\hline
            \text{Attempt} & \text{Length} & \text{Too short or Too long?} \\ \hline\hline
            1 & ?? & ?? \\ \hline
            2 & ?? & ??  \\ \hline
            \vdots & \vdots & \vdots \\ 
          \end{array}
          \]
  \end{tabular}}
  \begin{freeResponse}
    
    I'll make a table illustrating this
    \[
  \begin{array}{|c|c|c|c|}\hline
    \text{Attempt} & \text{Length} & \text{Too short or Too long?} \\ \hline\hline
    1 & 20 & \text{short} \\ \hline
    2 & 50 & \text{long} \\ \hline
    3 & 35 & \text{BINGO!} \\ \hline
  \end{array}
  \]
\end{freeResponse}
\end{question}


\begin{question}
  Let's automate this process with the following \snap\ script:

  
  \adjustbox{valign=t}{
    \begin{tabular}{lp{.7\textwidth}}
  \begin{scratch}[num blocks,scale=.8]
    \blockinit{when \greenflag clicked}
    \blockpen{pen down}
    \blockmove{move \ovalnum{80} steps}
    \blockmove{turn \turnright{} \ovalnum{90} degrees}
    \blockmove{move \ovalnum{150} steps}
    \blockmove{turn \turnright{} \ovalnum{??} degrees}
    \blockmove{move \ovalnum{70} steps}
  \end{scratch}
  & Clearly EXPLAIN what each line of the script is doing.  Use words,
  pictures, and so on, as needed/helpful in your explanation.
    \end{tabular}}
\end{question}



\end{document}
