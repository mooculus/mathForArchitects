\documentclass[noauthor,nooutcomes,hints,handout,12pt]{ximera}

%% Note, this is connected to Simpson's paradox and Arrow's Impossibility theorem.
%% See paper An Impossibility Theorem for Gerrymandering, by Alexeev and Mixon

%https://redistricting.ohio.gov/maps
%%https://www.lsc.ohio.gov/documents/reference/current/membersonlybriefs/134%20Redistricting%20in%20Ohio.pdf

%https://districtr.org/ohio

%https://projects.fivethirtyeight.com/redistricting-2022-maps/ohio/democratic_amendments/
%https://districtr.org/plan/120904

%https://districtr.org/plan/121411


%https://districtr.org/oh/congress

\graphicspath{  
{./}
{./whoAreYou/}
{./drawingWithTheTurtle/}
{./bisectionMethod/}
{./circles/}
{./anglesAndRightTriangles/}
{./lawOfSines/}
{./lawOfCosines/}
{./plotter/}
{./staircases/}
{./pitch/}
{./qualityControl/}
{./symmetry/}
{./nGonBlock/}
}


%% page layout
\usepackage[cm,headings]{fullpage}
\raggedright
\setlength\headheight{13.6pt}


%% fonts
\usepackage{euler}

\usepackage{FiraMono}
\renewcommand\familydefault{\ttdefault} 
\usepackage[defaultmathsizes]{mathastext}
\usepackage[htt]{hyphenat}

\usepackage[T1]{fontenc}
\usepackage[scaled=1]{FiraSans}

%\usepackage{wedn}
\usepackage{pbsi} %% Answer font


\usepackage{cancel} %% strike through in pitch/pitch.tex


%% \usepackage{ulem} %% 
%% \renewcommand{\ULthickness}{2pt}% changes underline thickness

\tikzset{>=stealth}

\usepackage{adjustbox}

\setcounter{titlenumber}{-1}

%% journal style
\makeatletter
\newcommand\journalstyle{%
  \def\activitystyle{activity-chapter}
  \def\maketitle{%
    \addtocounter{titlenumber}{1}%
                {\flushleft\small\sffamily\bfseries\@pretitle\par\vspace{-1.5em}}%
                {\flushleft\LARGE\sffamily\bfseries\thetitlenumber\hspace{1em}\@title \par }%
                {\vskip .6em\noindent\textit\theabstract\setcounter{question}{0}\setcounter{sectiontitlenumber}{0}}%
                    \par\vspace{2em}
                    \phantomsection\addcontentsline{toc}{section}{\thetitlenumber\hspace{1em}\textbf{\@title}}%
                     }}
\makeatother



%% thm like environments
\let\question\relax
\let\endquestion\relax

\newtheoremstyle{QuestionStyle}{\topsep}{\topsep}%%% space between body and thm
		{}                      %%% Thm body font
		{}                              %%% Indent amount (empty = no indent)
		{\bfseries}            %%% Thm head font
		{)}                              %%% Punctuation after thm head
		{ }                           %%% Space after thm head
		{\thmnumber{#2}\thmnote{ \bfseries(#3)}}%%% Thm head spec
\theoremstyle{QuestionStyle}
\newtheorem{question}{}



\let\freeResponse\relax
\let\endfreeResponse\relax

%% \newtheoremstyle{ResponseStyle}{\topsep}{\topsep}%%% space between body and thm
%% 		{\wedn\bfseries}                      %%% Thm body font
%% 		{}                              %%% Indent amount (empty = no indent)
%% 		{\wedn\bfseries}            %%% Thm head font
%% 		{}                              %%% Punctuation after thm head
%% 		{3ex}                           %%% Space after thm head
%% 		{\underline{\underline{\thmname{#1}}}}%%% Thm head spec
%% \theoremstyle{ResponseStyle}

\usepackage[tikz]{mdframed}
\mdfdefinestyle{ResponseStyle}{leftmargin=1cm,linecolor=black,roundcorner=5pt,
, font=\bsifamily,}%font=\wedn\bfseries\upshape,}


\ifhandout
\NewEnviron{freeResponse}{}
\else
%\newtheorem{freeResponse}{Response:}
\newenvironment{freeResponse}{\begin{mdframed}[style=ResponseStyle]}{\end{mdframed}}
\fi



%% attempting to automate outcomes.

%% \newwrite\outcomefile
%%   \immediate\openout\outcomefile=\jobname.oc
%% \renewcommand{\outcome}[1]{\edef\theoutcomes{\theoutcomes #1~}%
%% \immediate\write\outcomefile{\unexpanded{\outcome}{#1}}}

%% \newcommand{\outcomelist}{\begin{itemize}\theoutcomes\end{itemize}}

%% \NewEnviron{listOutcomes}{\small\sffamily
%% After answering the following questions, students should be able to:
%% \begin{itemize}
%% \BODY
%% \end{itemize}
%% }
\usepackage[tikz]{mdframed}
\mdfdefinestyle{OutcomeStyle}{leftmargin=2cm,rightmargin=2cm,linecolor=black,roundcorner=5pt,
, font=\small\sffamily,}%font=\wedn\bfseries\upshape,}
\newenvironment{listOutcomes}{\begin{mdframed}[style=OutcomeStyle]After answering the following questions, students should be able to:\begin{itemize}}{\end{itemize}\end{mdframed}}



%% my commands

\newcommand{\snap}{{\bfseries\itshape\textsf{Snap!}}}
\newcommand{\flavor}{\link[\snap]{https://snap.berkeley.edu/}}
\newcommand{\mooculus}{\textsf{\textbf{MOOC}\textnormal{\textsf{ULUS}}}}


\usepackage{tkz-euclide}
\tikzstyle geometryDiagrams=[rounded corners=.5pt,ultra thick,color=black]
\colorlet{penColor}{black} % Color of a curve in a plot



\ifhandout\newcommand{\mynewpage}{\newpage}\else\newcommand{\mynewpage}{}\fi


\title{Gerrymandering}
\author{Bart Snapp}

\begin{document}
\begin{abstract}
  We start to think about a real-world issue with representative government.
\end{abstract}
\maketitle

\begin{listOutcomes}
\item Describe what gerrymandering is.
\item Explain a governmental purpose for districts.
\item Explain reasons for district guidelines.
\end{listOutcomes}
\begin{mdframed}[style=OutcomeStyle]
\begin{quote}
  $\textbf{ger}\bullet\textbf{ry}\bullet\textbf{man}\bullet\textbf{der}$ (j$\check{\text{e}}$r$'\bar{\text{e}}-$m$\check{\text{a}}$n$'$d{\rotatebox[origin=c]{180}{e}}r)
  \\
  
  \textit{verb}\\

  
\quad to intentionally divide a region into districts, giving an
unfair advantage to a political party in elections.
\end{quote}
\end{mdframed}




\mynewpage






\begin{question}
  Play the game \textsl{GerryMander} at the website:
  \url{http://gametheorytest.com/gerry/} and complete up to level $10$
  (no need to give information).  \textbf{Demonstrate understanding}
  by EXPLAINING what gerrymandering is using pictures, words, and
  examples like those you saw in the game.
  
\end{question}


\mynewpage

\begin{question}
  Starting in $2022$, Ohio will have $15$ representatives in the
  U.S.\ House of Representatives. These representatives represent $15$
  districts in Ohio and are elected by people who live in these
  districts. However there is currently some disagreement as to what
  the districts should be. According to the document
  \textit{Redistricting in Ohio,} districts must be:
  \begin{enumerate}
  \item Compact.
  \item District populations must be substantially equal. No district
    may contain a population of less than $95\%$ or more than $105\%$
    of the ideal district population (total population)/(number of
    districts).
  \item The statewide proportion of districts that favor a particular
    party must correspond closely to the statewide preferences of the
    voters.
  \end{enumerate}
  Read the document \textit{Redistricting in Ohio} and EXPLAIN (using words/pictures, and so on) what each of these items means. 
\end{question}


\mynewpage

\begin{question}
  Considering that the $15$ districts elect representatives to the
  U.S.\ House of Representatives, DISCUSS the consequences of adopting
  a congressional district map that:
  \begin{enumerate}
  \item Is not compact.
  \item Has districts that are not equal in population.
  \item Where the statewide proportion of districts that favor a
    particular party does not correspond to the statewide preferences
    of the voters.
  \end{enumerate}
  \textbf{Address each issue above in turn, separately.} 
\end{question}



\end{document}
