\documentclass[nooutcomes,noauthor,handout]{../ximera}

%% page layout
\usepackage[in,headings]{fullpage}
\raggedright
\setlength\headheight{13.6pt}


%% fonts
\usepackage{euler}

\usepackage{FiraMono}
\renewcommand\familydefault{\ttdefault} 
\usepackage{mathastext}
\usepackage[htt]{hyphenat}

\usepackage[T1]{fontenc}
\usepackage[scaled=1]{FiraSans}

\usepackage{wedn}
\usepackage[T1]{fontenc}

%% wrap text around scripts
\usepackage{wrapfig}

\tikzset{>=stealth}
%% snap! scripts
\usepackage{scratch3}

\usepackage{adjustbox}

%% journal style
\makeatletter
\newcommand\journalstyle{%
  \def\activitystyle{activity-chapter}
  \def\maketitle{%
    \addtocounter{titlenumber}{1}%
                {\flushleft\small\sffamily\bfseries\@pretitle\par\vspace{-1.5em}}%
                {\flushleft\LARGE\sffamily\bfseries\thetitlenumber\hspace{1em}\@title \par }%
                {\vskip .6em\noindent\textit\theabstract\setcounter{question}{0}\setcounter{sectiontitlenumber}{0}}%
                    \par\vspace{2em}
                    \phantomsection\addcontentsline{toc}{section}{\thetitlenumber\hspace{1em}\textbf{\@title}}%
                     }}
\makeatother



%% thm like environments
\let\question\relax
\let\endquestion\relax

\newtheoremstyle{QuestionStyle}{\topsep}{\topsep}%%% space between body and thm
		{}                      %%% Thm body font
		{}                              %%% Indent amount (empty = no indent)
		{\bfseries}            %%% Thm head font
		{)}                              %%% Punctuation after thm head
		{ }                           %%% Space after thm head
		{\thmnumber{#2}\thmnote{ \bfseries(#3)}}%%% Thm head spec
\theoremstyle{QuestionStyle}
\newtheorem{question}{}



\let\freeResponse\relax
\let\endfreeResponse\relax

%% \newtheoremstyle{ResponseStyle}{\topsep}{\topsep}%%% space between body and thm
%% 		{\wedn\bfseries}                      %%% Thm body font
%% 		{}                              %%% Indent amount (empty = no indent)
%% 		{\wedn\bfseries}            %%% Thm head font
%% 		{}                              %%% Punctuation after thm head
%% 		{3ex}                           %%% Space after thm head
%% 		{\underline{\underline{\thmname{#1}}}}%%% Thm head spec
%% \theoremstyle{ResponseStyle}

\usepackage[tikz]{mdframed}
\mdfdefinestyle{ResponseStyle}{leftmargin=1cm,linecolor=black,roundcorner=5pt,frametitlefont=\wedn\bfseries,%frametitle={\underline{\underline{Response}}:}
, font=\wedn\bfseries,}%\begin{mdframed}[style=mystyle]foo\end{mdframed}


\ifhandout
\NewEnviron{freeResponse}{}
\else
%\newtheorem{freeResponse}{Response:}
\newenvironment{freeResponse}{\begin{mdframed}[style=ResponseStyle]}{\end{mdframed}}
\fi



%% attempting to automate outcomes.

\newwrite\outcomefile
  \immediate\openout\outcomefile=\jobname.oc
\renewcommand{\outcome}[1]{\edef\theoutcomes{\theoutcomes #1~}%
\immediate\write\outcomefile{\unexpanded{\outcome}{#1}}}

%% \newcommand{\outcomelist}{\begin{itemize}\theoutcomes\end{itemize}}



%% my commands

\newcommand{\snap}{{\bfseries\itshape\textsf{Snap!}}}
\newcommand{\flavor}{\link[\snap]{https://snap.berkeley.edu/}}


\usepackage{tkz-euclide}
\tikzstyle geometryDiagrams=[rounded corners=.5pt,ultra thick,color=black]
\colorlet{penColor}{black} % Color of a curve in a plot

\title{Louie Llama and the triangle}

\author{Jenny Sheldon \and Bart Snapp}

\begin{document}
\begin{abstract}
  Let's derive basic facts about angles of triangles.
\end{abstract}
\maketitle


\begin{listOutcomes}
\item Measure lengths with a ruler. 
\item Measure angles with a protractor.
\item Use \snap\ to verify hypotheses.
\item Work with interior angles.
\item Work with exterior angles.
\item Derive simple formulas for interior/exterior angles (and their
  sum) of basic shapes.
\end{listOutcomes}

We are going to investigate why the interior angles of a triangle sum
to $180^\circ$. We won't be alone on this journey; we'll have help from  Louie Llama:\index{Louie Llama}
\begin{center}
\includegraphics[height=1in]{llama.pdf}
\end{center}

Louie Llama is rather radical for a llama---he doesn't mind being
rotated one bit.

However, before we rotate Louie, we can try this out for ourselves!

Either in the classroom or right outside the building (depending on the weather), we will make a large triangle with painters tape. 

\mynewpage



\begin{question}
  Let's take Louie Llama for a walk.
  \begin{enumerate}
  \item Draw a copy of the triangle from class. Label the side lengths and the angle measure. 
  
  \item The volunteer walked around the triangle, starting in the middle of a side, keeping the inside of the triangle on their right. At each vertex of the triangle, label and draw the angle through which the volunteer turns.
  
  \item Use this activity to explain why the angles in the \snap\ code from last class are different from what you might have expected (ie, the code for the equilateral triangle says \texttt{turn 120 degrees}, not 60 degrees).  
 
 \end{enumerate}
 \end{question}

\mynewpage


\begin{question}
   Write a \snap\ script that repeats this process with Louie. \textbf{You should start Louie Llama out in the
    \emph{middle} of a side.}  Show off your work by giving screenshots of
    your SCRIPT and your STAGE. You may edit the file labeled \texttt{llama.xml}. 
  \begin{freeResponse}
    \begin{enumerate}
    \item Here it is:
      \begin{center}
        \includegraphics[width=.4\textwidth]{specificTri.jpg}
      \end{center}
    \item Here it is:
      \begin{center}
        \includegraphics[width=.4\textwidth]{llamaAndSpecificTri.jpg}
      \end{center}
    \item Here is my script and stage for the pentagon:
      \begin{center}
        \raisebox{-.2\height}{\includegraphics{SCRIPTllamaWalk.png}}
        \qquad
        \fbox{\includegraphics[width=.4\textwidth]{STAGEllamaWalk.png}}
      \end{center}
      
    \end{enumerate}
  \end{freeResponse}
\end{question}

\mynewpage


\begin{question}
  Let's take Louie Llama for a more general walk.
  \begin{enumerate}
  \item Draw some non-right scalene triangle, but this time label the
    angles $\alpha$, $\beta$, and $\gamma$. Again, imagine Louie Llama or yourself 
    starting at the middle of one side and walking around the
    triangle, like before. At each
    of the triangle, label the angle through which Louie Llama turns.
  \item In \emph{total}, how much did Louie Llama turn? Give a very brief and quick explanation.
  \item Write an equation where the right-hand side is Louie Llama's
    total rotation and the left-hand side is the sum of each rotation
    around the angle. SOLVE for $\alpha+\beta+\gamma$.
  \end{enumerate}
  \begin{freeResponse}
    \begin{enumerate}
    \item Here it is:
      \begin{center}
        \includegraphics[width=.4\textwidth]{generalTriAndLlama.jpg}
      \end{center}
    \item I can SEE he turned $360^\circ$!
    \item Write
      \[
      \underbrace{360}_{\text{whole turn}} = \underbrace{180-\alpha}_{\text{first turn}} + \underbrace{180-\beta}_{\text{second turn}} + \underbrace{180-\gamma}_{\text{third turn}}
      \]
      Solving for $\alpha + \beta + \gamma$, we have
      \[
      \alpha + \beta + \gamma = 180.
      \]
    \end{enumerate}
  \end{freeResponse}
\end{question}

\mynewpage


  Let's walk Louie Llama around other shapes and figure stuff
  out. Fill in the rest of the table below. I've filled in the top row. \textbf{Simplify} the bottom row.
  \[
  \renewcommand{\arraystretch}{3}
  \begin{array}{|c||c|c|c|}\hline
    \text{$n$-gon} & \begin{minipage}{1.3in}\center sum of \\ interior angles \end{minipage} &
    \begin{minipage}{1.5in}\center interior angles \\ of a regular $n$-gon\end{minipage} &
      \begin{minipage}{1.5in}\center exterior angles \\ of a regular $n$-gon\end{minipage} \\\hline\hline
        3 & 180^\circ & 60^\circ  & 120^\circ \\\hline
        4 & & & \\\hline
        5 & & & \\\hline
        6 & & & \\\hline
        7 & & & \\\hline
        8 & & & \\\hline\hline
        n & & & \\\hline
  \end{array}
  \]
  This table will be graded as 2 problems. You must explain your reasoning for both. 
    \begin{question}
    
\begin{enumerate}
 \item     For the exterior angles of a regular $n$-gon, try using a \emph{repeat} loop in \snap\ The exterior angle is the angle in this code.

 \item There are two ways to find the sum of interior angles and interior angles. Use one of them:
\begin{itemize}
\item  The interior angle and exterior angle always add to $180^\circ$. Once you know the size of the individual angles, you can determine the sum of all interior angles.
\item Each regular $n-$gon can be divided into triangles like this:
\begin{center}
    \begin{tikzpicture}[geometryDiagrams]
      \coordinate (A) at (0,0);
      \coordinate (B) at (2,0);
      \coordinate (C) at (2,2);
      \coordinate (D) at (0,2);
      \tkzDrawSegment (A,B)
      \tkzDrawSegment (D,B)
      \tkzDrawSegment[dashed](B,C)
      \tkzDrawSegment[dashed](C,D)
      \tkzDrawSegment (D,A)
    \end{tikzpicture}
  \end{center}

Once you know the sum of all interior angles, you can determine the size of the individual angles. 
\end{itemize}
\end{enumerate}
\end{question}

\begin{question}
Complete the last row of the table and explain your reasoning.
\end{question}

  \begin{freeResponse}
    Here's my table:
    \[
  \renewcommand{\arraystretch}{3}
  \begin{array}{|c||c|c|c|}\hline
    \text{$n$-gon} & \begin{minipage}{1.3in}\center sum of \\ interior angles \end{minipage} &
    \begin{minipage}{1.5in}\center interior angles \\ of a regular $n$-gon\end{minipage} &
      \begin{minipage}{1.5in}\center exterior angles \\ of a regular $n$-gon\end{minipage} \\\hline\hline
        3 & 180^\circ & 60^\circ  & 120^\circ \\\hline
        4 & 360^\circ & 90^\circ & 90^\circ \\\hline
        5 & 540^\circ & 108^\circ & 72^\circ \\\hline
        6 & 720^\circ & 120^\circ & 60^\circ\\\hline
        7 & 900^\circ & 128.6^\circ& 51.4^\circ\\\hline
        8 & 1080^\circ& 135^\circ& 45^\circ \\\hline\hline
        n & (n-2)180^\circ  & \frac{(n-2)180^\circ}{n} & \frac{360^\circ}{n}\\\hline
  \end{array}
  \]
  \end{freeResponse}


\end{document}
