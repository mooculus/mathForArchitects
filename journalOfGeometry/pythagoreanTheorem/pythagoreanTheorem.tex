\documentclass[noauthor,nooutcomes,handout,hints]{ximera}

\graphicspath{  
{./}
{./whoAreYou/}
{./drawingWithTheTurtle/}
{./bisectionMethod/}
{./circles/}
{./anglesAndRightTriangles/}
{./lawOfSines/}
{./lawOfCosines/}
{./plotter/}
{./staircases/}
{./pitch/}
{./qualityControl/}
{./symmetry/}
{./nGonBlock/}
}


%% page layout
\usepackage[cm,headings]{fullpage}
\raggedright
\setlength\headheight{13.6pt}


%% fonts
\usepackage{euler}

\usepackage{FiraMono}
\renewcommand\familydefault{\ttdefault} 
\usepackage[defaultmathsizes]{mathastext}
\usepackage[htt]{hyphenat}

\usepackage[T1]{fontenc}
\usepackage[scaled=1]{FiraSans}

%\usepackage{wedn}
\usepackage{pbsi} %% Answer font


\usepackage{cancel} %% strike through in pitch/pitch.tex


%% \usepackage{ulem} %% 
%% \renewcommand{\ULthickness}{2pt}% changes underline thickness

\tikzset{>=stealth}

\usepackage{adjustbox}

\setcounter{titlenumber}{-1}

%% journal style
\makeatletter
\newcommand\journalstyle{%
  \def\activitystyle{activity-chapter}
  \def\maketitle{%
    \addtocounter{titlenumber}{1}%
                {\flushleft\small\sffamily\bfseries\@pretitle\par\vspace{-1.5em}}%
                {\flushleft\LARGE\sffamily\bfseries\thetitlenumber\hspace{1em}\@title \par }%
                {\vskip .6em\noindent\textit\theabstract\setcounter{question}{0}\setcounter{sectiontitlenumber}{0}}%
                    \par\vspace{2em}
                    \phantomsection\addcontentsline{toc}{section}{\thetitlenumber\hspace{1em}\textbf{\@title}}%
                     }}
\makeatother



%% thm like environments
\let\question\relax
\let\endquestion\relax

\newtheoremstyle{QuestionStyle}{\topsep}{\topsep}%%% space between body and thm
		{}                      %%% Thm body font
		{}                              %%% Indent amount (empty = no indent)
		{\bfseries}            %%% Thm head font
		{)}                              %%% Punctuation after thm head
		{ }                           %%% Space after thm head
		{\thmnumber{#2}\thmnote{ \bfseries(#3)}}%%% Thm head spec
\theoremstyle{QuestionStyle}
\newtheorem{question}{}



\let\freeResponse\relax
\let\endfreeResponse\relax

%% \newtheoremstyle{ResponseStyle}{\topsep}{\topsep}%%% space between body and thm
%% 		{\wedn\bfseries}                      %%% Thm body font
%% 		{}                              %%% Indent amount (empty = no indent)
%% 		{\wedn\bfseries}            %%% Thm head font
%% 		{}                              %%% Punctuation after thm head
%% 		{3ex}                           %%% Space after thm head
%% 		{\underline{\underline{\thmname{#1}}}}%%% Thm head spec
%% \theoremstyle{ResponseStyle}

\usepackage[tikz]{mdframed}
\mdfdefinestyle{ResponseStyle}{leftmargin=1cm,linecolor=black,roundcorner=5pt,
, font=\bsifamily,}%font=\wedn\bfseries\upshape,}


\ifhandout
\NewEnviron{freeResponse}{}
\else
%\newtheorem{freeResponse}{Response:}
\newenvironment{freeResponse}{\begin{mdframed}[style=ResponseStyle]}{\end{mdframed}}
\fi



%% attempting to automate outcomes.

%% \newwrite\outcomefile
%%   \immediate\openout\outcomefile=\jobname.oc
%% \renewcommand{\outcome}[1]{\edef\theoutcomes{\theoutcomes #1~}%
%% \immediate\write\outcomefile{\unexpanded{\outcome}{#1}}}

%% \newcommand{\outcomelist}{\begin{itemize}\theoutcomes\end{itemize}}

%% \NewEnviron{listOutcomes}{\small\sffamily
%% After answering the following questions, students should be able to:
%% \begin{itemize}
%% \BODY
%% \end{itemize}
%% }
\usepackage[tikz]{mdframed}
\mdfdefinestyle{OutcomeStyle}{leftmargin=2cm,rightmargin=2cm,linecolor=black,roundcorner=5pt,
, font=\small\sffamily,}%font=\wedn\bfseries\upshape,}
\newenvironment{listOutcomes}{\begin{mdframed}[style=OutcomeStyle]After answering the following questions, students should be able to:\begin{itemize}}{\end{itemize}\end{mdframed}}



%% my commands

\newcommand{\snap}{{\bfseries\itshape\textsf{Snap!}}}
\newcommand{\flavor}{\link[\snap]{https://snap.berkeley.edu/}}
\newcommand{\mooculus}{\textsf{\textbf{MOOC}\textnormal{\textsf{ULUS}}}}


\usepackage{tkz-euclide}
\tikzstyle geometryDiagrams=[rounded corners=.5pt,ultra thick,color=black]
\colorlet{penColor}{black} % Color of a curve in a plot



\ifhandout\newcommand{\mynewpage}{\newpage}\else\newcommand{\mynewpage}{}\fi


\title{The Pythagorean theorem}
\author{Bart Snapp}

\begin{document}
\begin{abstract}
  We think about the most famous theorem of all.
\end{abstract}
\maketitle


\begin{listOutcomes}
\item{State the Pythagorean theorem.}
\item{Explain why the Pythagorean theorem is true.}
\item{Translate classroom mathematics into real world mathematics.}
\end{listOutcomes}
\mynewpage

\begin{question}
  State the \textbf{Pythagorean theorem}. EXPLAIN the result of the theorem as
  you would like to have it explained to you.  Use words, pictures,
  and so on, as needed/helpful in your explanation.
  \begin{hint}
    Be careful, don't answer too quickly, use the INTERNET to ensure
    you get this right. When I asked \textit{Geometry Giorgio} this
    question, he got it wrong.
  \end{hint}
  \begin{freeResponse}
    The \underline{Pythagorean theorem} says that given a right
    triangle,
    \begin{center}
      \begin{tikzpicture}[geometryDiagrams]
        \coordinate (A) at (0,0);
        \coordinate (B) at (0,3);
        \coordinate (C) at (7,0);
        \tkzDrawSegment (A,B)
        \tkzDrawSegment (A,C)
        \tkzDrawSegment (C,B)
        \tkzLabelSegment[left](A,B){$a$}
        \tkzLabelSegment[below](A,C){$b$}
        \tkzLabelSegment[above right](B,C){$c$}  

        \tkzMarkRightAngle[thin](C,A,B)
        %\tkzLabelAngle[pos=1.2](C,A,B){$?$}

        %% \tkzMarkAngle[size=0.8cm,thin,mark=](A,B,C)
        %% \tkzLabelAngle[pos=.5](A,B,C){$?$}

        %% \tkzMarkAngle[mark=,size=.9,thin](B,C,A)
        %% \tkzLabelAngle[pos=.6](B,C,A){$?$}
        
      \end{tikzpicture}
    \end{center}
    we will always have that
    \[
    a^2 + b^2 = c^2.
    \]
  \end{freeResponse}
\end{question}
\mynewpage


\begin{question}
  Consider the following pictures:
  \begin{center}
    \begin{tikzpicture}
        %% First square
        \coordinate (A) at (0,0);
        \coordinate (B) at (0,4);
        \coordinate (C) at (4,4);
        \coordinate (D) at (4,0);
        \coordinate (E) at (1,0);
        \coordinate (F) at (0,3);
        \coordinate (G) at (3,4);
        \coordinate (H) at (4,1);

        \tkzMarkRightAngle(A,B,C)
        \tkzMarkRightAngle(B,C,D)
        \tkzMarkRightAngle(C,D,A)
        \tkzMarkRightAngle(D,A,B)

        \tkzMarkRightAngle(E,F,G)
        \tkzMarkRightAngle(F,G,H)
        \tkzMarkRightAngle(G,H,E)
        \tkzMarkRightAngle(H,E,F)
        
        
        \draw (A)--(B)--(C)--(D)--cycle;
        \draw (E)--(F)--(G)--(H)--cycle;
        
        
        %% Second square
        \coordinate (AA) at (6,0);
        \coordinate (BB) at (6,4);
        \coordinate (CC) at (10,4);
        \coordinate (DD) at (10,0);
        \coordinate (EE) at (7,0);
        \coordinate (FF) at (6,3);
        \coordinate (GG) at (7,4);
        \coordinate (HH) at (10,3);
        \coordinate (II) at (7,3);

        \tkzMarkRightAngle(II,GG,CC)
        \tkzMarkRightAngle(II,HH,CC)
        \tkzMarkRightAngle(FF,II,EE)
        \tkzMarkRightAngle(FF,AA,EE)
        
        \draw (AA)--(BB)--(CC)--(DD)--cycle;
        \draw (EE)--(FF);
        \draw (II)--(CC);
        \draw (EE)--(GG);
        \draw (FF)--(HH);
       
      \end{tikzpicture}
  \end{center}
  EXPLAIN why the Pythagorean theorem is true.  Use the pictures
  above, words, perhaps additional pictures, and so on, as
  needed/helpful in your explanation.

  \begin{freeResponse}
    Given a right triangle with legs $a$ and $b$, and hypotenuse $c$,
    we can make the following squares each with side length $(a+b)$:
    \begin{center}
      \begin{tikzpicture}
        %% First square
        \coordinate (A) at (0,0);
        \coordinate (B) at (0,4);
        \coordinate (C) at (4,4);
        \coordinate (D) at (4,0);
        \coordinate (E) at (1,0);
        \coordinate (F) at (0,3);
        \coordinate (G) at (3,4);
        \coordinate (H) at (4,1);

        \tkzMarkRightAngle(A,B,C)
        \tkzMarkRightAngle(B,C,D)
        \tkzMarkRightAngle(C,D,A)
        \tkzMarkRightAngle(D,A,B)

        \tkzMarkRightAngle(E,F,G)
        \tkzMarkRightAngle(F,G,H)
        \tkzMarkRightAngle(G,H,E)
        \tkzMarkRightAngle(H,E,F)
        
        
        \draw (A)--(B)--(C)--(D)--cycle;
        \draw (E)--(F)--(G)--(H)--cycle;
        
        \node[right] at (.5,1.5) {$c$};
        \node[above] at (2.5,.5) {$c$};
        \node[left] at (3.5,2.5) {$c$};
        \node[below] at (1.5,3.5) {$c$};

        \node[left] at (0,1.5) {$b$};
        \node[below] at (2.5,0) {$b$};
        \node[right] at (4,2.5) {$b$};
        \node[above] at (1.5,4) {$b$};

        \node[left] at (0,3.5) {$a$};
        \node[right] at (4,.5) {$a$};
        \node[below] at (.5,0) {$a$};
        \node[above] at (3.5,4) {$a$};

        %% Second square
        \coordinate (AA) at (6,0);
        \coordinate (BB) at (6,4);
        \coordinate (CC) at (10,4);
        \coordinate (DD) at (10,0);
        \coordinate (EE) at (7,0);
        \coordinate (FF) at (6,3);
        \coordinate (GG) at (7,4);
        \coordinate (HH) at (10,3);
        \coordinate (II) at (7,3);

        \tkzMarkRightAngle(II,GG,CC)
        \tkzMarkRightAngle(II,HH,CC)
        \tkzMarkRightAngle(FF,II,EE)
        \tkzMarkRightAngle(FF,AA,EE)
        
        \draw (AA)--(BB)--(CC)--(DD)--cycle;
        \draw (EE)--(FF);
        \draw (II)--(CC);
        \draw (EE)--(GG);
        \draw (FF)--(HH);
        
        \node[below] at (6.5,3) {$a$};
        \node[right] at (7,3.5) {$a$};
        \node[left] at (6,3.5) {$a$};
        \node[above] at (6.5,4) {$a$};
        \node[below] at (6.5,0) {$a$};
        \node[right] at (10,3.5) {$a$};
        
        \node[below] at (8.5,3) {$b$};
        \node[left] at (10,1.5) {$b$};
        \node[right] at (7,1.5) {$b$};
        \node[above] at (8.5,0) {$b$};

        \node[left] at (6,1.5) {$b$};
        \node[above] at (8.5,4) {$b$};
      \end{tikzpicture}
    \end{center}
    Since both squares above have side length $(a+b)$, both large
    squares have the same area. Moreover, if we remove the triangles:
    \begin{center}
      \begin{tikzpicture}
        %% First square
        \coordinate (E) at (1,0);
        \coordinate (F) at (0,3);
        \coordinate (G) at (3,4);
        \coordinate (H) at (4,1);

        \draw (E)--(F)--(G)--(H)--cycle;
        
        \node[right] at (.5,1.5) {$c$};
        \node[above] at (2.5,.5) {$c$};
        \node[left] at (3.5,2.5) {$c$};
        \node[below] at (1.5,3.5) {$c$};

        \tkzMarkRightAngle(E,F,G)
        \tkzMarkRightAngle(F,G,H)
        \tkzMarkRightAngle(G,H,E)
        \tkzMarkRightAngle(H,E,F)
        
        %% Second square
        \coordinate (AA) at (6,0);
        \coordinate (BB) at (6,4);
        \coordinate (CC) at (10,4);
        \coordinate (DD) at (10,0);
        \coordinate (EE) at (7,0);
        \coordinate (FF) at (6,3);
        \coordinate (GG) at (7,4);
        \coordinate (HH) at (10,3);
        \coordinate (II) at (7,3);
        
        \tkzMarkRightAngle(AA,BB,CC)
        \tkzMarkRightAngle(CC,DD,AA)
        \tkzMarkRightAngle(BB,FF,II)
        \tkzMarkRightAngle(BB,GG,II)
        \tkzMarkRightAngle(FF,II,GG)
        \tkzMarkRightAngle(EE,II,HH)
        \tkzMarkRightAngle(II,HH,DD)
        \tkzMarkRightAngle(II,EE,DD)

        
        \draw (EE)--(GG)--(BB)--(FF)--(HH)--(DD)--cycle;

        \node[below] at (6.5,3) {$a$};
        \node[right] at (7,3.5) {$a$};
        \node[left] at (6,3.5) {$a$};
        \node[above] at (6.5,4) {$a$};
        
        \node[below] at (8.5,3) {$b$};
        \node[left] at (10,1.5) {$b$};
        \node[right] at (7,1.5) {$b$};
        \node[above] at (8.5,0) {$b$};
      \end{tikzpicture}
    \end{center}
    Both diagrams must still have the same area, since we removed an
    equal amount from each diagram.  Hence $c^2=a^2+b^2$.
  \end{freeResponse}
\end{question}
\mynewpage


\begin{question} %% TOY BLOCK PROBLEM
  Congratulations! You've been hired (for points, not hard cash) by
  the \mooculus\ toy company to design toy blocks! They currently have
  a $1\times 1\times 1$ cube and a $1\times 1\times 2$ rectangular prism:
  \begin{center}
    \raisebox{-.4\height}{\begin{tikzpicture}[geometryDiagrams]
      \draw (-.8,2.5)--(1.2,2.5)--(2,2);
      \draw (0,2)--(-.8,2.5)--(-.8,.5)--(0,0)--(0,2);
      \draw (0,0)--(2,0)--(2,2)--(0,2)--(0,0);
      \node[left] at (-.8,1.4) {$1$};
      \node[below] at (-.5,.3) {$1$};
      \node[below] at (1,0) {$1$};
    \end{tikzpicture}}
    \qquad\text{and}\qquad
    \raisebox{-.4\height}{\begin{tikzpicture}[geometryDiagrams]
      \draw (0,2)--(-.8,2.5)--(-.8,.5)--(0,0)--(0,2);
      \draw (-.8,2.5)--(3.2,2.5)--(4,2);
      \draw (0,0)--(4,0)--(4,2)--(0,2)--(0,0);
      \node[left] at (-.8,1.4) {$1$};
      \node[below] at (-.5,.3) {$1$};
      \node[below] at (2,0) {$2$};
    \end{tikzpicture}}
  \end{center}
  They want you to help design a SINGLE right-triangular prism
  \begin{center}
    \begin{tikzpicture}[geometryDiagrams]
      \draw[thin] (0,0)--(.3,0)--(.3,.3)--(0,.3)--(0,0);
      \draw (0,2)--(-.8,2.5)--(-.8,.5)--(0,0)--(0,2);
      \draw (-.8,2.5)--(3.2,.5)--(4,0);
      \draw (0,0)--(4,0)--(0,2)--(0,0);
      \node[left] at (-.8,1.4) {$1$};
      \node[below] at (-.5,.3) {$1$};
      \node[below] at (2,0) {$x$};
      \node[] at (1.6,.8) {$y$};
    \end{tikzpicture}
  \end{center}
  by finding $x$ and $y$ so that this shape can be best combined with
  the others in the following ways, as seen from the SIDE:
  \begin{center}
     \raisebox{-.4\height}{\begin{tikzpicture}[geometryDiagrams]
      \draw (0,0) -- (2,0) -- (2,1) -- (0,1) -- (0,0);
      \draw (0,2) --(2,2);
      \draw (1,1) -- (1,2);
      \draw (0,1) -- (0,2);
      \draw (2,1) -- (2,2);

      \draw (0,2) -- (0,3) -- (2,2);
     \end{tikzpicture}}
     \qquad \text{and}\qquad
          \raisebox{-.42\height}{\begin{tikzpicture}[geometryDiagrams]
      \draw (0,0) -- (2,0) -- (2,1) -- (0,1) -- (0,0);
      \draw (0,2) --(2,2);
      \draw (1,1) -- (1,2);
      \draw (0,1) -- (0,2);
      \draw (2,1) -- (2,2);

      \draw (0,2) -- (1.5,2.866) -- (2,2);
    \end{tikzpicture}}
  \end{center}
  SHOW YOUR WORK and EXPLAIN YOUR REASONING.
  \begin{freeResponse}
    We need to find $x$ and $y$ such that
    \[
    y^2 = x^2 + 1
    \]
    and $x$ and $y$ are as close to $2$ as possible. Let's do this set
    $x = 2-b$ and $y=2+b$ since $x<y$. Now write
    \begin{align*}
      y^2 &= x^2 + 1 \\
      (2+b)^2 &= (2-b)^2 + 1 \\
      4+4b+b^2 &= 4-4b+b^2 + 1 \\
      8b &=  1 \\
      b  &= 1/8.
    \end{align*}
    So we set $x  = 1.875$ and $y = 2.125$. Indeed
    \[
    1^2 + 1.875^2 = 2.125^2.
    \]
  \end{freeResponse}
\end{question}



\end{document}
