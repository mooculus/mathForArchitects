\documentclass[noauthor,nooutcomes,handout,12pt]{ximera}

%% page layout
\usepackage[in,headings]{fullpage}
\raggedright
\setlength\headheight{13.6pt}


%% fonts
\usepackage{euler}

\usepackage{FiraMono}
\renewcommand\familydefault{\ttdefault} 
\usepackage{mathastext}
\usepackage[htt]{hyphenat}

\usepackage[T1]{fontenc}
\usepackage[scaled=1]{FiraSans}

\usepackage{wedn}
\usepackage[T1]{fontenc}

%% wrap text around scripts
\usepackage{wrapfig}

\tikzset{>=stealth}
%% snap! scripts
\usepackage{scratch3}

\usepackage{adjustbox}

%% journal style
\makeatletter
\newcommand\journalstyle{%
  \def\activitystyle{activity-chapter}
  \def\maketitle{%
    \addtocounter{titlenumber}{1}%
                {\flushleft\small\sffamily\bfseries\@pretitle\par\vspace{-1.5em}}%
                {\flushleft\LARGE\sffamily\bfseries\thetitlenumber\hspace{1em}\@title \par }%
                {\vskip .6em\noindent\textit\theabstract\setcounter{question}{0}\setcounter{sectiontitlenumber}{0}}%
                    \par\vspace{2em}
                    \phantomsection\addcontentsline{toc}{section}{\thetitlenumber\hspace{1em}\textbf{\@title}}%
                     }}
\makeatother



%% thm like environments
\let\question\relax
\let\endquestion\relax

\newtheoremstyle{QuestionStyle}{\topsep}{\topsep}%%% space between body and thm
		{}                      %%% Thm body font
		{}                              %%% Indent amount (empty = no indent)
		{\bfseries}            %%% Thm head font
		{)}                              %%% Punctuation after thm head
		{ }                           %%% Space after thm head
		{\thmnumber{#2}\thmnote{ \bfseries(#3)}}%%% Thm head spec
\theoremstyle{QuestionStyle}
\newtheorem{question}{}



\let\freeResponse\relax
\let\endfreeResponse\relax

%% \newtheoremstyle{ResponseStyle}{\topsep}{\topsep}%%% space between body and thm
%% 		{\wedn\bfseries}                      %%% Thm body font
%% 		{}                              %%% Indent amount (empty = no indent)
%% 		{\wedn\bfseries}            %%% Thm head font
%% 		{}                              %%% Punctuation after thm head
%% 		{3ex}                           %%% Space after thm head
%% 		{\underline{\underline{\thmname{#1}}}}%%% Thm head spec
%% \theoremstyle{ResponseStyle}

\usepackage[tikz]{mdframed}
\mdfdefinestyle{ResponseStyle}{leftmargin=1cm,linecolor=black,roundcorner=5pt,frametitlefont=\wedn\bfseries,%frametitle={\underline{\underline{Response}}:}
, font=\wedn\bfseries,}%\begin{mdframed}[style=mystyle]foo\end{mdframed}


\ifhandout
\NewEnviron{freeResponse}{}
\else
%\newtheorem{freeResponse}{Response:}
\newenvironment{freeResponse}{\begin{mdframed}[style=ResponseStyle]}{\end{mdframed}}
\fi



%% attempting to automate outcomes.

\newwrite\outcomefile
  \immediate\openout\outcomefile=\jobname.oc
\renewcommand{\outcome}[1]{\edef\theoutcomes{\theoutcomes #1~}%
\immediate\write\outcomefile{\unexpanded{\outcome}{#1}}}

%% \newcommand{\outcomelist}{\begin{itemize}\theoutcomes\end{itemize}}



%% my commands

\newcommand{\snap}{{\bfseries\itshape\textsf{Snap!}}}
\newcommand{\flavor}{\link[\snap]{https://snap.berkeley.edu/}}


\usepackage{tkz-euclide}
\tikzstyle geometryDiagrams=[rounded corners=.5pt,ultra thick,color=black]
\colorlet{penColor}{black} % Color of a curve in a plot


\title{The Pythagorean theorem}
\author{Bart Snapp}

\begin{document}
\begin{abstract}
  We think about the most famous theorem of all.
\end{abstract}
\maketitle


\begin{listOutcomes}
\item{State the Pythagorean theorem.}
\item{Explain why the Pythagorean theorem is true.}
\item{Use a congruence theorm to explain why the converse of the
  Pythagorean theorem is true.}
\end{listOutcomes}
\mynewpage

\begin{question}
  State the \textbf{Pythagorean theorem}. EXPLAIN the result of the theorem as
  you would like to have it explained to you.  Use words, pictures,
  and so on, as needed/helpful in your explanation.
  \begin{freeResponse}
    The \underline{Pythagorean theorem} says that given a right
    triangle,
    \begin{center}
      \begin{tikzpicture}[geometryDiagrams]
        \coordinate (A) at (0,0);
        \coordinate (B) at (0,3);
        \coordinate (C) at (7,0);
        \tkzDrawSegment (A,B)
        \tkzDrawSegment (A,C)
        \tkzDrawSegment (C,B)
        \tkzLabelSegment[left](A,B){$a$}
        \tkzLabelSegment[below](A,C){$b$}
        \tkzLabelSegment[above right](B,C){$c$}  

        \tkzMarkRightAngle[thin](C,A,B)
        %\tkzLabelAngle[pos=1.2](C,A,B){$?$}

        %% \tkzMarkAngle[size=0.8cm,thin,mark=](A,B,C)
        %% \tkzLabelAngle[pos=.5](A,B,C){$?$}

        %% \tkzMarkAngle[mark=,size=.9,thin](B,C,A)
        %% \tkzLabelAngle[pos=.6](B,C,A){$?$}
        
      \end{tikzpicture}
    \end{center}
    we will always have that
    \[
    a^2 + b^2 = c^2.
    \]
  \end{freeResponse}
\end{question}
\mynewpage


\begin{question}
  Consider the following pictures:
  \begin{center}
    \begin{tikzpicture}
        %% First square
        \coordinate (A) at (0,0);
        \coordinate (B) at (0,4);
        \coordinate (C) at (4,4);
        \coordinate (D) at (4,0);
        \coordinate (E) at (1,0);
        \coordinate (F) at (0,3);
        \coordinate (G) at (3,4);
        \coordinate (H) at (4,1);

        \tkzMarkRightAngle(A,B,C)
        \tkzMarkRightAngle(B,C,D)
        \tkzMarkRightAngle(C,D,A)
        \tkzMarkRightAngle(D,A,B)

        \tkzMarkRightAngle(E,F,G)
        \tkzMarkRightAngle(F,G,H)
        \tkzMarkRightAngle(G,H,E)
        \tkzMarkRightAngle(H,E,F)
        
        
        \draw (A)--(B)--(C)--(D)--cycle;
        \draw (E)--(F)--(G)--(H)--cycle;
        
        
        %% Second square
        \coordinate (AA) at (6,0);
        \coordinate (BB) at (6,4);
        \coordinate (CC) at (10,4);
        \coordinate (DD) at (10,0);
        \coordinate (EE) at (7,0);
        \coordinate (FF) at (6,3);
        \coordinate (GG) at (7,4);
        \coordinate (HH) at (10,3);
        \coordinate (II) at (7,3);

        \tkzMarkRightAngle(II,GG,CC)
        \tkzMarkRightAngle(II,HH,CC)
        \tkzMarkRightAngle(FF,II,EE)
        \tkzMarkRightAngle(FF,AA,EE)
        
        \draw (AA)--(BB)--(CC)--(DD)--cycle;
        \draw (EE)--(FF);
        \draw (II)--(CC);
        \draw (EE)--(GG);
        \draw (FF)--(HH);
       
      \end{tikzpicture}
  \end{center}
  EXPLAIN why the Pythagorean theorem is true.  Use the pictures
  above, words, perhaps additional pictures, and so on, as
  needed/helpful in your explanation.

  \begin{freeResponse}
    Given a right triangle with legs $a$ and $b$, and hypotenuse $c$,
    we can make the following squares each with side length $(a+b)$:
    \begin{center}
      \begin{tikzpicture}
        %% First square
        \coordinate (A) at (0,0);
        \coordinate (B) at (0,4);
        \coordinate (C) at (4,4);
        \coordinate (D) at (4,0);
        \coordinate (E) at (1,0);
        \coordinate (F) at (0,3);
        \coordinate (G) at (3,4);
        \coordinate (H) at (4,1);

        \tkzMarkRightAngle(A,B,C)
        \tkzMarkRightAngle(B,C,D)
        \tkzMarkRightAngle(C,D,A)
        \tkzMarkRightAngle(D,A,B)

        \tkzMarkRightAngle(E,F,G)
        \tkzMarkRightAngle(F,G,H)
        \tkzMarkRightAngle(G,H,E)
        \tkzMarkRightAngle(H,E,F)
        
        
        \draw (A)--(B)--(C)--(D)--cycle;
        \draw (E)--(F)--(G)--(H)--cycle;
        
        \node[right] at (.5,1.5) {$c$};
        \node[above] at (2.5,.5) {$c$};
        \node[left] at (3.5,2.5) {$c$};
        \node[below] at (1.5,3.5) {$c$};

        \node[left] at (0,1.5) {$b$};
        \node[below] at (2.5,0) {$b$};
        \node[right] at (4,2.5) {$b$};
        \node[above] at (1.5,4) {$b$};

        \node[left] at (0,3.5) {$a$};
        \node[right] at (4,.5) {$a$};
        \node[below] at (.5,0) {$a$};
        \node[above] at (3.5,4) {$a$};

        %% Second square
        \coordinate (AA) at (6,0);
        \coordinate (BB) at (6,4);
        \coordinate (CC) at (10,4);
        \coordinate (DD) at (10,0);
        \coordinate (EE) at (7,0);
        \coordinate (FF) at (6,3);
        \coordinate (GG) at (7,4);
        \coordinate (HH) at (10,3);
        \coordinate (II) at (7,3);

        \tkzMarkRightAngle(II,GG,CC)
        \tkzMarkRightAngle(II,HH,CC)
        \tkzMarkRightAngle(FF,II,EE)
        \tkzMarkRightAngle(FF,AA,EE)
        
        \draw (AA)--(BB)--(CC)--(DD)--cycle;
        \draw (EE)--(FF);
        \draw (II)--(CC);
        \draw (EE)--(GG);
        \draw (FF)--(HH);
        
        \node[below] at (6.5,3) {$a$};
        \node[right] at (7,3.5) {$a$};
        \node[left] at (6,3.5) {$a$};
        \node[above] at (6.5,4) {$a$};
        \node[below] at (6.5,0) {$a$};
        \node[right] at (10,3.5) {$a$};
        
        \node[below] at (8.5,3) {$b$};
        \node[left] at (10,1.5) {$b$};
        \node[right] at (7,1.5) {$b$};
        \node[above] at (8.5,0) {$b$};

        \node[left] at (6,1.5) {$b$};
        \node[above] at (8.5,4) {$b$};
      \end{tikzpicture}
    \end{center}
    Since both squares above have side length $(a+b)$, both large
    squares have the same area. Moreover, if we remove the triangles:
    \begin{center}
      \begin{tikzpicture}
        %% First square
        \coordinate (E) at (1,0);
        \coordinate (F) at (0,3);
        \coordinate (G) at (3,4);
        \coordinate (H) at (4,1);

        \draw (E)--(F)--(G)--(H)--cycle;
        
        \node[right] at (.5,1.5) {$c$};
        \node[above] at (2.5,.5) {$c$};
        \node[left] at (3.5,2.5) {$c$};
        \node[below] at (1.5,3.5) {$c$};

        \tkzMarkRightAngle(E,F,G)
        \tkzMarkRightAngle(F,G,H)
        \tkzMarkRightAngle(G,H,E)
        \tkzMarkRightAngle(H,E,F)
        
        %% Second square
        \coordinate (BB) at (6,4);
        \coordinate (DD) at (10,0);
        \coordinate (EE) at (7,0);
        \coordinate (FF) at (6,3);
        \coordinate (GG) at (7,4);
        \coordinate (HH) at (10,3);
        \coordinate (II) at (7,3);
        
        \tkzMarkRightAngle(AA,BB,CC)
        \tkzMarkRightAngle(CC,DD,AA)
        \tkzMarkRightAngle(BB,FF,II)
        \tkzMarkRightAngle(BB,GG,II)
        \tkzMarkRightAngle(FF,II,GG)
        \tkzMarkRightAngle(EE,II,HH)
        \tkzMarkRightAngle(II,HH,DD)
        \tkzMarkRightAngle(II,EE,DD)

        
        \draw (EE)--(GG)--(BB)--(FF)--(HH)--(DD)--cycle;

        \node[below] at (6.5,3) {$a$};
        \node[right] at (7,3.5) {$a$};
        \node[left] at (6,3.5) {$a$};
        \node[above] at (6.5,4) {$a$};
        
        \node[below] at (8.5,3) {$b$};
        \node[left] at (10,1.5) {$b$};
        \node[right] at (7,1.5) {$b$};
        \node[above] at (8.5,0) {$b$};
      \end{tikzpicture}
    \end{center}
    Both diagrams must still have the same area, since we removed an
    equal amount from each diagram.  Hence $c^2=a^2+b^2$.
  \end{freeResponse}
\end{question}
\mynewpage

\begin{question}
  Suppose that I have three numbers $a$, $b$, and $c$ such that:
  \[
  a^2 + b^2 = c^2
  \]
  Does that mean that there is a cooresponding right triangle?
  \begin{hint}
    Think SAS. 
  \end{hint}
  \begin{freeResponse}
    Well as long as all of $a$, $b$, and $c$ are postive numbers there
    will be such a triangle, by SAS.  One side has length $a$, the
    other has length $b$, and the angle is the right angle between
    them. Since there is only one such triangle, and it's a right
    triangle, the Pythagorean theorem guarentees that the final side
    must have length $c = \sqrt{a^2+b^2}$.
  \end{freeResponse}
  
\end{question}



\end{document}
