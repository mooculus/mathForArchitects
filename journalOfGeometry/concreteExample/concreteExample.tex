\documentclass[handout,nooutcomes,noauthor]{ximera}

\graphicspath{  
{./}
{./whoAreYou/}
{./drawingWithTheTurtle/}
{./bisectionMethod/}
{./circles/}
{./anglesAndRightTriangles/}
{./lawOfSines/}
{./lawOfCosines/}
{./plotter/}
{./staircases/}
{./pitch/}
{./qualityControl/}
{./symmetry/}
{./nGonBlock/}
}


%% page layout
\usepackage[cm,headings]{fullpage}
\raggedright
\setlength\headheight{13.6pt}


%% fonts
\usepackage{euler}

\usepackage{FiraMono}
\renewcommand\familydefault{\ttdefault} 
\usepackage[defaultmathsizes]{mathastext}
\usepackage[htt]{hyphenat}

\usepackage[T1]{fontenc}
\usepackage[scaled=1]{FiraSans}

%\usepackage{wedn}
\usepackage{pbsi} %% Answer font


\usepackage{cancel} %% strike through in pitch/pitch.tex


%% \usepackage{ulem} %% 
%% \renewcommand{\ULthickness}{2pt}% changes underline thickness

\tikzset{>=stealth}

\usepackage{adjustbox}

\setcounter{titlenumber}{-1}

%% journal style
\makeatletter
\newcommand\journalstyle{%
  \def\activitystyle{activity-chapter}
  \def\maketitle{%
    \addtocounter{titlenumber}{1}%
                {\flushleft\small\sffamily\bfseries\@pretitle\par\vspace{-1.5em}}%
                {\flushleft\LARGE\sffamily\bfseries\thetitlenumber\hspace{1em}\@title \par }%
                {\vskip .6em\noindent\textit\theabstract\setcounter{question}{0}\setcounter{sectiontitlenumber}{0}}%
                    \par\vspace{2em}
                    \phantomsection\addcontentsline{toc}{section}{\thetitlenumber\hspace{1em}\textbf{\@title}}%
                     }}
\makeatother



%% thm like environments
\let\question\relax
\let\endquestion\relax

\newtheoremstyle{QuestionStyle}{\topsep}{\topsep}%%% space between body and thm
		{}                      %%% Thm body font
		{}                              %%% Indent amount (empty = no indent)
		{\bfseries}            %%% Thm head font
		{)}                              %%% Punctuation after thm head
		{ }                           %%% Space after thm head
		{\thmnumber{#2}\thmnote{ \bfseries(#3)}}%%% Thm head spec
\theoremstyle{QuestionStyle}
\newtheorem{question}{}



\let\freeResponse\relax
\let\endfreeResponse\relax

%% \newtheoremstyle{ResponseStyle}{\topsep}{\topsep}%%% space between body and thm
%% 		{\wedn\bfseries}                      %%% Thm body font
%% 		{}                              %%% Indent amount (empty = no indent)
%% 		{\wedn\bfseries}            %%% Thm head font
%% 		{}                              %%% Punctuation after thm head
%% 		{3ex}                           %%% Space after thm head
%% 		{\underline{\underline{\thmname{#1}}}}%%% Thm head spec
%% \theoremstyle{ResponseStyle}

\usepackage[tikz]{mdframed}
\mdfdefinestyle{ResponseStyle}{leftmargin=1cm,linecolor=black,roundcorner=5pt,
, font=\bsifamily,}%font=\wedn\bfseries\upshape,}


\ifhandout
\NewEnviron{freeResponse}{}
\else
%\newtheorem{freeResponse}{Response:}
\newenvironment{freeResponse}{\begin{mdframed}[style=ResponseStyle]}{\end{mdframed}}
\fi



%% attempting to automate outcomes.

%% \newwrite\outcomefile
%%   \immediate\openout\outcomefile=\jobname.oc
%% \renewcommand{\outcome}[1]{\edef\theoutcomes{\theoutcomes #1~}%
%% \immediate\write\outcomefile{\unexpanded{\outcome}{#1}}}

%% \newcommand{\outcomelist}{\begin{itemize}\theoutcomes\end{itemize}}

%% \NewEnviron{listOutcomes}{\small\sffamily
%% After answering the following questions, students should be able to:
%% \begin{itemize}
%% \BODY
%% \end{itemize}
%% }
\usepackage[tikz]{mdframed}
\mdfdefinestyle{OutcomeStyle}{leftmargin=2cm,rightmargin=2cm,linecolor=black,roundcorner=5pt,
, font=\small\sffamily,}%font=\wedn\bfseries\upshape,}
\newenvironment{listOutcomes}{\begin{mdframed}[style=OutcomeStyle]After answering the following questions, students should be able to:\begin{itemize}}{\end{itemize}\end{mdframed}}



%% my commands

\newcommand{\snap}{{\bfseries\itshape\textsf{Snap!}}}
\newcommand{\flavor}{\link[\snap]{https://snap.berkeley.edu/}}
\newcommand{\mooculus}{\textsf{\textbf{MOOC}\textnormal{\textsf{ULUS}}}}


\usepackage{tkz-euclide}
\tikzstyle geometryDiagrams=[rounded corners=.5pt,ultra thick,color=black]
\colorlet{penColor}{black} % Color of a curve in a plot



\ifhandout\newcommand{\mynewpage}{\newpage}\else\newcommand{\mynewpage}{}\fi

\title{A concrete example}

\author{Bart Snapp}

\begin{document}
\begin{abstract}
  Let's do a concrete example.
\end{abstract}
\maketitle


\begin{listOutcomes}
\item Convert standard units.
\item Critique and dismantle reasonable hypotheses in regard to
  geometry and arithmetic.
\item Find conversion factors for higher dimensional units, based on
  the linear unit conversion factors.
\item Scale one physical attribute of an object (length, area,
  volume), and find the scaled values for the other attributes.
\end{listOutcomes}



\mynewpage



\begin{question}
  Suppose you want to lay down a concrete sidewalk. We'll assume that
  \begin{itemize}
  \item sidewalks are $4$ feet wide,
  \item are poured using $2$-by-$4$s as forms.
  \end{itemize}
  So good so far, but there are some confounding issues:
  \begin{itemize}
  \item Concrete is sold in \textbf{cubic yards}.
  \item $2$-by-$4$s are actually $3$ and $\frac{1}{2}$ inches tall and
    $1$ and $\frac{1}{2}$ inches wide.
  \end{itemize}
  So, how much concrete (in cubic yards) is required to make a section
  of sidewalk $100$ feet long, $4$ feet wide, and $3.5$ inches tall?
  \begin{freeResponse}
    Let's find our volume in cubic yards. To do this, we'll convert
    all measurements to yards, and then multiply them together. Write with me
    \[
    \frac{100}{3} \cdot \frac{4}{3}\cdot \frac{3.5}{12\cdot 3} = \frac{1400}{324} = 4.32 \text{ cubic yards}
    \] 
  \end{freeResponse}
\end{question}
\mynewpage


\begin{question}
  Suppose you have $189$ cubic feet of concrete. \textit{Geometry
    Giorgio} suggests that this is
  \[
  \frac{189}{3} = 63 \text{ cubic yards}.
  \]
  Is he correct? If he is correct, explain why. If he is not correct,
  explain why and give the correct answer.
  \begin{freeResponse}
     \textit{Geometry Giorgio} is WAY OFF. He needs to divide by $27$,
     not $3$. This is because when finding the volume of our form in
     cubic yards, we convert EACH of the measurements into yards. This
     creates THREE factors of $\frac{1}{3}$, and hence the correct
     answer is
     \[
     \frac{189}{27} = 7 \text{ cubic yards}.
     \]
  \end{freeResponse}
\end{question}
\mynewpage


\begin{question}
  Suppose that you are pouring a square concrete patio with $1.5$
  cubic yards of concrete. Your client asks for the patio to have a
  side length THREE times as long.  \textit{Geometry Giorgio} suggests
  that this will require
  \[
  1.5\cdot  3 = 4.5 \text{ cubic yards of concrete}.
  \]
  Is he correct? If he is correct, explain why. If he is not correct,
  explain why and give the correct answer.
  \begin{freeResponse}
    Well, our square patio has some area $A$. To make the side of the
    square patio three times as long, we'll need NINE squares, hence
    the area will go up by $9$. If $A$ is the area of the patio and
    $h$ is the height,
    \[
    1.5 = A \cdot h
    \]
    Tripling the side length of the square multiply the area by nine, so
    \[
    9\cdot 1.5 = 9\cdot A \cdot h 
    \]
    hence, we'll need $13.5$ cubic yards of concrete.
  \end{freeResponse}
\end{question}





\end{document}
