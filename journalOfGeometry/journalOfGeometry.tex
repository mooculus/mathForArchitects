\documentclass[hints,numbers,nooutcomes]{xourse}

\graphicspath{  
{./}
{./whoAreYou/}
{./drawingWithTheTurtle/}
{./bisectionMethod/}
{./circles/}
{./anglesAndRightTriangles/}
{./lawOfSines/}
{./lawOfCosines/}
{./plotter/}
{./staircases/}
{./pitch/}
{./qualityControl/}
{./symmetry/}
{./nGonBlock/}
}


%% page layout
\usepackage[cm,headings]{fullpage}
\raggedright
\setlength\headheight{13.6pt}


%% fonts
\usepackage{euler}

\usepackage{FiraMono}
\renewcommand\familydefault{\ttdefault} 
\usepackage[defaultmathsizes]{mathastext}
\usepackage[htt]{hyphenat}

\usepackage[T1]{fontenc}
\usepackage[scaled=1]{FiraSans}

%\usepackage{wedn}
\usepackage{pbsi} %% Answer font


\usepackage{cancel} %% strike through in pitch/pitch.tex


%% \usepackage{ulem} %% 
%% \renewcommand{\ULthickness}{2pt}% changes underline thickness

\tikzset{>=stealth}

\usepackage{adjustbox}

\setcounter{titlenumber}{-1}

%% journal style
\makeatletter
\newcommand\journalstyle{%
  \def\activitystyle{activity-chapter}
  \def\maketitle{%
    \addtocounter{titlenumber}{1}%
                {\flushleft\small\sffamily\bfseries\@pretitle\par\vspace{-1.5em}}%
                {\flushleft\LARGE\sffamily\bfseries\thetitlenumber\hspace{1em}\@title \par }%
                {\vskip .6em\noindent\textit\theabstract\setcounter{question}{0}\setcounter{sectiontitlenumber}{0}}%
                    \par\vspace{2em}
                    \phantomsection\addcontentsline{toc}{section}{\thetitlenumber\hspace{1em}\textbf{\@title}}%
                     }}
\makeatother



%% thm like environments
\let\question\relax
\let\endquestion\relax

\newtheoremstyle{QuestionStyle}{\topsep}{\topsep}%%% space between body and thm
		{}                      %%% Thm body font
		{}                              %%% Indent amount (empty = no indent)
		{\bfseries}            %%% Thm head font
		{)}                              %%% Punctuation after thm head
		{ }                           %%% Space after thm head
		{\thmnumber{#2}\thmnote{ \bfseries(#3)}}%%% Thm head spec
\theoremstyle{QuestionStyle}
\newtheorem{question}{}



\let\freeResponse\relax
\let\endfreeResponse\relax

%% \newtheoremstyle{ResponseStyle}{\topsep}{\topsep}%%% space between body and thm
%% 		{\wedn\bfseries}                      %%% Thm body font
%% 		{}                              %%% Indent amount (empty = no indent)
%% 		{\wedn\bfseries}            %%% Thm head font
%% 		{}                              %%% Punctuation after thm head
%% 		{3ex}                           %%% Space after thm head
%% 		{\underline{\underline{\thmname{#1}}}}%%% Thm head spec
%% \theoremstyle{ResponseStyle}

\usepackage[tikz]{mdframed}
\mdfdefinestyle{ResponseStyle}{leftmargin=1cm,linecolor=black,roundcorner=5pt,
, font=\bsifamily,}%font=\wedn\bfseries\upshape,}


\ifhandout
\NewEnviron{freeResponse}{}
\else
%\newtheorem{freeResponse}{Response:}
\newenvironment{freeResponse}{\begin{mdframed}[style=ResponseStyle]}{\end{mdframed}}
\fi



%% attempting to automate outcomes.

%% \newwrite\outcomefile
%%   \immediate\openout\outcomefile=\jobname.oc
%% \renewcommand{\outcome}[1]{\edef\theoutcomes{\theoutcomes #1~}%
%% \immediate\write\outcomefile{\unexpanded{\outcome}{#1}}}

%% \newcommand{\outcomelist}{\begin{itemize}\theoutcomes\end{itemize}}

%% \NewEnviron{listOutcomes}{\small\sffamily
%% After answering the following questions, students should be able to:
%% \begin{itemize}
%% \BODY
%% \end{itemize}
%% }
\usepackage[tikz]{mdframed}
\mdfdefinestyle{OutcomeStyle}{leftmargin=2cm,rightmargin=2cm,linecolor=black,roundcorner=5pt,
, font=\small\sffamily,}%font=\wedn\bfseries\upshape,}
\newenvironment{listOutcomes}{\begin{mdframed}[style=OutcomeStyle]After answering the following questions, students should be able to:\begin{itemize}}{\end{itemize}\end{mdframed}}



%% my commands

\newcommand{\snap}{{\bfseries\itshape\textsf{Snap!}}}
\newcommand{\flavor}{\link[\snap]{https://snap.berkeley.edu/}}
\newcommand{\mooculus}{\textsf{\textbf{MOOC}\textnormal{\textsf{ULUS}}}}


\usepackage{tkz-euclide}
\tikzstyle geometryDiagrams=[rounded corners=.5pt,ultra thick,color=black]
\colorlet{penColor}{black} % Color of a curve in a plot



\ifhandout\newcommand{\mynewpage}{\newpage}\else\newcommand{\mynewpage}{}\fi


\renewcommand{\mynewpage}{}
%\renewcommand{\vfill}{}


\title{\textsf{Geometry Journal}}

\author{Merriman \quad $\bullet$\quad  Sheldon \quad$\bullet$\quad Snapp}


\begin{document}
\begin{abstract}
  Write your own geometry book.
\end{abstract}
\maketitle 

%\setcounter{tocdepth}{-1} %%% CURRENTLY DOES NOTHING!!!!



\partstyle
\activity{sectionHeads/introductionAU21.tex}
%%
%% ADD Greek letters and their names
%%

%% ADD "Check yourself" sections.
%%
%%

%% \partstyle
%% \activity{sectionHeads/sizeAndGrowthOfNumbers.tex}
%% \journalstyle
%%% THINK ABOUT NUMBERS
%
%%% payed a dollar every day, how long to make 100k?
%%% Use that to figure out lifetime, week month.
%%% Weathlest person in the world, how long?
%
%% \activity{dollarASecond/dollarASecond.tex}
%
%
%
%%% penniesADay
%% \activity{penniesADay/penniesADay.tex}
%%% Exponential growth
%%%%%% penny a day, two next, how much made in 1st day? 2nd day? 1st year? 2nd year? 
%%%%%% Use Snap?
%%%% Richter scale. 
%%% Feynman quote -- why so hard to understand?
%%% STUDENTS FIND EXAMPLES 
%
%
%%%% Abstracting away
%
%%% which is better
%%% summing? 1+ 2 + 4 +... + 2^n and pbp?
%%% then graphing
%
%
%
%%% COMPARING GROWTH
%
%\activity{comparingGrowth/comparingGrowth.tex}
%
%%% What is this speedy growth??? When the amount is proportial to the
%%% rate that it is growing.  EXPLAIN in CONTEXT OF BEING PAID.
%
%%R = A-1
%
%
%
%%% EXPLAIN VIRUS
%%% EPXLAIN ... 
%
%
%
%%% Find examples. 
%
%
%
%%% How long with exponetnial growth weathest person?
%%% adding numbers?
%%% multiplying numbers

\partstyle \activity{sectionHeads/scalingLengthAreaVolume} %% needs
 \journalstyle
%\activity{whatDoYaWannaKnow/whatDoYaWannaKnow.tex} % REQUIRED

\journalstyle
\activity{gettingStarted/gettingStarted.tex} %% DONE
\checkstyle
\activity{gettingStarted/checkYourself.tex} %% DONE
%% \journalstyle
%% \activity{roofEstimates/roofEstimates} 
\journalstyle
\activity{areaOfTriangles/areaOfTriangles.tex} %% DONE REQUIRED
\checkstyle
\activity{areaOfTriangles/checkYourself.tex}

%\skipjournalstyle
%\activity{isoperimetric/isoperimetric.tex} %% DONE (note could be skipped---not great for architects)
%% could be better accomplished by having them draw pictures of shapes that have
%% large and small perimertier for a given volume (SIC). MAYBE REMOVE
\journalstyle
\activity{volumeOfPyramids/volumeOfPyramids.tex} %% DONE  REQUIRED
\checkstyle
\activity{volumeOfPyramids/checkYourself.tex} %% DONE  
\journalstyle
\activity{roofEstimates/roofEstimates}  %%MOVE TO BEFORE AREA OF TRIANGLES!!!!!!!
\checkstyle
\activity{roofEstimates/checkYourself.tex} %% DONE  
\journalstyle
\activity{formulasGalore/formulasGalore.tex}  %% DONE  REQUIRED 
\checkstyle
\activity{formulasGalore/checkYourself.tex} %% DON

%\journalstyle
%\activity{problemSolving/problemSolving}
\journalstyle

%% TALK ABOUT NUMBERS.
%% 1) Think about size of numbers
%% 2) small* small is small, large* large  is large
%% 3) numbers of digits. 



%% BEST TRICK EVER
%% swapping percents
%% 50% of 17 is 17% of 50.



\journalstyle
\activity{hipToBeSquare/hipToBeSquare.tex}
\checkstyle
\activity{hipToBeSquare/checkyourself.tex} %% DONE
%%%% VERIFY THAT THE ANSWERS ARE CORRECT IN THE CHECKYOURSELF ABOVE


%%%%%%%%%%%%%%%%%% old stuff
%%
%% Reptiles could be used to prove infinte sums.
%%
%%
%\activity{reptiles/reptiles.tex} %% DONE REQUIRED
%\checkstyle
%\activity{reptiles/checkyourself.tex} %% DONE
%\journalstyle
%\activity{reptilesAndArea/reptilesAndArea.tex}  %% DONE REQUIRED
%\checkstyle
%\activity{reptilesAndArea/checkyourself.tex} %% DONE
%\journalstyle
%\activity{supersizeMe/supersizeMe.tex} %% DONE REQUIRED
%%%%%%%%%%%%%%%%%% old stuff


%% I think 1 and 2 of minifigsAndSuperFigs needs to be rewritten to
%% have the students compute (using formulas for boxes in 1, and
%% spheres in 2) the surface area and volume of the minifig,
%% middlefig, and superfigs, and then find that the scale factor for
%% the SA and V is always the scale factor squared/cubed regardless of
%% the shape.
%%
%% This should set them up for success in problem 3.
\activity{minifigsAndSuperfigs/minifigsAndSuperfigs.tex}
\checkstyle
\activity{minifigsAndSuperfigs/checkyourself.tex} %% DONE
%%%% VERIFY THAT THE ANSWERS ARE CORRECT IN THE CHECKYOURSELF ABOVE



%%
%% If it is 2024, remove supersizeme.
%%





\activity{concreteExample/concreteExample.tex} %% DONE REQUIRED
\checkstyle
\activity{concreteExample/checkyourself.tex} %% REWRITE with objective of scaling vol like area

%% A CRITICAL CONCEPT:
%% EVEN WHEN A MEASURMENT IS A VOLUME,
%% IT DOES NOT SCALE LIKE A VOLUME IF THE HEIGHT ISN'T CHANGING
%% EXAMPLE: SIDE WALK
%% EXAMPLE: PATIO
%% EXAMPLE: Rain Rain
%% Perhaps concrete example should come LATER... INFACT YES.
%% AND ITS CHECKYOURSELF  needs to be rewritten with the learning outcome
%% of "Volume scaling like volume vs volume scaling like area"






\journalstyle
\activity{quickQuestions/quickQuestions.tex} %% DONE REQUIRED
\checkstyle
\activity{quickQuestions/checkyourself.tex}
\journalstyle
\activity{inTheWild/inTheWild.tex} %% DONE REQUIRED
%% PIZZAS
%% TRASH CANS
%% FURNITURE
%% DOG BEDS
%% Mattresses

%%%%%%%%%%%%%%%%%%%%%%%%%%%%%%%%%%%%
%%%%%%%%%%%%%%%%%%%%%%%%%%%%%%%%%%%%
%% A CRITICAL CONCEPT:
%%
%% Warehouse problem
%%
%% n% of m is m% of n
%%
%% n% more than m followed by n% less is not m.
%%
%%%%%%%%%%%%%%%%%%%%%%%%%%%%%%%%%%%%
%%%%%%%%%%%%%%%%%%%%%%%%%%%%%%%%%%%%




\activity{rainRain/rainRain.tex}  %% DONE -- needs check

\activity{measureStairs/measureStairs.tex} %% DONE  REQUIRED
\checkstyle
\activity{measureStairs/checkYourself.tex}
\journalstyle
\activity{staircases/staircases.tex} %% DONE  REQUIRED
%https://www.calculator.net/stair-calculator.html?run=10&rununit=inch&totalrun=15&totalrununit=foot&ctype=total&totalheight=10&totalheightunit=foot&x=90&y=15
\checkstyle
\activity{staircases/checkYourself.tex} %% SEE PREVIOUS QUIZ
%%~/teaching/1118/quiz/quiz7/1118quiz7.pdf


%\activity{togetherStairs/togetherStairs.tex}
%
%
%
%
%
%
%%% REMOVED \activity{reflectingSVG/reflectingSVG.tex} %% OK IS THIS NEEDED?

\partstyle
\activity{sectionHeads/anglesAndTriangles.tex} %%%update
\journalstyle
\activity{whyTriangles/whyTriangles.tex} %% DONE REQUIRED
\checkstyle
\activity{whyTriangles/checkYourself.tex} %% DONE REQUIRED
%\activity{drawingWithTheTurtle/drawingWithTheTurtle.tex} %% DONE SNAP


%% perhaps there could be an activity
%% My turn
%%
%% discussing the "turns" seen in triangle and cone

\journalstyle
\activity{llamaAndTheTriangle/llamaAndTheTriangle.tex} %% DONE REQUIRED
\checkstyle
\activity{llamaAndTheTriangle/checkYourself.tex} %% DONE REQUIRED
\journalstyle
\activity{tilings/tilings.tex}%%%Requires tiles
\checkstyle
\activity{tilings/checkYourself.tex} %% DONE REQUIRED
\journalstyle
\activity{favoriteTilings/favoriteTilings.tex}
\checkstyle
\activity{favoriteTilings/checkYourself.tex} %% DONE REQUIRED
\skipjournalstyle
\activity{lifeOnACone/lifeOnACone.tex} %% DONE
\checkstyle
\activity{lifeOnACone/checkYourself.tex}
\journalstyle
\activity{whatsYourAngle/whatsYourAngle.tex} %% DONE REQUIRED
\checkstyle
\activity{whatsYourAngle/checkYourself.tex} %% DONE REQUIRED
%% artGalleryProblem??? %% theres a dir with notes.

%\activity{windFarm/windFarm.tex} %% DONE -- WRONG PLACE!!!!

%\activity{bisectionMethod/bisectionMethod.tex} %% DONE -- though maybe more table work should be added.
\journalstyle
\activity{pythagoreanTheorem/pythagoreanTheorem.tex} %% DONE REQUIRED


\activity{distanceFormula/distanceFormula.tex} %% DONE REQUIRED 
\activity{geometryInDisguise/geometryInDisguise.tex} %% DRAFTED -- CHECK SOLNS
\activity{applyOurselves/applyOurselves.tex}  %% DONE EXCEPT FOR PROPOSED SOLN  REQUIRED
\checkstyle
\activity{applyOurselves/checkYourself.tex}

%\skipjournalstyle
\journalstyle
\activity{moreTriangles/moreTriangles.tex}%% Added due to removing questions about vertical angles. Set up for earth
\checkstyle
\activity{moreTriangles/checkYourself.tex}

%% CRITICAL CONCEPT

%% 45-90-45 triangle est height
%% thumb measurement


\journalstyle
\activity{earth/earth.tex}  %% DRAFTED -- CHECK SOLNS SEE NOTES!!!!    REQUIRED
\checkstyle
\activity{earth/checkYourself.tex}

%https://wethestudy.com/engineering/length-eyeballing-measurements/ 
%https://www.mathsisfun.com/measure/estimate-distance.html 
\partstyle
\activity{sectionHeads/trigonometry.tex} %% 
\journalstyle



%% Sine and cosine and Right triangles and circles
%% are attempting to help students understand trig functions
%% when the angle is obtuse.

\activity{sineAndCosine/sineAndCosine.tex}  %%
\checkstyle
\activity{sineAndCosine/checkYourself.tex}  %% 
\journalstyle
%% STRING ART COULD GO HERE.



%% Wrong place?
%%
\activity{gerrymandering/gerrymandering.tex}
\activity{packCrackEfficiencyGap/packCrackEfficiencyGap.tex}
\activity{districtMaps/districtMaps.tex}
%%
%%



%%% WITHOUT PLOTTING --- NOT GREAT
%%%\activity{rightTrianglesAndCircles/rightTrianglesAndCircles.tex} %% Done
%%%

%% End cos/sin review
%%%%%%%%%%%%%%%%%%%%%%%


%%
%% BART REMOVED 3/2022
%% anglesAndRightTriangles
%%





%\activity{circles/circles.tex} %% REMOVE

%% LAW OF SINS IS NO LONGER REQUIRED.
%%\activity{lawOfSines/lawOfSines.tex} %% For ASA and SAA blocks, give a specific triangle to draw with the script. These triangles should distinguish 



%% REMOVED\activity{arccosine/arccosine.tex} %% REQUIRED

%%BADBAD SAS

%\activity{lawOfCosines/lawOfCosines.tex} %% %% SAS ORDER OF SSS VARIBLES
\activity{sunAndShadows/sunAndShadows.tex} %%Added by Claire, solar angle, windows, shadows
\activity{pitch/pitch.tex} %%  DONE
\activity{grade/grade.tex} %%% Added by Claire, ramps and road grade
% FANCY PITCH???  YES!
%% this needs to go after pitch
\activity{hipsAndValleys/hipsAndValleys.tex} %% 

\activity{scalingRoofs/scalingRoofs.tex} %% 
% FANCY PITCH???  YES!
%Find SA pyramid,  FOR SURE!!!

%Archimedies hatbox
%% PAPER AND NAILS. What curve does the right angle draw?



%%%% BELOW

%% BOUNCING BOX QUESTION!!!! (could come after tessellations)

%%%% ABOVE

%\activity{qualityControl/qualityControl.tex} %%  DELETE 
%\activity{dollhouse/dollhouse.tex} %%  DELETE




%%MAYBE PUT AFTER SINE AND COSINE? YES AFTER SINE AND COSINE
\activity{stringArt/stringArt.tex}  %% NEEDS A PICTURE!!!!! LAST QUESTION, NO ANSWER
%%% ACTIVITY ABOUT BEZIER CURVE????????






\partstyle
\activity{sectionHeads/symmetry.tex} %% 
\journalstyle
%\activity{symmetry/symmetry.tex} %% 
%%BADBAD

\activity{symTriangle/symTriangle.tex} %%  DONE
\activity{symSquare/symSquare.tex} %% DONE 
\skipjournalstyle
\activity{symHex/symHex.tex}  %%  DONE
\checkstyle
\activity{symTriangle/checkYourself.tex}  %% 
\journalstyle

%\activity{dihedralSymmetry/dihedralSymmetry.tex}  %% DDONE
\journalstyle
%\activity{stars/stars.tex} %% 

%\activity{substars/substars.tex}%% 

%\activity{lineArtAndSymmetry/lineArtAndSymmetry.tex}  %% WEAK
%note tutorial on bezier curves.... 


%\activity{coloringBook/coloringBook.tex} %% NEEDS WORK!!! MORE SUBGROUPS!!!


\activity{friezePatterns/friezePatterns.tex} %% REQUIRED


\activity{typesOfSymmetry/typesOfSymmetry.tex} %% DONE
%%BADBAD
%% this needs instructions for frieze explorer block, and thought and care about rotations.
%% BAD BAD FREIEZE EXPLORER  Unclear where the rotation is around.

\activity{combinationsOfSym/combinationsOfSym.tex} %%  DONE
\activity{fiveTypesOfSym/fiveTypesOfSym.tex}  %% DONE
\skipjournalstyle
%\activity{manyTypesOfPizza/manyTypesOfPizza.tex} %% DONE
\journalstyle
%\activity{sevenTypesOfFrieze/sevenTypesOfFriezeAU21.tex}  %% DONE
\activity{sevenTypesOfFrieze/sevenTypesOfFrieze.tex}  %% DONE
%\activity{identifyingSymmetry/identifyingSymmetry.tex}  SHOULD BE INFORMED BY QUIZZES



%% sub stars -- use the seq of building a fraction to build subgroups.
%% 1) about what a substar is
%% 2) about why it doesn't contain the other one for sure. Lagrange.
%% 3) 

%% Stars vs reg-n-gon?


%% Platonic solids?
%% Nets
%% Symmetry of scale (sq numbers area to linear,)
%% Golden ratio
%% INFINITE SERIES/connection to numbers.
%% fraciles





%% Numbers multiply and size
%% Llama and tri... scale invariant.


%% \newpage
%% \part{Growth}
%% \newpage
%% \activity{plotter/plotter.tex}





\end{document}
