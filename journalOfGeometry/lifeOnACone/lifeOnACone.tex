\documentclass[nooutcomes,noauthor,handout]{ximera}

%% page layout
\usepackage[in,headings]{fullpage}
\raggedright
\setlength\headheight{13.6pt}


%% fonts
\usepackage{euler}

\usepackage{FiraMono}
\renewcommand\familydefault{\ttdefault} 
\usepackage{mathastext}
\usepackage[htt]{hyphenat}

\usepackage[T1]{fontenc}
\usepackage[scaled=1]{FiraSans}

\usepackage{wedn}
\usepackage[T1]{fontenc}

%% wrap text around scripts
\usepackage{wrapfig}

\tikzset{>=stealth}
%% snap! scripts
\usepackage{scratch3}

\usepackage{adjustbox}

%% journal style
\makeatletter
\newcommand\journalstyle{%
  \def\activitystyle{activity-chapter}
  \def\maketitle{%
    \addtocounter{titlenumber}{1}%
                {\flushleft\small\sffamily\bfseries\@pretitle\par\vspace{-1.5em}}%
                {\flushleft\LARGE\sffamily\bfseries\thetitlenumber\hspace{1em}\@title \par }%
                {\vskip .6em\noindent\textit\theabstract\setcounter{question}{0}\setcounter{sectiontitlenumber}{0}}%
                    \par\vspace{2em}
                    \phantomsection\addcontentsline{toc}{section}{\thetitlenumber\hspace{1em}\textbf{\@title}}%
                     }}
\makeatother



%% thm like environments
\let\question\relax
\let\endquestion\relax

\newtheoremstyle{QuestionStyle}{\topsep}{\topsep}%%% space between body and thm
		{}                      %%% Thm body font
		{}                              %%% Indent amount (empty = no indent)
		{\bfseries}            %%% Thm head font
		{)}                              %%% Punctuation after thm head
		{ }                           %%% Space after thm head
		{\thmnumber{#2}\thmnote{ \bfseries(#3)}}%%% Thm head spec
\theoremstyle{QuestionStyle}
\newtheorem{question}{}



\let\freeResponse\relax
\let\endfreeResponse\relax

%% \newtheoremstyle{ResponseStyle}{\topsep}{\topsep}%%% space between body and thm
%% 		{\wedn\bfseries}                      %%% Thm body font
%% 		{}                              %%% Indent amount (empty = no indent)
%% 		{\wedn\bfseries}            %%% Thm head font
%% 		{}                              %%% Punctuation after thm head
%% 		{3ex}                           %%% Space after thm head
%% 		{\underline{\underline{\thmname{#1}}}}%%% Thm head spec
%% \theoremstyle{ResponseStyle}

\usepackage[tikz]{mdframed}
\mdfdefinestyle{ResponseStyle}{leftmargin=1cm,linecolor=black,roundcorner=5pt,frametitlefont=\wedn\bfseries,%frametitle={\underline{\underline{Response}}:}
, font=\wedn\bfseries,}%\begin{mdframed}[style=mystyle]foo\end{mdframed}


\ifhandout
\NewEnviron{freeResponse}{}
\else
%\newtheorem{freeResponse}{Response:}
\newenvironment{freeResponse}{\begin{mdframed}[style=ResponseStyle]}{\end{mdframed}}
\fi



%% attempting to automate outcomes.

\newwrite\outcomefile
  \immediate\openout\outcomefile=\jobname.oc
\renewcommand{\outcome}[1]{\edef\theoutcomes{\theoutcomes #1~}%
\immediate\write\outcomefile{\unexpanded{\outcome}{#1}}}

%% \newcommand{\outcomelist}{\begin{itemize}\theoutcomes\end{itemize}}



%% my commands

\newcommand{\snap}{{\bfseries\itshape\textsf{Snap!}}}
\newcommand{\flavor}{\link[\snap]{https://snap.berkeley.edu/}}


\usepackage{tkz-euclide}
\tikzstyle geometryDiagrams=[rounded corners=.5pt,ultra thick,color=black]
\colorlet{penColor}{black} % Color of a curve in a plot

\title{Life on a cone}

\author{Herb Clemens \and Jenny Sheldon \and Bart Snapp}

\begin{document}
\begin{abstract}
  Geometry changes when we work on different surfaces.
\end{abstract}
\maketitle


\begin{listOutcomes}
\item Construct a paper cone.
\item Draw a triangle on a cone.
\item Measure angles with a protractor.
\item Work with interior angles.
\item Work with exterior angles.
\item Witness that the ``facts'' of geometry depend on the nature of
  space.
\end{listOutcomes}

We're going to study geometry on a cone. To do this, we need to
make a.

\begin{enumerate}
\item Get a blank sheet of paper.
\item Use a protractor to draw a $50^\circ$ angle whose
  \begin{description}
    \item[\textbf{vertex}] is in the \textbf{center of the paper}, and whose
    \item[\textbf{legs}] extend to the \textbf{edges} of the paper.
  \end{description}
  \item {Lightly shade the interior of the angle.}
  \item {Use scissors to cut the paper along \textbf{one leg} of the
    angle and one leg only.} 
    
    Fold the paper along the line you drew.
  \item Make a cone by moving the new side to the other leg of the
    angle. When you do this, the \textbf{shaded region should be covered}.
\end{enumerate}

Now we'll draw a triangle on the cone that surrounds the vertex of the
cone. To do this,
\begin{enumerate}
\item Squish your cone and draw a line across the seem.
\item Unfold your cone and lay it flat. Label the line that crosses the seam so that you remember it is one line on the cone. 
\item Draw two more sides of the triangle.
\end{enumerate}




\mynewpage



\begin{question}
   You measure angles on your cone by laying the paper flat and
   measuring the angles on the paper.
   \begin{enumerate}
   \item Draw a triangle on your cone so that the vertex is in the
     interior of the triangle. Remember when a line gets to the shaded
     region, fold the paper to \textbf{hide the shaded region} and
     keep drawing. Measure the angles, label them, and take a
     photograph of your triangle on the cone with the angles
     measured. Share the photograph here.
     \item Compute the sum of the interior angles of the triangle.
   \end{enumerate}
   \begin{freeResponse}
     \begin{enumerate}
     \item Here is my PHOTO:
       \begin{center}
         \includegraphics[width=.4\textwidth]{conePhoto.jpg}
       \end{center}
       
     \item The angles are $74.5^\circ$, $65.5^\circ$ and
       $90^\circ$. Hence the interior angles sum to $230^\circ$.
     \end{enumerate}
   \end{freeResponse}
\end{question}

\mynewpage


\begin{question}
  Let's see if we can explain this. Do you know who is eager to help
you? That's right: Louie Llama.
\begin{center}
\includegraphics[height=1in]{llama.pdf}
\end{center}

We would like to parade Louie around the triangle.
\begin{enumerate}
\item Through what angle does Louie rotate as he strolls around the
  entire triangle?
\item How many degrees did the ``cut'' of $50^\circ$ rotate Louie
  Llama?
\end{enumerate}
In each case, EXPLAIN why your answer is correct. 
\begin{freeResponse}
  \begin{enumerate}
  \item Louie Llama again rotates a full $360^\circ$. He starts
    right-side-up and then spins all the way around as he walks,
    exactly once.
  \item The cut rotates Louie Llama $50^\circ$. Unlike the interior
    angles of the triangle, this angle actually shows how Louie Llama
    turns.
  \end{enumerate}
\end{freeResponse}
\end{question}

\mynewpage


\begin{question}
If a cone is made on a sheet of paper with a cut of $\theta$ degrees,
and a triangle is made with angles $\alpha$, $\beta$, and $\gamma$
surrounding the point of the cone, what is the sum of the measures of
the interior angles of this triangle? COMPARE this result with your
answer for Question 1. Explain your reasoning using words, pictures,
numbers, and so on.
\begin{freeResponse}
  Well, walking Louie Llama around the triangle we find:
  \[
  (180-\alpha) + (180-\beta) + (180-\gamma) + \theta = 360.
  \]
  Hence
  \[
  \alpha + \beta + \gamma = 180 + \theta.
  \]
  This makes total sense when compared to Question 1, because my
  answer was $230^\circ = 180+50$.
\end{freeResponse}
\end{question}



\end{document}
