\documentclass[nooutcomes,noauthor,handout]{ximera}

%% page layout
\usepackage[in,headings]{fullpage}
\raggedright
\setlength\headheight{13.6pt}


%% fonts
\usepackage{euler}

\usepackage{FiraMono}
\renewcommand\familydefault{\ttdefault} 
\usepackage{mathastext}
\usepackage[htt]{hyphenat}

\usepackage[T1]{fontenc}
\usepackage[scaled=1]{FiraSans}

\usepackage{wedn}
\usepackage[T1]{fontenc}

%% wrap text around scripts
\usepackage{wrapfig}

\tikzset{>=stealth}
%% snap! scripts
\usepackage{scratch3}

\usepackage{adjustbox}

%% journal style
\makeatletter
\newcommand\journalstyle{%
  \def\activitystyle{activity-chapter}
  \def\maketitle{%
    \addtocounter{titlenumber}{1}%
                {\flushleft\small\sffamily\bfseries\@pretitle\par\vspace{-1.5em}}%
                {\flushleft\LARGE\sffamily\bfseries\thetitlenumber\hspace{1em}\@title \par }%
                {\vskip .6em\noindent\textit\theabstract\setcounter{question}{0}\setcounter{sectiontitlenumber}{0}}%
                    \par\vspace{2em}
                    \phantomsection\addcontentsline{toc}{section}{\thetitlenumber\hspace{1em}\textbf{\@title}}%
                     }}
\makeatother



%% thm like environments
\let\question\relax
\let\endquestion\relax

\newtheoremstyle{QuestionStyle}{\topsep}{\topsep}%%% space between body and thm
		{}                      %%% Thm body font
		{}                              %%% Indent amount (empty = no indent)
		{\bfseries}            %%% Thm head font
		{)}                              %%% Punctuation after thm head
		{ }                           %%% Space after thm head
		{\thmnumber{#2}\thmnote{ \bfseries(#3)}}%%% Thm head spec
\theoremstyle{QuestionStyle}
\newtheorem{question}{}



\let\freeResponse\relax
\let\endfreeResponse\relax

%% \newtheoremstyle{ResponseStyle}{\topsep}{\topsep}%%% space between body and thm
%% 		{\wedn\bfseries}                      %%% Thm body font
%% 		{}                              %%% Indent amount (empty = no indent)
%% 		{\wedn\bfseries}            %%% Thm head font
%% 		{}                              %%% Punctuation after thm head
%% 		{3ex}                           %%% Space after thm head
%% 		{\underline{\underline{\thmname{#1}}}}%%% Thm head spec
%% \theoremstyle{ResponseStyle}

\usepackage[tikz]{mdframed}
\mdfdefinestyle{ResponseStyle}{leftmargin=1cm,linecolor=black,roundcorner=5pt,frametitlefont=\wedn\bfseries,%frametitle={\underline{\underline{Response}}:}
, font=\wedn\bfseries,}%\begin{mdframed}[style=mystyle]foo\end{mdframed}


\ifhandout
\NewEnviron{freeResponse}{}
\else
%\newtheorem{freeResponse}{Response:}
\newenvironment{freeResponse}{\begin{mdframed}[style=ResponseStyle]}{\end{mdframed}}
\fi



%% attempting to automate outcomes.

\newwrite\outcomefile
  \immediate\openout\outcomefile=\jobname.oc
\renewcommand{\outcome}[1]{\edef\theoutcomes{\theoutcomes #1~}%
\immediate\write\outcomefile{\unexpanded{\outcome}{#1}}}

%% \newcommand{\outcomelist}{\begin{itemize}\theoutcomes\end{itemize}}



%% my commands

\newcommand{\snap}{{\bfseries\itshape\textsf{Snap!}}}
\newcommand{\flavor}{\link[\snap]{https://snap.berkeley.edu/}}


\usepackage{tkz-euclide}
\tikzstyle geometryDiagrams=[rounded corners=.5pt,ultra thick,color=black]
\colorlet{penColor}{black} % Color of a curve in a plot

\title{An isoperimetric inequality of your own}

\author{Bart Snapp}

\begin{document}
\begin{abstract}
  We think about the isoperimetric inequality.
\end{abstract}
\maketitle


\begin{listOutcomes}
\item Understand the statement of the isoperimetric inequality.
\item Specialize the isoperimetric inequality to dimensions $2$ and
  $3$.
\item Use basic number sense to find a reasonable estimate for the
  value of an expression.
\end{listOutcomes}

The \textit{isoperimetric inequality} relates an objects
$n$-dimensional ``perimeter'' to its $n$-dimensional ``volume.'' Here
if
\begin{itemize}
\item $V$ is the $n$-dimensional ``volume'' of a shape and 
\item $P$ is the $n$-dimensional ``perimeter'' of a shape,
\end{itemize}
then:
\[
n^n \cdot  (\text{``volume'' of $n$-dimensional unit sphere})\cdot  V^{(n-1)} \le P^n 
\]



\mynewpage


\begin{question}
  Tell me:
  \begin{enumerate}
  \item What do we usually call $2$-dimensional ``volume?''
  \item What do we usually call $2$-dimensional ``perimeter?''
  \item State the isoperimetric inequality in TWO dimensions.
  \item Confirm that equality holds when the shape in question is a
    circle.
  \end{enumerate}
  \begin{freeResponse}
    \begin{enumerate}
    \item $2$-dimensional ``volume'' is usually called \underline{area}.
    \item $2$-dimensional ``perimeter'' is usually called \underline{perimeter}.
    \item In two dimensions, the isoperimetric inequality states
      \[
      4 \pi A \le P^2
      \]
      where $A$ is the area and $P$ is the perimeter.
    \item If the shape is a circle, then $A = \pi r^2$. Write with me:
      \begin{align*}
        4 \pi A &= 4 \pi \cdot \pi r^2 \\
        &= 2^2 \pi^2 r^2.
      \end{align*}
      But the perimeter is $A = 2\pi r$ and
      \[
      P^2 = 2^2 \pi^2 r^2.
      \]
    \end{enumerate}
  \end{freeResponse}
\end{question}
\mynewpage


\begin{question}
  Tell me:
  \begin{enumerate}
  \item What do we usually call $3$-dimensional ``volume?''
  \item What do we usually call $3$-dimensional ``perimeter?''
  \item State the isoperimetric inequality in THREE dimensions.
  \item Confirm that equality holds when the shape in question is a
    sphere.
  \end{enumerate}
  \begin{freeResponse}
    \begin{enumerate}
    \item $3$-dimensional ``volume'' is usually called \underline{volume}.
    \item $3$-dimensional ``perimeter'' is usually called
      \underline{surface area}.
    \item In three dimensions, the isoperimetric inequality states
      \[
      36 \pi V^2 \le A^3
      \]
      where $V$ is the volume and $A$ is the surface area.
    \end{enumerate}
  \item If the shape is a sphere, then $V= \left(\frac{4}{3}\right)\pi
    r^3$. Write with me:
    \begin{align*}
      36 \pi V^2 &= 36 \pi \left(\frac{4}{3}\right)^2\pi^2
      r^5\\
      &= 4^3 \pi^3 r^5.
    \end{align*}
    But the surface area is $A = 4\pi r^2$ and
    \[
    A^3 = 4^3 \pi^3 r^5.
    \]
  \end{freeResponse}
\end{question}
\mynewpage

\begin{question}
  The isoperimetric inequality can be found by assuming that the
  $n$-dimensional sphere IS the shape that for a given $n$-dimensional
  volume, has the least $n$-dimensional perimeter. Here's how you do
  it:
  \begin{enumerate}
    \item Take the given volume and assume it is from the shape of
      least perimeter.
    \item Use formulas to compute the perimeter of the assumed shape.
    \item Any other shape must have greater perimeter.
  \end{enumerate}
  Let's think about this. What happens if we WRONGLY assume that the
  the $n$-dimensional CUBE IS the shape that for a given
  $n$-dimensional volume, has the least $n$-dimensional perimeter?
  \begin{enumerate}
  \item Follow the algorithm above to develop a $2$-dimensional
    CUBICAL isoperimetric inequality.
  \item Follow the algorithm above to develop a $3$-dimensional
    CUBICAL isoperimetric inequality.
  \item Compare contrast the (WRONG) cubical isopermetric inequality
    to the standard isoperimetric inequality.
  \end{enumerate}
  \begin{freeResponse}
    \begin{enumerate}
    \item Suppose I have a square of area $A$. In this case its side length is $\sqrt{A}$. So the perimeter is
      \[
      4\sqrt{A} = P.
      \]
      Thus our $2$-dimensional CUBICAL isopermetric inequality is
      \[
      4A \le P^2.
      \]
    \item Suppose I have a cube of volume $V$. In this case its side
      length is $\sqrt[3]{V}$. So the surface area is
      \[
      6\left(\sqrt[3]{V}\right)^2 = A.
      \]
      Thus our $3$-dimensional CUBICAL isopermetric inequality is
      \[
      216 V^2 \le A^3.
      \]
    \end{enumerate}
  \item In the cubical isopermetric inequality, we are comparing to a
    cube, which is WRONG, but may be good for a rule-of-thumb.
  \end{freeResponse}
\end{question}


\end{document}
