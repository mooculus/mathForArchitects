\documentclass[nooutcomes,noauthor,hints,handout,12pt]{ximera}

\graphicspath{  
{./}
{./whoAreYou/}
{./drawingWithTheTurtle/}
{./bisectionMethod/}
{./circles/}
{./anglesAndRightTriangles/}
{./lawOfSines/}
{./lawOfCosines/}
{./plotter/}
{./staircases/}
{./pitch/}
{./qualityControl/}
{./symmetry/}
{./nGonBlock/}
}


%% page layout
\usepackage[cm,headings]{fullpage}
\raggedright
\setlength\headheight{13.6pt}


%% fonts
\usepackage{euler}

\usepackage{FiraMono}
\renewcommand\familydefault{\ttdefault} 
\usepackage[defaultmathsizes]{mathastext}
\usepackage[htt]{hyphenat}

\usepackage[T1]{fontenc}
\usepackage[scaled=1]{FiraSans}

%\usepackage{wedn}
\usepackage{pbsi} %% Answer font


\usepackage{cancel} %% strike through in pitch/pitch.tex


%% \usepackage{ulem} %% 
%% \renewcommand{\ULthickness}{2pt}% changes underline thickness

\tikzset{>=stealth}

\usepackage{adjustbox}

\setcounter{titlenumber}{-1}

%% journal style
\makeatletter
\newcommand\journalstyle{%
  \def\activitystyle{activity-chapter}
  \def\maketitle{%
    \addtocounter{titlenumber}{1}%
                {\flushleft\small\sffamily\bfseries\@pretitle\par\vspace{-1.5em}}%
                {\flushleft\LARGE\sffamily\bfseries\thetitlenumber\hspace{1em}\@title \par }%
                {\vskip .6em\noindent\textit\theabstract\setcounter{question}{0}\setcounter{sectiontitlenumber}{0}}%
                    \par\vspace{2em}
                    \phantomsection\addcontentsline{toc}{section}{\thetitlenumber\hspace{1em}\textbf{\@title}}%
                     }}
\makeatother



%% thm like environments
\let\question\relax
\let\endquestion\relax

\newtheoremstyle{QuestionStyle}{\topsep}{\topsep}%%% space between body and thm
		{}                      %%% Thm body font
		{}                              %%% Indent amount (empty = no indent)
		{\bfseries}            %%% Thm head font
		{)}                              %%% Punctuation after thm head
		{ }                           %%% Space after thm head
		{\thmnumber{#2}\thmnote{ \bfseries(#3)}}%%% Thm head spec
\theoremstyle{QuestionStyle}
\newtheorem{question}{}



\let\freeResponse\relax
\let\endfreeResponse\relax

%% \newtheoremstyle{ResponseStyle}{\topsep}{\topsep}%%% space between body and thm
%% 		{\wedn\bfseries}                      %%% Thm body font
%% 		{}                              %%% Indent amount (empty = no indent)
%% 		{\wedn\bfseries}            %%% Thm head font
%% 		{}                              %%% Punctuation after thm head
%% 		{3ex}                           %%% Space after thm head
%% 		{\underline{\underline{\thmname{#1}}}}%%% Thm head spec
%% \theoremstyle{ResponseStyle}

\usepackage[tikz]{mdframed}
\mdfdefinestyle{ResponseStyle}{leftmargin=1cm,linecolor=black,roundcorner=5pt,
, font=\bsifamily,}%font=\wedn\bfseries\upshape,}


\ifhandout
\NewEnviron{freeResponse}{}
\else
%\newtheorem{freeResponse}{Response:}
\newenvironment{freeResponse}{\begin{mdframed}[style=ResponseStyle]}{\end{mdframed}}
\fi



%% attempting to automate outcomes.

%% \newwrite\outcomefile
%%   \immediate\openout\outcomefile=\jobname.oc
%% \renewcommand{\outcome}[1]{\edef\theoutcomes{\theoutcomes #1~}%
%% \immediate\write\outcomefile{\unexpanded{\outcome}{#1}}}

%% \newcommand{\outcomelist}{\begin{itemize}\theoutcomes\end{itemize}}

%% \NewEnviron{listOutcomes}{\small\sffamily
%% After answering the following questions, students should be able to:
%% \begin{itemize}
%% \BODY
%% \end{itemize}
%% }
\usepackage[tikz]{mdframed}
\mdfdefinestyle{OutcomeStyle}{leftmargin=2cm,rightmargin=2cm,linecolor=black,roundcorner=5pt,
, font=\small\sffamily,}%font=\wedn\bfseries\upshape,}
\newenvironment{listOutcomes}{\begin{mdframed}[style=OutcomeStyle]After answering the following questions, students should be able to:\begin{itemize}}{\end{itemize}\end{mdframed}}



%% my commands

\newcommand{\snap}{{\bfseries\itshape\textsf{Snap!}}}
\newcommand{\flavor}{\link[\snap]{https://snap.berkeley.edu/}}
\newcommand{\mooculus}{\textsf{\textbf{MOOC}\textnormal{\textsf{ULUS}}}}


\usepackage{tkz-euclide}
\tikzstyle geometryDiagrams=[rounded corners=.5pt,ultra thick,color=black]
\colorlet{penColor}{black} % Color of a curve in a plot



\ifhandout\newcommand{\mynewpage}{\newpage}\else\newcommand{\mynewpage}{}\fi

\title{Simpson's paradox}


%% https://en.wikipedia.org/wiki/Simpson%27s_paradox
%% https://www.amazon.com/Mathematician-Ballpark-Odds-Probabilities-Baseball/dp/0452287820
%% https://aperiodical.com/2013/01/as-maths-results-and-batting-averages/gyz
% https://pubmed.ncbi.nlm.nih.gov/16931545/


%%% https://stats.stackexchange.com/questions/518146/how-to-handle-simpsons-paradox%% \[


\author{Bart Snapp}

\begin{document}
\begin{abstract}
  Let's examine some confusing data.
\end{abstract}
\maketitle

\begin{listOutcomes}
\item Work with real-world numbers.
\item Critically analyze data.
\item Explain patterns in data from a mathematical point of view.
\item Explain patterns in data from a qualitative point of view.
\end{listOutcomes}


A \textbf{paradox} in mathematics is something that \textit{seems
contradictory} but isn't.



\textbf{Simpson's paradox} occurs when a data set seems to
simultaneously \textit{support and refute} a hypothesis, based on how
the data is organized.



Let's investigate some examples of Simpson's paradox. 


\mynewpage



\begin{question}

  \link[Studies on mortality and tobacco
    use]{https://www.valueinhealthjournal.com/article/S1098-3015(15)02902-2/fulltext}
  show that at each age, smokers are more likely to die of
  cardiovascular disease than nonsmokers.  However, when examining
  deaths due to cardiovascular disease at all ages together, smokers
  are less likely to die of cardiovascular disease than nonsmokers.

\begin{enumerate}
\item This is an example of Simpson's paradox. What's the paradox? Explain.
\item How can we reconcile this data?
\item Does smoking reduce your risk of dying from cardiovascular disease? Explain.
\end{enumerate}
\end{question}
\mynewpage




\begin{question}
In Ken Ross's book \textit{A Mathematician at the Ballpark: Odds and Probabilities for Baseball Fans}, he points out the following about batting averages:
\[
\begin{array}{|c||c|c|c|} \hline
 & 1995~\text{Batting Avg.} & 1996~\text{Batting Avg.}& \text{Combined}\\ \hline\hline
\text{Derek Jeter} & 12/48 = 0.250 & 183/582 = 0.314 & 195/630 =0.310\\ \hline
\text{David Justice} & 104/411=0.253 & 45/140=0.321  & 149/551 = 0.270\\ \hline
\end{array}
\]
\begin{enumerate}
\item This is an example of Simpson's paradox. What's the paradox? Explain.
\item How can we reconcile this data?
\item Does having the best average each year ensure you'll be best overall? Explain.
\end{enumerate}
\end{question}
\mynewpage








\begin{question}
In the 1970's a scandal arose at the University of California,
Berkeley. There was a distinct gender bias against women in the number
of graduate students admitted among the 6 largest departments:
\begin{center}
\begin{tabular}{|c||c|c|} \hline
 & Applicants & Admitted \\ \hline\hline
Men & 2590 & 46\% \\ \hline
Women & 1835 & 30\% \\ \hline
\end{tabular}
\end{center}
Data from individual departments was collected for further investigation:
\[
\begin{array}{|c||c|c||c|c|} \hline
\multirow{2}{*}{\text{Department}} & \multicolumn{2}{c||}{\text{Men}} & \multicolumn{2}{c|}{\text{Women}}  \\\cline{2-5}
 & \text{Applicants} & \text{Admitted} & \text{Applicants} & \text{Admitted} \\\hline\hline
A & 825 & 62\% & 108 & 82\%\\ \hline
B & 560 & 63\% & 25 & 68\% \\ \hline
C & 325 & 37\% & 593 & 34\% \\ \hline
D & 417 & 33\% & 375 & 35\% \\ \hline
E & 191 & 28\% & 393 & 24\% \\ \hline
F & 272 & 6\% & 341 & 7\% \\ \hline
\end{array}
\]

\begin{enumerate}
\item This is an example of Simpson's paradox. What's the paradox? Explain.
\item How can we reconcile this data?
\item Does not discriminating in every department ensure there is not discrimination? Explain.
\end{enumerate}

\end{question}



\end{document}
