\documentclass[nooutcomes,noauthor,hints]{ximera}

%% page layout
\usepackage[in,headings]{fullpage}
\raggedright
\setlength\headheight{13.6pt}


%% fonts
\usepackage{euler}

\usepackage{FiraMono}
\renewcommand\familydefault{\ttdefault} 
\usepackage{mathastext}
\usepackage[htt]{hyphenat}

\usepackage[T1]{fontenc}
\usepackage[scaled=1]{FiraSans}

\usepackage{wedn}
\usepackage[T1]{fontenc}

%% wrap text around scripts
\usepackage{wrapfig}

\tikzset{>=stealth}
%% snap! scripts
\usepackage{scratch3}

\usepackage{adjustbox}

%% journal style
\makeatletter
\newcommand\journalstyle{%
  \def\activitystyle{activity-chapter}
  \def\maketitle{%
    \addtocounter{titlenumber}{1}%
                {\flushleft\small\sffamily\bfseries\@pretitle\par\vspace{-1.5em}}%
                {\flushleft\LARGE\sffamily\bfseries\thetitlenumber\hspace{1em}\@title \par }%
                {\vskip .6em\noindent\textit\theabstract\setcounter{question}{0}\setcounter{sectiontitlenumber}{0}}%
                    \par\vspace{2em}
                    \phantomsection\addcontentsline{toc}{section}{\thetitlenumber\hspace{1em}\textbf{\@title}}%
                     }}
\makeatother



%% thm like environments
\let\question\relax
\let\endquestion\relax

\newtheoremstyle{QuestionStyle}{\topsep}{\topsep}%%% space between body and thm
		{}                      %%% Thm body font
		{}                              %%% Indent amount (empty = no indent)
		{\bfseries}            %%% Thm head font
		{)}                              %%% Punctuation after thm head
		{ }                           %%% Space after thm head
		{\thmnumber{#2}\thmnote{ \bfseries(#3)}}%%% Thm head spec
\theoremstyle{QuestionStyle}
\newtheorem{question}{}



\let\freeResponse\relax
\let\endfreeResponse\relax

%% \newtheoremstyle{ResponseStyle}{\topsep}{\topsep}%%% space between body and thm
%% 		{\wedn\bfseries}                      %%% Thm body font
%% 		{}                              %%% Indent amount (empty = no indent)
%% 		{\wedn\bfseries}            %%% Thm head font
%% 		{}                              %%% Punctuation after thm head
%% 		{3ex}                           %%% Space after thm head
%% 		{\underline{\underline{\thmname{#1}}}}%%% Thm head spec
%% \theoremstyle{ResponseStyle}

\usepackage[tikz]{mdframed}
\mdfdefinestyle{ResponseStyle}{leftmargin=1cm,linecolor=black,roundcorner=5pt,frametitlefont=\wedn\bfseries,%frametitle={\underline{\underline{Response}}:}
, font=\wedn\bfseries,}%\begin{mdframed}[style=mystyle]foo\end{mdframed}


\ifhandout
\NewEnviron{freeResponse}{}
\else
%\newtheorem{freeResponse}{Response:}
\newenvironment{freeResponse}{\begin{mdframed}[style=ResponseStyle]}{\end{mdframed}}
\fi



%% attempting to automate outcomes.

\newwrite\outcomefile
  \immediate\openout\outcomefile=\jobname.oc
\renewcommand{\outcome}[1]{\edef\theoutcomes{\theoutcomes #1~}%
\immediate\write\outcomefile{\unexpanded{\outcome}{#1}}}

%% \newcommand{\outcomelist}{\begin{itemize}\theoutcomes\end{itemize}}



%% my commands

\newcommand{\snap}{{\bfseries\itshape\textsf{Snap!}}}
\newcommand{\flavor}{\link[\snap]{https://snap.berkeley.edu/}}


\usepackage{tkz-euclide}
\tikzstyle geometryDiagrams=[rounded corners=.5pt,ultra thick,color=black]
\colorlet{penColor}{black} % Color of a curve in a plot

\title{Simpsons paradox}

\author{Bart Snapp}

\begin{document}
\begin{abstract}
  Let's examine some data.
\end{abstract}
\maketitle

\begin{listOutcomes}
\item Critically analyze data.
\end{listOutcomes}


A \textit{paradox} in mathematics is something that \textbf{seems
contradictory} but isn't.



%% https://en.wikipedia.org/wiki/Simpson%27s_paradox
%% https://www.amazon.com/Mathematician-Ballpark-Odds-Probabilities-Baseball/dp/0452287820
%% https://aperiodical.com/2013/01/as-maths-results-and-batting-averages/gyz
In Ken Ross's book \textit{A Mathematician at the Ballpark: Odds and Probabilities for Baseball Fans}, he points out the following:
\[
\begin{array}{|c||c|c|c|} \hline
 & 1995 & 1996 & \text{Combined}\\ \hline\hline
\text{Derek Jeter} & 12/48 = 0.250 & 183/582 = 0.314 & 195/630 =0.310\\ \hline
\text{David Justice} & 104/411=0.253 & 45/140=0.321  & 149/551 = 0.270\\ \hline
\end{array}
\]

There's a paradox happening here. What is it?
\mynewpage

In the 1970's a scandal arose at the University of California,
Berkeley. There was a distinct gender bias against women in the number
of graduate students admitted among the 6 largest departments:
\begin{center}
\begin{tabular}{|c||c|c|} \hline
 & Applicants & Admitted \\ \hline\hline
Men & 2590 & 46\% \\ \hline
Women & 1835 & 30\% \\ \hline
\end{tabular}
\end{center}
Data from individual departments was collected for further investigation:
\[
\begin{array}{|c||c|c||c|c|} \hline
\multirow{2}{*}{\text{Department}} & \multicolumn{2}{c||}{\text{Men}} & \multicolumn{2}{c|}{\text{Women}}  \\\cline{2-5}
 & \text{Applicants} & \text{Admitted} & \text{Applicants} & \text{Admitted} \\\hline\hline
A & 825 & 62\% & 108 & 82\%\\ \hline
B & 560 & 63\% & 25 & 68\% \\ \hline
C & 325 & 37\% & 593 & 34\% \\ \hline
D & 417 & 33\% & 375 & 35\% \\ \hline
E & 191 & 28\% & 393 & 24\% \\ \hline
F & 272 & 6\% & 341 & 7\% \\ \hline
\end{array}
\]


\mynewpage



\begin{question}
Examine this data critically. What seems to be the case? Does this
jive with your intuition? What is actually happening? Can you explain
when this will happen?
\end{question}
\mynewpage
% https://pubmed.ncbi.nlm.nih.gov/16931545/



\[
\begin{array}{|c||c|c||c|c|} \hline
Birth
weight
(g)
&
Non-smoking
(no. of infants)
& 
Infant
mortality
(per 1,000
livebirths)
&
Population B
(no. of infants)
& Infant
mortality
(per 1,000
livebirths) \\
1,000 &  0 & -- &  40 & 175.0\\
1,500 & 40 & 100.0 & 630 & 72.0\\
2,000 & 630 & 42.0 & 6,230 & 30.2\\
2,500 & 6,230 & 17.6 & 24,100 & 12.7\\
3,000 & 24,100 & 7.4 & 38,000 & 5.3\\
3,500 & 38,000 & 3.1 & 24,100 & 2.2\\
4,000 & 24,100 & 1.3 & 6,230 & 0.9\\
4,500 & 6,230 & 0.6 & 630 & 0.4\\
5,000 & 630 & 0.2 & 40 & 0.2\\
5,500 & 40 & 0.1 & 0 & \\
Total & 100,000 & 4.7 & 100,000 & 8.1\\
\end{array}
\]
\end{document}


\end{document}
