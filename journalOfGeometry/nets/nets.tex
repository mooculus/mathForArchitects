\documentclass[noauthor,nooutcomes,hints,handout]{ximera}

%% page layout
\usepackage[in,headings]{fullpage}
\raggedright
\setlength\headheight{13.6pt}


%% fonts
\usepackage{euler}

\usepackage{FiraMono}
\renewcommand\familydefault{\ttdefault} 
\usepackage{mathastext}
\usepackage[htt]{hyphenat}

\usepackage[T1]{fontenc}
\usepackage[scaled=1]{FiraSans}

\usepackage{wedn}
\usepackage[T1]{fontenc}

%% wrap text around scripts
\usepackage{wrapfig}

\tikzset{>=stealth}
%% snap! scripts
\usepackage{scratch3}

\usepackage{adjustbox}

%% journal style
\makeatletter
\newcommand\journalstyle{%
  \def\activitystyle{activity-chapter}
  \def\maketitle{%
    \addtocounter{titlenumber}{1}%
                {\flushleft\small\sffamily\bfseries\@pretitle\par\vspace{-1.5em}}%
                {\flushleft\LARGE\sffamily\bfseries\thetitlenumber\hspace{1em}\@title \par }%
                {\vskip .6em\noindent\textit\theabstract\setcounter{question}{0}\setcounter{sectiontitlenumber}{0}}%
                    \par\vspace{2em}
                    \phantomsection\addcontentsline{toc}{section}{\thetitlenumber\hspace{1em}\textbf{\@title}}%
                     }}
\makeatother



%% thm like environments
\let\question\relax
\let\endquestion\relax

\newtheoremstyle{QuestionStyle}{\topsep}{\topsep}%%% space between body and thm
		{}                      %%% Thm body font
		{}                              %%% Indent amount (empty = no indent)
		{\bfseries}            %%% Thm head font
		{)}                              %%% Punctuation after thm head
		{ }                           %%% Space after thm head
		{\thmnumber{#2}\thmnote{ \bfseries(#3)}}%%% Thm head spec
\theoremstyle{QuestionStyle}
\newtheorem{question}{}



\let\freeResponse\relax
\let\endfreeResponse\relax

%% \newtheoremstyle{ResponseStyle}{\topsep}{\topsep}%%% space between body and thm
%% 		{\wedn\bfseries}                      %%% Thm body font
%% 		{}                              %%% Indent amount (empty = no indent)
%% 		{\wedn\bfseries}            %%% Thm head font
%% 		{}                              %%% Punctuation after thm head
%% 		{3ex}                           %%% Space after thm head
%% 		{\underline{\underline{\thmname{#1}}}}%%% Thm head spec
%% \theoremstyle{ResponseStyle}

\usepackage[tikz]{mdframed}
\mdfdefinestyle{ResponseStyle}{leftmargin=1cm,linecolor=black,roundcorner=5pt,frametitlefont=\wedn\bfseries,%frametitle={\underline{\underline{Response}}:}
, font=\wedn\bfseries,}%\begin{mdframed}[style=mystyle]foo\end{mdframed}


\ifhandout
\NewEnviron{freeResponse}{}
\else
%\newtheorem{freeResponse}{Response:}
\newenvironment{freeResponse}{\begin{mdframed}[style=ResponseStyle]}{\end{mdframed}}
\fi



%% attempting to automate outcomes.

\newwrite\outcomefile
  \immediate\openout\outcomefile=\jobname.oc
\renewcommand{\outcome}[1]{\edef\theoutcomes{\theoutcomes #1~}%
\immediate\write\outcomefile{\unexpanded{\outcome}{#1}}}

%% \newcommand{\outcomelist}{\begin{itemize}\theoutcomes\end{itemize}}



%% my commands

\newcommand{\snap}{{\bfseries\itshape\textsf{Snap!}}}
\newcommand{\flavor}{\link[\snap]{https://snap.berkeley.edu/}}


\usepackage{tkz-euclide}
\tikzstyle geometryDiagrams=[rounded corners=.5pt,ultra thick,color=black]
\colorlet{penColor}{black} % Color of a curve in a plot


\title{Nothing but nets}
\author{Bart Snapp and Jenny Sheldon}

\begin{document}
\begin{abstract}
  We study solids, and learn facts whose origins are both old and new.
\end{abstract}
\maketitle

\begin{listOutcomes}
\item Draw nets of Platonic solids.
\item Evaluate nets of polyhedra in regards to real-world issues.
\item Discover basic facts about Platonic solids.
\end{listOutcomes}



We are going to think about building polyhedra out
of poster-board and tape. To do this, we're going to draw
\textit{nets} of the various regular convex polyhedra.

\begin{definition}
  A \textbf{net}\index{net of a polyhedron} of a polyhedron is a
  single piece arrangement of polygons that are connected along their
  edges so that they can be folded into the polyhedron.
\end{definition}


\mynewpage


\begin{question}
  Use the INTERNET (or your brain) to find nets for each of the five
  Platonic solids. Display them here.
  
\end{question}
\mynewpage

\begin{question}
  Time to think about nets.
  \begin{enumerate}
  \item Draw all possible nets of regular tetrahedra.
  \item Think about building these nets using poster-board and
    tape. Which nets would be ``best'' to use and why? Think about
    ease of construction, strength of final solid, number of folds,
    number of cuts, amount of tape, and \textbf{at least one more}
    consideration.
  \end{enumerate}
\end{question}
\mynewpage

\begin{question}
  Most of the mathematics you learn in school is very old. The concept
  of Platonic solids is thousands of years old. However, there is
  always new knowledge to be created. As far as I know, this question
  \begin{quote}
    ``Consider a Platonic solid. Given a side length, what's the
    \textbf{average} width of the solid?''
  \end{quote}
  wasn't completely addressed until the late Twentieth Century.
  %% https://mathoverflow.net/questions/306318/average-caliper-diameter-mean-width-of-a-polyhedron
  %% [1] J.W. Cahn, The significance of average mean curvature and its determination by quantitative metallography, Trans. Met. Soc. AIME 239 (1967) 610-616. behind a paywall
  %% [2] R.E. Miles, Poisson flats in Euclidean spaces. Part I: A finite number of random uniform flats, Adv. App. Prob. 1 (1969) 211-237.
  %% [3] R.E. Miles, Direct Derivations of Certain Surface Integral Formulae for the Mean Projections of a Convex Set, Adv. Applied Prob. 7 (1975), 818-829.
  The answer is given by the formula
  \[
  n \cdot l \cdot (180-\theta)/720
  \]
  where $n$ is the number of edges, $l$ is the edge-length, and
  $\theta$ is the interior angle of the regular $n$-gon for the given
  Platonic solid. Suppose you wish to make a model of each of the
  Platonic solids so that their average width is $10$
  centimeters. \textbf{What should the edge length be in each case? Show your
  work.}
\end{question}

\end{document}
