\documentclass[nooutcomes,noauthor,hints]{ximera}

\graphicspath{  
{./}
{./whoAreYou/}
{./drawingWithTheTurtle/}
{./bisectionMethod/}
{./circles/}
{./anglesAndRightTriangles/}
{./lawOfSines/}
{./lawOfCosines/}
{./plotter/}
{./staircases/}
{./pitch/}
{./qualityControl/}
{./symmetry/}
{./nGonBlock/}
}


%% page layout
\usepackage[cm,headings]{fullpage}
\raggedright
\setlength\headheight{13.6pt}


%% fonts
\usepackage{euler}

\usepackage{FiraMono}
\renewcommand\familydefault{\ttdefault} 
\usepackage[defaultmathsizes]{mathastext}
\usepackage[htt]{hyphenat}

\usepackage[T1]{fontenc}
\usepackage[scaled=1]{FiraSans}

%\usepackage{wedn}
\usepackage{pbsi} %% Answer font


\usepackage{cancel} %% strike through in pitch/pitch.tex


%% \usepackage{ulem} %% 
%% \renewcommand{\ULthickness}{2pt}% changes underline thickness

\tikzset{>=stealth}

\usepackage{adjustbox}

\setcounter{titlenumber}{-1}

%% journal style
\makeatletter
\newcommand\journalstyle{%
  \def\activitystyle{activity-chapter}
  \def\maketitle{%
    \addtocounter{titlenumber}{1}%
                {\flushleft\small\sffamily\bfseries\@pretitle\par\vspace{-1.5em}}%
                {\flushleft\LARGE\sffamily\bfseries\thetitlenumber\hspace{1em}\@title \par }%
                {\vskip .6em\noindent\textit\theabstract\setcounter{question}{0}\setcounter{sectiontitlenumber}{0}}%
                    \par\vspace{2em}
                    \phantomsection\addcontentsline{toc}{section}{\thetitlenumber\hspace{1em}\textbf{\@title}}%
                     }}
\makeatother



%% thm like environments
\let\question\relax
\let\endquestion\relax

\newtheoremstyle{QuestionStyle}{\topsep}{\topsep}%%% space between body and thm
		{}                      %%% Thm body font
		{}                              %%% Indent amount (empty = no indent)
		{\bfseries}            %%% Thm head font
		{)}                              %%% Punctuation after thm head
		{ }                           %%% Space after thm head
		{\thmnumber{#2}\thmnote{ \bfseries(#3)}}%%% Thm head spec
\theoremstyle{QuestionStyle}
\newtheorem{question}{}



\let\freeResponse\relax
\let\endfreeResponse\relax

%% \newtheoremstyle{ResponseStyle}{\topsep}{\topsep}%%% space between body and thm
%% 		{\wedn\bfseries}                      %%% Thm body font
%% 		{}                              %%% Indent amount (empty = no indent)
%% 		{\wedn\bfseries}            %%% Thm head font
%% 		{}                              %%% Punctuation after thm head
%% 		{3ex}                           %%% Space after thm head
%% 		{\underline{\underline{\thmname{#1}}}}%%% Thm head spec
%% \theoremstyle{ResponseStyle}

\usepackage[tikz]{mdframed}
\mdfdefinestyle{ResponseStyle}{leftmargin=1cm,linecolor=black,roundcorner=5pt,
, font=\bsifamily,}%font=\wedn\bfseries\upshape,}


\ifhandout
\NewEnviron{freeResponse}{}
\else
%\newtheorem{freeResponse}{Response:}
\newenvironment{freeResponse}{\begin{mdframed}[style=ResponseStyle]}{\end{mdframed}}
\fi



%% attempting to automate outcomes.

%% \newwrite\outcomefile
%%   \immediate\openout\outcomefile=\jobname.oc
%% \renewcommand{\outcome}[1]{\edef\theoutcomes{\theoutcomes #1~}%
%% \immediate\write\outcomefile{\unexpanded{\outcome}{#1}}}

%% \newcommand{\outcomelist}{\begin{itemize}\theoutcomes\end{itemize}}

%% \NewEnviron{listOutcomes}{\small\sffamily
%% After answering the following questions, students should be able to:
%% \begin{itemize}
%% \BODY
%% \end{itemize}
%% }
\usepackage[tikz]{mdframed}
\mdfdefinestyle{OutcomeStyle}{leftmargin=2cm,rightmargin=2cm,linecolor=black,roundcorner=5pt,
, font=\small\sffamily,}%font=\wedn\bfseries\upshape,}
\newenvironment{listOutcomes}{\begin{mdframed}[style=OutcomeStyle]After answering the following questions, students should be able to:\begin{itemize}}{\end{itemize}\end{mdframed}}



%% my commands

\newcommand{\snap}{{\bfseries\itshape\textsf{Snap!}}}
\newcommand{\flavor}{\link[\snap]{https://snap.berkeley.edu/}}
\newcommand{\mooculus}{\textsf{\textbf{MOOC}\textnormal{\textsf{ULUS}}}}


\usepackage{tkz-euclide}
\tikzstyle geometryDiagrams=[rounded corners=.5pt,ultra thick,color=black]
\colorlet{penColor}{black} % Color of a curve in a plot



\ifhandout\newcommand{\mynewpage}{\newpage}\else\newcommand{\mynewpage}{}\fi

\title{The art gallery problem}

\author{Bart Snapp}

\begin{document}
\begin{abstract}
  Triangulate FTW!
\end{abstract}
\maketitle


\begin{listOutcomes}
\item 
\item 
\item 
\end{listOutcomes}

The \textit{art gallery problem} is a classic
mathematics/computer-science problem.  Imagine an art gallery that has
the shape of some polygon. We wish to \textbf{place as few cameras as
  possible so that we may view every point inside the gallery.} We'll
make the following assumptions:
\begin{itemize}
\item Cameras have a $360^\circ$ view, and
\item cameras can view any point, as long as they are not obstructed
    by a wall.
\end{itemize}
Here is our plan to solve this problem:
\begin{itemize}
\item We will argue that any polygon can be triangulated.
\item We will argue that through our process of triangulating, an
  $n$-gon is triangulated without adding any vertices.
\item Each triangle only requires $1$ guard, at one of its $3$
  vertices, so we require no more than $n/3$ cameras.
\end{itemize}

\mynewpage


\begin{question}%http://cgm.cs.mcgill.ca/~godfried/teaching/cg-projects/97/Thierry/thierry507webprj/artgallery.html
  It's a FACT: Every polygon has two ``ears!''
  \begin{enumerate}
    \item On
      \link[Wikipedia]{https://en.wikipedia.org/wiki/Two_ears_theorem}
      it says:
      \begin{quote}
        An \textbf{ear} of a polygon is defined as a vertex $v$ such that the
        line segment between the two neighbors of $v$ lies entirely in
        the interior of the polygon.
      \end{quote}
      DRAW a picture of a polygon that has at least $7$ sides and no
      more than $2$ ears.
    \item Here is an ALGORITHM for triangulating any polygon.
    \item Now imagine an $n$-gon where $n> 4$. INSTEAD OF DOING INDUCTION TRY TO MAKE IT AN ALGORITHM.
    \item If it has an ear, remove it.
    \item If there are no ears...
      \item Triangulate and then worst case. 
  \end{enumerate}
  
  
\end{question}

\mynewpage


\begin{question} POLYGON CAMERAS
\end{question}

\mynewpage


\begin{question} WORST CASE // GIVE IT A TRY
\end{question}








\end{document}
