\documentclass[noauthor,nooutcomes,hints,handout]{ximera}

%% page layout
\usepackage[in,headings]{fullpage}
\raggedright
\setlength\headheight{13.6pt}


%% fonts
\usepackage{euler}

\usepackage{FiraMono}
\renewcommand\familydefault{\ttdefault} 
\usepackage{mathastext}
\usepackage[htt]{hyphenat}

\usepackage[T1]{fontenc}
\usepackage[scaled=1]{FiraSans}

\usepackage{wedn}
\usepackage[T1]{fontenc}

%% wrap text around scripts
\usepackage{wrapfig}

\tikzset{>=stealth}
%% snap! scripts
\usepackage{scratch3}

\usepackage{adjustbox}

%% journal style
\makeatletter
\newcommand\journalstyle{%
  \def\activitystyle{activity-chapter}
  \def\maketitle{%
    \addtocounter{titlenumber}{1}%
                {\flushleft\small\sffamily\bfseries\@pretitle\par\vspace{-1.5em}}%
                {\flushleft\LARGE\sffamily\bfseries\thetitlenumber\hspace{1em}\@title \par }%
                {\vskip .6em\noindent\textit\theabstract\setcounter{question}{0}\setcounter{sectiontitlenumber}{0}}%
                    \par\vspace{2em}
                    \phantomsection\addcontentsline{toc}{section}{\thetitlenumber\hspace{1em}\textbf{\@title}}%
                     }}
\makeatother



%% thm like environments
\let\question\relax
\let\endquestion\relax

\newtheoremstyle{QuestionStyle}{\topsep}{\topsep}%%% space between body and thm
		{}                      %%% Thm body font
		{}                              %%% Indent amount (empty = no indent)
		{\bfseries}            %%% Thm head font
		{)}                              %%% Punctuation after thm head
		{ }                           %%% Space after thm head
		{\thmnumber{#2}\thmnote{ \bfseries(#3)}}%%% Thm head spec
\theoremstyle{QuestionStyle}
\newtheorem{question}{}



\let\freeResponse\relax
\let\endfreeResponse\relax

%% \newtheoremstyle{ResponseStyle}{\topsep}{\topsep}%%% space between body and thm
%% 		{\wedn\bfseries}                      %%% Thm body font
%% 		{}                              %%% Indent amount (empty = no indent)
%% 		{\wedn\bfseries}            %%% Thm head font
%% 		{}                              %%% Punctuation after thm head
%% 		{3ex}                           %%% Space after thm head
%% 		{\underline{\underline{\thmname{#1}}}}%%% Thm head spec
%% \theoremstyle{ResponseStyle}

\usepackage[tikz]{mdframed}
\mdfdefinestyle{ResponseStyle}{leftmargin=1cm,linecolor=black,roundcorner=5pt,frametitlefont=\wedn\bfseries,%frametitle={\underline{\underline{Response}}:}
, font=\wedn\bfseries,}%\begin{mdframed}[style=mystyle]foo\end{mdframed}


\ifhandout
\NewEnviron{freeResponse}{}
\else
%\newtheorem{freeResponse}{Response:}
\newenvironment{freeResponse}{\begin{mdframed}[style=ResponseStyle]}{\end{mdframed}}
\fi



%% attempting to automate outcomes.

\newwrite\outcomefile
  \immediate\openout\outcomefile=\jobname.oc
\renewcommand{\outcome}[1]{\edef\theoutcomes{\theoutcomes #1~}%
\immediate\write\outcomefile{\unexpanded{\outcome}{#1}}}

%% \newcommand{\outcomelist}{\begin{itemize}\theoutcomes\end{itemize}}



%% my commands

\newcommand{\snap}{{\bfseries\itshape\textsf{Snap!}}}
\newcommand{\flavor}{\link[\snap]{https://snap.berkeley.edu/}}


\usepackage{tkz-euclide}
\tikzstyle geometryDiagrams=[rounded corners=.5pt,ultra thick,color=black]
\colorlet{penColor}{black} % Color of a curve in a plot


\title{Grade}
\author{Claire Merriman}

\begin{document}
\begin{abstract}
  We think about pitch, slope, and grade.
\end{abstract}
\maketitle

\begin{listOutcomes}
\item Distinguish between pitch and slope.
\item Solve a problem where slope is given.
\item Use slope to determine road grade.
\item Use road grade to determine slope.
\end{listOutcomes}
\mynewpage

\begin{question}
  Roofs aren't the only things measured with pitch. Ramps are also measure using pitch. 
  
  The ADA says that ramps should have a slope $1:12$, meaning rise $1 \ foot$ for every $12\ feet$ in horizontal length. Also, ramps should be no more that $30\ feet$ long without a landing. 
  
  
\begin{enumerate}
 \item What is the vertical height for a $30\ foot$ long ramp? Explain your reasoning. 
 \item What is the angle between the ramp and the ground?
\end{enumerate}

\end{question}

\mynewpage

\begin{question}
The entrance to the Math Tower from 18th Avenue is $7\ stairs$ tall. Each stair is $13\ inches$ long and $5.5\ inches$ tall, with a $52\ inch$ landing. If OSU wanted to add a ramp at this entrance, how long would it need to be?

  \end{question}
\mynewpage

\begin{question}
Roads are typically measured with \textbf{grade}. Grade is simply
turning the slope into a percentage; that is, a slope with a
denominator of $100$.  Here are two road grade warning signs. Each
give a grade and a mileage.
\begin{center}
\includegraphics[width=.4\textwidth]{gradeWarning}
\includegraphics[width=.3\textwidth]{truckGradeWarning}
\end{center}
In each case, find the change in elevation over the warning area. As a reminder, $1\ mile=5280\ ft.$
\end{question}

\end{document}
