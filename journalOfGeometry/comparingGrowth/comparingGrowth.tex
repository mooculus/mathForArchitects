\documentclass[nooutcomes,noauthor,hints]{ximera}

%% page layout
\usepackage[in,headings]{fullpage}
\raggedright
\setlength\headheight{13.6pt}


%% fonts
\usepackage{euler}

\usepackage{FiraMono}
\renewcommand\familydefault{\ttdefault} 
\usepackage{mathastext}
\usepackage[htt]{hyphenat}

\usepackage[T1]{fontenc}
\usepackage[scaled=1]{FiraSans}

\usepackage{wedn}
\usepackage[T1]{fontenc}

%% wrap text around scripts
\usepackage{wrapfig}

\tikzset{>=stealth}
%% snap! scripts
\usepackage{scratch3}

\usepackage{adjustbox}

%% journal style
\makeatletter
\newcommand\journalstyle{%
  \def\activitystyle{activity-chapter}
  \def\maketitle{%
    \addtocounter{titlenumber}{1}%
                {\flushleft\small\sffamily\bfseries\@pretitle\par\vspace{-1.5em}}%
                {\flushleft\LARGE\sffamily\bfseries\thetitlenumber\hspace{1em}\@title \par }%
                {\vskip .6em\noindent\textit\theabstract\setcounter{question}{0}\setcounter{sectiontitlenumber}{0}}%
                    \par\vspace{2em}
                    \phantomsection\addcontentsline{toc}{section}{\thetitlenumber\hspace{1em}\textbf{\@title}}%
                     }}
\makeatother



%% thm like environments
\let\question\relax
\let\endquestion\relax

\newtheoremstyle{QuestionStyle}{\topsep}{\topsep}%%% space between body and thm
		{}                      %%% Thm body font
		{}                              %%% Indent amount (empty = no indent)
		{\bfseries}            %%% Thm head font
		{)}                              %%% Punctuation after thm head
		{ }                           %%% Space after thm head
		{\thmnumber{#2}\thmnote{ \bfseries(#3)}}%%% Thm head spec
\theoremstyle{QuestionStyle}
\newtheorem{question}{}



\let\freeResponse\relax
\let\endfreeResponse\relax

%% \newtheoremstyle{ResponseStyle}{\topsep}{\topsep}%%% space between body and thm
%% 		{\wedn\bfseries}                      %%% Thm body font
%% 		{}                              %%% Indent amount (empty = no indent)
%% 		{\wedn\bfseries}            %%% Thm head font
%% 		{}                              %%% Punctuation after thm head
%% 		{3ex}                           %%% Space after thm head
%% 		{\underline{\underline{\thmname{#1}}}}%%% Thm head spec
%% \theoremstyle{ResponseStyle}

\usepackage[tikz]{mdframed}
\mdfdefinestyle{ResponseStyle}{leftmargin=1cm,linecolor=black,roundcorner=5pt,frametitlefont=\wedn\bfseries,%frametitle={\underline{\underline{Response}}:}
, font=\wedn\bfseries,}%\begin{mdframed}[style=mystyle]foo\end{mdframed}


\ifhandout
\NewEnviron{freeResponse}{}
\else
%\newtheorem{freeResponse}{Response:}
\newenvironment{freeResponse}{\begin{mdframed}[style=ResponseStyle]}{\end{mdframed}}
\fi



%% attempting to automate outcomes.

\newwrite\outcomefile
  \immediate\openout\outcomefile=\jobname.oc
\renewcommand{\outcome}[1]{\edef\theoutcomes{\theoutcomes #1~}%
\immediate\write\outcomefile{\unexpanded{\outcome}{#1}}}

%% \newcommand{\outcomelist}{\begin{itemize}\theoutcomes\end{itemize}}



%% my commands

\newcommand{\snap}{{\bfseries\itshape\textsf{Snap!}}}
\newcommand{\flavor}{\link[\snap]{https://snap.berkeley.edu/}}


\usepackage{tkz-euclide}
\tikzstyle geometryDiagrams=[rounded corners=.5pt,ultra thick,color=black]
\colorlet{penColor}{black} % Color of a curve in a plot

\title{Comparing growth}

\author{Bart Snapp}

\begin{document}
\begin{abstract}
  Now we'll compare linear and exponential growth.
\end{abstract}
\maketitle

\begin{listOutcomes}
\item 
\end{listOutcomes}


Again, literally repeating ourselves for probably a third time: It is
difficult to grasp the size of large numbers without a context that
seems real to us. We'll use the context of ``cold, hard, cash'' and
the ``cruel curse of time'' to help us understand the magnitude of
numbers.  Dream this daydream with me:
\begin{enumerate}
\item[DAYDREAM 1]
\begin{quote}
   Suppose you were paid the secondly wage of $\$1$, meaning you are
  paid a dollar \text{every second} of every minute, of every hour, of
  every day, and so on.
\end{quote}
\item[DAYDREAM 2]
  \begin{quote}
     Suppose you were paid the following way: Just a penny on your first
  day, double that (two pennies) the next, double that (four pennies)
  the next, and so on, each day doubling the payment amount.
  \end{quote}
\end{enumerate}




Let's think about this seemingly impossible situation.



\mynewpage


\begin{question}
  Assume you are collecting money through the impossible situation
  above.
  \begin{enumerate}
  \item How much money would you collect in the first week? Show your
    work.
  \item \textit{Geometry Giorgio} suggest that in two weeks you would
    collect
    \[
    (\text{amount from first week}) + 2\cdot (\text{amount from first
      week})
    \]
    Is he correct? If he is correct, explain why. If he is not
    correct, explain why and give a correct analysis.
  \end{enumerate}
  \begin{freeResponse}
    \begin{enumerate}
    \item In one week, I'd collect
      \[
      0.01+0.02+0.04+0.08+0.16+0.32+0.64 = 1.27
      \]
      dollars.
    \end{enumerate}
  \end{freeResponse}
\end{question}
\mynewpage




%% TABLE
%% formula?
%% dfn of exponential growth.



\begin{question}
  One way to conceptualize data is with a table. Fill in the table
  below:
\[
\renewcommand{\arraystretch}{3}
\begin{array}{|c||c|c|c|c|c|c|c|c|c|c|}
  \hline
  \text{Day}       & 1 & 2 & 3 & 4 & 5 & 6 & 7 & 8 & 9 & 10 \\ \hline\hline
  \text{Collected} & \phantom{1023} & \phantom{1023}  & \phantom{1023}  & \phantom{1023}  & \phantom{1023}  & \phantom{1023}  & \phantom{1023}  & \phantom{1023}  & \phantom{1023}  & \phantom{1023}   \\ \hline
  \text{Total}     &   &   &   &   &   &   &   &   &   &    \\ \hline
\end{array}
\]
\begin{freeResponse}
  Here is my table:
  \[
\renewcommand{\arraystretch}{3}
\begin{array}{|c||c|c|c|c|c|c|c|c|c|c|}
  \hline
  \text{Day}       & 1  & 2  & 3  & 4 & 5 & 6 & 7 & 8 & 9 & 10 \\ \hline\hline
  \text{Collected} & 0.01  & 0.02  & 0.04  & 0.08 & 0.16& 0.32& 0.64& 1.28 & 2.56  & 5.12   \\ \hline
  \text{Total}     & 0.01  & 0.03  & 0.07  & 0.15  & 0.31  & 0.63  & 1.27  & 2.55  & 5.11  & 10.23   \\ \hline
\end{array}
\]
\end{freeResponse}

  
\end{question}
\mynewpage


\begin{question}
  Now we need to REFLECT on our work.
  \begin{enumerate}
    \item Use your table above to guess a for the total amount
      collected from day $1$ to day $n$. Explain how you found your
      guess USING THE TABLE.
    \item Use ALGEBRAIC REASONING to EXPLAIN WHY your formula is true.
  \end{enumerate}
  \begin{freeResponse}
    \begin{enumerate}
      \item From the table, it looks like the amount collected from
        day $1$ to day $n$ is equal to how much you'll collect on the
        next day minus one. Since on day $n$ we collect $.01\cdot
        2^n$, we guess:
        \[
        \mathrm{Total}(n) = 0.01\cdot 2^{n+1} -0.01
        \]
      \item To see why this is true with algebra, write
        \begin{align*}
        \mathrm{Total}(n) &= 0.01 +0.01\cdot 2+0.01\cdot 2^2 +0.01 \cdot 2^3 + \dots + 0.01 \cdot 2^n\\
        &= 0.01 + 2 ( 0.01 +0.01\cdot 2+0.01\cdot 2^2 +0.01 \cdot 2^3 + \dots + 0.01 \cdot 2^{n-1})\\
        &= 0.01 + 2 ( \mathrm{Total}(n) - 0.01\cdot 2^n)\\
        &= 0.01 + 2\cdot\mathrm{Total}(n) - 0.01\cdot 2^{n+1}
        \end{align*}
    Now we have:
    \[
    \mathrm{Total}(n)= 0.01 + 2\cdot\mathrm{Total}(n) - 0.01\cdot 2^{n+1}
    \]
    Solving for $\mathrm{Total}(n)$ we find
    \[
    \mathrm{Total}(n) = 0.01\cdot 2^{n+1} -0.01.
    \]
    \end{enumerate}
  \end{freeResponse}
\end{question}



\end{document}
