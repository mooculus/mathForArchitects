\documentclass[handout,nooutcomes,noauthor]{ximera}

%% page layout
\usepackage[in,headings]{fullpage}
\raggedright
\setlength\headheight{13.6pt}


%% fonts
\usepackage{euler}

\usepackage{FiraMono}
\renewcommand\familydefault{\ttdefault} 
\usepackage{mathastext}
\usepackage[htt]{hyphenat}

\usepackage[T1]{fontenc}
\usepackage[scaled=1]{FiraSans}

\usepackage{wedn}
\usepackage[T1]{fontenc}

%% wrap text around scripts
\usepackage{wrapfig}

\tikzset{>=stealth}
%% snap! scripts
\usepackage{scratch3}

\usepackage{adjustbox}

%% journal style
\makeatletter
\newcommand\journalstyle{%
  \def\activitystyle{activity-chapter}
  \def\maketitle{%
    \addtocounter{titlenumber}{1}%
                {\flushleft\small\sffamily\bfseries\@pretitle\par\vspace{-1.5em}}%
                {\flushleft\LARGE\sffamily\bfseries\thetitlenumber\hspace{1em}\@title \par }%
                {\vskip .6em\noindent\textit\theabstract\setcounter{question}{0}\setcounter{sectiontitlenumber}{0}}%
                    \par\vspace{2em}
                    \phantomsection\addcontentsline{toc}{section}{\thetitlenumber\hspace{1em}\textbf{\@title}}%
                     }}
\makeatother



%% thm like environments
\let\question\relax
\let\endquestion\relax

\newtheoremstyle{QuestionStyle}{\topsep}{\topsep}%%% space between body and thm
		{}                      %%% Thm body font
		{}                              %%% Indent amount (empty = no indent)
		{\bfseries}            %%% Thm head font
		{)}                              %%% Punctuation after thm head
		{ }                           %%% Space after thm head
		{\thmnumber{#2}\thmnote{ \bfseries(#3)}}%%% Thm head spec
\theoremstyle{QuestionStyle}
\newtheorem{question}{}



\let\freeResponse\relax
\let\endfreeResponse\relax

%% \newtheoremstyle{ResponseStyle}{\topsep}{\topsep}%%% space between body and thm
%% 		{\wedn\bfseries}                      %%% Thm body font
%% 		{}                              %%% Indent amount (empty = no indent)
%% 		{\wedn\bfseries}            %%% Thm head font
%% 		{}                              %%% Punctuation after thm head
%% 		{3ex}                           %%% Space after thm head
%% 		{\underline{\underline{\thmname{#1}}}}%%% Thm head spec
%% \theoremstyle{ResponseStyle}

\usepackage[tikz]{mdframed}
\mdfdefinestyle{ResponseStyle}{leftmargin=1cm,linecolor=black,roundcorner=5pt,frametitlefont=\wedn\bfseries,%frametitle={\underline{\underline{Response}}:}
, font=\wedn\bfseries,}%\begin{mdframed}[style=mystyle]foo\end{mdframed}


\ifhandout
\NewEnviron{freeResponse}{}
\else
%\newtheorem{freeResponse}{Response:}
\newenvironment{freeResponse}{\begin{mdframed}[style=ResponseStyle]}{\end{mdframed}}
\fi



%% attempting to automate outcomes.

\newwrite\outcomefile
  \immediate\openout\outcomefile=\jobname.oc
\renewcommand{\outcome}[1]{\edef\theoutcomes{\theoutcomes #1~}%
\immediate\write\outcomefile{\unexpanded{\outcome}{#1}}}

%% \newcommand{\outcomelist}{\begin{itemize}\theoutcomes\end{itemize}}



%% my commands

\newcommand{\snap}{{\bfseries\itshape\textsf{Snap!}}}
\newcommand{\flavor}{\link[\snap]{https://snap.berkeley.edu/}}


\usepackage{tkz-euclide}
\tikzstyle geometryDiagrams=[rounded corners=.5pt,ultra thick,color=black]
\colorlet{penColor}{black} % Color of a curve in a plot

\title{Rep-tiles and area}

\author{Bart Snapp}

\begin{document}
\begin{abstract}
  We will investigate more interesting rep-tiles.
\end{abstract}
\maketitle


\begin{listOutcomes}
\item Demonstrate given shapes are rep-n-tiles.
\item Understand a relationship between scaling, perimeter, and area.
\end{listOutcomes}
\mynewpage

\begin{question}
Consider the following polygons:
\[
\includegraphics[width=.2\textwidth]{rhomb.png} \qquad \includegraphics[width=.15\textwidth]{P.png}
\]
\[
\includegraphics[width=.2\textwidth]{wedge.png} \qquad \includegraphics[width=.2\textwidth]{sphinx.png}
\]
Show that each of these polygons is a rep-4-tile.
\end{question}

\mynewpage


\begin{question}
  Show that a $(1,2,\sqrt{5})$ right triangle is a rep-$5$-tile.
  \begin{hint}
    I suggest you rotate $90^\circ$.
  \end{hint}
\end{question}

\mynewpage

\begin{question}
  Now we'll connect rep-tiles to SCALING area.
  \begin{enumerate}
    \item EXPLAIN why given ANY rep-2-tile,
      \[
      \frac{\text{new perimeter}}{\text{old perimeter}}\qquad\text{and}\qquad \frac{\text{new area}}{\text{old area}}
      \]
      will be constants that are invariant, regardless of the shape of
      the given rep-2-tile.
    \item EXPLAIN why given ANY rep-$n$-tile,
      \[
      \frac{\text{new perimeter}}{\text{old perimeter}}\qquad\text{and}\qquad \frac{\text{new area}}{\text{old area}}
      \]
      will be constants that are invariant, regardless of the shape of
      the given rep-$n$-tile.
    \item
      Start make a table like the one below:
      \begin{center}
        \begin{tabular}{c|c|c}
          rep-$n$-tile & $\frac{\text{new perimeter}}{\text{old perimeter}}$ & $\frac{\text{new area}}{\text{old area}}$  \\
          \hline\hline
          2 & $\vdots$  &  $\vdots$  \\ 
          3 & $\vdots$  &  $\vdots$  \\ 
        \end{tabular}
      \end{center}
      COMPLETE the table up to $n= 10$.
  \end{enumerate}
\end{question}



\end{document}
