\documentclass[handout,nooutcomes,noauthor,hints]{ximera}

%% page layout
\usepackage[in,headings]{fullpage}
\raggedright
\setlength\headheight{13.6pt}


%% fonts
\usepackage{euler}

\usepackage{FiraMono}
\renewcommand\familydefault{\ttdefault} 
\usepackage{mathastext}
\usepackage[htt]{hyphenat}

\usepackage[T1]{fontenc}
\usepackage[scaled=1]{FiraSans}

\usepackage{wedn}
\usepackage[T1]{fontenc}

%% wrap text around scripts
\usepackage{wrapfig}

\tikzset{>=stealth}
%% snap! scripts
\usepackage{scratch3}

\usepackage{adjustbox}

%% journal style
\makeatletter
\newcommand\journalstyle{%
  \def\activitystyle{activity-chapter}
  \def\maketitle{%
    \addtocounter{titlenumber}{1}%
                {\flushleft\small\sffamily\bfseries\@pretitle\par\vspace{-1.5em}}%
                {\flushleft\LARGE\sffamily\bfseries\thetitlenumber\hspace{1em}\@title \par }%
                {\vskip .6em\noindent\textit\theabstract\setcounter{question}{0}\setcounter{sectiontitlenumber}{0}}%
                    \par\vspace{2em}
                    \phantomsection\addcontentsline{toc}{section}{\thetitlenumber\hspace{1em}\textbf{\@title}}%
                     }}
\makeatother



%% thm like environments
\let\question\relax
\let\endquestion\relax

\newtheoremstyle{QuestionStyle}{\topsep}{\topsep}%%% space between body and thm
		{}                      %%% Thm body font
		{}                              %%% Indent amount (empty = no indent)
		{\bfseries}            %%% Thm head font
		{)}                              %%% Punctuation after thm head
		{ }                           %%% Space after thm head
		{\thmnumber{#2}\thmnote{ \bfseries(#3)}}%%% Thm head spec
\theoremstyle{QuestionStyle}
\newtheorem{question}{}



\let\freeResponse\relax
\let\endfreeResponse\relax

%% \newtheoremstyle{ResponseStyle}{\topsep}{\topsep}%%% space between body and thm
%% 		{\wedn\bfseries}                      %%% Thm body font
%% 		{}                              %%% Indent amount (empty = no indent)
%% 		{\wedn\bfseries}            %%% Thm head font
%% 		{}                              %%% Punctuation after thm head
%% 		{3ex}                           %%% Space after thm head
%% 		{\underline{\underline{\thmname{#1}}}}%%% Thm head spec
%% \theoremstyle{ResponseStyle}

\usepackage[tikz]{mdframed}
\mdfdefinestyle{ResponseStyle}{leftmargin=1cm,linecolor=black,roundcorner=5pt,frametitlefont=\wedn\bfseries,%frametitle={\underline{\underline{Response}}:}
, font=\wedn\bfseries,}%\begin{mdframed}[style=mystyle]foo\end{mdframed}


\ifhandout
\NewEnviron{freeResponse}{}
\else
%\newtheorem{freeResponse}{Response:}
\newenvironment{freeResponse}{\begin{mdframed}[style=ResponseStyle]}{\end{mdframed}}
\fi



%% attempting to automate outcomes.

\newwrite\outcomefile
  \immediate\openout\outcomefile=\jobname.oc
\renewcommand{\outcome}[1]{\edef\theoutcomes{\theoutcomes #1~}%
\immediate\write\outcomefile{\unexpanded{\outcome}{#1}}}

%% \newcommand{\outcomelist}{\begin{itemize}\theoutcomes\end{itemize}}



%% my commands

\newcommand{\snap}{{\bfseries\itshape\textsf{Snap!}}}
\newcommand{\flavor}{\link[\snap]{https://snap.berkeley.edu/}}


\usepackage{tkz-euclide}
\tikzstyle geometryDiagrams=[rounded corners=.5pt,ultra thick,color=black]
\colorlet{penColor}{black} % Color of a curve in a plot

\title{What's your angle}

\author{Bart Snapp}

\begin{document}
\begin{abstract}
  Let's put our knowledge of angles to work. 
\end{abstract}
\maketitle


\begin{listOutcomes}
\item Apply basic facts to solve for angles.
\item Apply triangle congruence theorems.
\item 
\end{listOutcomes}

\mynewpage



\begin{question}
  In the diagram below, we see the intersection of $\bar{BC}$ and
  $\bar{AE}$. Suppose that $\left\vert BD\right\vert =\left\vert
  CD\right\vert $ and $\left\vert AD\right\vert =\left\vert
  ED\right\vert $.
  \begin{center}
    \begin{tikzpicture}[geometryDiagrams]
      \coordinate (A) at (0,2);
      \coordinate (B) at (2,5);
      \coordinate (C) at (6.5,.5);
      \coordinate (E) at (8,4);
      \coordinate (D) at (4,3);
      \draw (A)--(B)--(C)--(E)--(D)--cycle;
      \tkzMarkSegments[mark=|](B,D D,C)
      \tkzMarkSegments[mark=||](A,D D,E)
      \tkzLabelPoints[above](B,D,E)
      \tkzLabelPoints[below](A,C)
      %\draw[step=.5cm] (0,0) grid (10,5);
    \end{tikzpicture}
  \end{center}
  Show that triangle $\triangle BDA$ and triangle $\triangle CDE$ are
  congruent.
  
  \begin{hint}
    First you should explain why $\angle BDA = \angle CDE$.
  \end{hint}
  \begin{freeResponse}
    To start, we claim that $\angle BDA = \angle CDE$. Labeling our
    diagram above,
    \begin{center}
      \begin{tikzpicture}[geometryDiagrams]
        \coordinate (A) at (0,2);
        \coordinate (B) at (2,5);
        \coordinate (C) at (6.5,.5);
        \coordinate (E) at (8,4);
        \coordinate (D) at (4,3);
        \draw (A)--(B)--(C)--(E)--(D)--cycle;
        
        \tkzMarkAngle[size=0.7cm,thin](B,D,A)
        \tkzLabelAngle[pos = -0.4](B,D,A){$\alpha$}
        
        \tkzMarkAngle[arc=ll,size=0.5cm,thin](E,D,B)
        \tkzLabelAngle[pos = 0.25](E,D,B){$\beta$}
        
        \tkzMarkAngle[size=0.7cm,thin](C,D,E)
        \tkzLabelAngle[pos = 0.4](C,D,E){$\gamma$}
        
        \tkzMarkAngle[arc=ll,size=0.5cm,thin](A,D,C)
        \tkzLabelAngle[pos = 0.25](A,D,C){$\delta$}
        
        %\draw[step=.5cm] (0,0) grid (10,5);
      \end{tikzpicture}
    \end{center}
    we see that 
    \begin{align*}
      \alpha+\beta &= 180^\circ\\
      \beta + \gamma &= 180^\circ.
\end{align*}
Subtracting the equations above we find that $\alpha=\gamma = 0$.
This means that $\alpha = \gamma$ and hence $\angle BDA = \angle
CDE$. Since we know that $\left\vert BD\right\vert =\left\vert
CD\right\vert $ and $\left\vert AD\right\vert =\left\vert
ED\right\vert $ we may now apply SAS to prove that triangle $\triangle
BDA$ and triangle $\triangle CDE$ are congruent.
\end{freeResponse}
\end{question}

\mynewpage


\begin{question}
  Consider the crazy shape below:
  \begin{center}
    \begin{tikzpicture}[geometryDiagrams]
      \coordinate (A) at (0,4.5);
      \coordinate (B) at (2.5,2.5);
      \coordinate (C) at (0,.5);
      \coordinate (D) at (4,2.5);
      \coordinate (E) at (2.5,0);
      \coordinate (F) at (10,2.5);
      \coordinate (G) at (9.5,5);
      \coordinate (H) at (7,2);
      \coordinate (I) at (3.5,5);

      \draw (A)--(B)--(C)--(D)--(E)--(F)--(G)--(H)--(I)--cycle;
      
      %% \tkzMarkAngle[size=0.7cm,thin](B,D,A)
      \tkzLabelAngle[pos = -0.5](I,A,B){$\alpha$}

      \tkzLabelAngle[pos = -0.2](A,B,C){$\beta$}

      \tkzLabelAngle[pos = -1.9](B,C,D){$\gamma$}
      
      \tkzLabelAngle[pos = -0.2](C,D,E){$\delta$}

      \tkzLabelAngle[pos = -0.5](D,E,F){$\varepsilon$}

      \tkzLabelAngle[pos = -0.3](E,F,G){$\zeta$}

      \tkzLabelAngle[pos = -0.5](F,G,H){$\eta$}

      \tkzLabelAngle[pos = -0.2](G,H,I){$\theta$}

      \tkzLabelAngle[pos = -0.3](H,I,A){$\iota$}

      %\draw[step=.5cm,thin,gray] (0,0) grid (10,5);
      \end{tikzpicture}
  \end{center}
  \begin{enumerate}
  \item Using a protractor, measure the interior angles of the shape
    above and use this table to record your findings:
  \[
    {\renewcommand{\arraystretch}{1.5}
      \begin{array}{|c|c|c|c|c|c|c|c|c||c|}\hline
        \alpha & \beta & \gamma & \delta & \varepsilon & \zeta & \eta & \theta & \iota & SUM\\\hline\hline
        \rule[7mm]{10mm}{0mm}  & \rule[7mm]{10mm}{0mm}    & \rule[7mm]{10mm}{0mm}   & \rule[7mm]{10mm}{0mm}   &  \rule[7mm]{10mm}{0mm}   & \rule[7mm]{10mm}{0mm}    & \rule[7mm]{10mm}{0mm}   & \rule[7mm]{10mm}{0mm}   & \rule[7mm]{10mm}{0mm} & \rule[7mm]{10mm}{0mm} \\ \hline
    \end{array}}
    \]
\end{question}

\mynewpage


\begin{question}

  POLYGON TRIANGULATION? 

  MORE LLAMA ESK QUESTIONS OR PROTRACTOR LLAMA QUESTIONS
\end{question}








\end{document}
