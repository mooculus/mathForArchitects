\documentclass[nooutcomes,noauthor,hints,handout]{ximera}

\graphicspath{  
{./}
{./whoAreYou/}
{./drawingWithTheTurtle/}
{./bisectionMethod/}
{./circles/}
{./anglesAndRightTriangles/}
{./lawOfSines/}
{./lawOfCosines/}
{./plotter/}
{./staircases/}
{./pitch/}
{./qualityControl/}
{./symmetry/}
{./nGonBlock/}
}


%% page layout
\usepackage[cm,headings]{fullpage}
\raggedright
\setlength\headheight{13.6pt}


%% fonts
\usepackage{euler}

\usepackage{FiraMono}
\renewcommand\familydefault{\ttdefault} 
\usepackage[defaultmathsizes]{mathastext}
\usepackage[htt]{hyphenat}

\usepackage[T1]{fontenc}
\usepackage[scaled=1]{FiraSans}

%\usepackage{wedn}
\usepackage{pbsi} %% Answer font


\usepackage{cancel} %% strike through in pitch/pitch.tex


%% \usepackage{ulem} %% 
%% \renewcommand{\ULthickness}{2pt}% changes underline thickness

\tikzset{>=stealth}

\usepackage{adjustbox}

\setcounter{titlenumber}{-1}

%% journal style
\makeatletter
\newcommand\journalstyle{%
  \def\activitystyle{activity-chapter}
  \def\maketitle{%
    \addtocounter{titlenumber}{1}%
                {\flushleft\small\sffamily\bfseries\@pretitle\par\vspace{-1.5em}}%
                {\flushleft\LARGE\sffamily\bfseries\thetitlenumber\hspace{1em}\@title \par }%
                {\vskip .6em\noindent\textit\theabstract\setcounter{question}{0}\setcounter{sectiontitlenumber}{0}}%
                    \par\vspace{2em}
                    \phantomsection\addcontentsline{toc}{section}{\thetitlenumber\hspace{1em}\textbf{\@title}}%
                     }}
\makeatother



%% thm like environments
\let\question\relax
\let\endquestion\relax

\newtheoremstyle{QuestionStyle}{\topsep}{\topsep}%%% space between body and thm
		{}                      %%% Thm body font
		{}                              %%% Indent amount (empty = no indent)
		{\bfseries}            %%% Thm head font
		{)}                              %%% Punctuation after thm head
		{ }                           %%% Space after thm head
		{\thmnumber{#2}\thmnote{ \bfseries(#3)}}%%% Thm head spec
\theoremstyle{QuestionStyle}
\newtheorem{question}{}



\let\freeResponse\relax
\let\endfreeResponse\relax

%% \newtheoremstyle{ResponseStyle}{\topsep}{\topsep}%%% space between body and thm
%% 		{\wedn\bfseries}                      %%% Thm body font
%% 		{}                              %%% Indent amount (empty = no indent)
%% 		{\wedn\bfseries}            %%% Thm head font
%% 		{}                              %%% Punctuation after thm head
%% 		{3ex}                           %%% Space after thm head
%% 		{\underline{\underline{\thmname{#1}}}}%%% Thm head spec
%% \theoremstyle{ResponseStyle}

\usepackage[tikz]{mdframed}
\mdfdefinestyle{ResponseStyle}{leftmargin=1cm,linecolor=black,roundcorner=5pt,
, font=\bsifamily,}%font=\wedn\bfseries\upshape,}


\ifhandout
\NewEnviron{freeResponse}{}
\else
%\newtheorem{freeResponse}{Response:}
\newenvironment{freeResponse}{\begin{mdframed}[style=ResponseStyle]}{\end{mdframed}}
\fi



%% attempting to automate outcomes.

%% \newwrite\outcomefile
%%   \immediate\openout\outcomefile=\jobname.oc
%% \renewcommand{\outcome}[1]{\edef\theoutcomes{\theoutcomes #1~}%
%% \immediate\write\outcomefile{\unexpanded{\outcome}{#1}}}

%% \newcommand{\outcomelist}{\begin{itemize}\theoutcomes\end{itemize}}

%% \NewEnviron{listOutcomes}{\small\sffamily
%% After answering the following questions, students should be able to:
%% \begin{itemize}
%% \BODY
%% \end{itemize}
%% }
\usepackage[tikz]{mdframed}
\mdfdefinestyle{OutcomeStyle}{leftmargin=2cm,rightmargin=2cm,linecolor=black,roundcorner=5pt,
, font=\small\sffamily,}%font=\wedn\bfseries\upshape,}
\newenvironment{listOutcomes}{\begin{mdframed}[style=OutcomeStyle]After answering the following questions, students should be able to:\begin{itemize}}{\end{itemize}\end{mdframed}}



%% my commands

\newcommand{\snap}{{\bfseries\itshape\textsf{Snap!}}}
\newcommand{\flavor}{\link[\snap]{https://snap.berkeley.edu/}}
\newcommand{\mooculus}{\textsf{\textbf{MOOC}\textnormal{\textsf{ULUS}}}}


\usepackage{tkz-euclide}
\tikzstyle geometryDiagrams=[rounded corners=.5pt,ultra thick,color=black]
\colorlet{penColor}{black} % Color of a curve in a plot



\ifhandout\newcommand{\mynewpage}{\newpage}\else\newcommand{\mynewpage}{}\fi

\title{What's your angle}

\author{Herb Clemens \and Brad Findell \and Jenny Sheldon \and Bart Snapp}


\begin{document}
\begin{abstract}
  Let's put our knowledge to work. 
\end{abstract}
\maketitle


\begin{listOutcomes}
\item Apply basic facts to solve for angles.
\item Apply triangle congruence theorems.
\item Measure angles with a protractor.
\item Work with interior angles.
\item Work with exterior angles.
\item Derive simple formulas for interior/exterior angles (and their
  sum) of basic shapes.
\end{listOutcomes}

%\mynewpage



%\begin{question}
%  In the diagram below, we see the intersection of $\bar{BC}$ and
%  $\bar{AE}$. Suppose that $\left\vert BD\right\vert =\left\vert
%  CD\right\vert $ and $\left\vert AD\right\vert =\left\vert
%  ED\right\vert $.
%  \begin{center}
%    \begin{tikzpicture}[geometryDiagrams]
%      \coordinate (A) at (0,2);
%      \coordinate (B) at (2,5);
%      \coordinate (C) at (6.5,.5);
%      \coordinate (E) at (8,4);
%      \coordinate (D) at (4,3);
%      \draw (A)--(B)--(C)--(E)--(D)--cycle;
%      \tkzMarkSegments[mark=|](B,D D,C)
%      \tkzMarkSegments[mark=||](A,D D,E)
%      \tkzLabelPoints[above](B,D,E)
%      \tkzLabelPoints[below](A,C)
%      %\draw[step=.5cm] (0,0) grid (10,5);
%    \end{tikzpicture}
%  \end{center}
%  Show that triangle $\triangle BDA$ and triangle $\triangle CDE$ are
%  congruent.
%  
%  \begin{hint}
%    First you should explain why $\angle BDA = \angle CDE$.
%  \end{hint}
%  \begin{freeResponse}
%    To start, we claim that $\angle BDA = \angle CDE$. Labeling our
%    diagram above,
%    \begin{center}
%      \begin{tikzpicture}[geometryDiagrams]
%        \coordinate (A) at (0,2);
%        \coordinate (B) at (2,5);
%        \coordinate (C) at (6.5,.5);
%        \coordinate (E) at (8,4);
%        \coordinate (D) at (4,3);
%        \draw (A)--(B)--(C)--(E)--(D)--cycle;
%        
%        \tkzMarkAngle[size=0.7cm,thin](B,D,A)
%        \tkzLabelAngle[pos = 0.4](B,D,A){$\alpha$}
%        
%        \tkzMarkAngle[arc=ll,size=0.5cm,thin](E,D,B)
%        \tkzLabelAngle[pos = 0.25](E,D,B){$\beta$}
%        
%        \tkzMarkAngle[size=0.7cm,thin](C,D,E)
%        \tkzLabelAngle[pos = 0.4](C,D,E){$\gamma$}
%        
%        \tkzMarkAngle[arc=ll,size=0.5cm,thin](A,D,C)
%        \tkzLabelAngle[pos = 0.25](A,D,C){$\delta$}
%        
%        %\draw[step=.5cm] (0,0) grid (10,5);
%      \end{tikzpicture}
%    \end{center}
%    we see that 
%    \begin{align*}
%      \alpha+\beta &= 180^\circ\\
%      \beta + \gamma &= 180^\circ.
%\end{align*}
%Subtracting the equations above we find that $\alpha=\gamma = 0$.
%This means that $\alpha = \gamma$ and hence $\angle BDA = \angle
%CDE$. Since we know that $\left\vert BD\right\vert =\left\vert
%CD\right\vert $ and $\left\vert AD\right\vert =\left\vert
%ED\right\vert $ we may now apply SAS to prove that triangle $\triangle
%BDA$ and triangle $\triangle CDE$ are congruent.
%\end{freeResponse}
%\end{question}
%
%\mynewpage
%
%
%\begin{question}
%  Consider the crazy shape below:
%  \begin{center}
%    \begin{tikzpicture}[geometryDiagrams]
%      \coordinate (A) at (0,4.5);
%      \coordinate (B) at (2.5,2.5);
%      \coordinate (C) at (0,.5);
%      \coordinate (D) at (4,2.5);
%      \coordinate (E) at (2.5,0);
%      \coordinate (F) at (10,2.5);
%      \coordinate (G) at (9.5,5);
%      \coordinate (H) at (7,2);
%      \coordinate (I) at (3.5,5);
%
%      \draw (A)--(B)--(C)--(D)--(E)--(F)--(G)--(H)--(I)--cycle;
%      
%      %% \tkzMarkAngle[size=0.7cm,thin](B,D,A)
%      \tkzLabelAngle[pos = -0.5](I,A,B){$\alpha$}
%
%      \tkzLabelAngle[pos = -0.2](A,B,C){$\beta$}
%
%      \tkzLabelAngle[pos = -1.9](B,C,D){$\gamma$}
%      
%      \tkzLabelAngle[pos = -0.2](C,D,E){$\delta$}
%
%      \tkzLabelAngle[pos = -0.5](D,E,F){$\varepsilon$}
%
%      \tkzLabelAngle[pos = -0.3](E,F,G){$\zeta$}
%
%      \tkzLabelAngle[pos = -0.5](F,G,H){$\eta$}
%
%      \tkzLabelAngle[pos = -0.2](G,H,I){$\theta$}
%
%      \tkzLabelAngle[pos = -0.3](H,I,A){$\iota$}
%
%      %\draw[step=.5cm,thin,gray] (0,0) grid (10,5);
%      \end{tikzpicture}
%  \end{center}
%  \begin{enumerate}
%  \item Using a protractor, or an online tool like
%    \link[\textsl{Geogebra}]{https://www.geogebra.org/classic},
%    measure the interior angles of the shape above (to the nearest
%    degree) and use this table to record your findings, along with the
%    SUM of YOUR MEASUREMENTS:
%  \[
%    {\renewcommand{\arraystretch}{1.5}
%      \begin{array}{|c|c|c|c|c|c|c|c|c||c|}\hline
%        \alpha & \beta & \gamma & \delta & \varepsilon & \zeta & \eta & \theta & \iota & SUM\\\hline\hline
%        \rule[7mm]{10mm}{0mm}  & \rule[7mm]{10mm}{0mm}    & \rule[7mm]{10mm}{0mm}   & \rule[7mm]{10mm}{0mm}   &  \rule[7mm]{10mm}{0mm}   & \rule[7mm]{10mm}{0mm}    & \rule[7mm]{10mm}{0mm}   & \rule[7mm]{10mm}{0mm}   & \rule[7mm]{10mm}{0mm} & \rule[7mm]{10mm}{0mm} \\ \hline
%    \end{array}}
%    \]
%  \item Use Louie Llama to help you find the sum of the interior
%    angles WITHOUT appealing to your measurements above. SHOW YOUR
%    WORK, and/or EXPLAIN your reasoning.
%  \end{enumerate}
%  \begin{freeResponse}
%    \begin{enumerate}
%      \item Here is my table:
%      \[
%        {\renewcommand{\arraystretch}{1.5}
%          \begin{array}{|c|c|c|c|c|c|c|c|c||c|}\hline
%            \alpha & \beta & \gamma & \delta & \varepsilon & \zeta & \eta & \theta & \iota & SUM\\\hline\hline
%            48^\circ & 283^\circ & 12^\circ & 327^\circ & 41^\circ & 96^\circ & 52^\circ & 270^\circ & 131^\circ & 1260^\circ\\ \hline
%        \end{array}}
%        \]
%      \item Walking Louie Llama around this shape will spin him $360^\circ$. Each, time he goes around an angle, he rotates
%        \[
%        180 - (\text{interior angle}) ~\text{degrees}.
%        \]
%        So we may write:
%        \[
%        (180 - \alpha) + (180 - \beta) + (180 - \gamma)+ (180 - \delta)+ (180 - \varepsilon) + (180 - \zeta) + (180 - \eta) + (180 - \theta) + (180 - \iota) = 360
%        \]
%        Solving, we find
%        \[
%        \alpha + \beta + \gamma + \delta + \varepsilon + \zeta + \eta + \theta + \iota = 1260.
%        \]
%    \end{enumerate}
%  \end{freeResponse}
%\end{question}
%
%\mynewpage
  Let's go for a few more walks with the ever insightful Louie Llama. For these problems, it would probably help to try physically moving around the shape.

\begin{question}
  Here's a regular five pointed star:
    \begin{center}
      \begin{tikzpicture}
        \foreach \i in {1,...,5}
                 {
                   \draw ({2*sin(\i *360/5)},{2*cos(\i * 360/5)}) -- ({2*sin((\i+2)*360/(5))},{2*cos((\i+2)*360/(5))});
                   \node (1) at ({1.5*sin(\i *360/5)},{1.5*cos(\i *360/5)}) {$\alpha$};
                 }
                 
      \end{tikzpicture}
    \end{center}
    Use Louie Llama to find the measure of one of the angles labeled
    $\alpha$. Explain how you arrived at your conclusion using words,
    pictures, llamas, and so on as needed/helpful.
\end{question}

\mynewpage


\begin{question}
Here's regular seven pointed star:
    \begin{center}
      \begin{tikzpicture}
        \foreach \i in {1,...,7}
                 {
                   \draw ({2*sin(\i *360/7)},{2*cos(\i * 360/7)}) -- ({2*sin((\i+3)*360/(7))},{2*cos((\i+3)*360/(7))});
                   \node (1) at ({1.3*sin(\i *360/7)},{1.3*cos(\i *360/7)}) {$\beta$};
                 }
                 
      \end{tikzpicture}
    \end{center}
    Use Louie Llama to find the measure of one of the angles labeled
    $\beta$. EXPLAIN how you arrived at your conclusion using words,
    pictures, llamas, and so on as needed/helpful.  
\end{question}

\mynewpage


\begin{question}
 Here's another regular seven pointed star:
    \begin{center}
      \begin{tikzpicture}
        \foreach \i in {1,...,7}
                 {
                   \draw ({2*sin(\i *360/7)},{2*cos(\i * 360/7)}) -- ({2*sin((\i+2)*360/(7))},{2*cos((\i+2)*360/(7))});
                   \node (1) at ({1.7*sin(\i *360/7)},{1.7*cos(\i *360/7)}) {$\gamma$};
                 }
                 
      \end{tikzpicture}
    \end{center}
    Use Louie Llama to find the measure of one of the angles labeled
    $\gamma$. Explain how you arrived at your conclusion using words,
    pictures, llamas, and so on as needed/helpful.
  \begin{freeResponse}
    \begin{enumerate}
    \item Here, Louie makes $5$ turns of $180-\alpha$ each. In
      the process, he makes TWO complete rotations. This means
      \[
      5\cdot (180-\alpha) = 2\cdot 360
      \]
      and so
      \[
      \alpha= 36^\circ.
      \]
    \item Here, Louie makes $7$ turns of $180-\beta$ each. In
      the process, he makes THREE complete rotations. This means
      \[
      7\cdot (180-\beta) = 3\cdot 360
      \]
      and so
      \[
      \beta\approx 25.7^\circ.
      \]
    \item Here, Louie makes $7$ turns of $180-\gamma$ each. In
      the process, he makes TWO complete rotations. This means
      \[
      7\cdot (180-\gamma) = 2\cdot 360
      \]
      and so
      \[
      \gamma\approx 77.1^\circ.
      \]
    \end{enumerate}
  \end{freeResponse}
\end{question}



%% Question about triangulation of a polygon.
%%
%% \begin{question}
%%   In mathematics, we like multiple explanations for the same
%%   facts. Another way to understand the sum of the interior angles of a
%%   polygon, is to ``triangulate'' the polygon. 
%%   \begin{enumerate}
%%   \item Explain what it means to ``triangulate'' a polygon
%%     \textbf{without} extra vertices.
%%   \item Triangulate a quadrilateral to find the sum of the measures of
%%     the interior angles. 
%%   \item Triangulate a pentagon to find the sum of the measures of
%%     the interior angles. 
%%   \item Triangulate this crazy shape
%%     \begin{center}
%%       \begin{tikzpicture}[geometryDiagrams]
%%         \coordinate (A) at (0,4.5);
%%         \coordinate (B) at (2.5,2.5);
%%         \coordinate (C) at (0,.5);
%%         \coordinate (D) at (4,2.5);
%%         \coordinate (E) at (2.5,0);
%%         \coordinate (F) at (10,2.5);
%%         \coordinate (G) at (9.5,5);
%%         \coordinate (H) at (7,2);
%%         \coordinate (I) at (3.5,5);
        
%%         \draw (A)--(B)--(C)--(D)--(E)--(F)--(G)--(H)--(I)--cycle;
        
%%         %% \tkzMarkAngle[size=0.7cm,thin](B,D,A)
%%         %% \tkzLabelAngle[pos = -0.5](I,A,B){$\alpha$}
        
%%         %% \tkzLabelAngle[pos = -0.2](A,B,C){$\beta$}
        
%%         %% \tkzLabelAngle[pos = -1.9](B,C,D){$\gamma$}
        
%%         %% \tkzLabelAngle[pos = -0.2](C,D,E){$\delta$}
        
%%         %% \tkzLabelAngle[pos = -0.5](D,E,F){$\varepsilon$}
        
%%         %% \tkzLabelAngle[pos = -0.3](E,F,G){$\zeta$}
        
%%         %% \tkzLabelAngle[pos = -0.5](F,G,H){$\eta$}
        
%%         %% \tkzLabelAngle[pos = -0.2](G,H,I){$\theta$}
        
%%         %% \tkzLabelAngle[pos = -0.3](H,I,A){$\iota$}
        
%%         %\draw[step=.5cm,thin,gray] (0,0) grid (10,5);
%%       \end{tikzpicture}
%%     \end{center}
%%     to find the sum of the measures of the interior angles.
%%   \end{enumerate}
%%   \begin{freeResponse}
%%     \begin{enumerate}
%%       \item 
%%     \end{enumerate}
%%   \end{freeResponse}
%% \end{question}








\end{document}
