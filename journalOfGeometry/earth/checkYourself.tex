\documentclass{../ximera}

\graphicspath{  
{./}
{./whoAreYou/}
{./drawingWithTheTurtle/}
{./bisectionMethod/}
{./circles/}
{./anglesAndRightTriangles/}
{./lawOfSines/}
{./lawOfCosines/}
{./plotter/}
{./staircases/}
{./pitch/}
{./qualityControl/}
{./symmetry/}
{./nGonBlock/}
}


%% page layout
\usepackage[cm,headings]{fullpage}
\raggedright
\setlength\headheight{13.6pt}


%% fonts
\usepackage{euler}

\usepackage{FiraMono}
\renewcommand\familydefault{\ttdefault} 
\usepackage[defaultmathsizes]{mathastext}
\usepackage[htt]{hyphenat}

\usepackage[T1]{fontenc}
\usepackage[scaled=1]{FiraSans}

%\usepackage{wedn}
\usepackage{pbsi} %% Answer font


\usepackage{cancel} %% strike through in pitch/pitch.tex


%% \usepackage{ulem} %% 
%% \renewcommand{\ULthickness}{2pt}% changes underline thickness

\tikzset{>=stealth}

\usepackage{adjustbox}

\setcounter{titlenumber}{-1}

%% journal style
\makeatletter
\newcommand\journalstyle{%
  \def\activitystyle{activity-chapter}
  \def\maketitle{%
    \addtocounter{titlenumber}{1}%
                {\flushleft\small\sffamily\bfseries\@pretitle\par\vspace{-1.5em}}%
                {\flushleft\LARGE\sffamily\bfseries\thetitlenumber\hspace{1em}\@title \par }%
                {\vskip .6em\noindent\textit\theabstract\setcounter{question}{0}\setcounter{sectiontitlenumber}{0}}%
                    \par\vspace{2em}
                    \phantomsection\addcontentsline{toc}{section}{\thetitlenumber\hspace{1em}\textbf{\@title}}%
                     }}
\makeatother



%% thm like environments
\let\question\relax
\let\endquestion\relax

\newtheoremstyle{QuestionStyle}{\topsep}{\topsep}%%% space between body and thm
		{}                      %%% Thm body font
		{}                              %%% Indent amount (empty = no indent)
		{\bfseries}            %%% Thm head font
		{)}                              %%% Punctuation after thm head
		{ }                           %%% Space after thm head
		{\thmnumber{#2}\thmnote{ \bfseries(#3)}}%%% Thm head spec
\theoremstyle{QuestionStyle}
\newtheorem{question}{}



\let\freeResponse\relax
\let\endfreeResponse\relax

%% \newtheoremstyle{ResponseStyle}{\topsep}{\topsep}%%% space between body and thm
%% 		{\wedn\bfseries}                      %%% Thm body font
%% 		{}                              %%% Indent amount (empty = no indent)
%% 		{\wedn\bfseries}            %%% Thm head font
%% 		{}                              %%% Punctuation after thm head
%% 		{3ex}                           %%% Space after thm head
%% 		{\underline{\underline{\thmname{#1}}}}%%% Thm head spec
%% \theoremstyle{ResponseStyle}

\usepackage[tikz]{mdframed}
\mdfdefinestyle{ResponseStyle}{leftmargin=1cm,linecolor=black,roundcorner=5pt,
, font=\bsifamily,}%font=\wedn\bfseries\upshape,}


\ifhandout
\NewEnviron{freeResponse}{}
\else
%\newtheorem{freeResponse}{Response:}
\newenvironment{freeResponse}{\begin{mdframed}[style=ResponseStyle]}{\end{mdframed}}
\fi



%% attempting to automate outcomes.

%% \newwrite\outcomefile
%%   \immediate\openout\outcomefile=\jobname.oc
%% \renewcommand{\outcome}[1]{\edef\theoutcomes{\theoutcomes #1~}%
%% \immediate\write\outcomefile{\unexpanded{\outcome}{#1}}}

%% \newcommand{\outcomelist}{\begin{itemize}\theoutcomes\end{itemize}}

%% \NewEnviron{listOutcomes}{\small\sffamily
%% After answering the following questions, students should be able to:
%% \begin{itemize}
%% \BODY
%% \end{itemize}
%% }
\usepackage[tikz]{mdframed}
\mdfdefinestyle{OutcomeStyle}{leftmargin=2cm,rightmargin=2cm,linecolor=black,roundcorner=5pt,
, font=\small\sffamily,}%font=\wedn\bfseries\upshape,}
\newenvironment{listOutcomes}{\begin{mdframed}[style=OutcomeStyle]After answering the following questions, students should be able to:\begin{itemize}}{\end{itemize}\end{mdframed}}



%% my commands

\newcommand{\snap}{{\bfseries\itshape\textsf{Snap!}}}
\newcommand{\flavor}{\link[\snap]{https://snap.berkeley.edu/}}
\newcommand{\mooculus}{\textsf{\textbf{MOOC}\textnormal{\textsf{ULUS}}}}


\usepackage{tkz-euclide}
\tikzstyle geometryDiagrams=[rounded corners=.5pt,ultra thick,color=black]
\colorlet{penColor}{black} % Color of a curve in a plot



\ifhandout\newcommand{\mynewpage}{\newpage}\else\newcommand{\mynewpage}{}\fi


\author{Bart Snapp}

\checkYourselfAbstract

\begin{document}
\maketitle



\begin{center}
\begin{tikzpicture}[x=.5cm, y=.5cm]
  \coordinate (A) at (-2,0);
  \coordinate (B) at (6,0);
  \coordinate (C) at (-2,-4);
  \coordinate (D) at (6,4);
  \coordinate (E) at (2,0);
  \coordinate (F) at (8,-4);
  \coordinate (G) at (-4,0);
  \coordinate (H) at (8,0);
  \coordinate (I) at (-4,-4);
  \draw (G)--(H);
  \draw (I)--(F);
  \draw (B)--(D)--(C)--(A);
  \tkzLabelPoints[above](D,A,E)
  \tkzLabelPoints[below](B,C,F)
  \tkzMarkSegments[mark=|](A,E E,B)
  \tkzMarkSegments[mark=||](C,E E,D)
  \tkzMarkAngle[size=0.75cm,mark=](B,E,D)
  \tkzMarkAngle[size=0.5cm,mark=](D,E,A)
  \tkzMarkAngle[size=0.75cm,mark=](A,E,C)
  \tkzMarkAngle[size=0.5cm,mark=](C,E,B)
  \tkzMarkAngle[size=0.75,mark=](F,C,D)
  \tkzLabelAngle[pos = -1](A,E,C){$\alpha$}
  \tkzLabelAngle[pos = 1](C,E,B){$\beta$}
  \tkzLabelAngle[pos = 1](D,E,A){$\gamma$}
  \tkzLabelAngle[pos = -1](C,E,A){$\delta$}
  \tkzLabelAngle[pos = 1](F,C,D){$\epsilon$}
\end{tikzpicture}
\end{center}

\begin{exercise}
 The formula for $\beta$ in terms of $\alpha$ is:
 
\begin{enumerate}
\begin{multicols}{2}
 \item $180^\circ-\alpha$
 \item $90^\circ-\beta$
 \item $\alpha$
 \item $360-\alpha$
\end{multicols}
\end{enumerate}
\end{exercise}

\begin{exercise}
 The formula for $\gamma$ in terms of $\alpha$ is:
 
\begin{enumerate}
\begin{multicols}{2}
 \item $180^\circ-\alpha$
 \item $90^\circ-\beta$
 \item $\alpha$
 \item $360-\alpha$
\end{multicols}
\end{enumerate}
\end{exercise}

\begin{exercise}
 The formula for $\delta$ in terms of $\alpha$ is:
 
\begin{enumerate}
\begin{multicols}{2}
 \item $180^\circ-\alpha$
 \item $90^\circ-\beta$
 \item $\alpha$
 \item $360-\alpha$
\end{multicols}
\end{enumerate}
\end{exercise}

\begin{exercise}
 The formula for $\epsilon$ in terms of $\alpha$ is:
 
\begin{enumerate}
\begin{multicols}{2}
 \item $180^\circ-\alpha$
 \item $90^\circ-\beta$
 \item $\alpha$
 \item $360-\alpha$
\end{multicols}
\end{enumerate}
\end{exercise}

\begin{exercise}
 You find yourself on a planet like Earth, but the size is very different. 
 
 On the longest day of summer the Sun is directly above your city, there are no shadows and light reaches the bottom of even the deepest open wells.

Your friend lives in a city 500 miles directly North of you. Here is a schematic diagram of a stick and a shadow in your friend's city:
  \begin{center}
      \begin{tikzpicture}[x=.5cm, y=.5cm]
        \coordinate (A) at (0,0);
        \coordinate (B) at (0,9.7);
        \coordinate (C) at (2,0);
        \tkzDrawSegment (A,B)
        \tkzDrawSegment (A,C)
        \tkzDrawSegment (C,B)

        \tkzMarkAngle[mark=,size=1.2cm,thin](A,B,C)
        \tkzLabelAngle[pos=3](A,B,C){$12^\circ$}
      \end{tikzpicture}
    \end{center}
    
What is the approximate circumference of the planet?
\begin{enumerate}
\begin{multicols}{2}
 \item about 20000 miles
 \item about 10000 miles
 \item about 5000 miles
 \item about 15000 miles
\end{multicols}
 \end{enumerate}
\end{exercise}

\answerlistbox{(a)}{(a)}{(c)}{(c)}{(d)}
\end{document}
