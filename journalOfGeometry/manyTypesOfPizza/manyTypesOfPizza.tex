\documentclass[noauthor,nooutcomes,hints,handout]{ximera}

\graphicspath{  
{./}
{./whoAreYou/}
{./drawingWithTheTurtle/}
{./bisectionMethod/}
{./circles/}
{./anglesAndRightTriangles/}
{./lawOfSines/}
{./lawOfCosines/}
{./plotter/}
{./staircases/}
{./pitch/}
{./qualityControl/}
{./symmetry/}
{./nGonBlock/}
}


%% page layout
\usepackage[cm,headings]{fullpage}
\raggedright
\setlength\headheight{13.6pt}


%% fonts
\usepackage{euler}

\usepackage{FiraMono}
\renewcommand\familydefault{\ttdefault} 
\usepackage[defaultmathsizes]{mathastext}
\usepackage[htt]{hyphenat}

\usepackage[T1]{fontenc}
\usepackage[scaled=1]{FiraSans}

%\usepackage{wedn}
\usepackage{pbsi} %% Answer font


\usepackage{cancel} %% strike through in pitch/pitch.tex


%% \usepackage{ulem} %% 
%% \renewcommand{\ULthickness}{2pt}% changes underline thickness

\tikzset{>=stealth}

\usepackage{adjustbox}

\setcounter{titlenumber}{-1}

%% journal style
\makeatletter
\newcommand\journalstyle{%
  \def\activitystyle{activity-chapter}
  \def\maketitle{%
    \addtocounter{titlenumber}{1}%
                {\flushleft\small\sffamily\bfseries\@pretitle\par\vspace{-1.5em}}%
                {\flushleft\LARGE\sffamily\bfseries\thetitlenumber\hspace{1em}\@title \par }%
                {\vskip .6em\noindent\textit\theabstract\setcounter{question}{0}\setcounter{sectiontitlenumber}{0}}%
                    \par\vspace{2em}
                    \phantomsection\addcontentsline{toc}{section}{\thetitlenumber\hspace{1em}\textbf{\@title}}%
                     }}
\makeatother



%% thm like environments
\let\question\relax
\let\endquestion\relax

\newtheoremstyle{QuestionStyle}{\topsep}{\topsep}%%% space between body and thm
		{}                      %%% Thm body font
		{}                              %%% Indent amount (empty = no indent)
		{\bfseries}            %%% Thm head font
		{)}                              %%% Punctuation after thm head
		{ }                           %%% Space after thm head
		{\thmnumber{#2}\thmnote{ \bfseries(#3)}}%%% Thm head spec
\theoremstyle{QuestionStyle}
\newtheorem{question}{}



\let\freeResponse\relax
\let\endfreeResponse\relax

%% \newtheoremstyle{ResponseStyle}{\topsep}{\topsep}%%% space between body and thm
%% 		{\wedn\bfseries}                      %%% Thm body font
%% 		{}                              %%% Indent amount (empty = no indent)
%% 		{\wedn\bfseries}            %%% Thm head font
%% 		{}                              %%% Punctuation after thm head
%% 		{3ex}                           %%% Space after thm head
%% 		{\underline{\underline{\thmname{#1}}}}%%% Thm head spec
%% \theoremstyle{ResponseStyle}

\usepackage[tikz]{mdframed}
\mdfdefinestyle{ResponseStyle}{leftmargin=1cm,linecolor=black,roundcorner=5pt,
, font=\bsifamily,}%font=\wedn\bfseries\upshape,}


\ifhandout
\NewEnviron{freeResponse}{}
\else
%\newtheorem{freeResponse}{Response:}
\newenvironment{freeResponse}{\begin{mdframed}[style=ResponseStyle]}{\end{mdframed}}
\fi



%% attempting to automate outcomes.

%% \newwrite\outcomefile
%%   \immediate\openout\outcomefile=\jobname.oc
%% \renewcommand{\outcome}[1]{\edef\theoutcomes{\theoutcomes #1~}%
%% \immediate\write\outcomefile{\unexpanded{\outcome}{#1}}}

%% \newcommand{\outcomelist}{\begin{itemize}\theoutcomes\end{itemize}}

%% \NewEnviron{listOutcomes}{\small\sffamily
%% After answering the following questions, students should be able to:
%% \begin{itemize}
%% \BODY
%% \end{itemize}
%% }
\usepackage[tikz]{mdframed}
\mdfdefinestyle{OutcomeStyle}{leftmargin=2cm,rightmargin=2cm,linecolor=black,roundcorner=5pt,
, font=\small\sffamily,}%font=\wedn\bfseries\upshape,}
\newenvironment{listOutcomes}{\begin{mdframed}[style=OutcomeStyle]After answering the following questions, students should be able to:\begin{itemize}}{\end{itemize}\end{mdframed}}



%% my commands

\newcommand{\snap}{{\bfseries\itshape\textsf{Snap!}}}
\newcommand{\flavor}{\link[\snap]{https://snap.berkeley.edu/}}
\newcommand{\mooculus}{\textsf{\textbf{MOOC}\textnormal{\textsf{ULUS}}}}


\usepackage{tkz-euclide}
\tikzstyle geometryDiagrams=[rounded corners=.5pt,ultra thick,color=black]
\colorlet{penColor}{black} % Color of a curve in a plot



\ifhandout\newcommand{\mynewpage}{\newpage}\else\newcommand{\mynewpage}{}\fi


\author{Bart Snapp and Jenny Sheldon and Betsy McNeal and Vic Ferdinand}

\title{Many types of pizza}

\begin{document}
\begin{abstract}
  How many different types of pizza can we make given some toppings?
\end{abstract}
\maketitle

\begin{listOutcomes}
\item Count combinations of items.
\item Recognize Pascal's Triangle.
\item Produce any row of Pascal's Triangle.
\item Explain patterns in Pascal's Triangle in terms of a concrete
  setting.
\end{listOutcomes}


\textit{Pascal's Pizza Shop} makes a variety of pizzas, all which come
with cheese.  However, they do not offer double
toppings, no ``double cheese'' and so on.


\mynewpage


\begin{question}
  We'll call a cheese pizza a $0$-topping pizza. Suppose the shop also
  has basil, garlic, and spinach toppings available.
\begin{enumerate}
\item How many different $0$-topping pizzas can be made?
\item How many different $1$-topping pizzas can be made?
\item How many different $2$-topping pizzas can be made?
\item How many different $3$-topping pizzas can be made?
\item In total, how many different pizzas can be made?
\end{enumerate}
In each case, give a brief EXPLANATION of your REASONING.
\begin{freeResponse}
  \begin{enumerate}
  \item You can make $1$ $0$-topping pizza. There's only one, plain
    cheese.
  \item You can make $3$ $1$-topping pizzas. We can make one with each
    of our three toppings.
  \item You can make $3$ $2$-topping pizzas. It's a works minus one topping!
  \item You can only make $1$ $3$-topping pizza.
  \item In total you can make $8$ different pizzas.
  \end{enumerate}
\end{freeResponse}
\end{question}
\mynewpage

\begin{question}
  Complete the following chart by listing and counting the
possibilities, where $n$ is the number of toppings Pascal has
available, and $k$ is the number of toppings used.\\
\[
{\renewcommand{\arraystretch}{2}
\begin{array}{|c||c|c|c|c|c|c||c|}
    \hline
          & k=0 & k=1 & k=2 & k=3 & k=4 & k=5 & \text{Total}\\
    \hline\hline
    n=0 &       &       &       &       &       &    &  \\
    \hline
    n=1 &       &       &       &       &       &    &  \\
    \hline
    n=2 &       &       &       &       &       &    &  \\
    \hline
    n=3 &       &       &       &       &       &    &  \\
    \hline
    n=4 &       &       &       &       &       &    &  \\
    \hline
    n=5 &       &       &       &       &       &    &  \\
    \hline
\end{array}}
\]
\begin{freeResponse}
  Here is my table:
  \[
{\renewcommand{\arraystretch}{2}
\begin{array}{|c||c|c|c|c|c|c||c|}
    \hline
          & k=0 & k=1 & k=2 & k=3 & k=4 & k=5 & \text{Total}\\
    \hline\hline
    n=0 &   1   &  0    &  0    &  0    &  0    &  0 & 1\\
    \hline
    n=1 &   1   &   1   &  0    &   0   &  0    &  0 & 2\\
    \hline
    n=2 &   1   &   2   &  1    &    0  &   0   &  0 & 4\\
    \hline
    n=3 &  1    &   3   &    3  &   1   &  0    &  0 & 8\\
    \hline
    n=4 &  1    &   4   &   6   &   4   &   1   &  0 & 16\\
    \hline
    n=5 &  1    &   5   &   10  &   10  &   5   &  1 & 32\\
    \hline
\end{array}}
\]
\end{freeResponse}
\end{question}
\mynewpage


\begin{question}
  Often the numbers in the table you filled in above are arranged a
  bit differently, as below, where the top ``$1$'' counts the number
  of $0$-topping pizzas that can be made if zero toppings are
  available.
  \[
  \begin{array}{c c c c c c c}
    &   &   & 1 &   &   &  \\
    &   & 1 &   & 1 &   &  \\
    & 1 &   & 2 &   & 1 &  \\
    1 &   & 3 &   & 3 &   & 1
  \end{array}
  \]
This triangle of numbers is usually called \textit{Pascal's
  Triangle}.\index{Pascal's Triangle} There are lots of patterns in
Pascal's Triangle.
\begin{enumerate}
  \item\label{PPS:s} How does someone produce the ``next'' row of Pascal's
    Triangle? 
  \item Complete the first $9$ rows of the triangle, above we have the
    first $4$ rows shown.
  \item Explain, in terms of pizza, why you can produce the next row
    from the previous row as in Part \ref{PPS:s}.
  \item Explain, in terms of pizza, why the numbers in the $(n+1)$th row
    must sum to $2^n$.
\end{enumerate}
\begin{freeResponse}
  \begin{enumerate}
  \item To make the next row of Pascal's Triangle, you add
    ``adjacent'' numbers in the previous row, and put the sum in the
    next row, between the two previous numbers.
  \item
    \[%https://tex.stackexchange.com/questions/17522/pascals-triangle-in-tikz
    \begin{tikzpicture}
      \foreach \n in {0,...,8} {
        \foreach \k in {0,...,\n} {
          \node at ({1.5*(\k-\n/2)},-\n*.7) {\binomialCoefficient{\n}{\k}};
        }
      }
    \end{tikzpicture}
    \]
  \item Consider this, you have a bunch of pizzas with one less
    topping on the left, and one more topping on the right. When we
    add a NEW topping, and want to know how many we can make, we
    simply take the pile with one less, add the new topping.  Now, the
    two piles combined will be the entire you want.
  \item So the numbers in the $n$th row tell us how many of each type
    of pizza there is. However, for each topping, it is either ``on''
    or ``off,'' and so if you have $n$ toppings, you can make a total
    of $2^n$ different pizzas.
  \end{enumerate}
\end{freeResponse}
\end{question}

\end{document}
