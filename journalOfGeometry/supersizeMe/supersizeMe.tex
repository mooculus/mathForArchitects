\documentclass[handout,nooutcomes,noauthor]{ximera}

%% page layout
\usepackage[in,headings]{fullpage}
\raggedright
\setlength\headheight{13.6pt}


%% fonts
\usepackage{euler}

\usepackage{FiraMono}
\renewcommand\familydefault{\ttdefault} 
\usepackage{mathastext}
\usepackage[htt]{hyphenat}

\usepackage[T1]{fontenc}
\usepackage[scaled=1]{FiraSans}

\usepackage{wedn}
\usepackage[T1]{fontenc}

%% wrap text around scripts
\usepackage{wrapfig}

\tikzset{>=stealth}
%% snap! scripts
\usepackage{scratch3}

\usepackage{adjustbox}

%% journal style
\makeatletter
\newcommand\journalstyle{%
  \def\activitystyle{activity-chapter}
  \def\maketitle{%
    \addtocounter{titlenumber}{1}%
                {\flushleft\small\sffamily\bfseries\@pretitle\par\vspace{-1.5em}}%
                {\flushleft\LARGE\sffamily\bfseries\thetitlenumber\hspace{1em}\@title \par }%
                {\vskip .6em\noindent\textit\theabstract\setcounter{question}{0}\setcounter{sectiontitlenumber}{0}}%
                    \par\vspace{2em}
                    \phantomsection\addcontentsline{toc}{section}{\thetitlenumber\hspace{1em}\textbf{\@title}}%
                     }}
\makeatother



%% thm like environments
\let\question\relax
\let\endquestion\relax

\newtheoremstyle{QuestionStyle}{\topsep}{\topsep}%%% space between body and thm
		{}                      %%% Thm body font
		{}                              %%% Indent amount (empty = no indent)
		{\bfseries}            %%% Thm head font
		{)}                              %%% Punctuation after thm head
		{ }                           %%% Space after thm head
		{\thmnumber{#2}\thmnote{ \bfseries(#3)}}%%% Thm head spec
\theoremstyle{QuestionStyle}
\newtheorem{question}{}



\let\freeResponse\relax
\let\endfreeResponse\relax

%% \newtheoremstyle{ResponseStyle}{\topsep}{\topsep}%%% space between body and thm
%% 		{\wedn\bfseries}                      %%% Thm body font
%% 		{}                              %%% Indent amount (empty = no indent)
%% 		{\wedn\bfseries}            %%% Thm head font
%% 		{}                              %%% Punctuation after thm head
%% 		{3ex}                           %%% Space after thm head
%% 		{\underline{\underline{\thmname{#1}}}}%%% Thm head spec
%% \theoremstyle{ResponseStyle}

\usepackage[tikz]{mdframed}
\mdfdefinestyle{ResponseStyle}{leftmargin=1cm,linecolor=black,roundcorner=5pt,frametitlefont=\wedn\bfseries,%frametitle={\underline{\underline{Response}}:}
, font=\wedn\bfseries,}%\begin{mdframed}[style=mystyle]foo\end{mdframed}


\ifhandout
\NewEnviron{freeResponse}{}
\else
%\newtheorem{freeResponse}{Response:}
\newenvironment{freeResponse}{\begin{mdframed}[style=ResponseStyle]}{\end{mdframed}}
\fi



%% attempting to automate outcomes.

\newwrite\outcomefile
  \immediate\openout\outcomefile=\jobname.oc
\renewcommand{\outcome}[1]{\edef\theoutcomes{\theoutcomes #1~}%
\immediate\write\outcomefile{\unexpanded{\outcome}{#1}}}

%% \newcommand{\outcomelist}{\begin{itemize}\theoutcomes\end{itemize}}



%% my commands

\newcommand{\snap}{{\bfseries\itshape\textsf{Snap!}}}
\newcommand{\flavor}{\link[\snap]{https://snap.berkeley.edu/}}


\usepackage{tkz-euclide}
\tikzstyle geometryDiagrams=[rounded corners=.5pt,ultra thick,color=black]
\colorlet{penColor}{black} % Color of a curve in a plot

\title{Supersize me}

\author{Bart Snapp}

\begin{document}
\begin{abstract}
  We will investigate the physical effects of scaling.
\end{abstract}
\maketitle


\begin{listOutcomes}
\item Apply the relation between linear scaling and area scaling.
\item Apply the relation between linear scaling and volume scaling.
\item Connect scaling of length, area, and volume to real-world
  attributes of objects.  
\item Apply understanding of scaling to evaluate thought experiments.
\end{listOutcomes}

This is Dr.\ Zoidberg:

\begin{center}
  \includegraphics[width=.3\textwidth]{zoidberg.png}
\end{center}

In the \textit{Futurama} episode \link[\textit{Anthology of Interest
    I}]{https://youtu.be/-IOXv1ysae4} Zoidberg is enlarged to be
approximately $100$ times his normal height. Let's use what we know
about scaling to think about this.

\mynewpage


 
\begin{question}
  Compute the following:
    \begin{enumerate}
    \item $\frac{\text{Dr.\ Zoidberg's enlarged
        height}}{\text{Dr.\ Zoidberg's normal height}}$
    \item $\frac{\text{Dr.\ Zoidberg's enlarged
        surface area}}{\text{Dr.\ Zoidberg's normal surface area}}$
    \item $\frac{\text{Dr.\ Zoidberg's enlarged
        volume}}{\text{Dr.\ Zoidberg's normal volume}}$
    \end{enumerate}
    In each case, EXPLAIN your reasoning.
    \begin{freeResponse}
      \begin{enumerate}
      \item Well,
        \[
        \frac{\text{Dr.\ Zoidberg's enlarged
            height}}{\text{Dr.\ Zoidberg's normal height}} = 100.
        \]
      \item And,
        \[
        \frac{\text{Dr.\ Zoidberg's enlarged
            surface area}}{\text{Dr.\ Zoidberg's normal surface area}} = 100^2 = 10000
        \]
        because when you scale linear lengths by a factor, you scale
        the area by the square of the factor.
      \item Finally,
        \[
        \frac{\text{Dr.\ Zoidberg's enlarged
            volume}}{\text{Dr.\ Zoidberg's normal volume}} = 100^3 = 1000000
        \]
        because when you scale linear lengths by a factor, you scale
        the volume by the cube of the factor.
      \end{enumerate}
    \end{freeResponse}
\end{question}
\mynewpage


\begin{question}
  Now estimate the following:
  \begin{enumerate}
    \item $\frac{\text{Dr.\ Zoidberg's enlarged
        weight}}{\text{Dr.\ Zoidberg's normal weight}}$
    \item $\frac{\text{Amount of Dr.\ Zoidberg's enlarged ``blood''}}{\text{Amount of Dr.\ Zoidberg's normal ``blood''}}$
    \item $\frac{\text{Weight of Dr.\ Zoidberg's enlarged
        shell}}{\text{Weight of Dr.\ Zoidberg's normal shell}}$
  \end{enumerate}
  In each case, DISCUSS how you arrived at your answer.
  \begin{freeResponse}
    \begin{enumerate}
    \item So,
      \[
      \frac{\text{Dr.\ Zoidberg's enlarged
        weight}}{\text{Dr.\ Zoidberg's normal weight}} = 100^3
      \]
      because volume determines weight!
    \item In this case:
      \[
      \frac{\text{Amount of Dr.\ Zoidberg's enlarged ``blood''}}{\text{Amount of Dr.\ Zoidberg's normal ``blood''}} = 100^3
      \]
      because volume determines the amount of blood!
    \item This one is tricky. If Zoidberg's shell is very thin, the answer will be closer to
      \[
      \frac{\text{Weight of Dr.\ Zoidberg's enlarged
          shell}}{\text{Weight of Dr.\ Zoidberg's normal shell}} = 100^2
      \]
      because the weight of a thin shell will be determined by its
      area. On the other hand, if there is a significant thickness to the shell then  the answer will be closer to:
      \[
      \frac{\text{Weight of Dr.\ Zoidberg's enlarged
          shell}}{\text{Weight of Dr.\ Zoidberg's normal shell}} = 100^3
      \]
      because volume determines weight.
    \end{enumerate}
  \end{freeResponse}
\end{question}
\mynewpage



\begin{question}
  I think if I were made $100$ times larger:
  \begin{enumerate}
  \item I would suffocate!
  \item Also I would not be able to lift my arms or legs.
  \end{enumerate}
  Can you EXPLAIN why I might think that?

  \begin{freeResponse}
    \begin{enumerate}
    \item The amount of blood one has is determined by volume, but the
      amount of oxygen our lungs can absorb is determined by surface
      area. So I would have $1000000$ times more blood, but only
      $10000$ times as much surface area to acquire oxygen. Hence it
      would be as if I was getting $1/100$ of the amount of air I
      needed.
    \item So, your muscles work by contracting ``strings,'' and
      stronger muscles have more of these strings. Hence the strength
      of a muscle is determined by its cross-sectional area. Since the
      weight of me, $100$ times larger is $1000000$ more, and my
      strength is only $10000$ more, I would be $100$ times weaker
      than I am. I don't think I would be able to move. 
    \end{enumerate}
  \end{freeResponse}
  
\end{question}



\end{document}
