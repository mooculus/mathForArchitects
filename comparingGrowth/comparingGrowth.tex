\documentclass[nooutcomes,noauthor,hints]{ximera}

\graphicspath{  
{./}
{./whoAreYou/}
{./drawingWithTheTurtle/}
{./bisectionMethod/}
{./circles/}
{./anglesAndRightTriangles/}
{./lawOfSines/}
{./lawOfCosines/}
{./plotter/}
{./staircases/}
{./pitch/}
{./qualityControl/}
{./symmetry/}
{./nGonBlock/}
}


%% page layout
\usepackage[cm,headings]{fullpage}
\raggedright
\setlength\headheight{13.6pt}


%% fonts
\usepackage{euler}

\usepackage{FiraMono}
\renewcommand\familydefault{\ttdefault} 
\usepackage[defaultmathsizes]{mathastext}
\usepackage[htt]{hyphenat}

\usepackage[T1]{fontenc}
\usepackage[scaled=1]{FiraSans}

%\usepackage{wedn}
\usepackage{pbsi} %% Answer font


\usepackage{cancel} %% strike through in pitch/pitch.tex


%% \usepackage{ulem} %% 
%% \renewcommand{\ULthickness}{2pt}% changes underline thickness

\tikzset{>=stealth}

\usepackage{adjustbox}

\setcounter{titlenumber}{-1}

%% journal style
\makeatletter
\newcommand\journalstyle{%
  \def\activitystyle{activity-chapter}
  \def\maketitle{%
    \addtocounter{titlenumber}{1}%
                {\flushleft\small\sffamily\bfseries\@pretitle\par\vspace{-1.5em}}%
                {\flushleft\LARGE\sffamily\bfseries\thetitlenumber\hspace{1em}\@title \par }%
                {\vskip .6em\noindent\textit\theabstract\setcounter{question}{0}\setcounter{sectiontitlenumber}{0}}%
                    \par\vspace{2em}
                    \phantomsection\addcontentsline{toc}{section}{\thetitlenumber\hspace{1em}\textbf{\@title}}%
                     }}
\makeatother



%% thm like environments
\let\question\relax
\let\endquestion\relax

\newtheoremstyle{QuestionStyle}{\topsep}{\topsep}%%% space between body and thm
		{}                      %%% Thm body font
		{}                              %%% Indent amount (empty = no indent)
		{\bfseries}            %%% Thm head font
		{)}                              %%% Punctuation after thm head
		{ }                           %%% Space after thm head
		{\thmnumber{#2}\thmnote{ \bfseries(#3)}}%%% Thm head spec
\theoremstyle{QuestionStyle}
\newtheorem{question}{}



\let\freeResponse\relax
\let\endfreeResponse\relax

%% \newtheoremstyle{ResponseStyle}{\topsep}{\topsep}%%% space between body and thm
%% 		{\wedn\bfseries}                      %%% Thm body font
%% 		{}                              %%% Indent amount (empty = no indent)
%% 		{\wedn\bfseries}            %%% Thm head font
%% 		{}                              %%% Punctuation after thm head
%% 		{3ex}                           %%% Space after thm head
%% 		{\underline{\underline{\thmname{#1}}}}%%% Thm head spec
%% \theoremstyle{ResponseStyle}

\usepackage[tikz]{mdframed}
\mdfdefinestyle{ResponseStyle}{leftmargin=1cm,linecolor=black,roundcorner=5pt,
, font=\bsifamily,}%font=\wedn\bfseries\upshape,}


\ifhandout
\NewEnviron{freeResponse}{}
\else
%\newtheorem{freeResponse}{Response:}
\newenvironment{freeResponse}{\begin{mdframed}[style=ResponseStyle]}{\end{mdframed}}
\fi



%% attempting to automate outcomes.

%% \newwrite\outcomefile
%%   \immediate\openout\outcomefile=\jobname.oc
%% \renewcommand{\outcome}[1]{\edef\theoutcomes{\theoutcomes #1~}%
%% \immediate\write\outcomefile{\unexpanded{\outcome}{#1}}}

%% \newcommand{\outcomelist}{\begin{itemize}\theoutcomes\end{itemize}}

%% \NewEnviron{listOutcomes}{\small\sffamily
%% After answering the following questions, students should be able to:
%% \begin{itemize}
%% \BODY
%% \end{itemize}
%% }
\usepackage[tikz]{mdframed}
\mdfdefinestyle{OutcomeStyle}{leftmargin=2cm,rightmargin=2cm,linecolor=black,roundcorner=5pt,
, font=\small\sffamily,}%font=\wedn\bfseries\upshape,}
\newenvironment{listOutcomes}{\begin{mdframed}[style=OutcomeStyle]After answering the following questions, students should be able to:\begin{itemize}}{\end{itemize}\end{mdframed}}



%% my commands

\newcommand{\snap}{{\bfseries\itshape\textsf{Snap!}}}
\newcommand{\flavor}{\link[\snap]{https://snap.berkeley.edu/}}
\newcommand{\mooculus}{\textsf{\textbf{MOOC}\textnormal{\textsf{ULUS}}}}


\usepackage{tkz-euclide}
\tikzstyle geometryDiagrams=[rounded corners=.5pt,ultra thick,color=black]
\colorlet{penColor}{black} % Color of a curve in a plot



\ifhandout\newcommand{\mynewpage}{\newpage}\else\newcommand{\mynewpage}{}\fi

\title{Comparing growth}

\author{Bart Snapp}

\begin{document}
\begin{abstract}
  Now we'll compare linear and exponential growth.
\end{abstract}
\maketitle

\begin{listOutcomes}
\item 
\end{listOutcomes}
Check out this dialogue between two students (based on a true story):

\begin{mdframed}[style=OutcomeStyle]
  \begin{quote}
\begin{dialogue}
\item[Devyn] Riley, let's play a game: I agree to pay you a secondly
  wage of $\$1$, meaning I'll paid you a dollar \text{every second} of
  every minute, of every hour, of every day, and so on.
\item[Riley] Ok, what's the catch? 
\item[Devyn] Well, you must agree to pay me a penny on the first day,
  double that (two pennies) the next, double that (four pennies) the
  next, and so on, each day doubling the payment amount.
\item[Riley] Okey dokey. Hmm\dots\ it seems I'm getting the better deal. 
\item[Devyn] It does certaily seem that way\dots
\end{dialogue}
  \end{quote}
\end{mdframed}
We'll think about this conversation.




\mynewpage


\begin{question}
 Who is in a better position to collect the most money, Devyn or
 Riley? What happens if they play their game:
 \begin{enumerate}
 \item A week,
 \item two weeks,
 \item three weeks,
 \item a month,
 \item a year.
 \end{enumerate}
 In each case show your work EXPLAIN your thoughts on who has more money.
 \end{question}
 
\mynewpage



\begin{question}
  Let's start with two definitions:
  \begin{mdframed}[style=OutcomeStyle]
\begin{quote}
  $\textbf{pro}\bullet\textbf{por}\bullet\textbf{tion}\bullet\textbf{al}$
  (pr{\rotatebox[origin=c]{180}{e}}$'$p\^orSH({\rotatebox[origin=c]{180}{e}})n({\rotatebox[origin=c]{180}{e}})l)
  \\
  
  \textit{adjective}\\

  
\quad having the same or a constant ratio; if $y$ is \textbf{proportional} to
$x$, then $y = k x$, where $k$ is a constant
\end{quote}
  \end{mdframed}

  
  \begin{mdframed}[style=OutcomeStyle]
\begin{quote}
  $\textbf{ex}\bullet\textbf{po}\bullet\textbf{nen}\bullet\textbf{tial}~\textbf{growth}$
  ({\rotatebox[origin=c]{180}{${}^{\textstyle '}$}}eksp{\rotatebox[origin=c]{180}{e}}$'$nen(t)SH({\rotatebox[origin=c]{180}{e}})l gr$\overline{\mbox{o}}$TH)
  \\
  
  \textit{adjective}\\

  
\quad when the \textbf{growth rate} of something is approximatly
\textbf{proportional} to the \textbf{amount}
\end{quote}
  \end{mdframed}
  \begin{enumerate}
  \item Explain why being paid a penny on the first day, double
    that (two pennies) the next, double that (four pennies) the next,
    and so on, each day doubling the payment amount is an example of
    exponetial growth.

  \item Explain why being a paid secondly wage of $\$1$ is not
    exponential growth.
\end{enumerate}


\end{question}
\mynewpage


\begin{question}
  Let's do some research and think about what we find.
  \begin{enumerate}
    \item Use the INTERNET to find a REAL-WORLD example of exponential growth. 
    \item Use the definition of exponential growth to explain WHY your example  has this property. 
    \item What would happen if your example of exponential growth is left ``unchecked?'' Explain your reasoning. 
  \end{enumerate}


\end{question}



\end{document}
