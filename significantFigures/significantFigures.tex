\documentclass[handout,noauthor,nooutcomes,hints,12pt]{ximera}
\graphicspath{  
{./}
{./whoAreYou/}
{./drawingWithTheTurtle/}
{./bisectionMethod/}
{./circles/}
{./anglesAndRightTriangles/}
{./lawOfSines/}
{./lawOfCosines/}
{./plotter/}
{./staircases/}
{./pitch/}
{./qualityControl/}
{./symmetry/}
{./nGonBlock/}
}


%% page layout
\usepackage[cm,headings]{fullpage}
\raggedright
\setlength\headheight{13.6pt}


%% fonts
\usepackage{euler}

\usepackage{FiraMono}
\renewcommand\familydefault{\ttdefault} 
\usepackage[defaultmathsizes]{mathastext}
\usepackage[htt]{hyphenat}

\usepackage[T1]{fontenc}
\usepackage[scaled=1]{FiraSans}

%\usepackage{wedn}
\usepackage{pbsi} %% Answer font


\usepackage{cancel} %% strike through in pitch/pitch.tex


%% \usepackage{ulem} %% 
%% \renewcommand{\ULthickness}{2pt}% changes underline thickness

\tikzset{>=stealth}

\usepackage{adjustbox}

\setcounter{titlenumber}{-1}

%% journal style
\makeatletter
\newcommand\journalstyle{%
  \def\activitystyle{activity-chapter}
  \def\maketitle{%
    \addtocounter{titlenumber}{1}%
                {\flushleft\small\sffamily\bfseries\@pretitle\par\vspace{-1.5em}}%
                {\flushleft\LARGE\sffamily\bfseries\thetitlenumber\hspace{1em}\@title \par }%
                {\vskip .6em\noindent\textit\theabstract\setcounter{question}{0}\setcounter{sectiontitlenumber}{0}}%
                    \par\vspace{2em}
                    \phantomsection\addcontentsline{toc}{section}{\thetitlenumber\hspace{1em}\textbf{\@title}}%
                     }}
\makeatother



%% thm like environments
\let\question\relax
\let\endquestion\relax

\newtheoremstyle{QuestionStyle}{\topsep}{\topsep}%%% space between body and thm
		{}                      %%% Thm body font
		{}                              %%% Indent amount (empty = no indent)
		{\bfseries}            %%% Thm head font
		{)}                              %%% Punctuation after thm head
		{ }                           %%% Space after thm head
		{\thmnumber{#2}\thmnote{ \bfseries(#3)}}%%% Thm head spec
\theoremstyle{QuestionStyle}
\newtheorem{question}{}



\let\freeResponse\relax
\let\endfreeResponse\relax

%% \newtheoremstyle{ResponseStyle}{\topsep}{\topsep}%%% space between body and thm
%% 		{\wedn\bfseries}                      %%% Thm body font
%% 		{}                              %%% Indent amount (empty = no indent)
%% 		{\wedn\bfseries}            %%% Thm head font
%% 		{}                              %%% Punctuation after thm head
%% 		{3ex}                           %%% Space after thm head
%% 		{\underline{\underline{\thmname{#1}}}}%%% Thm head spec
%% \theoremstyle{ResponseStyle}

\usepackage[tikz]{mdframed}
\mdfdefinestyle{ResponseStyle}{leftmargin=1cm,linecolor=black,roundcorner=5pt,
, font=\bsifamily,}%font=\wedn\bfseries\upshape,}


\ifhandout
\NewEnviron{freeResponse}{}
\else
%\newtheorem{freeResponse}{Response:}
\newenvironment{freeResponse}{\begin{mdframed}[style=ResponseStyle]}{\end{mdframed}}
\fi



%% attempting to automate outcomes.

%% \newwrite\outcomefile
%%   \immediate\openout\outcomefile=\jobname.oc
%% \renewcommand{\outcome}[1]{\edef\theoutcomes{\theoutcomes #1~}%
%% \immediate\write\outcomefile{\unexpanded{\outcome}{#1}}}

%% \newcommand{\outcomelist}{\begin{itemize}\theoutcomes\end{itemize}}

%% \NewEnviron{listOutcomes}{\small\sffamily
%% After answering the following questions, students should be able to:
%% \begin{itemize}
%% \BODY
%% \end{itemize}
%% }
\usepackage[tikz]{mdframed}
\mdfdefinestyle{OutcomeStyle}{leftmargin=2cm,rightmargin=2cm,linecolor=black,roundcorner=5pt,
, font=\small\sffamily,}%font=\wedn\bfseries\upshape,}
\newenvironment{listOutcomes}{\begin{mdframed}[style=OutcomeStyle]After answering the following questions, students should be able to:\begin{itemize}}{\end{itemize}\end{mdframed}}



%% my commands

\newcommand{\snap}{{\bfseries\itshape\textsf{Snap!}}}
\newcommand{\flavor}{\link[\snap]{https://snap.berkeley.edu/}}
\newcommand{\mooculus}{\textsf{\textbf{MOOC}\textnormal{\textsf{ULUS}}}}


\usepackage{tkz-euclide}
\tikzstyle geometryDiagrams=[rounded corners=.5pt,ultra thick,color=black]
\colorlet{penColor}{black} % Color of a curve in a plot



\ifhandout\newcommand{\mynewpage}{\newpage}\else\newcommand{\mynewpage}{}\fi

\title{Significant digits}
\author{Bart Snapp}

\begin{document}
\begin{abstract}
We will not delude ourselves with a false sense of precision. 
\end{abstract}
\maketitle

%% \begin{listObjectives}
%%  \item{Determine a reasonable estimate before preforming a calculation.}
%%  \item{Increase student confidence in their ability to solve difficult math problems by using previous results, trying different methods, asking questions, and work- ing with others.}
%% \end{listObjectives}

\begin{listOutcomes}
  \item Identify the number of significant digits in an expression.
  \item Apply the rules of significant digits in addition, subtraction.
  \item Apply the rules of significant digits in multiplication, and division.
  \item{Reflect on past work.}
\end{listOutcomes}

\begin{description}
  \item[Rules for determining the number of significant digits]\hfil
\begin{itemize}
    \item All nonzero digits are significant.
    \item Zeros between nonzero digits are significant.
    \item Leading zeros are not significant.
    \item Trailing zeros are significant only if there is a decimal point.
\end{itemize}
    \item[Addition and Subtraction]\hfil
    \begin{itemize}
      \item The sum or difference should have the same number of decimal places as the measurement with the fewest decimal places.
    \end{itemize}
    \item[Multiplication and Division]\hfil
    \begin{itemize}
      \item The product or quotient should have the same number of significant digits as the measurement with the fewest significant digits.
    \end{itemize}
\end{description}
\mynewpage

\begin{question}
  Let's see if you understand the rules above. Go to:
\begin{center}
  \url{https://chemquiz.net/sig/}
\end{center}
and quiz yourself!
\begin{itemize}
  \item Select 5 problems.
  \item Select all types of questions: Counting sig figs in numbers, Rounding numbers by sig figs, Multiplication \& division problems with sig figs, Addition \& subtraction problems with sig figs.
  \item Select decimal (regular) notation
  \item Choose your thousands separator.
  \item No units. 
  \item Fill-in-the-blank
  \item Start!
\end{itemize}
Take the quiz until you achieve a perfect score. 

\textbf{BELOW, DESCRIBE A PROBLEM YOU MISSED OR THOUGHT WAS TRICKY.} 
\end{question}
\mynewpage

\begin{question}
Demonstrate skill here by answering the following questions respecting significant digits. \textbf{In each case, explain your reasoning.}
\begin{enumerate}
  \item Compute: $32.8000 \cdot 7200$
  \vfill

  \item How many significant digits are in $191.000$?

  \vfill 
  \item Round  $3363750$ to $4$ significant digits.
  \vfill

  \item Compute:  $5290 - 0.3100$
  \vfill
  \item Compute: $87 900 000 + 10 000$
\vfill
\end{enumerate}


\end{question}
\mynewpage

\begin{question}
Let's reflect on the activity we did the previous class day. Express your answers from Problems 1 and 2 from that activity 
respecting significant digits. Explain your reasoning. 
\end{question}

\end{document}
