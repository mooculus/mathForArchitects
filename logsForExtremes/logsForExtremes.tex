\documentclass[noauthor,nooutcomes,handout,hints,12pt]{ximera}

%% page layout
\usepackage[in,headings]{fullpage}
\raggedright
\setlength\headheight{13.6pt}


%% fonts
\usepackage{euler}

\usepackage{FiraMono}
\renewcommand\familydefault{\ttdefault} 
\usepackage{mathastext}
\usepackage[htt]{hyphenat}

\usepackage[T1]{fontenc}
\usepackage[scaled=1]{FiraSans}

\usepackage{wedn}
\usepackage[T1]{fontenc}

%% wrap text around scripts
\usepackage{wrapfig}

\tikzset{>=stealth}
%% snap! scripts
\usepackage{scratch3}

\usepackage{adjustbox}

%% journal style
\makeatletter
\newcommand\journalstyle{%
  \def\activitystyle{activity-chapter}
  \def\maketitle{%
    \addtocounter{titlenumber}{1}%
                {\flushleft\small\sffamily\bfseries\@pretitle\par\vspace{-1.5em}}%
                {\flushleft\LARGE\sffamily\bfseries\thetitlenumber\hspace{1em}\@title \par }%
                {\vskip .6em\noindent\textit\theabstract\setcounter{question}{0}\setcounter{sectiontitlenumber}{0}}%
                    \par\vspace{2em}
                    \phantomsection\addcontentsline{toc}{section}{\thetitlenumber\hspace{1em}\textbf{\@title}}%
                     }}
\makeatother



%% thm like environments
\let\question\relax
\let\endquestion\relax

\newtheoremstyle{QuestionStyle}{\topsep}{\topsep}%%% space between body and thm
		{}                      %%% Thm body font
		{}                              %%% Indent amount (empty = no indent)
		{\bfseries}            %%% Thm head font
		{)}                              %%% Punctuation after thm head
		{ }                           %%% Space after thm head
		{\thmnumber{#2}\thmnote{ \bfseries(#3)}}%%% Thm head spec
\theoremstyle{QuestionStyle}
\newtheorem{question}{}



\let\freeResponse\relax
\let\endfreeResponse\relax

%% \newtheoremstyle{ResponseStyle}{\topsep}{\topsep}%%% space between body and thm
%% 		{\wedn\bfseries}                      %%% Thm body font
%% 		{}                              %%% Indent amount (empty = no indent)
%% 		{\wedn\bfseries}            %%% Thm head font
%% 		{}                              %%% Punctuation after thm head
%% 		{3ex}                           %%% Space after thm head
%% 		{\underline{\underline{\thmname{#1}}}}%%% Thm head spec
%% \theoremstyle{ResponseStyle}

\usepackage[tikz]{mdframed}
\mdfdefinestyle{ResponseStyle}{leftmargin=1cm,linecolor=black,roundcorner=5pt,frametitlefont=\wedn\bfseries,%frametitle={\underline{\underline{Response}}:}
, font=\wedn\bfseries,}%\begin{mdframed}[style=mystyle]foo\end{mdframed}


\ifhandout
\NewEnviron{freeResponse}{}
\else
%\newtheorem{freeResponse}{Response:}
\newenvironment{freeResponse}{\begin{mdframed}[style=ResponseStyle]}{\end{mdframed}}
\fi



%% attempting to automate outcomes.

\newwrite\outcomefile
  \immediate\openout\outcomefile=\jobname.oc
\renewcommand{\outcome}[1]{\edef\theoutcomes{\theoutcomes #1~}%
\immediate\write\outcomefile{\unexpanded{\outcome}{#1}}}

%% \newcommand{\outcomelist}{\begin{itemize}\theoutcomes\end{itemize}}



%% my commands

\newcommand{\snap}{{\bfseries\itshape\textsf{Snap!}}}
\newcommand{\flavor}{\link[\snap]{https://snap.berkeley.edu/}}


\usepackage{tkz-euclide}
\tikzstyle geometryDiagrams=[rounded corners=.5pt,ultra thick,color=black]
\colorlet{penColor}{black} % Color of a curve in a plot
 % <- Adjust this to your local setup if needed.

\author{Bart Snapp}
\title{Logs for extremes}

\begin{document}
\begin{abstract}
    We use logarithms to compare and visualize quantities that range
    across many orders of magnitude.
\end{abstract}
\maketitle

\begin{listOutcomes}
    \item Recall the definition of the logarithm.
    \item Read and interpret log scales.
    \item Explore how log scales can help us compare vastly different numbers.
    \item Plot data on a log-scale.
\end{listOutcomes}

\begin{definition}
    We define the \textbf{logarithm} as
    \[
        \log_b(a) = x
        \quad\Leftrightarrow\quad
        b^x = a.
    \]
\end{definition}

In everyday life (and in architecture, science, and economics!), we
often run into situations where numbers differ by factors of millions,
billions, or more. A \textbf{logarithmic} scale helps us see, plot,
and compare these extremes more easily.

\mynewpage

%%%%%%%%%%%%%%%%%%%%%%%%%%%%%%%%%%%%%%%%%%%%%%%%%%%%
% QUESTION 1: SHOW RICHTER SCALE, ASK ABOUT BASE, AND CONNECTION TO EXPONENTIAL GROWTH
%%%%%%%%%%%%%%%%%%%%%%%%%%%%%%%%%%%%%%%%%%%%%%%%%%%%
\begin{question}
The Richter scale is a logarithmic scale used to measure the
intensity of earthquakes.
\begin{center}%%% https://civil-engg-world.blogspot.com/2009/03/magnitude-of-earthquake.html
    \begin{tabular}{|c|c|c|}\hline
        \textbf{Magnitude} & \textbf{TNT Energy} & \textbf{Example}                   \\\hline
        1                  & 6 ounces            & Small blast at a construction site \\\hline
        2                  & 13 pounds           & Small explosion                    \\\hline
        3                  & 397 pounds          & Military bomb                      \\\hline
        4                  & 6 tons              & 2001 El Salvador earthquake        \\\hline
        5                  & 199 tons            & 2011 Lorca earthquake              \\\hline
        6                  & 6270 tons           & 2022 Afghanistan earthquake        \\\hline
        7                  & 199000 tons         & 2010 Haiti earthquake              \\\hline
        8                  & 6270000 tons        & 2008 Sichuan earthquake            \\\hline
        9                  & 199999000 tons      & 2004 Indian Ocean earthquake and
        tsunami                                                                       \\\hline
    \end{tabular}
\end{center}
\begin{enumerate}
    \item What is the base of the logarithm in the Richter scale?
    \item How does the Richter scale relate to the concept of exponential
          growth?
\end{enumerate}
\end{question}

\mynewpage
%%%%%%%%%%%%%%%%%%%%%%%%%%%%%%%%%%%%%%%%%%%%%%%%%%%%
% QUESTION 2: INTERPRETING LOG SCALES
%%%%%%%%%%%%%%%%%%%%%%%%%%%%%%%%%%%%%%%%%%%%%%%%%%%%
\begin{question}
    The edge of the \emph{observable} universe is around $47$ billion
    light-years from Earth, which is an unfathomably large distance. However,
    if we use what is called a \textbf{log-scale} on the vertical axis, meaning
    we take some log (in this case $\log_{463}$) of the vertical values, we can
    easily visualize the data, behold:
    \begin{center}
        \begin{tikzpicture}[x=1.5cm,y=1cm]
            % A simple blank axis for the students to label
            \draw[thin,->] (-.5,0)--(10.5,0);
            \draw[thin,->] (0,-.5)--(0,10.5);
            % Example tick marks (unlabeled, for students to fill in)
            \foreach \y in {1,...,10}{
                    \draw (0,\y) -- ++(0.2,0);
                    %\node at (-1.5,\y) {$\log_{463}\left(463^{\y}\right)=\y$};
                    \node[anchor=east] at (-.1,\y) {$\y$};
                }

            \node[anchor=west] at (1,.2) {{\scriptsize Height of a human}};
            \draw[fill=black] (1,.1) circle (.05);

            \node[anchor=west] at (2,.95) {{\scriptsize Height of Eiffel
                        Tower}};
            \draw[fill=black] (2,.95) circle (.05);

            \node[anchor=west] at (3,1.5) {{\scriptsize Height of Mount
                        Everest}};
            \draw[fill=black] (3,1.5) circle (.05);

            \node[anchor=west] at (4,2.1) {{\scriptsize Distance to the
                        International Space Station}};
            \draw[fill=black] (4,2.1) circle (.05);

            \node[anchor=west] at (5,3.2) {{\scriptsize Distance to the Moon}};
            \draw[fill=black] (5,3.2) circle (.05);

            \node[anchor=west] at (6,4.2) {{\scriptsize Distance to the Sun}};
            \draw[fill=black] (6,4.2) circle (.05);

            \node[anchor=east] at (7,5.2) {{\scriptsize Distance to the edge of
                        the solar system}};
            \draw[fill=black] (7,5.2) circle (.05);

            \node[anchor=east] at (8,7.8) {{\scriptsize Distance to the edge of
                        the Milky Way}};
            \draw[fill=black] (8,7.8) circle (.05);

            \node[anchor=east] at (9,8.4) {{\scriptsize Distance to the
                        Andromeda galaxy}};
            \draw[fill=black] (9,8.4) circle (.05);

            \node[anchor=east] at (10,10){{\scriptsize Distance to the edge of
                        the universe}};
            \draw[fill=black] (10,10) circle (.05);
        \end{tikzpicture}
    \end{center}
    \begin{enumerate}
        \item Starting with the same first tick-mark, how tall would this plot
              be if we used a linear-scale instead of a log scale? Explain your reasoning.
        \item Starting with the same first tick-mark, how tall would this plot
              be if we used a $\log_2$-scale? Explain your reasoning.
    \end{enumerate}
\end{question}

\mynewpage

%%%%%%%%%%%%%%%%%%%%%%%%%%%%%%%%%%%%%%%%%%%%%%%%%%%%
% QUESTION 3: WEALTH DISTRIBUTION
%%%%%%%%%%%%%%%%%%%%%%%%%%%%%%%%%%%%%%%%%%%%%%%%%%%%
\begin{question}
    \textbf{Wealth Distribution: Comparing Everyday Net Worth to Extreme
        Fortunes}

    According to data compiled at
    \begin{center}
        \texttt{http://www.righto.com/2024/10/wealth-distribution-in-united-states.html},
    \end{center}
    net worth in the United States varies widely, from a median of
    roughly \(\$100{,}000\) to the \(\$200\) billion range for some of the
    richest individuals (e.g., Elon Musk).

    \begin{enumerate}
        \item On a single \textbf{logarithmic-scale}, plot typical net worth
              amounts such as:
              \[
                  \$1{,}000,\quad \$100{,}000,\quad \$1{,}000{,}000,\quad
                  \$1{,}000{,}000{,}000,\quad
                  \$200{,}000{,}000{,}000.
              \]
              Use the blank axis below and add your own labels or ticks as
              needed.

        \item Briefly discuss why it might be more intuitive to use a log scale
              than a linear scale when comparing a typical family's net worth
              to
              that of someone with \(\$200\) billion.
    \end{enumerate}

    \begin{center}
        \begin{tikzpicture}[x=1cm,y=1cm,scale=0.8]
            % A simple blank axis for the students to label
            \draw[thick,->] (0,0)--(10,0);
            \foreach \x in {0,...,10}
            \draw (\x,0) -- ++(0,0.2);

            \node at (5,-1) {\small (Place your net-worth labels here)};
        \end{tikzpicture}
    \end{center}
\end{question}

%%%%%%%%%%%%%%%%%%%%%%%%%%%%%%%%%%%%%%%%%%%%%%%%%%%%
% QUESTION 3: DISCOVER THE RICHTER SCALE
%%%%%%%%%%%%%%%%%%%%%%%%%%%%%%%%%%%%%%%%%%%%%%%%%%%%
\begin{question}
    \textbf{Richter Scale for Earthquakes READING?}

    Earthquake magnitudes are often reported using the (common) logarithm
    base \(10\). For instance, a magnitude \(M\) earthquake satisfies
    \[
        M = \log_{10}\!\Bigl(\tfrac{A}{A_0}\Bigr),
    \]
    where \(A\) is the measured seismic wave amplitude and \(A_0\) is a
    standard reference amplitude.

    \begin{enumerate}
        \item Suppose an earthquake measures \(M=7\). By what factor is its
              seismic wave amplitude \(A\) bigger than the reference amplitude
              \(A_0\)?

        \item If another earthquake measures \(M=8\), how many times larger is
              \emph{its} amplitude compared to the \(M=7\) quake?

        \item Draw an axis below to illustrate how quickly earthquake
              amplitudes grow as \(M\) increases. Label some hypothetical
              magnitudes (like 2, 5, 7, 9) along the axis.

    \end{enumerate}

    \begin{center}
        \begin{tikzpicture}[x=1cm,y=1cm,scale=0.8]
            % A simple blank axis for the students to label
            \draw[thick,->] (0,0)--(10,0);
            \foreach \x in {0,...,10}
            \draw (\x,0) -- ++(0,0.2);

            \node at (5,-1) {\small (Your Richter-scale labels here)};
        \end{tikzpicture}
    \end{center}
\end{question}

\mynewpage

\end{document}
