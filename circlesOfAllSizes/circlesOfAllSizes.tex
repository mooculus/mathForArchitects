\documentclass[noauthor,hints,nooutcomes,handout]{ximera}

\graphicspath{  
{./}
{./whoAreYou/}
{./drawingWithTheTurtle/}
{./bisectionMethod/}
{./circles/}
{./anglesAndRightTriangles/}
{./lawOfSines/}
{./lawOfCosines/}
{./plotter/}
{./staircases/}
{./pitch/}
{./qualityControl/}
{./symmetry/}
{./nGonBlock/}
}


%% page layout
\usepackage[cm,headings]{fullpage}
\raggedright
\setlength\headheight{13.6pt}


%% fonts
\usepackage{euler}

\usepackage{FiraMono}
\renewcommand\familydefault{\ttdefault} 
\usepackage[defaultmathsizes]{mathastext}
\usepackage[htt]{hyphenat}

\usepackage[T1]{fontenc}
\usepackage[scaled=1]{FiraSans}

%\usepackage{wedn}
\usepackage{pbsi} %% Answer font


\usepackage{cancel} %% strike through in pitch/pitch.tex


%% \usepackage{ulem} %% 
%% \renewcommand{\ULthickness}{2pt}% changes underline thickness

\tikzset{>=stealth}

\usepackage{adjustbox}

\setcounter{titlenumber}{-1}

%% journal style
\makeatletter
\newcommand\journalstyle{%
  \def\activitystyle{activity-chapter}
  \def\maketitle{%
    \addtocounter{titlenumber}{1}%
                {\flushleft\small\sffamily\bfseries\@pretitle\par\vspace{-1.5em}}%
                {\flushleft\LARGE\sffamily\bfseries\thetitlenumber\hspace{1em}\@title \par }%
                {\vskip .6em\noindent\textit\theabstract\setcounter{question}{0}\setcounter{sectiontitlenumber}{0}}%
                    \par\vspace{2em}
                    \phantomsection\addcontentsline{toc}{section}{\thetitlenumber\hspace{1em}\textbf{\@title}}%
                     }}
\makeatother



%% thm like environments
\let\question\relax
\let\endquestion\relax

\newtheoremstyle{QuestionStyle}{\topsep}{\topsep}%%% space between body and thm
		{}                      %%% Thm body font
		{}                              %%% Indent amount (empty = no indent)
		{\bfseries}            %%% Thm head font
		{)}                              %%% Punctuation after thm head
		{ }                           %%% Space after thm head
		{\thmnumber{#2}\thmnote{ \bfseries(#3)}}%%% Thm head spec
\theoremstyle{QuestionStyle}
\newtheorem{question}{}



\let\freeResponse\relax
\let\endfreeResponse\relax

%% \newtheoremstyle{ResponseStyle}{\topsep}{\topsep}%%% space between body and thm
%% 		{\wedn\bfseries}                      %%% Thm body font
%% 		{}                              %%% Indent amount (empty = no indent)
%% 		{\wedn\bfseries}            %%% Thm head font
%% 		{}                              %%% Punctuation after thm head
%% 		{3ex}                           %%% Space after thm head
%% 		{\underline{\underline{\thmname{#1}}}}%%% Thm head spec
%% \theoremstyle{ResponseStyle}

\usepackage[tikz]{mdframed}
\mdfdefinestyle{ResponseStyle}{leftmargin=1cm,linecolor=black,roundcorner=5pt,
, font=\bsifamily,}%font=\wedn\bfseries\upshape,}


\ifhandout
\NewEnviron{freeResponse}{}
\else
%\newtheorem{freeResponse}{Response:}
\newenvironment{freeResponse}{\begin{mdframed}[style=ResponseStyle]}{\end{mdframed}}
\fi



%% attempting to automate outcomes.

%% \newwrite\outcomefile
%%   \immediate\openout\outcomefile=\jobname.oc
%% \renewcommand{\outcome}[1]{\edef\theoutcomes{\theoutcomes #1~}%
%% \immediate\write\outcomefile{\unexpanded{\outcome}{#1}}}

%% \newcommand{\outcomelist}{\begin{itemize}\theoutcomes\end{itemize}}

%% \NewEnviron{listOutcomes}{\small\sffamily
%% After answering the following questions, students should be able to:
%% \begin{itemize}
%% \BODY
%% \end{itemize}
%% }
\usepackage[tikz]{mdframed}
\mdfdefinestyle{OutcomeStyle}{leftmargin=2cm,rightmargin=2cm,linecolor=black,roundcorner=5pt,
, font=\small\sffamily,}%font=\wedn\bfseries\upshape,}
\newenvironment{listOutcomes}{\begin{mdframed}[style=OutcomeStyle]After answering the following questions, students should be able to:\begin{itemize}}{\end{itemize}\end{mdframed}}



%% my commands

\newcommand{\snap}{{\bfseries\itshape\textsf{Snap!}}}
\newcommand{\flavor}{\link[\snap]{https://snap.berkeley.edu/}}
\newcommand{\mooculus}{\textsf{\textbf{MOOC}\textnormal{\textsf{ULUS}}}}


\usepackage{tkz-euclide}
\tikzstyle geometryDiagrams=[rounded corners=.5pt,ultra thick,color=black]
\colorlet{penColor}{black} % Color of a curve in a plot



\ifhandout\newcommand{\mynewpage}{\newpage}\else\newcommand{\mynewpage}{}\fi


\title{Circles of all sizes (REWRITE DOG WALK 1m away)}
\author{Claire Merriman \and Bart Snapp}

%% ONLY REQUIRES WHAT's YOUR ANGLE...



\begin{document}
\begin{abstract}
  We'll explore some more properties of circles.
\end{abstract}
\maketitle

\begin{listOutcomes}
  \item Think about the isoperimetric inequality.
\item Use angles to find the length of an arc of a circle.
\item Use formulas to solve word problems.
\end{listOutcomes}

%% \begin{listObjectives}
%% \item Learn and apply basic geometric formulas,
%% \item Explain why presented concepts and formulas are true,
%% \end{listObjectives}


\mynewpage

\begin{question}
  I was reading some fancy math the other day, it said for any shape
  of area $A$, the perimeter of this shape is $P$ where:
  \[
  P \ge 2\pi\sqrt{A/\pi}
  \]
  Use the fact that a circle is the two dimensional shape of greatest
  area for a given perimeter to deduce the inequality above.
  \begin{hint} Recall $A = \pi r^2$ and $C = 2\pi r$.
  \end{hint}
\end{question}
\mynewpage




\begin{question}
\textbf{Angles} and the \textbf{arc length} of a circle are closely
related. Suppose we have a circle and an angle is made:

\begin{center}
\begin{tikzpicture}[geometryDiagrams]
  \tkzDefPoint(0,0){O}
  \tkzDefPoint(4,0){A}
  \tkzDefPoint(4*cos(1),4*sin(1)){B}
  \tkzDrawArc[line width = 4,black](O,A)(B)
  \tkzDrawArc[](O,B)(A)
  \tkzDrawLines[add = 0 and .5](O,A O,B)
 % \tkzLablSegments[mark=|](O,A)
  \tkzDrawPoints(O,A,B)
  \tkzMarkAngle[size=0.75cm,mark=](A,O,B)
   \tkzLabelAngle[pos = .8,right](A,O,B){$\theta$}
 \end{tikzpicture}
 
\end{center}
Answer these questions:
\begin{enumerate}
\item If the thick segment has a length of $45$ units and the angle is $45^\circ$, what is the circumference of the circle?
\item If the thick segment has a length of $10$ units and the angle is $180^\circ$, what is the circumference of the circle?
\item If the thick segment has a length of $25$ units and the angle is $60^\circ$, what is the circumference of the circle?
\item If the thick segment has a length of $50$ units and the angle is $7^\circ$, what is the circumference of the circle?
\item If the thick segment has a length of $x$ units and the angle is $\theta^\circ$, what is the circumference of the circle?
\end{enumerate}
and explain in words how to think about such problems.
\end{question}
\mynewpage












\begin{question} %% https://en.wikipedia.org/wiki/String_girdling_Earth 
  Here a problem that is at least $300$ years old, so let's call it an  ``oldie, but a goodie!''
  \begin{quote}
    Imagine a chain, pulled tight around the Earth's equator. Suppose
    you add an extra $10'$ to the chain. The chain is then raised up
    uniformly, all around the Earth, as high as possible. How high is
    that?
  \end{quote}
  Supposing that
  \begin{itemize}
  \item There are $5280$ feet in a mile, and
  \item the radius of the Earth is $4000$ miles,
  \end{itemize}
  it is now time for you to try you hand at this classic.
  \begin{enumerate}
  \item Give a solution in feet. Show all work.
  \item Now solve the problem AGAIN but for a table that is $2'$ in
    radius; that is imagine a chain, pulled tight around a circular
    table. Suppose you add an extra $10'$ to the chain. The chain is
    then pulled away uniformly, all around the table, as far as
    possible. How far is that?
  \item Is the answer to the previous parts surprising? Explain why or
    why not.
  \end{enumerate}
  \begin{freeResponse}
    It is easiest to work with letters. If $R$ is the radius of the
    circle, then we have
    \begin{align*}
      2\pi (R + d) &= 2\pi R +10,\\
      2\pi R + 2\pi d &= 2\pi R + 10,\\
      2\pi d &= 10,\\
      d &= \frac{10}{2\pi}.
    \end{align*}
  \end{freeResponse}
\end{question}

\end{document}
