\documentclass{ximera}


\graphicspath{  
{./}
{./whoAreYou/}
{./drawingWithTheTurtle/}
{./bisectionMethod/}
{./circles/}
{./anglesAndRightTriangles/}
{./lawOfSines/}
{./lawOfCosines/}
{./plotter/}
{./staircases/}
{./pitch/}
{./qualityControl/}
{./symmetry/}
{./nGonBlock/}
}


%% page layout
\usepackage[cm,headings]{fullpage}
\raggedright
\setlength\headheight{13.6pt}


%% fonts
\usepackage{euler}

\usepackage{FiraMono}
\renewcommand\familydefault{\ttdefault} 
\usepackage[defaultmathsizes]{mathastext}
\usepackage[htt]{hyphenat}

\usepackage[T1]{fontenc}
\usepackage[scaled=1]{FiraSans}

%\usepackage{wedn}
\usepackage{pbsi} %% Answer font


\usepackage{cancel} %% strike through in pitch/pitch.tex


%% \usepackage{ulem} %% 
%% \renewcommand{\ULthickness}{2pt}% changes underline thickness

\tikzset{>=stealth}

\usepackage{adjustbox}

\setcounter{titlenumber}{-1}

%% journal style
\makeatletter
\newcommand\journalstyle{%
  \def\activitystyle{activity-chapter}
  \def\maketitle{%
    \addtocounter{titlenumber}{1}%
                {\flushleft\small\sffamily\bfseries\@pretitle\par\vspace{-1.5em}}%
                {\flushleft\LARGE\sffamily\bfseries\thetitlenumber\hspace{1em}\@title \par }%
                {\vskip .6em\noindent\textit\theabstract\setcounter{question}{0}\setcounter{sectiontitlenumber}{0}}%
                    \par\vspace{2em}
                    \phantomsection\addcontentsline{toc}{section}{\thetitlenumber\hspace{1em}\textbf{\@title}}%
                     }}
\makeatother



%% thm like environments
\let\question\relax
\let\endquestion\relax

\newtheoremstyle{QuestionStyle}{\topsep}{\topsep}%%% space between body and thm
		{}                      %%% Thm body font
		{}                              %%% Indent amount (empty = no indent)
		{\bfseries}            %%% Thm head font
		{)}                              %%% Punctuation after thm head
		{ }                           %%% Space after thm head
		{\thmnumber{#2}\thmnote{ \bfseries(#3)}}%%% Thm head spec
\theoremstyle{QuestionStyle}
\newtheorem{question}{}



\let\freeResponse\relax
\let\endfreeResponse\relax

%% \newtheoremstyle{ResponseStyle}{\topsep}{\topsep}%%% space between body and thm
%% 		{\wedn\bfseries}                      %%% Thm body font
%% 		{}                              %%% Indent amount (empty = no indent)
%% 		{\wedn\bfseries}            %%% Thm head font
%% 		{}                              %%% Punctuation after thm head
%% 		{3ex}                           %%% Space after thm head
%% 		{\underline{\underline{\thmname{#1}}}}%%% Thm head spec
%% \theoremstyle{ResponseStyle}

\usepackage[tikz]{mdframed}
\mdfdefinestyle{ResponseStyle}{leftmargin=1cm,linecolor=black,roundcorner=5pt,
, font=\bsifamily,}%font=\wedn\bfseries\upshape,}


\ifhandout
\NewEnviron{freeResponse}{}
\else
%\newtheorem{freeResponse}{Response:}
\newenvironment{freeResponse}{\begin{mdframed}[style=ResponseStyle]}{\end{mdframed}}
\fi



%% attempting to automate outcomes.

%% \newwrite\outcomefile
%%   \immediate\openout\outcomefile=\jobname.oc
%% \renewcommand{\outcome}[1]{\edef\theoutcomes{\theoutcomes #1~}%
%% \immediate\write\outcomefile{\unexpanded{\outcome}{#1}}}

%% \newcommand{\outcomelist}{\begin{itemize}\theoutcomes\end{itemize}}

%% \NewEnviron{listOutcomes}{\small\sffamily
%% After answering the following questions, students should be able to:
%% \begin{itemize}
%% \BODY
%% \end{itemize}
%% }
\usepackage[tikz]{mdframed}
\mdfdefinestyle{OutcomeStyle}{leftmargin=2cm,rightmargin=2cm,linecolor=black,roundcorner=5pt,
, font=\small\sffamily,}%font=\wedn\bfseries\upshape,}
\newenvironment{listOutcomes}{\begin{mdframed}[style=OutcomeStyle]After answering the following questions, students should be able to:\begin{itemize}}{\end{itemize}\end{mdframed}}



%% my commands

\newcommand{\snap}{{\bfseries\itshape\textsf{Snap!}}}
\newcommand{\flavor}{\link[\snap]{https://snap.berkeley.edu/}}
\newcommand{\mooculus}{\textsf{\textbf{MOOC}\textnormal{\textsf{ULUS}}}}


\usepackage{tkz-euclide}
\tikzstyle geometryDiagrams=[rounded corners=.5pt,ultra thick,color=black]
\colorlet{penColor}{black} % Color of a curve in a plot



\ifhandout\newcommand{\mynewpage}{\newpage}\else\newcommand{\mynewpage}{}\fi


\author{Bart Snapp}
 
\title{What is going on?}



\begin{document}
\begin{abstract}
  Let me explain. 
\end{abstract}
\maketitle

At this point, we need to rethink
\begin{itemize}
\item what is teaching,
\item what is assessment, and
\item what it means to learn.
\end{itemize}
Our typical notions of the items in the list often go something like
this:
\begin{itemize}
\item A teacher is someone standing at the front of the room talking
  and writing on a backboard,
\item Assessment is an in-class exam, where students work without the
  aid of peers or resources such as books or (gasp!) the internet.
\item Learning is the process of copying down whatever is written on
  the blackboard and then reproducing similar work on the in-class
  exam.
\end{itemize}

Of course this is all BOGUS. We all know we learn best by doing.
So as a TEACHER, I'm going to try an make you DO THINGS.

In the real-world, you are basically assessed on two things:
\begin{itemize}
\item Can you immediately tell when something is BOGUS?
\item If given time and resources, can you complete a project.
\end{itemize}

So our assessments will be just that. Quick easy questions (not
actually \textit{easy}!) and projects.

Finally, we wish to cultivate a sense of ``life-long learning.''  I
this class, you will develop skills for acquiring new knowledge, when
it is needed.



So where does this leave me? I'm the curator of knowledge and your
guide through this journey. I have gathered the



Here is a simplified version of Bloom's Taxonomy of Learning.


\begin{center}
\scalebox{.5}{
\begin{tikzpicture}
\coordinate (A) at (-4.5,1) {};
\coordinate (B) at ( 4.5,1) {};
\coordinate (C) at (0,5) {};
\draw[name path=AC,ultra thick] (A) -- (C);
\draw[name path=BC,ultra thick] (B) -- (C);

\path[name path=horiz] (A|-0,1) -- (B|-0,1);
\path[name path=horizTop] (A|-0,2) -- (B|-0,2);
\draw[ultra thick, black,
  name intersections={of=AC and horiz,by=P},
  name intersections={of=BC and horiz,by=Q},
  name intersections={of=AC and horizTop,by=S},
  name intersections={of=BC and horizTop,by=R},
  fill=purple!50!blue!50!white]
(P) -- (Q) -- (R) -- (S) --(P) -- (Q)
node[midway,above] {\textbf{Remember/Describe}};

\path[name path=horiz] (A|-0,2) -- (B|-0,2);
\path[name path=horizTop] (A|-0,3) -- (B|-0,3);
\draw[ultra thick, black,
  name intersections={of=AC and horiz,by=P},
  name intersections={of=BC and horiz,by=Q},
  name intersections={of=AC and horizTop,by=S},
  name intersections={of=BC and horizTop,by=R},fill=blue!30!green!50!white]
(P) -- (Q) -- (R) -- (S) --(P) -- (Q)
node[midway,above] {\textbf{Apply/Analyze}};


\path[name path=horiz] (A|-0,3) -- (B|-0,3);
\draw[ultra thick, black,
  name intersections={of=AC and horiz,by=P},
  name intersections={of=BC and horiz,by=Q},
  fill=yellow!50!red!50!white]
(P) -- (Q) -- (0,5) --(P) -- (Q)
node[midway,above,black] {\textbf{Evaluate/Create}};
\end{tikzpicture}}
\end{center}






Why is this ``teaching?'' And if it is teaching, why can we not
simply replace the ``teacher'' with a television?

and the skip the whole copying process, and simply give the students a
pre-written book?


, we've been trying to replace teachers with televisions
for at least the last 50 years.

In the past, teaching was the only method of content delivery, 

Blooms taxonomy




\end{document}
