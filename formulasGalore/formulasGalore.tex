\documentclass[handout,nooutcomes,noauthor,hints1,12pt]{ximera}

%% page layout
\usepackage[in,headings]{fullpage}
\raggedright
\setlength\headheight{13.6pt}


%% fonts
\usepackage{euler}

\usepackage{FiraMono}
\renewcommand\familydefault{\ttdefault} 
\usepackage{mathastext}
\usepackage[htt]{hyphenat}

\usepackage[T1]{fontenc}
\usepackage[scaled=1]{FiraSans}

\usepackage{wedn}
\usepackage[T1]{fontenc}

%% wrap text around scripts
\usepackage{wrapfig}

\tikzset{>=stealth}
%% snap! scripts
\usepackage{scratch3}

\usepackage{adjustbox}

%% journal style
\makeatletter
\newcommand\journalstyle{%
  \def\activitystyle{activity-chapter}
  \def\maketitle{%
    \addtocounter{titlenumber}{1}%
                {\flushleft\small\sffamily\bfseries\@pretitle\par\vspace{-1.5em}}%
                {\flushleft\LARGE\sffamily\bfseries\thetitlenumber\hspace{1em}\@title \par }%
                {\vskip .6em\noindent\textit\theabstract\setcounter{question}{0}\setcounter{sectiontitlenumber}{0}}%
                    \par\vspace{2em}
                    \phantomsection\addcontentsline{toc}{section}{\thetitlenumber\hspace{1em}\textbf{\@title}}%
                     }}
\makeatother



%% thm like environments
\let\question\relax
\let\endquestion\relax

\newtheoremstyle{QuestionStyle}{\topsep}{\topsep}%%% space between body and thm
		{}                      %%% Thm body font
		{}                              %%% Indent amount (empty = no indent)
		{\bfseries}            %%% Thm head font
		{)}                              %%% Punctuation after thm head
		{ }                           %%% Space after thm head
		{\thmnumber{#2}\thmnote{ \bfseries(#3)}}%%% Thm head spec
\theoremstyle{QuestionStyle}
\newtheorem{question}{}



\let\freeResponse\relax
\let\endfreeResponse\relax

%% \newtheoremstyle{ResponseStyle}{\topsep}{\topsep}%%% space between body and thm
%% 		{\wedn\bfseries}                      %%% Thm body font
%% 		{}                              %%% Indent amount (empty = no indent)
%% 		{\wedn\bfseries}            %%% Thm head font
%% 		{}                              %%% Punctuation after thm head
%% 		{3ex}                           %%% Space after thm head
%% 		{\underline{\underline{\thmname{#1}}}}%%% Thm head spec
%% \theoremstyle{ResponseStyle}

\usepackage[tikz]{mdframed}
\mdfdefinestyle{ResponseStyle}{leftmargin=1cm,linecolor=black,roundcorner=5pt,frametitlefont=\wedn\bfseries,%frametitle={\underline{\underline{Response}}:}
, font=\wedn\bfseries,}%\begin{mdframed}[style=mystyle]foo\end{mdframed}


\ifhandout
\NewEnviron{freeResponse}{}
\else
%\newtheorem{freeResponse}{Response:}
\newenvironment{freeResponse}{\begin{mdframed}[style=ResponseStyle]}{\end{mdframed}}
\fi



%% attempting to automate outcomes.

\newwrite\outcomefile
  \immediate\openout\outcomefile=\jobname.oc
\renewcommand{\outcome}[1]{\edef\theoutcomes{\theoutcomes #1~}%
\immediate\write\outcomefile{\unexpanded{\outcome}{#1}}}

%% \newcommand{\outcomelist}{\begin{itemize}\theoutcomes\end{itemize}}



%% my commands

\newcommand{\snap}{{\bfseries\itshape\textsf{Snap!}}}
\newcommand{\flavor}{\link[\snap]{https://snap.berkeley.edu/}}


\usepackage{tkz-euclide}
\tikzstyle geometryDiagrams=[rounded corners=.5pt,ultra thick,color=black]
\colorlet{penColor}{black} % Color of a curve in a plot


\title{Formulas galore}

\author{Bart Snapp}

\begin{document}
\begin{abstract}
We make estimates using basic formulas and determine what is, and is
not, a reasonable estimate.
\end{abstract}
\maketitle

\begin{listOutcomes}
\item Distinguish between reasonable and unreasonable estimates.
\item Determine a reasonable estimate before preforming a calculation.
\item Apply basic geometric formulas.
\end{listOutcomes}

Here are a bunch of formulas Dr.\ Snapp has found useful in the past. %% found on the internet at this site
%% \begin{center}
%%   \textit{https://sites.google.com/site/standardbasicengineering/home/useful-formula-for-finding-area-volume-of-solid-figures}
%% \end{center}
%% \begin{center}
%%   \includegraphics[width=.4\textwidth]{photoFormula1.jpg} \qquad \includegraphics[width=.4\textwidth]{photoFormula2.jpg}
%% \end{center}
%\begin{mdframed}[style=OutcomeStyle]
  \textbf{Formulas:} For three-dimensional solids, $A$ is the area of
  the base.

  \begin{description}
  \item[Rectangle]$A= b\cdot h$
    \hfil\raisebox{-2cm}{
      \begin{tikzpicture}
      \draw[ultra thick] (0,0) rectangle (3,1.5);
      \node[below] at (1.5,0) {$b$};
      \node[right] at (3,.75) {$h$};
    \end{tikzpicture}}
    
  \item[Triangle] $A=\frac{b\cdot h}{2}$
    \hfil\raisebox{-2cm}{
     \begin{tikzpicture}
      \draw[ultra thick] (0,0) -- (2,0) -- (3,1.5) -- (0,0) -- cycle;
      \draw[dashed] (3,1.5) -- (3,0);
      \node[below] at (,0) {$b$};
      \node[right] at (3,.75) {$h$};
    \end{tikzpicture}}
  \item[Sheared Prism] $V = A \cdot h$.
    \hfil\raisebox{-2cm}{
      \begin{tikzpicture}
      \draw[ultra thick] plot [smooth cycle] coordinates  {(0,0) (1,.5) (2,.6) (3,.5) (3,1) (2,1.5) (0,1) };

      \draw[ultra thick] plot [smooth cycle] coordinates  {(1,1.5) (2,2) (3,2.1) (4,2) (4,2.5) (3,3) (1,2.5) };

      \draw[ultra thick] (0,0) -- (1,1.5);

      %% \draw[ultra thick] (1,.5) -- (2,2);

      %% \draw[ultra thick] (2,.6) -- (3,2.1);

      \draw[ultra thick] (3,.5) -- (4,2);

      \draw[ultra thick] (0,1) -- (1,2.5);

      
      \draw[ultra thick,dashed] (3,1) -- (4,2.5);
     
      \draw[ultra thick,dashed] (2,1.5) -- (3,3);
      

      
      \draw[dashed] (4,2) -- (4,.5);
      \node at (2,1) {$A$};
      \node[right] at (4,1.25) {$h$};
    \end{tikzpicture}}
  \item[Cone] $V = \frac{A\cdot h}{3}$ %$SA = A \cdot \sqrt{1 + \left(\frac{h}{r}\right)^2}$
    \hfil\raisebox{-2cm}{
      \begin{tikzpicture}
      \draw[ultra thick] plot [smooth cycle] coordinates  {(0,0) (1,.5) (2,.6) (3,.5) (3,1) (2,1.5) (0,1) };

      

      \draw[ultra thick] (0,0) -- (4,2.5);

      %% \draw[ultra thick] (1,.5) -- (2,2);

      %% \draw[ultra thick] (2,.6) -- (3,2.1);

      \draw[ultra thick] (3,.5) -- (4,2.5);

      \draw[ultra thick] (0,1) -- (4,2.5);

      
      \draw[ultra thick,dashed] (3,1) -- (4,2.5);
     
      \draw[ultra thick,dashed] (2,1.5) -- (4,2.5);
      \draw [fill=black] (4,2.5) circle (.05);


      
      \draw[dashed] (4,2) -- (4,.5);
      \node at (2,1) {$A$};
      \node[right] at (4,1.25) {$h$};
    \end{tikzpicture}}
  \item[Sphere] $V = \frac{4\pi r^3}{3}$ \qquad $SA = 4\pi r^2$
    \hfil\raisebox{-2cm}{\begin{tikzpicture}
      %% sphere
      \draw[ultra thick] (1.75,0) arc (0:360:1.75);
      %%%% Equator
      \draw[ultra thick] (-1.75,0) arc (180:360:1.75 and .5);
      \draw[ultra thick,dashed] (1.75,0) arc (0:180:1.75 and .5);

      \draw [fill=black] (0,0) circle (.05);

      \draw (0,0) -- (1.75,0);
      \node[above] at (.8,0) {$r$};
    \end{tikzpicture}}
  \end{description}
% \end{mdframed}




\mynewpage




\begin{question} \label{FG1:1}
  Let's compare estimates made with formulas and estimates made with
  simple boxes. Consider the \link[\textit{Ericsson
      Globe}]{https://en.wikipedia.org/wiki/Ericsson_Globe}:
   \begin{center}
    \includegraphics[width=.4\textwidth]{dome.png} %https://www.researchgate.net/figure/Example-of-a-hemispherical-dome-in-modern-civil-engineering-Stockholm-Globe-Arena-by_fig2_308655043
   \end{center}
   Assuming that this building has a diameter of $110$ meters and a
   height of $85$ meters.
   
   Use the formulas above to find the surface area
   and the volume of the \textit{Ericsson Globe}.
\begin{enumerate}
\item Estimate these values by finding the surface area and volume of
  a box with dimensions $100\times 100 \times 80$ meters. Show all work.
\item Compare the estimates above with estimates using the formulas
  found at the beginning of this activity. Explain your work and
  describe the percentage difference between the two estimates.
\end{enumerate}
\begin{freeResponse}
  \begin{enumerate}
  \item So if we use a box with dimensions $100\times 100 \times 80$ meters,
    we find the surface area is
    \[
    4\times 80 \times 100 +100\times 100= 42000~\text{square meters}
    \]
    and the volume is
    \[
    100\times 100 \times 80 = 800000~\text{cubic meters.}
    \]
    The box estimate is around $40\%$ greater than the formula
    estimate.
  \item If we use the formulas, the surface area will be
    \[
    2\pi rh
    \]
    where $h=85$ and $r= 110/2 = 55$, so
    \[
    2\pi 55\cdot 85 \approx  29000~\text{square meters}.
    \]
    The volume will be
    \[
    \pi h^2 \left( r- \frac{h}{3}\right)
    \]
    so
    \[
    \pi \times 85^2 (55 - 85/3) \approx 600000 ~\text{cubic meters.}
    \]
    The box estimate is around $33\%$ bigger than the formula
    estimate.
  \end{enumerate}
\end{freeResponse}
   \end{question}
\mynewpage

 \begin{question}   Consider the \link[\textit{Luxor Las Vegas}]{https://en.wikipedia.org/wiki/Luxor_Las_Vegas}.  Assume that the (square) base of this building has a side length of
 $646$ feet and that the pyramid is $350$ feet tall:
  \begin{center}
    \includegraphics[width=.4\textwidth]{pyramid.jpg} 
  \end{center}

  \begin{enumerate}
  \item Which is \textbf{easier} to compute:
    \begin{itemize}
    \item the surface area (not including the bottom!) OR
    \item the volume
    \end{itemize}
    of the \textit{Luxor Las Vegas}?
    Explain why one computation is more difficult than the other.
    \item Compute the EASIER computation of the two above. Show all
      work.
  \end{enumerate}

   \begin{freeResponse}
    \begin{enumerate}
      \item The volume is MUCH EASIER to compute than the surface
        area. To compute the surface area, you have to compute the
        area of each face of the pyramid, this requires GEOMETRY.

      \item The volume is
        \[
        V = \frac{646\cdot 646\cdot 350}{3} \approx = 48686867 \text{ cubic feet}.
        \]
    \end{enumerate}
  \end{freeResponse}
\end{question}

\mynewpage

 \begin{question}%
    You ask six friends what the surface area of the \textit{Luxor Las
      Vegas} is (not counting the floor!). They give you six wrong
    answers, but you love them anyway, because they are your
    friends. Here are their answers:
    \begin{enumerate}
    \item $650$ square feet
    \item $6500$ square feet
    \item $65000$ square feet
    \item\label{fg:correct} $650000$ square feet
    \item $6500000$ square feet
    \item $65000000$ square feet
    \end{enumerate}
      Which answer above is \textbf{closest} to the correct answer?
      Explain your reasoning.
      \begin{freeResponse}
        Answer $\ref{fg:correct}$ is correct, here's how I know. The pyramid has a
        base, and four more sides. The BIGGEST the surface area could
        be is
        \[
        4\cdot 350\cdot 646 = 904400.
        \]
        On the other hand, the area of the SMALLEST the surface area could be is
        the area of the base, or
        \[
        646\cdot 646 = 417316.
        \]
        Note, $417316 < 650000 <  904400$.        
      \end{freeResponse}
\end{question}





\end{document}
