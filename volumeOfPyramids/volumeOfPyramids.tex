\documentclass[nooutcomes,noauthor,handout,hints,12pt]{ximera}

\graphicspath{  
{./}
{./whoAreYou/}
{./drawingWithTheTurtle/}
{./bisectionMethod/}
{./circles/}
{./anglesAndRightTriangles/}
{./lawOfSines/}
{./lawOfCosines/}
{./plotter/}
{./staircases/}
{./pitch/}
{./qualityControl/}
{./symmetry/}
{./nGonBlock/}
}


%% page layout
\usepackage[cm,headings]{fullpage}
\raggedright
\setlength\headheight{13.6pt}


%% fonts
\usepackage{euler}

\usepackage{FiraMono}
\renewcommand\familydefault{\ttdefault} 
\usepackage[defaultmathsizes]{mathastext}
\usepackage[htt]{hyphenat}

\usepackage[T1]{fontenc}
\usepackage[scaled=1]{FiraSans}

%\usepackage{wedn}
\usepackage{pbsi} %% Answer font


\usepackage{cancel} %% strike through in pitch/pitch.tex


%% \usepackage{ulem} %% 
%% \renewcommand{\ULthickness}{2pt}% changes underline thickness

\tikzset{>=stealth}

\usepackage{adjustbox}

\setcounter{titlenumber}{-1}

%% journal style
\makeatletter
\newcommand\journalstyle{%
  \def\activitystyle{activity-chapter}
  \def\maketitle{%
    \addtocounter{titlenumber}{1}%
                {\flushleft\small\sffamily\bfseries\@pretitle\par\vspace{-1.5em}}%
                {\flushleft\LARGE\sffamily\bfseries\thetitlenumber\hspace{1em}\@title \par }%
                {\vskip .6em\noindent\textit\theabstract\setcounter{question}{0}\setcounter{sectiontitlenumber}{0}}%
                    \par\vspace{2em}
                    \phantomsection\addcontentsline{toc}{section}{\thetitlenumber\hspace{1em}\textbf{\@title}}%
                     }}
\makeatother



%% thm like environments
\let\question\relax
\let\endquestion\relax

\newtheoremstyle{QuestionStyle}{\topsep}{\topsep}%%% space between body and thm
		{}                      %%% Thm body font
		{}                              %%% Indent amount (empty = no indent)
		{\bfseries}            %%% Thm head font
		{)}                              %%% Punctuation after thm head
		{ }                           %%% Space after thm head
		{\thmnumber{#2}\thmnote{ \bfseries(#3)}}%%% Thm head spec
\theoremstyle{QuestionStyle}
\newtheorem{question}{}



\let\freeResponse\relax
\let\endfreeResponse\relax

%% \newtheoremstyle{ResponseStyle}{\topsep}{\topsep}%%% space between body and thm
%% 		{\wedn\bfseries}                      %%% Thm body font
%% 		{}                              %%% Indent amount (empty = no indent)
%% 		{\wedn\bfseries}            %%% Thm head font
%% 		{}                              %%% Punctuation after thm head
%% 		{3ex}                           %%% Space after thm head
%% 		{\underline{\underline{\thmname{#1}}}}%%% Thm head spec
%% \theoremstyle{ResponseStyle}

\usepackage[tikz]{mdframed}
\mdfdefinestyle{ResponseStyle}{leftmargin=1cm,linecolor=black,roundcorner=5pt,
, font=\bsifamily,}%font=\wedn\bfseries\upshape,}


\ifhandout
\NewEnviron{freeResponse}{}
\else
%\newtheorem{freeResponse}{Response:}
\newenvironment{freeResponse}{\begin{mdframed}[style=ResponseStyle]}{\end{mdframed}}
\fi



%% attempting to automate outcomes.

%% \newwrite\outcomefile
%%   \immediate\openout\outcomefile=\jobname.oc
%% \renewcommand{\outcome}[1]{\edef\theoutcomes{\theoutcomes #1~}%
%% \immediate\write\outcomefile{\unexpanded{\outcome}{#1}}}

%% \newcommand{\outcomelist}{\begin{itemize}\theoutcomes\end{itemize}}

%% \NewEnviron{listOutcomes}{\small\sffamily
%% After answering the following questions, students should be able to:
%% \begin{itemize}
%% \BODY
%% \end{itemize}
%% }
\usepackage[tikz]{mdframed}
\mdfdefinestyle{OutcomeStyle}{leftmargin=2cm,rightmargin=2cm,linecolor=black,roundcorner=5pt,
, font=\small\sffamily,}%font=\wedn\bfseries\upshape,}
\newenvironment{listOutcomes}{\begin{mdframed}[style=OutcomeStyle]After answering the following questions, students should be able to:\begin{itemize}}{\end{itemize}\end{mdframed}}



%% my commands

\newcommand{\snap}{{\bfseries\itshape\textsf{Snap!}}}
\newcommand{\flavor}{\link[\snap]{https://snap.berkeley.edu/}}
\newcommand{\mooculus}{\textsf{\textbf{MOOC}\textnormal{\textsf{ULUS}}}}


\usepackage{tkz-euclide}
\tikzstyle geometryDiagrams=[rounded corners=.5pt,ultra thick,color=black]
\colorlet{penColor}{black} % Color of a curve in a plot



\ifhandout\newcommand{\mynewpage}{\newpage}\else\newcommand{\mynewpage}{}\fi

\title{Volume of pyramids}

\author{Jenny Sheldon \and Bart Snapp}

\begin{document}
\begin{abstract}
  We build a formula for the volume of a pyramid based on the volume
  of a box and observe relationships between surface area and volume.
\end{abstract}
\maketitle


\begin{listOutcomes}
\item Dissect a cube into three right pyramids.
\item Find the volume of right pyramid.
\item State Cavalieri's shearing principle in three dimensions.
\item Apply Cavalieri's shearing principle to derive the volume of ANY pyramid
  of height $s$ over an $s\times s$ square base.
\end{listOutcomes}

%% \begin{listObjectives}
%%  \item Explain why presented concepts and formulas are true,
%%  \item Take a solution to a problem in a given context, and applying that same solution to another problem,
%%  \item Increase student confidence in their ability to solve difficult math problems by using previous results, trying different methods, asking questions, and working with others,
%%  \item Improve student’s mathematical communication skills.

%% \end{listObjectives}

\mynewpage


\begin{question}
  Give detailed diagrams with explanation showing that a cube can be
  constructed from three identical square-based pyramids.
  \begin{freeResponse}
    Here they are:
    \begin{center}
    \begin{tikzpicture}
      \coordinate (A) at (0,0,0);
      \coordinate (B) at (4,0,0);
      \coordinate (C) at (4,4,0);
      \coordinate (D) at (0,4,0);
      \coordinate (A') at (0,0,4);
      \coordinate (B') at (4,0,4);
      \coordinate (C') at (4,4,4);
      \coordinate (D') at (0,4,4);
      
      
      
      %\draw[fill=red,opacity=0.3] (A)--(B)--(C)--(D)--cycle; % base
      %\draw[fill=blue,opacity=0.3] (A')--(B')--(C')--(D')--cycle; % base
      %% \draw[fill=orange,opacity=0.3] (B1)--(B2)--(peak);
      %% \draw[fill=yellow,opacity=0.3] (B2)--(B3)--(peak);
      %% \draw[fill=green,opacity=0.3] (B3)--(B4)--(peak);
      %% \draw[fill=blue,opacity=0.3] (B4)--(B5)--(peak);
      %% \draw[fill=purple,opacity=0.3] (B5)--(B1)--(peak);

      
      \draw (A)--(B)--(C)--(D)--cycle;
      \draw (A)--(A');
      \draw (B)--(B');
      \draw (C)--(C');
      \draw (D)--(D');
      \draw (A')--(B')--(C')--(D')--cycle;
      
      \draw[line width=3.5pt,red] (A)--(A');
      \draw[line width=3.5pt,red] (A')--(B');
      \draw[line width=3.5pt,red] (B)--(B');
      \draw[line width=3.5pt,red] (B)--(A);
      \draw[line width=3.5pt,red] (A')--(D);
      \draw[line width=3.5pt,red] (B)--(D);
      \draw[line width=3.5pt, red] (B')--(D);
      \draw[line width=3.5pt, red] (A)--(D);

      \draw[line width=2pt,blue] (C')--(C);
      \draw[line width=2pt,blue] (C')--(D);
      \draw[line width=2pt,blue] (C)--(D);
      \draw[line width=2pt,blue] (B)--(D);
      \draw[line width=2pt,blue] (B')--(D); %% diagonal
      \draw[line width=2pt,blue] (B')--(B);
      \draw[line width=2pt, blue] (B')--(C');
      \draw[line width=2pt, blue] (B)--(C);

      
    \end{tikzpicture}
  \end{center}
\end{freeResponse}
\end{question}
\mynewpage


\begin{question} 
  Use your work above to derive a formula for the volume of an
  individual pyramid from the previous problem. Assume that the height
  is $s$ with an $s\times s$ square base.
  \begin{freeResponse}
    The volume of a cube is $s^3$. Since $3$ right pyramids make up
    this cube, a right pyramid of side length $s$ and height $s$ must
    have a volume of:
    \[
    \frac{s^3}{3}
    \]
  \end{freeResponse}
\end{question}
\mynewpage

\begin{question}
  Let us work analogously to what we did in the triangle case.
  \begin{enumerate}
  \item State Cavalieri's principle in three dimensions. Explain what Cavalieri's principle means using words, pictures etc.
  \item Give a formula for the volume of \emph{any} pyramid of height $s$ over an $s\times
    s$ square base.
 % \item Give a formula for the volume of \emph{any} pyramid of height
 %   $s$ over a base of area $A$.
  \end{enumerate}
    \begin{freeResponse}
      \begin{enumerate}
      \item In three dimensions, Cavalieri's principle states:
        \begin{quote}
          Shearing parallel to a fixed direction does not change the
          volume of a three dimensional object.
        \end{quote}
      \item Easy, just shear the pyramid! By Cavalieri it must have
        the same volume as a right pyramid. Hence it is still $s^3/3$.
      \end{enumerate}
    \end{freeResponse}
\end{question}


\end{document}
