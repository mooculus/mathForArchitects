\documentclass{ximera}

\graphicspath{  
{./}
{./whoAreYou/}
{./drawingWithTheTurtle/}
{./bisectionMethod/}
{./circles/}
{./anglesAndRightTriangles/}
{./lawOfSines/}
{./lawOfCosines/}
{./plotter/}
{./staircases/}
{./pitch/}
{./qualityControl/}
{./symmetry/}
{./nGonBlock/}
}


%% page layout
\usepackage[cm,headings]{fullpage}
\raggedright
\setlength\headheight{13.6pt}


%% fonts
\usepackage{euler}

\usepackage{FiraMono}
\renewcommand\familydefault{\ttdefault} 
\usepackage[defaultmathsizes]{mathastext}
\usepackage[htt]{hyphenat}

\usepackage[T1]{fontenc}
\usepackage[scaled=1]{FiraSans}

%\usepackage{wedn}
\usepackage{pbsi} %% Answer font


\usepackage{cancel} %% strike through in pitch/pitch.tex


%% \usepackage{ulem} %% 
%% \renewcommand{\ULthickness}{2pt}% changes underline thickness

\tikzset{>=stealth}

\usepackage{adjustbox}

\setcounter{titlenumber}{-1}

%% journal style
\makeatletter
\newcommand\journalstyle{%
  \def\activitystyle{activity-chapter}
  \def\maketitle{%
    \addtocounter{titlenumber}{1}%
                {\flushleft\small\sffamily\bfseries\@pretitle\par\vspace{-1.5em}}%
                {\flushleft\LARGE\sffamily\bfseries\thetitlenumber\hspace{1em}\@title \par }%
                {\vskip .6em\noindent\textit\theabstract\setcounter{question}{0}\setcounter{sectiontitlenumber}{0}}%
                    \par\vspace{2em}
                    \phantomsection\addcontentsline{toc}{section}{\thetitlenumber\hspace{1em}\textbf{\@title}}%
                     }}
\makeatother



%% thm like environments
\let\question\relax
\let\endquestion\relax

\newtheoremstyle{QuestionStyle}{\topsep}{\topsep}%%% space between body and thm
		{}                      %%% Thm body font
		{}                              %%% Indent amount (empty = no indent)
		{\bfseries}            %%% Thm head font
		{)}                              %%% Punctuation after thm head
		{ }                           %%% Space after thm head
		{\thmnumber{#2}\thmnote{ \bfseries(#3)}}%%% Thm head spec
\theoremstyle{QuestionStyle}
\newtheorem{question}{}



\let\freeResponse\relax
\let\endfreeResponse\relax

%% \newtheoremstyle{ResponseStyle}{\topsep}{\topsep}%%% space between body and thm
%% 		{\wedn\bfseries}                      %%% Thm body font
%% 		{}                              %%% Indent amount (empty = no indent)
%% 		{\wedn\bfseries}            %%% Thm head font
%% 		{}                              %%% Punctuation after thm head
%% 		{3ex}                           %%% Space after thm head
%% 		{\underline{\underline{\thmname{#1}}}}%%% Thm head spec
%% \theoremstyle{ResponseStyle}

\usepackage[tikz]{mdframed}
\mdfdefinestyle{ResponseStyle}{leftmargin=1cm,linecolor=black,roundcorner=5pt,
, font=\bsifamily,}%font=\wedn\bfseries\upshape,}


\ifhandout
\NewEnviron{freeResponse}{}
\else
%\newtheorem{freeResponse}{Response:}
\newenvironment{freeResponse}{\begin{mdframed}[style=ResponseStyle]}{\end{mdframed}}
\fi



%% attempting to automate outcomes.

%% \newwrite\outcomefile
%%   \immediate\openout\outcomefile=\jobname.oc
%% \renewcommand{\outcome}[1]{\edef\theoutcomes{\theoutcomes #1~}%
%% \immediate\write\outcomefile{\unexpanded{\outcome}{#1}}}

%% \newcommand{\outcomelist}{\begin{itemize}\theoutcomes\end{itemize}}

%% \NewEnviron{listOutcomes}{\small\sffamily
%% After answering the following questions, students should be able to:
%% \begin{itemize}
%% \BODY
%% \end{itemize}
%% }
\usepackage[tikz]{mdframed}
\mdfdefinestyle{OutcomeStyle}{leftmargin=2cm,rightmargin=2cm,linecolor=black,roundcorner=5pt,
, font=\small\sffamily,}%font=\wedn\bfseries\upshape,}
\newenvironment{listOutcomes}{\begin{mdframed}[style=OutcomeStyle]After answering the following questions, students should be able to:\begin{itemize}}{\end{itemize}\end{mdframed}}



%% my commands

\newcommand{\snap}{{\bfseries\itshape\textsf{Snap!}}}
\newcommand{\flavor}{\link[\snap]{https://snap.berkeley.edu/}}
\newcommand{\mooculus}{\textsf{\textbf{MOOC}\textnormal{\textsf{ULUS}}}}


\usepackage{tkz-euclide}
\tikzstyle geometryDiagrams=[rounded corners=.5pt,ultra thick,color=black]
\colorlet{penColor}{black} % Color of a curve in a plot



\ifhandout\newcommand{\mynewpage}{\newpage}\else\newcommand{\mynewpage}{}\fi


\author{Jenny Sheldon \and Bart Snapp}

\begin{document}

\begin{exercise}
  If someone wanted to plot the graph of $y=x^2$, they might start
  by filling in the following table:
  \[
  \begin{array}{r | c}
    x & x^2 \\
    \hline\hline
    0  & \answer{0} \\\hline
    1  & \answer{1} \\\hline
    -1 & \answer{1} \\\hline
    2  & \answer{4} \\\hline
    -2 & \answer{4} \\\hline
    3  & \answer{9} \\\hline
    -3 & \answer{9} 
  \end{array}
  \]
  \begin{exercise}
  Reflect each point you obtain from the table above about the line
  $y=x$.
  \begin{prompt}
    \begin{align*}
      \mat{F}_{y=x} (0,0) &= \left(\answer{0},\answer{0}\right)\\
      \mat{F}_{y=x} (1,1) &= \left(\answer{1},\answer{1}\right)\\
      \mat{F}_{y=x} (-1,1) &= \left(\answer{1},\answer{-1}\right)\\
      \mat{F}_{y=x} (2,4) &= \left(\answer{4},\answer{2}\right)\\
      \mat{F}_{y=x} (-2,4) &= \left(\answer{4},\answer{-2}\right)\\
      \mat{F}_{y=x} (3,9) &= \left(\answer{9},\answer{3}\right)\\
      \mat{F}_{y=x} (-3,9) &= \left(\answer{9},\answer{-3}\right)
    \end{align*}
  \end{prompt}
  \begin{exercise}
    Give a plot of the new points and connect the dots in a reasonable
    way. (You should do this with pencil and paper.) What curve do you
    obtain?
    \begin{multipleChoice}
      \choice[correct]{I've plotted this.}
    \end{multipleChoice}
    \begin{feedback}
    \begin{image}
      \begin{tikzpicture}  
        \begin{axis}[  
            xmin=-10.5,  
            xmax=10.5,  
            ymin=-10.5,  
            ymax=10.5,  
        axis lines=center,  
        xlabel=$x$,  
        ylabel=$y$,  
        every axis y label/.style={at=(current axis.above origin),anchor=south},  
        every axis x label/.style={at=(current axis.right of origin),anchor=west},  
          ]  
          %\addplot [ultra thick, blue, smooth] {x^2};
          \addplot [ultra thick, red, smooth] ({x^2},{x});  
        \end{axis}  
      \end{tikzpicture}  
    \end{image}
    You get a horizontal parabola. This is the inverse of $y=x^2$ (note, it is not a function of $x$).
    \end{feedback}
  \end{exercise}
  \end{exercise}
\end{exercise}
\end{document}
