\documentclass{ximera}

%% page layout
\usepackage[in,headings]{fullpage}
\raggedright
\setlength\headheight{13.6pt}


%% fonts
\usepackage{euler}

\usepackage{FiraMono}
\renewcommand\familydefault{\ttdefault} 
\usepackage{mathastext}
\usepackage[htt]{hyphenat}

\usepackage[T1]{fontenc}
\usepackage[scaled=1]{FiraSans}

\usepackage{wedn}
\usepackage[T1]{fontenc}

%% wrap text around scripts
\usepackage{wrapfig}

\tikzset{>=stealth}
%% snap! scripts
\usepackage{scratch3}

\usepackage{adjustbox}

%% journal style
\makeatletter
\newcommand\journalstyle{%
  \def\activitystyle{activity-chapter}
  \def\maketitle{%
    \addtocounter{titlenumber}{1}%
                {\flushleft\small\sffamily\bfseries\@pretitle\par\vspace{-1.5em}}%
                {\flushleft\LARGE\sffamily\bfseries\thetitlenumber\hspace{1em}\@title \par }%
                {\vskip .6em\noindent\textit\theabstract\setcounter{question}{0}\setcounter{sectiontitlenumber}{0}}%
                    \par\vspace{2em}
                    \phantomsection\addcontentsline{toc}{section}{\thetitlenumber\hspace{1em}\textbf{\@title}}%
                     }}
\makeatother



%% thm like environments
\let\question\relax
\let\endquestion\relax

\newtheoremstyle{QuestionStyle}{\topsep}{\topsep}%%% space between body and thm
		{}                      %%% Thm body font
		{}                              %%% Indent amount (empty = no indent)
		{\bfseries}            %%% Thm head font
		{)}                              %%% Punctuation after thm head
		{ }                           %%% Space after thm head
		{\thmnumber{#2}\thmnote{ \bfseries(#3)}}%%% Thm head spec
\theoremstyle{QuestionStyle}
\newtheorem{question}{}



\let\freeResponse\relax
\let\endfreeResponse\relax

%% \newtheoremstyle{ResponseStyle}{\topsep}{\topsep}%%% space between body and thm
%% 		{\wedn\bfseries}                      %%% Thm body font
%% 		{}                              %%% Indent amount (empty = no indent)
%% 		{\wedn\bfseries}            %%% Thm head font
%% 		{}                              %%% Punctuation after thm head
%% 		{3ex}                           %%% Space after thm head
%% 		{\underline{\underline{\thmname{#1}}}}%%% Thm head spec
%% \theoremstyle{ResponseStyle}

\usepackage[tikz]{mdframed}
\mdfdefinestyle{ResponseStyle}{leftmargin=1cm,linecolor=black,roundcorner=5pt,frametitlefont=\wedn\bfseries,%frametitle={\underline{\underline{Response}}:}
, font=\wedn\bfseries,}%\begin{mdframed}[style=mystyle]foo\end{mdframed}


\ifhandout
\NewEnviron{freeResponse}{}
\else
%\newtheorem{freeResponse}{Response:}
\newenvironment{freeResponse}{\begin{mdframed}[style=ResponseStyle]}{\end{mdframed}}
\fi



%% attempting to automate outcomes.

\newwrite\outcomefile
  \immediate\openout\outcomefile=\jobname.oc
\renewcommand{\outcome}[1]{\edef\theoutcomes{\theoutcomes #1~}%
\immediate\write\outcomefile{\unexpanded{\outcome}{#1}}}

%% \newcommand{\outcomelist}{\begin{itemize}\theoutcomes\end{itemize}}



%% my commands

\newcommand{\snap}{{\bfseries\itshape\textsf{Snap!}}}
\newcommand{\flavor}{\link[\snap]{https://snap.berkeley.edu/}}


\usepackage{tkz-euclide}
\tikzstyle geometryDiagrams=[rounded corners=.5pt,ultra thick,color=black]
\colorlet{penColor}{black} % Color of a curve in a plot


\author{Jenny Sheldon \and Bart Snapp}


\outcome{Explain what is meant by a reflection.}
\outcome{Compute reflections using matrix multiplication.}
\outcome{Use the distance formula to show that a given reflection is an isometry.}

\title{Reflections}

\begin{document}
\begin{abstract}
  We view reflections as matrices.
\end{abstract}
\maketitle

The act of reflection has fascinated humanity for millennia.  It has a
strong effect on our perception of beauty and has a defined place in
art---not to mention how useful it is for the application
of make-up. Here is our definition of a reflection:

\begin{definition}
The \dfn{reflection} across a line $\l$, denoted by $\mat{F}_\l$, is the
function that maps a point $\vec{p}$ to a point $\mat{F}_\l
\vec{p}$ such that:
\begin{enumerate}
\item If $\vec{p}$ is on $\l$, then $\mat{F}_\l \vec{p} = \vec{p}$.
\item If $\vec{p}$ is not on $\l$, then $\l$ is the perpendicular
  bisector of the segment connecting $\vec{p}$ and $\mat{F}_\l \vec{p}$. 
\end{enumerate}
\end{definition}

You might be saying, ``Huh?''  It's not as hard as it looks.  Check
out this picture of the situation; again Louie Llama\index{Louie Llama} 
will help us out:
\begin{image}
\includegraphics{refIdeaEg.pdf}
\end{image}

\subsubsection{A Collection of Reflections}

We are going to begin with a trio of reflections. We'll start with a
\textbf{horizontal reflection}\index{horizontal reflection}\index{reflection!horizontal} across the $y$-axis.  Using
our matrix notation, we write:
\[
\mat{F}_{x=0} =
\begin{bmatrix}
-1 & 0 & 0\\
 0 & 1 & 0\\
 0 & 0 & 1
\end{bmatrix}
\]

The next reflection in our collection is a \textbf{vertical
  reflection}\index{vertical reflection}\index{reflection!vertical}
across the $x$-axis.  Using our matrix notation, we write:
\[
\mat{F}_{y=0} =
\begin{bmatrix}
1 &  0 & 0\\
0 & -1 & 0\\
0 &  0 & 1
\end{bmatrix}
\]
The final reflection to add to our collection is a \textbf{diagonal
  reflection}\index{diagonal reflection}\index{reflection!diagonal}
across the line $y=x$.  Using our matrix notation, we write:
\[
\mat{F}_{y=x} =
\begin{bmatrix}
0 & 1 & 0\\
1 & 0 & 0\\
0 & 0 & 1
\end{bmatrix}
\]


\begin{example}
Consider the point $\vec{p} =(3,-1)$.  Use a matrix to reflect
$\vec{p}$ across the line $y = x$.
\begin{image}
\includegraphics{refEg1.pdf}
\end{image}
\begin{explanation}
Here is how you do it:
\begin{align*}
\mat{F}_{y = x} \vec{p} &= 
\begin{bmatrix}
0 & 1 & 0 \\ 
1 & 0 & 0 \\
0 & 0 & 1
\end{bmatrix}
\begin{bmatrix}
3 \\
-1 \\
1
\end{bmatrix}\\
&=
\begin{bmatrix}
\answer[given]{-1}\\
\answer[given]{3} \\
\answer[given]{1}
\end{bmatrix}
\end{align*}
Hence we end up with the point $\left(\answer[given]{-1},\answer[given]{3}\right)$.
\end{explanation}
\end{example}

\begin{question}
Let $\vec{p}$ be some point in Quadrant I of the $(x,y)$-plane. What
reflection will map this point to Quadrant II?
\begin{hint}
  If you have forgotten about the quadrants of hte $(x,y)$-plane, checkout \link[this]{https://en.wikipedia.org/wiki/Quadrant_(plane_geometry)}.
\end{hint}
\begin{prompt}
  To map a point in Quadrant I to Quadrant II, use the reflection $\mat{F}_{\answer{x}=\answer{0}}$
\end{prompt}
\begin{question}
  What about mapping a point in Quadrant I to Quadrant IV?
  \begin{prompt}
    To map a point in Quadrant I to Quadrant IV, use the reflection $\mat{F}_{\answer{y}=\answer{0}}$
  \end{prompt}
  \begin{question}
  What about Quadrant III?
  \begin{prompt}
    None of the reflections explicitly given in this section will do
    the trick. Instead you need to find a line of the form $y=
    \text{something}$ that will work. Thinking a bit, we see
    $\mat{F}_{y= \answer{-x}}$ does the trick.
  \end{prompt}
  \end{question}
\end{question}
\end{question}

\begin{question} 
  Can you demonstrate with algebra why $\mat{F}_{y=x}$ is an isometry?


  \begin{prompt}
    To start we need two points, say $\vec{a} = (a,b)$ and $\vec{p} =
    (p,q)$. Now compute the distance between these two points using
    the distance formula:
    \[
    d(\vec{a},\vec{p}) = \answer{\sqrt{(a-p)^2 + (b-q)^2}}
    \]
    Now we will reflect these two points across the line $y=x$ using
    $\mat{F}_{y=x}$ and show that the distance between the reflected
    points is the same distance found above. Write with me:
    \[
    \mat{F}_{y=x} \vec{a} =
    \begin{bmatrix}
      \answer{b}\\
      \answer{a}
    \end{bmatrix}
    \]
    and
    \[
    \mat{F}_{y=x} \vec{p} =
    \begin{bmatrix}
      \answer{q}\\
      \answer{p}
    \end{bmatrix}
    \]
    Now, compute the distance between $\mat{F}_{y=x} \vec{a}$ and
    $\mat{F}_{y=x} \vec{p}$:
    \[
    d(\mat{F}_{y=x} \vec{a}, \mat{F}_{y=x} \vec{p}) =
    \answer{\sqrt{(b-q)^2 + (a-p)^2}}.
    \]
    Since the distances found \wordChoice{\choice{are
        different}\choice[correct]{are the same}}, we see that $\mat{F}_{y=x}$ is
    an isometry.
  \end{prompt}
\end{question}



\end{document}
