\documentclass[noauthor,nooutcomes,12pt]{ximera}

%% page layout
\usepackage[in,headings]{fullpage}
\raggedright
\setlength\headheight{13.6pt}


%% fonts
\usepackage{euler}

\usepackage{FiraMono}
\renewcommand\familydefault{\ttdefault} 
\usepackage{mathastext}
\usepackage[htt]{hyphenat}

\usepackage[T1]{fontenc}
\usepackage[scaled=1]{FiraSans}

\usepackage{wedn}
\usepackage[T1]{fontenc}

%% wrap text around scripts
\usepackage{wrapfig}

\tikzset{>=stealth}
%% snap! scripts
\usepackage{scratch3}

\usepackage{adjustbox}

%% journal style
\makeatletter
\newcommand\journalstyle{%
  \def\activitystyle{activity-chapter}
  \def\maketitle{%
    \addtocounter{titlenumber}{1}%
                {\flushleft\small\sffamily\bfseries\@pretitle\par\vspace{-1.5em}}%
                {\flushleft\LARGE\sffamily\bfseries\thetitlenumber\hspace{1em}\@title \par }%
                {\vskip .6em\noindent\textit\theabstract\setcounter{question}{0}\setcounter{sectiontitlenumber}{0}}%
                    \par\vspace{2em}
                    \phantomsection\addcontentsline{toc}{section}{\thetitlenumber\hspace{1em}\textbf{\@title}}%
                     }}
\makeatother



%% thm like environments
\let\question\relax
\let\endquestion\relax

\newtheoremstyle{QuestionStyle}{\topsep}{\topsep}%%% space between body and thm
		{}                      %%% Thm body font
		{}                              %%% Indent amount (empty = no indent)
		{\bfseries}            %%% Thm head font
		{)}                              %%% Punctuation after thm head
		{ }                           %%% Space after thm head
		{\thmnumber{#2}\thmnote{ \bfseries(#3)}}%%% Thm head spec
\theoremstyle{QuestionStyle}
\newtheorem{question}{}



\let\freeResponse\relax
\let\endfreeResponse\relax

%% \newtheoremstyle{ResponseStyle}{\topsep}{\topsep}%%% space between body and thm
%% 		{\wedn\bfseries}                      %%% Thm body font
%% 		{}                              %%% Indent amount (empty = no indent)
%% 		{\wedn\bfseries}            %%% Thm head font
%% 		{}                              %%% Punctuation after thm head
%% 		{3ex}                           %%% Space after thm head
%% 		{\underline{\underline{\thmname{#1}}}}%%% Thm head spec
%% \theoremstyle{ResponseStyle}

\usepackage[tikz]{mdframed}
\mdfdefinestyle{ResponseStyle}{leftmargin=1cm,linecolor=black,roundcorner=5pt,frametitlefont=\wedn\bfseries,%frametitle={\underline{\underline{Response}}:}
, font=\wedn\bfseries,}%\begin{mdframed}[style=mystyle]foo\end{mdframed}


\ifhandout
\NewEnviron{freeResponse}{}
\else
%\newtheorem{freeResponse}{Response:}
\newenvironment{freeResponse}{\begin{mdframed}[style=ResponseStyle]}{\end{mdframed}}
\fi



%% attempting to automate outcomes.

\newwrite\outcomefile
  \immediate\openout\outcomefile=\jobname.oc
\renewcommand{\outcome}[1]{\edef\theoutcomes{\theoutcomes #1~}%
\immediate\write\outcomefile{\unexpanded{\outcome}{#1}}}

%% \newcommand{\outcomelist}{\begin{itemize}\theoutcomes\end{itemize}}



%% my commands

\newcommand{\snap}{{\bfseries\itshape\textsf{Snap!}}}
\newcommand{\flavor}{\link[\snap]{https://snap.berkeley.edu/}}


\usepackage{tkz-euclide}
\tikzstyle geometryDiagrams=[rounded corners=.5pt,ultra thick,color=black]
\colorlet{penColor}{black} % Color of a curve in a plot

\usepackage{fullpage}
\makeatletter
%% no number for activity
\newcommand\logostyle{%
  \def\activitystyle{activity-chapter}
  \def\maketitle{%
                {\flushleft\small\sffamily\bfseries\@pretitle\par\vspace{-1.5em}}%
                {\flushleft\LARGE\sffamily\bfseries\@title \par }%
                {\vskip .6em\noindent\textit\theabstract\setcounter{problem}{0}\setcounter{sectiontitlenumber}{0}}%
                    \par\vspace{2em}
                    \phantomsection\addcontentsline{toc}{section}{\textbf{\@title}}%
                     \setcounter{titlenumber}{0}}}
\makeatother
\newcommand{\nameblankgen}{\noindent\textbf{Name(s) (please print):}\ \hrulefill \\

\hrulefill}
\logostyle



%% %%% Graph paper background
%% \def\mygraphpaper{%
%%   \begin{tikzpicture}
%%     %\draw[line width=.4pt,draw=black!30] (0,0) grid[step=1mm] (\paperwidth,\paperheight);
%%     \draw[line width=.4pt,draw=blue!50] (0,0) grid[step=.2in] (\paperwidth,\paperheight);
%%   \end{tikzpicture}%
%% }
%% \usepackage{background}
%% \backgroundsetup{
%%   angle=0,
%%   contents=\mygraphpaper,
%%   color=black,
%%   scale=1,
%% }





\pagenumbering{gobble}



\title{Curves and headings}
\author{Bart Snapp}

\begin{document}
\begin{abstract}
  We give ways to draw curves and point the turtle.
\end{abstract}
\maketitle

\nameblankgen

\begin{multicols*}{2}
  So far, we've been drawing polygons. We can draw a regular $n$-gon
  with
\begin{logo}
  repeat :n [ fd :s rt 360/:n ]
\end{logo}
where the character \lc{:n} represents the number of sides of the
regular $n$-gon and \lc{:s} represents the side-length of the regular
polygon. Both \lc{:n} and \lc{:s} are \emph{numbers} that you should
\emph{fill-in}. So for example, \lc{repeat 17 [ fd 20 rt 360/17 ]}, draws
\begin{logoout}
\begin{tikzpicture}[turtle/distance=.3cm]
  \draw [thick,black,turtle={home,forward,right=21,fd,right=21,
      fd,right=21,
      fd,right=21,
      fd,right=21,
      fd,right=21,
      fd,right=21,
      fd,right=21,
      fd,right=21,
      fd,right=21,
      fd,right=21,
      fd,right=21,
      fd,right=21,
      fd,right=21,
      fd,right=21,
      fd,right=21,
      fd,right=21,
      fd,right=21,
  }];
  \node at (0,0) {\turtle};
\end{tikzpicture}
\end{logoout}
So one way to draw a ``circle'' is by making a regular $n$-gon with
many sides. So something like \lc{repeat 360 [fd :s rt 1]} for some
number \lc{:s}. As a gesture of friendship, I offer you:
\begin{logo}
to circle
pu rt 90 bk 100 lt 90 pd
repeat 360 [fd 1.74538 rt 1]
pu rt 90 fd 100 lt 90 pd
end
\end{logo}
This program draws a ``circle'' of radius $100$ around the point you
start at. The value $1.74538$ was found by trial-and-error.


Now that we can draw a ``circle,'' we can use \lc{seth \#} to set the
heading of the turtle to different points on the circle. Basically
\lc{seth \#} points the turtle at a heading of \lc{\#} degrees when
measured from the \emph{top of the circle} in a \emph{clockwise
  direction}. This is especially useful if you have a shape that needs
to be the same orientation every time. Check out this starry night.

\begin{logo}
to star 
  filled 7 [
    seth 90
    repeat 5 [ fd 50 rt 144 ]
  ]
end

ht setcolor [10 10 50] fill setcolor 7 pu
repeat 10 [ pu
  rt random 50 + 20 fd random 200 + 100
  pd star ]
\end{logo}
\emph{Note:} The stars are all at the same orientation, all thanks to \lc{seth}.
\begin{logoout}
  \includegraphics[width=.3\textwidth]{starryNight.png}
\end{logoout}



\section{Commands to know}
\begin{tabular}{lll}
  \lc{CMD}   & Description                 \\ \hlinewd{1pt}
  \lc{seth \#}   & set the heading to \lc{\#}\\
  \lc{filled \# [ BODY ]} & fill \lc{BODY} with color \lc{\#}
\end{tabular}


\end{multicols*}

\newpage

\begin{problem}
  Draw and label a picture (by hand or using a computer) showing how
  \lc{seth \#} points the turtle.
\end{problem}

\mynewpage

\begin{problem}
  \emph{First:}  Explain in words what the \lc{circle} program is doing line-by-line.


  \emph{Second:} Based on the \lc{circle} program above, make a new program
  \lc{circle.200} that will draw a circle of radius $200$ centered at
  the origin. Don't get freaked out by the \lc{.} in \lc{circle.200},
  it's just part of the name. Show-off your work with your code and a
  picture.
\end{problem}

\mynewpage


\begin{problem}
  The following code produces a simple ``house.''
\begin{logo}
filled 3 [ repeat 4 [ rt 90 fd 50 ] ]
filled 4 [ rt 30 fd 50 rt 120 fd 50 ]
\end{logo}
Modify this code so that the house is $\approx 50$ units tall, the
colors are randomized, and the house can be moved \emph{without} its
orientation changing. Call this function \lc{house}. Show-off your
work with your code and a picture.
\end{problem}

\mynewpage

\begin{problem}
  Use your command \lc{house} to make a command \lc{village} that will
  place three small houses on a green circle. Show-off your work with
  your code and a picture.
\end{problem}

\mynewpage

\begin{problem}
  Use your command \lc{village} to make a command \lc{country} that
  will place $5$ villages on a green background, with the villages
  connected by roads. Make sure that the roads don't pass through the
  houses! Show-off your work with your code and a picture.
\end{problem}


\end{document}
