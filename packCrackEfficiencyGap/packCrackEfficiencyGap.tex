\documentclass[noauthor,nooutcomes,hints,handout,12pt]{ximera}

%% Note, this is connected to Simpson's paradox and Arrow's Impossibility theorem.
%% See paper An Impossibility Theorem for Gerrymandering, by Alexeev and Mixon

%https://redistricting.ohio.gov/maps
%%https://www.lsc.ohio.gov/documents/reference/current/membersonlybriefs/134%20Redistricting%20in%20Ohio.pdf

%https://districtr.org/ohio

%https://projects.fivethirtyeight.com/redistricting-2022-maps/ohio/democratic_amendments/
%https://districtr.org/plan/120904

%https://districtr.org/plan/121411


%https://districtr.org/oh/congress

\graphicspath{  
{./}
{./whoAreYou/}
{./drawingWithTheTurtle/}
{./bisectionMethod/}
{./circles/}
{./anglesAndRightTriangles/}
{./lawOfSines/}
{./lawOfCosines/}
{./plotter/}
{./staircases/}
{./pitch/}
{./qualityControl/}
{./symmetry/}
{./nGonBlock/}
}


%% page layout
\usepackage[cm,headings]{fullpage}
\raggedright
\setlength\headheight{13.6pt}


%% fonts
\usepackage{euler}

\usepackage{FiraMono}
\renewcommand\familydefault{\ttdefault} 
\usepackage[defaultmathsizes]{mathastext}
\usepackage[htt]{hyphenat}

\usepackage[T1]{fontenc}
\usepackage[scaled=1]{FiraSans}

%\usepackage{wedn}
\usepackage{pbsi} %% Answer font


\usepackage{cancel} %% strike through in pitch/pitch.tex


%% \usepackage{ulem} %% 
%% \renewcommand{\ULthickness}{2pt}% changes underline thickness

\tikzset{>=stealth}

\usepackage{adjustbox}

\setcounter{titlenumber}{-1}

%% journal style
\makeatletter
\newcommand\journalstyle{%
  \def\activitystyle{activity-chapter}
  \def\maketitle{%
    \addtocounter{titlenumber}{1}%
                {\flushleft\small\sffamily\bfseries\@pretitle\par\vspace{-1.5em}}%
                {\flushleft\LARGE\sffamily\bfseries\thetitlenumber\hspace{1em}\@title \par }%
                {\vskip .6em\noindent\textit\theabstract\setcounter{question}{0}\setcounter{sectiontitlenumber}{0}}%
                    \par\vspace{2em}
                    \phantomsection\addcontentsline{toc}{section}{\thetitlenumber\hspace{1em}\textbf{\@title}}%
                     }}
\makeatother



%% thm like environments
\let\question\relax
\let\endquestion\relax

\newtheoremstyle{QuestionStyle}{\topsep}{\topsep}%%% space between body and thm
		{}                      %%% Thm body font
		{}                              %%% Indent amount (empty = no indent)
		{\bfseries}            %%% Thm head font
		{)}                              %%% Punctuation after thm head
		{ }                           %%% Space after thm head
		{\thmnumber{#2}\thmnote{ \bfseries(#3)}}%%% Thm head spec
\theoremstyle{QuestionStyle}
\newtheorem{question}{}



\let\freeResponse\relax
\let\endfreeResponse\relax

%% \newtheoremstyle{ResponseStyle}{\topsep}{\topsep}%%% space between body and thm
%% 		{\wedn\bfseries}                      %%% Thm body font
%% 		{}                              %%% Indent amount (empty = no indent)
%% 		{\wedn\bfseries}            %%% Thm head font
%% 		{}                              %%% Punctuation after thm head
%% 		{3ex}                           %%% Space after thm head
%% 		{\underline{\underline{\thmname{#1}}}}%%% Thm head spec
%% \theoremstyle{ResponseStyle}

\usepackage[tikz]{mdframed}
\mdfdefinestyle{ResponseStyle}{leftmargin=1cm,linecolor=black,roundcorner=5pt,
, font=\bsifamily,}%font=\wedn\bfseries\upshape,}


\ifhandout
\NewEnviron{freeResponse}{}
\else
%\newtheorem{freeResponse}{Response:}
\newenvironment{freeResponse}{\begin{mdframed}[style=ResponseStyle]}{\end{mdframed}}
\fi



%% attempting to automate outcomes.

%% \newwrite\outcomefile
%%   \immediate\openout\outcomefile=\jobname.oc
%% \renewcommand{\outcome}[1]{\edef\theoutcomes{\theoutcomes #1~}%
%% \immediate\write\outcomefile{\unexpanded{\outcome}{#1}}}

%% \newcommand{\outcomelist}{\begin{itemize}\theoutcomes\end{itemize}}

%% \NewEnviron{listOutcomes}{\small\sffamily
%% After answering the following questions, students should be able to:
%% \begin{itemize}
%% \BODY
%% \end{itemize}
%% }
\usepackage[tikz]{mdframed}
\mdfdefinestyle{OutcomeStyle}{leftmargin=2cm,rightmargin=2cm,linecolor=black,roundcorner=5pt,
, font=\small\sffamily,}%font=\wedn\bfseries\upshape,}
\newenvironment{listOutcomes}{\begin{mdframed}[style=OutcomeStyle]After answering the following questions, students should be able to:\begin{itemize}}{\end{itemize}\end{mdframed}}



%% my commands

\newcommand{\snap}{{\bfseries\itshape\textsf{Snap!}}}
\newcommand{\flavor}{\link[\snap]{https://snap.berkeley.edu/}}
\newcommand{\mooculus}{\textsf{\textbf{MOOC}\textnormal{\textsf{ULUS}}}}


\usepackage{tkz-euclide}
\tikzstyle geometryDiagrams=[rounded corners=.5pt,ultra thick,color=black]
\colorlet{penColor}{black} % Color of a curve in a plot



\ifhandout\newcommand{\mynewpage}{\newpage}\else\newcommand{\mynewpage}{}\fi


\title{Pack, crack, efficiency gap}
\author{Bart Snapp}

\begin{document}
\begin{abstract}
  We think about strategies and measurements of gerrymandering.
\end{abstract}
\maketitle

\begin{listOutcomes}
\item Describe what gerrymandering is.
\item Understand what ``packing'' is in reference to gerrymandering.
\item Understand what ``cracking'' is in reference to gerrymandering.
\item Understand what ``efficiency gap'' is in reference to
  districting.
\end{listOutcomes}





\mynewpage






\begin{question}
  Let's talk about ``packing.'' Consider the following array of squares and circles
   \[
   \renewcommand{\arraystretch}{1.5}
   \huge
  \begin{array}{|c|c|c|c|c|c|}\hline
   \hspace*{.5cm}\bigcirc\hspace*{.5cm} & \hspace*{.5cm} \blacksquare \hspace*{.5cm} & \hspace*{.5cm} \blacksquare \hspace*{.5cm} & \hspace*{.5cm}\blacksquare\hspace*{.5cm} & \hspace*{.5cm}\bigcirc\hspace*{.5cm} & \hspace*{.5cm}\blacksquare\hspace*{.5cm} \\\hline
    \blacksquare & \bigcirc & \bigcirc & \blacksquare & \bigcirc & \blacksquare \\\hline
    \bigcirc & \bigcirc & \blacksquare & \bigcirc & \blacksquare & \bigcirc \\\hline
    \blacksquare & \blacksquare & \blacksquare & \bigcirc & \bigcirc & \bigcirc \\\hline
    \blacksquare & \bigcirc & \bigcirc & \blacksquare & \bigcirc & \blacksquare \\\hline
    \bigcirc & \bigcirc & \blacksquare & \blacksquare & \blacksquare & \bigcirc \\\hline
  \end{array}
  \]
  \begin{enumerate}
  \item Gerrymander the array above into $12$ districts ($3$ in each
    district) so that the \textbf{squares} win.
  \item Look up ``packing'' in reference to gerrymandering. Explain it
    here, and give examples where you ``packed'' above.
  \end{enumerate}
\end{question}


\mynewpage

\begin{question}
Let's talk about ``cracking.'' Consider the following array of squares and
circles
   \[
  \renewcommand{\arraystretch}{1.5}\huge
  \begin{array}{|c|c|c|c|c|c|}\hline
   \hspace*{.5cm} \bigcirc\hspace*{.5cm} & \hspace*{.5cm} \blacksquare\hspace*{.5cm} & \hspace*{.5cm} \blacksquare\hspace*{.5cm} & \hspace*{.5cm} \blacksquare\hspace*{.5cm} & \hspace*{.5cm} \bigcirc\hspace*{.5cm} & \hspace*{.5cm} \blacksquare\hspace*{.5cm} \\\hline
    \blacksquare & \bigcirc & \bigcirc & \blacksquare & \bigcirc & \blacksquare \\\hline
    \bigcirc & \bigcirc & \blacksquare & \bigcirc & \blacksquare & \bigcirc \\\hline
    \blacksquare & \blacksquare & \blacksquare & \bigcirc & \bigcirc & \bigcirc \\\hline
    \blacksquare & \bigcirc & \bigcirc & \blacksquare & \bigcirc & \blacksquare \\\hline
    \bigcirc & \bigcirc & \blacksquare & \blacksquare & \blacksquare & \bigcirc \\\hline
  \end{array}
  \]
  \begin{enumerate}
  \item Gerrymander the array above into $12$ districts ($3$ in each
    district) so that the \textbf{circles} win.
  \item Look up ``cracking'' in reference to gerrymandering. Explain it
    here, and give examples where you ``cracked'' above.
  \end{enumerate}
\end{question}

\mynewpage


\begin{question}
  The ``efficiency gap'' of a district map is
  \[
  \frac{\left|\right(\text{wasted votes for }\blacksquare\left) - \right(\text{wasted votes for }\bigcirc\left)\right|}{(\text{total number of votes})}
  \]
  Within a district, all losing votes are wasted. Moreover, some ``winning'' votes could be wasted.
  \begin{enumerate}
    \item How many votes are needed to ``win'' a district with a
      population of $3$? How many ``wasted'' votes are there in a district of $3$?
    \item What's your efficiency gap for Question 1? Show work and
      explain your reasoning.
    \item What's your efficiency gap for Question 2? Show work and
      explain your reasoning.
  \end{enumerate}
\end{question}
See:
\begin{center}
  \url{https://www.brennancenter.org/sites/default/files/legal-work/How\_the\_Efficiency\_Gap\_Standard\_Works.pdf}
\end{center}
for more information on the efficiency gap.


\end{document}
