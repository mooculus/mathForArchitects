\documentclass{ximera}

%% page layout
\usepackage[in,headings]{fullpage}
\raggedright
\setlength\headheight{13.6pt}


%% fonts
\usepackage{euler}

\usepackage{FiraMono}
\renewcommand\familydefault{\ttdefault} 
\usepackage{mathastext}
\usepackage[htt]{hyphenat}

\usepackage[T1]{fontenc}
\usepackage[scaled=1]{FiraSans}

\usepackage{wedn}
\usepackage[T1]{fontenc}

%% wrap text around scripts
\usepackage{wrapfig}

\tikzset{>=stealth}
%% snap! scripts
\usepackage{scratch3}

\usepackage{adjustbox}

%% journal style
\makeatletter
\newcommand\journalstyle{%
  \def\activitystyle{activity-chapter}
  \def\maketitle{%
    \addtocounter{titlenumber}{1}%
                {\flushleft\small\sffamily\bfseries\@pretitle\par\vspace{-1.5em}}%
                {\flushleft\LARGE\sffamily\bfseries\thetitlenumber\hspace{1em}\@title \par }%
                {\vskip .6em\noindent\textit\theabstract\setcounter{question}{0}\setcounter{sectiontitlenumber}{0}}%
                    \par\vspace{2em}
                    \phantomsection\addcontentsline{toc}{section}{\thetitlenumber\hspace{1em}\textbf{\@title}}%
                     }}
\makeatother



%% thm like environments
\let\question\relax
\let\endquestion\relax

\newtheoremstyle{QuestionStyle}{\topsep}{\topsep}%%% space between body and thm
		{}                      %%% Thm body font
		{}                              %%% Indent amount (empty = no indent)
		{\bfseries}            %%% Thm head font
		{)}                              %%% Punctuation after thm head
		{ }                           %%% Space after thm head
		{\thmnumber{#2}\thmnote{ \bfseries(#3)}}%%% Thm head spec
\theoremstyle{QuestionStyle}
\newtheorem{question}{}



\let\freeResponse\relax
\let\endfreeResponse\relax

%% \newtheoremstyle{ResponseStyle}{\topsep}{\topsep}%%% space between body and thm
%% 		{\wedn\bfseries}                      %%% Thm body font
%% 		{}                              %%% Indent amount (empty = no indent)
%% 		{\wedn\bfseries}            %%% Thm head font
%% 		{}                              %%% Punctuation after thm head
%% 		{3ex}                           %%% Space after thm head
%% 		{\underline{\underline{\thmname{#1}}}}%%% Thm head spec
%% \theoremstyle{ResponseStyle}

\usepackage[tikz]{mdframed}
\mdfdefinestyle{ResponseStyle}{leftmargin=1cm,linecolor=black,roundcorner=5pt,frametitlefont=\wedn\bfseries,%frametitle={\underline{\underline{Response}}:}
, font=\wedn\bfseries,}%\begin{mdframed}[style=mystyle]foo\end{mdframed}


\ifhandout
\NewEnviron{freeResponse}{}
\else
%\newtheorem{freeResponse}{Response:}
\newenvironment{freeResponse}{\begin{mdframed}[style=ResponseStyle]}{\end{mdframed}}
\fi



%% attempting to automate outcomes.

\newwrite\outcomefile
  \immediate\openout\outcomefile=\jobname.oc
\renewcommand{\outcome}[1]{\edef\theoutcomes{\theoutcomes #1~}%
\immediate\write\outcomefile{\unexpanded{\outcome}{#1}}}

%% \newcommand{\outcomelist}{\begin{itemize}\theoutcomes\end{itemize}}



%% my commands

\newcommand{\snap}{{\bfseries\itshape\textsf{Snap!}}}
\newcommand{\flavor}{\link[\snap]{https://snap.berkeley.edu/}}


\usepackage{tkz-euclide}
\tikzstyle geometryDiagrams=[rounded corners=.5pt,ultra thick,color=black]
\colorlet{penColor}{black} % Color of a curve in a plot


\author{Jenny Sheldon \and Bart Snapp}


\outcome{}


\title{Matrix multiplication}

\begin{document}
\begin{abstract}
  We introduce how to multiply, or compose, matrices.
\end{abstract}
\maketitle

We know how to multiply a matrix and a point. Multiplying two
matrices is a similar procedure:
\[
\begin{bmatrix}
a & b & c \\ 
d & e & f \\
g & h & i
\end{bmatrix}
\begin{bmatrix}
j & k & l \\ 
m & n & o \\
p & q & r
\end{bmatrix}
= \begin{bmatrix}
aj + bm + cp & ak + bn + cq & al + bo + cr \\
dj + em + fp & dk + en + fq & dl + eo + fr \\
gj + hm + ip & gk + hn + iq & gl + ho + ir 
\end{bmatrix}
\]

Variables are all good and well, but let's do this with actual
numbers. Consider the following two matrices:
\[
\mat{M} =
\begin{bmatrix}
1 & 2 & 3\\
4 & 5 & 6\\
7 & 8 & 9
\end{bmatrix}
\qquad\text{and}\qquad
\mat{I} = 
\begin{bmatrix}
1 & 0 & 0\\
0 & 1 & 0\\
0 & 0 & 1
\end{bmatrix}
\]
Let's multiply them together and see what we get:
\begin{align*}
\mat{M}\mat{I} &= \begin{bmatrix}
1 & 2 & 3\\
4 & 5 & 6\\
7 & 8 & 9
\end{bmatrix}
\begin{bmatrix}
1 & 0 & 0\\
0 & 1 & 0\\
0 & 0 & 1
\end{bmatrix} \\
&=
\begin{bmatrix}
1\cdot 1 + 2\cdot 0 + 3\cdot 0 & 1\cdot 0 + 2\cdot 1 + 3\cdot 0 & 1\cdot 0 + 2\cdot 0 + 3\cdot 1\\
4\cdot 1 + 5\cdot 0 + 6\cdot 0 & 4\cdot 0 + 5\cdot 1 + 6\cdot 0 & 4\cdot 0 + 5\cdot 0 + 6\cdot 1\\
7\cdot 1 + 8\cdot 0 + 9\cdot 0 & 7\cdot 0 + 8\cdot 1 + 9\cdot 0 & 7\cdot 0 + 8\cdot 0 + 9\cdot 1 
\end{bmatrix} \\
&=
\begin{bmatrix}
1 & 2 & 3\\
4 & 5 & 6\\
7 & 8 & 9
\end{bmatrix} \\
&= \mat{M}
\end{align*}

\begin{question}
What is $\mat{I}\mat{M}$ equal to?
\end{question}

It turns out that we have a special name for $\mat{I}$.  We call it the \dfn{identity matrix}.

\begin{warning}
Matrix multiplication is \textbf{not} generally commutative. Check it out:
\[
\mat{F} =
\begin{bmatrix}
1 & 0 & 0\\
0 & -1 & 0\\
0 & 0 & 1
\end{bmatrix}
\qquad\text{and}\qquad
\mat{R} = 
\begin{bmatrix}
0 & -1 & 0\\
1 & 0 & 0\\
0 & 0 & 1
\end{bmatrix}
\]
When we multiply these matrices, we get:
\begin{align*}
\mat{F}\mat{R} &= \begin{bmatrix}
1 & 0 & 0\\
0 & -1 & 0\\
0 & 0 & 1\\
\end{bmatrix}
\begin{bmatrix}
0 & -1 & 0\\
1 & 0 & 0\\
0 & 0 & 1\\
\end{bmatrix} \\
&=
\begin{bmatrix}
1\cdot 0 +  0\cdot 1 + 0\cdot 0 & 1\cdot (-1) +  0\cdot 0 + 0\cdot 0 & 1\cdot 0 +  0\cdot 0 + 0\cdot 1\\
0\cdot 0 + (-1)\cdot 1 + 0\cdot 0 & 0\cdot (-1) + (-1)\cdot 0 + 0\cdot 0 & 0\cdot 0 + (-1)\cdot 0 + 0\cdot 1\\
0\cdot 0 +  0\cdot 1 + 1\cdot 0 & 0\cdot (-1) +  0\cdot 0 + 1\cdot 0 & 0\cdot 0 +  0\cdot 0 + 1\cdot 1\\
\end{bmatrix} \\
&=
\begin{bmatrix}
0 & -1 & 0\\
-1 & 0 & 0\\
0 & 0 & 1\\
\end{bmatrix}
\end{align*}
On the other hand, we get:
\[
\mat{R}\mat{F} = \begin{bmatrix}
0 & 1 & 0\\
1 & 0 & 0\\
0 & 0 & 1
\end{bmatrix}
\]
\end{warning}

\begin{question}
Can you draw some nice pictures showing geometrically that matrix
multiplication is not commutative?
\end{question}


\begin{question}
Is it always the case that $(\mat{L}\mat{M})\mat{N} = \mat{L}(\mat{M}\mat{N})$?
\end{question}


\end{document}
