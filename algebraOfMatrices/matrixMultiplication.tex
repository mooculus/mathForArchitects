\documentclass{ximera}

\graphicspath{  
{./}
{./whoAreYou/}
{./drawingWithTheTurtle/}
{./bisectionMethod/}
{./circles/}
{./anglesAndRightTriangles/}
{./lawOfSines/}
{./lawOfCosines/}
{./plotter/}
{./staircases/}
{./pitch/}
{./qualityControl/}
{./symmetry/}
{./nGonBlock/}
}


%% page layout
\usepackage[cm,headings]{fullpage}
\raggedright
\setlength\headheight{13.6pt}


%% fonts
\usepackage{euler}

\usepackage{FiraMono}
\renewcommand\familydefault{\ttdefault} 
\usepackage[defaultmathsizes]{mathastext}
\usepackage[htt]{hyphenat}

\usepackage[T1]{fontenc}
\usepackage[scaled=1]{FiraSans}

%\usepackage{wedn}
\usepackage{pbsi} %% Answer font


\usepackage{cancel} %% strike through in pitch/pitch.tex


%% \usepackage{ulem} %% 
%% \renewcommand{\ULthickness}{2pt}% changes underline thickness

\tikzset{>=stealth}

\usepackage{adjustbox}

\setcounter{titlenumber}{-1}

%% journal style
\makeatletter
\newcommand\journalstyle{%
  \def\activitystyle{activity-chapter}
  \def\maketitle{%
    \addtocounter{titlenumber}{1}%
                {\flushleft\small\sffamily\bfseries\@pretitle\par\vspace{-1.5em}}%
                {\flushleft\LARGE\sffamily\bfseries\thetitlenumber\hspace{1em}\@title \par }%
                {\vskip .6em\noindent\textit\theabstract\setcounter{question}{0}\setcounter{sectiontitlenumber}{0}}%
                    \par\vspace{2em}
                    \phantomsection\addcontentsline{toc}{section}{\thetitlenumber\hspace{1em}\textbf{\@title}}%
                     }}
\makeatother



%% thm like environments
\let\question\relax
\let\endquestion\relax

\newtheoremstyle{QuestionStyle}{\topsep}{\topsep}%%% space between body and thm
		{}                      %%% Thm body font
		{}                              %%% Indent amount (empty = no indent)
		{\bfseries}            %%% Thm head font
		{)}                              %%% Punctuation after thm head
		{ }                           %%% Space after thm head
		{\thmnumber{#2}\thmnote{ \bfseries(#3)}}%%% Thm head spec
\theoremstyle{QuestionStyle}
\newtheorem{question}{}



\let\freeResponse\relax
\let\endfreeResponse\relax

%% \newtheoremstyle{ResponseStyle}{\topsep}{\topsep}%%% space between body and thm
%% 		{\wedn\bfseries}                      %%% Thm body font
%% 		{}                              %%% Indent amount (empty = no indent)
%% 		{\wedn\bfseries}            %%% Thm head font
%% 		{}                              %%% Punctuation after thm head
%% 		{3ex}                           %%% Space after thm head
%% 		{\underline{\underline{\thmname{#1}}}}%%% Thm head spec
%% \theoremstyle{ResponseStyle}

\usepackage[tikz]{mdframed}
\mdfdefinestyle{ResponseStyle}{leftmargin=1cm,linecolor=black,roundcorner=5pt,
, font=\bsifamily,}%font=\wedn\bfseries\upshape,}


\ifhandout
\NewEnviron{freeResponse}{}
\else
%\newtheorem{freeResponse}{Response:}
\newenvironment{freeResponse}{\begin{mdframed}[style=ResponseStyle]}{\end{mdframed}}
\fi



%% attempting to automate outcomes.

%% \newwrite\outcomefile
%%   \immediate\openout\outcomefile=\jobname.oc
%% \renewcommand{\outcome}[1]{\edef\theoutcomes{\theoutcomes #1~}%
%% \immediate\write\outcomefile{\unexpanded{\outcome}{#1}}}

%% \newcommand{\outcomelist}{\begin{itemize}\theoutcomes\end{itemize}}

%% \NewEnviron{listOutcomes}{\small\sffamily
%% After answering the following questions, students should be able to:
%% \begin{itemize}
%% \BODY
%% \end{itemize}
%% }
\usepackage[tikz]{mdframed}
\mdfdefinestyle{OutcomeStyle}{leftmargin=2cm,rightmargin=2cm,linecolor=black,roundcorner=5pt,
, font=\small\sffamily,}%font=\wedn\bfseries\upshape,}
\newenvironment{listOutcomes}{\begin{mdframed}[style=OutcomeStyle]After answering the following questions, students should be able to:\begin{itemize}}{\end{itemize}\end{mdframed}}



%% my commands

\newcommand{\snap}{{\bfseries\itshape\textsf{Snap!}}}
\newcommand{\flavor}{\link[\snap]{https://snap.berkeley.edu/}}
\newcommand{\mooculus}{\textsf{\textbf{MOOC}\textnormal{\textsf{ULUS}}}}


\usepackage{tkz-euclide}
\tikzstyle geometryDiagrams=[rounded corners=.5pt,ultra thick,color=black]
\colorlet{penColor}{black} % Color of a curve in a plot



\ifhandout\newcommand{\mynewpage}{\newpage}\else\newcommand{\mynewpage}{}\fi


\author{Jenny Sheldon \and Bart Snapp}


\outcome{Multiply $3\times 3$ matrices.}
\outcome{Identify the identity matrix.}
\outcome{Algebraically show that matrix multiplication is not necessarily commutative.}
\outcome{Geometrically show that matrix multiplication is not necessarily commutative.}



\title{Matrix multiplication}

\begin{document}
\begin{abstract}
  We introduce how to multiply, or compose, matrices.
\end{abstract}
\maketitle

We know how to multiply a matrix and a point. Multiplying two
matrices is a similar procedure:
\[
\begin{bmatrix}
a & b & c \\ 
d & e & f \\
g & h & i
\end{bmatrix}
\begin{bmatrix}
j & k & l \\ 
m & n & o \\
p & q & r
\end{bmatrix}
= \begin{bmatrix}
aj + bm + cp & ak + bn + cq & al + bo + cr \\
dj + em + fp & dk + en + fq & dl + eo + fr \\
gj + hm + ip & gk + hn + iq & gl + ho + ir 
\end{bmatrix}
\]

Variables are all good and well, but let's do this with actual
numbers.


\begin{example}
  Consider the following two matrices:
  \[
  \mat{M} =
  \begin{bmatrix}
    2 & 0 & 3\\
    1 & 2 & 0\\
    0 & 4 & -2
  \end{bmatrix}
  \qquad\text{and}\qquad
  \mat{N} = 
  \begin{bmatrix}
    3 & 2 & 1\\
    0 & 0 & -2\\
    -2 & 0 & 1
  \end{bmatrix}
  \]
  Compute: $\mat{M}\mat{N}$
  \begin{explanation}
    We just use the formula given above for matrix
    multiplication. Write with me:
    \begin{align*}
      \mat{M}\mat{N} &=
      \begin{bmatrix}
        2 & 0 & 3\\
        1 & 2 & 0\\
        0 & 4 & -2
      \end{bmatrix}
      \begin{bmatrix}
        3 & 2 & 1\\
        0 & 0 & -2\\
        -2 & 0 & 1
      \end{bmatrix}\\
      &=
      \begin{bmatrix}
        \answer[given]{0} & \answer[given]{4} & \answer[given]{5} \\
        \answer[given]{3} & \answer[given]{2} & \answer[given]{-3} \\
        \answer[given]{4} & \answer[given]{0} & \answer[given]{-10}
      \end{bmatrix}
    \end{align*}
  \end{explanation}
\end{example}

Some matrices have a lot of zeros in them. This makes your life
easier. Check out the next example.

\begin{example}
Consider the following two matrices:
\[
\mat{M} =
\begin{bmatrix}
1 & 2 & 3\\
4 & 5 & 6\\
7 & 8 & 9
\end{bmatrix}
\qquad\text{and}\qquad
\mat{I} = 
\begin{bmatrix}
1 & 0 & 0\\
0 & 1 & 0\\
0 & 0 & 1
\end{bmatrix}
\]
Compute: $\mat{M}\mat{I}$ 
\begin{explanation}
Write with me:
\begin{align*}
\mat{M}\mat{I} &= \begin{bmatrix}
1 & 2 & 3\\
4 & 5 & 6\\
7 & 8 & 9
\end{bmatrix}
\begin{bmatrix}
1 & 0 & 0\\
0 & 1 & 0\\
0 & 0 & 1
\end{bmatrix} \\
&=
\begin{bmatrix}
\answer[given]{1} & \answer[given]{2} & \answer[given]{3}\\
\answer[given]{4} & \answer[given]{5} & \answer[given]{6}\\
\answer[given]{7} & \answer[given]{8} & \answer[given]{9}
\end{bmatrix}
\end{align*}
But this is just $\mat{M}$!
\end{explanation}
\end{example}

\begin{question}
  What is $\mat{I}\mat{M}$ equal to?

  \begin{prompt}
    \[
    \mat{I}\mat{M} =
    \begin{bmatrix}
      \answer{1} & \answer{2} & \answer{3}\\
      \answer{4} & \answer{5} & \answer{6}\\
      \answer{7} & \answer{8} & \answer{9}
    \end{bmatrix}
    \]
  \end{prompt}
\end{question}

It turns out that we have a special name for $\mat{I}$.  We call it the \dfn{identity matrix}.

\begin{warning}
Matrix multiplication is \textbf{not} generally commutative. Check it out:
\[
\mat{F} =
\begin{bmatrix}
1 & 0 & 0\\
0 & -1 & 0\\
0 & 0 & 1
\end{bmatrix}
\qquad\text{and}\qquad
\mat{R} = 
\begin{bmatrix}
0 & -1 & 0\\
1 & 0 & 0\\
0 & 0 & 1
\end{bmatrix}
\]
When we multiply these matrices, we get:
\begin{align*}
\mat{F}\mat{R} &= \begin{bmatrix}
1 & 0 & 0\\
0 & -1 & 0\\
0 & 0 & 1
\end{bmatrix}
\begin{bmatrix}
0 & -1 & 0\\
1 & 0 & 0\\
0 & 0 & 1
\end{bmatrix}\\
&=
\begin{bmatrix}
\answer[given]{0} & \answer[given]{-1} & \answer[given]{0}\\
\answer[given]{-1} & \answer[given]{0} & \answer[given]{0}\\
\answer[given]{0} & \answer[given]{0} & \answer[given]{1}
\end{bmatrix}
\end{align*}
On the other hand, we get:
\begin{align*}
\mat{R}\mat{F} &=
\begin{bmatrix}
0 & -1 & 0\\
1 & 0 & 0\\
0 & 0 & 1
\end{bmatrix}
\begin{bmatrix}
1 & 0 & 0\\
0 & -1 & 0\\
0 & 0 & 1
\end{bmatrix}\\
&=
\begin{bmatrix}
\answer[given]{0} & \answer[given]{1} & \answer[given]{0}\\
\answer[given]{1} & \answer[given]{0} & \answer[given]{0}\\
\answer[given]{0} & \answer[given]{0} & \answer[given]{1}
\end{bmatrix}
\end{align*}
\end{warning}

\begin{question}
Can you draw some nice pictures showing geometrically that matrix
multiplication is not commutative?
\end{question}


\begin{question}
  Is it always the case that $(\mat{L}\mat{M})\mat{N} = \mat{L}(\mat{M}\mat{N})$?
  
\begin{prompt}
\begin{multipleChoice}
  \choice[correct]{I've thought about this.}
  \choice{I've not thought about this.}
\end{multipleChoice}
\begin{idea}
  It turns out that whenever you have matrices $\mat{L}$, $\mat{M}$,
  and $\mat{N}$ it will be the case that:
  \[
  (\mat{L}\mat{M})\mat{N} = \mat{L}(\mat{M}\mat{N})
  \]
  This is called \dfn{associativity}. It is difficult to explain this
  working purely algebraically. Instead, we should realize that
  matrices are functions, and work abstractly.  In essence, the
  following diagram ``proves'' that functional composition is
  \textit{always} associative---and hence matrix multiplication is
  always associative. Note, we are writing our functions on the
  \textit{left} of the argument.
\begin{image}
  \begin{tikzpicture}[scale=.5,geometryDiagrams]

    \node at (-2,8) {$\vec{p}$};
    \draw[|->] (-1.5,8)--(-.5,8);
    \draw[fill=black!10!white] (0,8) ellipse (0.166 and 0.5);% ellipse
    \draw (.5,7) rectangle (2.5,9);
    \draw (3,8) [partial ellipse=270:450:0.166 and 0.5];
    \node at (1.5,8) {\Large $\mat{N}$};
    \draw (.5,8.5) [partial ellipse=180:270:.5 and 0.166];
    \draw (.5,7.5) [partial ellipse=90:180:.5 and 0.166];
    \draw (2.5,8.5) [partial ellipse=270:360:.5 and 0.166];
    \draw (2.5,7.5) [partial ellipse=0:90:.5 and 0.166];

    %\node at (5,8.6) {$\vec{p} \mat{N}$};
    \draw[|->] (3.5,8)--(6.5,8);
    
    \draw[fill=black!10!white] (7,8) ellipse (0.166 and 0.5);% ellipse
    \draw (7.5,7) rectangle (9.5,9);
    \draw (10,8) [partial ellipse=270:450:0.166 and 0.5];
    \node at (8.5,8) {\Large $\mat{M}$};
    \draw (7.5,8.5) [partial ellipse=180:270:.5 and 0.166];
    \draw (7.5,7.5) [partial ellipse=90:180:.5 and 0.166];
    \draw (9.5,8.5) [partial ellipse=270:360:.5 and 0.166];
    \draw (9.5,7.5) [partial ellipse=0:90:.5 and 0.166];

    \node at (5,8.6) {$\mat{N}\vec{p}$};

    \draw[|->] (10.5,8)--(13.5,8);
    
    \draw[fill=black!10!white] (14,8) ellipse (0.166 and 0.5);% ellipse
    \draw (14.5,7) rectangle (16.5,9);
    \draw (17,8) [partial ellipse=270:450:0.166 and 0.5];
    \node at (15.5,8) {\Large $\mat{L}$};
    \draw (14.5,8.5) [partial ellipse=180:270:.5 and 0.166];
    \draw (14.5,7.5) [partial ellipse=90:180:.5 and 0.166];
    \draw (16.5,8.5) [partial ellipse=270:360:.5 and 0.166];
    \draw (16.5,7.5) [partial ellipse=0:90:.5 and 0.166];

    \node at (20.5,8) {$(\mat{L} \mat{M}) \mat{N}\vec{p}$};
    \draw[|->] (17.5,8)--(18.5,8);

    %\draw[dashed,gray] (-.4,6.6) rectangle (10.35,9.4);
    %\draw[dashed,gray] (6.6,2.6) rectangle (17.35,5.4);

    \draw[dashed,gray] (-.4,2.6) rectangle (10.35,5.4);
    \draw[dashed,gray] (6.6,6.6) rectangle (17.35,9.4);    
    
    %%%%%%%%%%%%%%%%%
    
    \node at (-2,4) {$\vec{p}$};
    \draw[|->] (-1.5,4)--(-.5,4);
    \draw[fill=black!10!white] (0,4) ellipse (0.166 and 0.5);% ellipse
    \draw (.5,3) rectangle (2.5,5);
    \draw (3,4) [partial ellipse=270:450:0.166 and 0.5];
    \node at (1.5,4) {\Large $\mat{N}$};
    \draw (.5,4.5) [partial ellipse=180:270:.5 and 0.166];
    \draw (.5,3.5) [partial ellipse=90:180:.5 and 0.166];
    \draw (2.5,4.5) [partial ellipse=270:360:.5 and 0.166];
    \draw (2.5,3.5) [partial ellipse=0:90:.5 and 0.166];

    %\node at (5,4.6) {$\mat{N}\vec{p}$};
    \node at (12,4.6) {$\mat{M} \mat{N}\vec{p}$};
    \draw[|->] (3.5,4)--(6.5,4);
    
    \draw[fill=black!10!white] (7,4) ellipse (0.166 and 0.5);% ellipse
    \draw (7.5,3) rectangle (9.5,5);
    \draw (10,4) [partial ellipse=270:450:0.166 and 0.5];
    \node at (8.5,4) {\Large $\mat{M}$};
    \draw (7.5,4.5) [partial ellipse=180:270:.5 and 0.166];
    \draw (7.5,3.5) [partial ellipse=90:180:.5 and 0.166];
    \draw (9.5,4.5) [partial ellipse=270:360:.5 and 0.166];
    \draw (9.5,3.5) [partial ellipse=0:90:.5 and 0.166];

    %\node at (12,4.6) {$\vec{p} \mat{N} \mat{M}$};
    \draw[|->] (10.5,4)--(13.5,4);
    
    \draw[fill=black!10!white] (14,4) ellipse (0.166 and 0.5);% ellipse
    \draw (14.5,3) rectangle (16.5,5);
    \draw (17,4) [partial ellipse=270:450:0.166 and 0.5];
    \node at (15.5,4) {\Large $\mat{L}$};
    \draw (14.5,4.5) [partial ellipse=180:270:.5 and 0.166];
    \draw (14.5,3.5) [partial ellipse=90:180:.5 and 0.166];
    \draw (16.5,4.5) [partial ellipse=270:360:.5 and 0.166];
    \draw (16.5,3.5) [partial ellipse=0:90:.5 and 0.166];

    \node at (20.5,4) {$\mat{L}(\mat{M}\mat{N})\vec{p}$};
    \draw[|->] (17.5,4)--(18.5,4);

    %%%%%%%%%%%%%%%%%

    \node at (-2,0) {$\vec{p}$};
    \draw[|->] (-1.5,0)--(-.5,0);
    \draw[fill=black!10!white] (0,0) ellipse (0.166 and 0.5);% ellipse
    \draw (.5,-1) rectangle (2.5,1);
    \draw (3,0) [partial ellipse=270:450:0.166 and 0.5];
    \node at (1.5,0) {\Large $\mat{N}$};
    \draw (.5,.5) [partial ellipse=180:270:.5 and 0.166];
    \draw (.5,-.5) [partial ellipse=90:180:.5 and 0.166];
    \draw (2.5,.5) [partial ellipse=270:360:.5 and 0.166];
    \draw (2.5,-.5) [partial ellipse=0:90:.5 and 0.166];

    \node at (5,.6) {$\mat{N}\vec{p}$};
    \draw[|->] (3.5,0)--(6.5,0);
    
    \draw[fill=black!10!white] (7,0) ellipse (0.166 and 0.5);% ellipse
    \draw (7.5,-1) rectangle (9.5,1);
    \draw (10,0) [partial ellipse=270:450:0.166 and 0.5];
    \node at (8.5,0) {\Large $\mat{M}$};
    \draw (7.5,.5) [partial ellipse=180:270:.5 and 0.166];
    \draw (7.5,-.5) [partial ellipse=90:180:.5 and 0.166];
    \draw (9.5,.5) [partial ellipse=270:360:.5 and 0.166];
    \draw (9.5,-.5) [partial ellipse=0:90:.5 and 0.166];

    \node at (12,.6) {$\mat{M} \mat{N}\vec{p}$};
    \draw[|->] (10.5,0)--(13.5,0);
    
    \draw[fill=black!10!white] (14,0) ellipse (0.166 and 0.5);% ellipse
    \draw (14.5,-1) rectangle (16.5,1);
    \draw (17,0) [partial ellipse=270:450:0.166 and 0.5];
    \node at (15.5,0) {\Large $\mat{L}$};
    \draw (14.5,.5) [partial ellipse=180:270:.5 and 0.166];
    \draw (14.5,-.5) [partial ellipse=90:180:.5 and 0.166];
    \draw (16.5,.5) [partial ellipse=270:360:.5 and 0.166];
    \draw (16.5,-.5) [partial ellipse=0:90:.5 and 0.166];

    \node at (20.5,0) {$\mat{L} \mat{M} \mat{N}\vec{p}$};
    \draw[|->] (17.5,0)--(18.5,0);

    %%%%%%%%%%%%%%%%%

    

  \end{tikzpicture}
\end{image}


In the first row we see $\mat{N}$ applied to $\vec{p}$ and then
$\mat{L}$ composed with $\mat{M}$ applied to $\mat{N}\vec{p}$,
meaning:
\[
(\mat{L}\mat{M})\mat{N}\vec{p}
\]
In the second row we see $\mat{M}$ composed with $\mat{N}$ applied to
$\vec{p}$ and then $\mat{L}$ applied to $\mat{M}\mat{N}\vec{p}$,
meaning:
\[
\mat{L}(\mat{M}\mat{N})\vec{p}
\]
But all we are really doing is successively applying these functions
to $\vec{p}$, as show in the third row. The diagram shows that the
three functions are equivalent, and hence function composition is
associative.
\end{idea}
\end{prompt}
\end{question}


\end{document}
