\documentclass{ximera}

%% page layout
\usepackage[in,headings]{fullpage}
\raggedright
\setlength\headheight{13.6pt}


%% fonts
\usepackage{euler}

\usepackage{FiraMono}
\renewcommand\familydefault{\ttdefault} 
\usepackage{mathastext}
\usepackage[htt]{hyphenat}

\usepackage[T1]{fontenc}
\usepackage[scaled=1]{FiraSans}

\usepackage{wedn}
\usepackage[T1]{fontenc}

%% wrap text around scripts
\usepackage{wrapfig}

\tikzset{>=stealth}
%% snap! scripts
\usepackage{scratch3}

\usepackage{adjustbox}

%% journal style
\makeatletter
\newcommand\journalstyle{%
  \def\activitystyle{activity-chapter}
  \def\maketitle{%
    \addtocounter{titlenumber}{1}%
                {\flushleft\small\sffamily\bfseries\@pretitle\par\vspace{-1.5em}}%
                {\flushleft\LARGE\sffamily\bfseries\thetitlenumber\hspace{1em}\@title \par }%
                {\vskip .6em\noindent\textit\theabstract\setcounter{question}{0}\setcounter{sectiontitlenumber}{0}}%
                    \par\vspace{2em}
                    \phantomsection\addcontentsline{toc}{section}{\thetitlenumber\hspace{1em}\textbf{\@title}}%
                     }}
\makeatother



%% thm like environments
\let\question\relax
\let\endquestion\relax

\newtheoremstyle{QuestionStyle}{\topsep}{\topsep}%%% space between body and thm
		{}                      %%% Thm body font
		{}                              %%% Indent amount (empty = no indent)
		{\bfseries}            %%% Thm head font
		{)}                              %%% Punctuation after thm head
		{ }                           %%% Space after thm head
		{\thmnumber{#2}\thmnote{ \bfseries(#3)}}%%% Thm head spec
\theoremstyle{QuestionStyle}
\newtheorem{question}{}



\let\freeResponse\relax
\let\endfreeResponse\relax

%% \newtheoremstyle{ResponseStyle}{\topsep}{\topsep}%%% space between body and thm
%% 		{\wedn\bfseries}                      %%% Thm body font
%% 		{}                              %%% Indent amount (empty = no indent)
%% 		{\wedn\bfseries}            %%% Thm head font
%% 		{}                              %%% Punctuation after thm head
%% 		{3ex}                           %%% Space after thm head
%% 		{\underline{\underline{\thmname{#1}}}}%%% Thm head spec
%% \theoremstyle{ResponseStyle}

\usepackage[tikz]{mdframed}
\mdfdefinestyle{ResponseStyle}{leftmargin=1cm,linecolor=black,roundcorner=5pt,frametitlefont=\wedn\bfseries,%frametitle={\underline{\underline{Response}}:}
, font=\wedn\bfseries,}%\begin{mdframed}[style=mystyle]foo\end{mdframed}


\ifhandout
\NewEnviron{freeResponse}{}
\else
%\newtheorem{freeResponse}{Response:}
\newenvironment{freeResponse}{\begin{mdframed}[style=ResponseStyle]}{\end{mdframed}}
\fi



%% attempting to automate outcomes.

\newwrite\outcomefile
  \immediate\openout\outcomefile=\jobname.oc
\renewcommand{\outcome}[1]{\edef\theoutcomes{\theoutcomes #1~}%
\immediate\write\outcomefile{\unexpanded{\outcome}{#1}}}

%% \newcommand{\outcomelist}{\begin{itemize}\theoutcomes\end{itemize}}



%% my commands

\newcommand{\snap}{{\bfseries\itshape\textsf{Snap!}}}
\newcommand{\flavor}{\link[\snap]{https://snap.berkeley.edu/}}


\usepackage{tkz-euclide}
\tikzstyle geometryDiagrams=[rounded corners=.5pt,ultra thick,color=black]
\colorlet{penColor}{black} % Color of a curve in a plot


\author{Jenny Sheldon \and Bart Snapp}


\outcome{}


\title{Compositions of matrices}

\begin{document}
\begin{abstract}
  We study repeated applications of our isometries.
\end{abstract}
\maketitle


It is often the case that we wish to apply several isometries
successively to a point. Consider the following:
\[
\mat{M} =
\begin{bmatrix}
a & b & c\\
d & e & f \\
0 & 0 & 1
\end{bmatrix}
\qquad
\mat{N} =
\begin{bmatrix}
g & h & i\\
j & k & l\\
0 & 0 & 1
\end{bmatrix}
\qquad\text{and}\qquad \vec{p} =
\begin{bmatrix}
x \\
y \\
1
\end{bmatrix}
\]
Now let's compute 
\begin{align*}
\mat{M}(\mat{N}\vec{p}) &= \begin{bmatrix}
a & b & c\\
d & e & f \\
0 & 0 & 1
\end{bmatrix} 
\left(
\begin{bmatrix}
g & h & i\\
j & k & l\\
0 & 0 & 1
\end{bmatrix}
\begin{bmatrix}
x \\
y \\
1
\end{bmatrix}
\right)\\
&= 
\begin{bmatrix}
a & b & c\\
d & e & f \\
0 & 0 & 1
\end{bmatrix} 
\begin{bmatrix}
gx+hy + i\\
jx+ky +l\\
1
\end{bmatrix} \\
&=
\begin{bmatrix}
agx+ahy +ai + bjx+bky+bl+c\\
dgx+dhy+ di + ejx+eky+el+f\\
1
\end{bmatrix}
\end{align*}
Now \textit{you} compute $(\mat{M}\mat{N})\vec{p}$ and compare what
\textit{you} get to what we got above.



\section{Compositions of translations}

A composition of translations occurs when two or more successive
translations are applied to the same point. Check it out:
\begin{align*}
\mat{T}_{(5,-4)}\mat{T}_{(-3,2)} &= \begin{bmatrix}
1 & 0 &  5\\
0 & 1 & -4\\
0 & 0 &  1
\end{bmatrix}
\begin{bmatrix}
1 & 0 & -3\\
0 & 1 &  2\\
0 & 0 &  1
\end{bmatrix}\\
&=\begin{bmatrix}
1 & 0 &  2\\
0 & 1 & -2\\
0 & 0 &  1
\end{bmatrix}\\
&=\mat{T}_{(5+(-3),(-4)+2)}\\
&=\mat{T}_{(2,-2)}
\end{align*}

\begin{theorem}
The composition of two translations $\mat{T}_{(u,v)}$ and
$\mat{T}_{(s,t)}$ is equal to the translation $\mat{T}_{(u+s,v+t)}$.
\end{theorem}

\begin{question} How do you prove the theorem above?

  \begin{prompt}
    A proof is nothing but a fancy word for an explanation. So what we
    do is work generally, with
    \[
    \mat{T}_{(u,v)} =
    \begin{bmatrix}
      \answer{1} & \answer{0} & \answer{u} \\
      \answer{0} & \answer{1} & \answer{v} \\
      \answer{0} & \answer{0} & \answer{1}
    \end{bmatrix}
    \]
    and
    \[
    \mat{T}_{(s,t)} =
    \begin{bmatrix}
      \answer{1} & \answer{0} & \answer{s} \\
      \answer{0} & \answer{1} & \answer{t} \\
      \answer{0} & \answer{0} & \answer{1}
    \end{bmatrix}
    \]
    so, write with me
    \begin{align*}
      \mat{T}_{(u,v)} \mat{T}_{(s,t)} &=
      \begin{bmatrix}
      \answer{1} & \answer{0} & \answer{u} \\
      \answer{0} & \answer{1} & \answer{v} \\
      \answer{0} & \answer{0} & \answer{1}
    \end{bmatrix}
    \begin{bmatrix}
      \answer{1} & \answer{0} & \answer{s} \\
      \answer{0} & \answer{1} & \answer{t} \\
      \answer{0} & \answer{0} & \answer{1}
    \end{bmatrix} \\
    &=\begin{bmatrix}
      \answer{1} & \answer{0} & \answer{u+s} \\
      \answer{0} & \answer{1} & \answer{v+t} \\
      \answer{0} & \answer{0} & \answer{1}
    \end{bmatrix}\\
    &=\mat{T}_{(u+s,v+t)}
    \end{align*}
    And boom! We're done.
  \end{prompt}
\end{question}


\begin{question}
Can you give a single translation that is equal to the following composition?
\[
\mat{T}_{(-7,5)}\mat{T}_{(0,-6)}\mat{T}_{(2,8)}\mat{T}_{(5,-4)}
\]

\begin{prompt}
  Absolutely, it's:
  \[
  T_{\answer{0},\answer{3}}
  \]
\end{prompt}
\end{question}


\begin{question}
Are compositions of translations commutative?  Are they
associative?
\begin{prompt}
\begin{selectAll}
\choice[correct]{Translations are commutative.}
\choice{Translations are not commutative.}
\choice[correct]{Translations are associative.}
\choice{Translations are not associative.}
\end{selectAll}
\begin{feedback}
Translations are commutative because addition is commutative. Translations are associative because function composition is always associative. 
\end{feedback}
\end{prompt}
\end{question}



\section{Compositions of reflections}

A composition of reflections occurs when two or more successive
reflections are applied to the same point. Check it out:

\begin{align*}
\mat{F}_{y=0}\mat{F}_{y=x} &= \begin{bmatrix}
1 &  0 & 0\\
0 & -1 & 0\\
0 &  0 & 1
\end{bmatrix}
\begin{bmatrix}
0 & 1 & 0\\
1 & 0 & 0\\
0 & 0 & 1
\end{bmatrix}\\
&= \begin{bmatrix}
 0 & 1 & 0\\
-1 & 0 & 0\\
 0 & 0 & 1
\end{bmatrix}
\end{align*}


\begin{question}
Is the composition $\mat{F}_{y=0}\mat{F}_{y=x}$ still a reflection?
\end{question}


\begin{question}
Are compositions of reflections commutative?  Are they
associative?
\end{question}




\section{Compositions of rotations}


A composition of rotations occurs when two or more successive
rotations are applied to the same point. Check it out: 
\begin{align*}
\mat{R}_{60}\mat{R}_{60} &= \begin{bmatrix}
\frac{1}{2} & \frac{-\sqrt{3}}{2} & 0\\
\frac{\sqrt{3}}{2} & \frac{1}{2} & 0\\
0 & 0 & 1
\end{bmatrix}
\begin{bmatrix}
\frac{1}{2} & \frac{-\sqrt{3}}{2} & 0\\
\frac{\sqrt{3}}{2} & \frac{1}{2} & 0\\
0 & 0 & 1
\end{bmatrix}\\
&= \begin{bmatrix}
\frac{-1}{2} & \frac{-\sqrt{3}}{2} & 0\\
\frac{\sqrt{3}}{2} & \frac{-1}{2} & 0\\
0 & 0 & 1
\end{bmatrix}
\end{align*}

\begin{theorem}
The product of two rotations $\mat{R}_\theta$ and $\mat{R}_\ph$ is
equal to the rotation $\mat{R}_{\theta+\ph}$.
\end{theorem}

From this we see that:
\[
\mat{R}_{120} =
\begin{bmatrix}
\frac{-1}{2} & \frac{-\sqrt{3}}{2} & 0\\
\frac{\sqrt{3}}{2} & \frac{-1}{2} & 0\\
0 & 0 & 1
\end{bmatrix}
\]

\begin{question} What is the rotation matrix for a $360^\circ$ rotation? What about a $405^\circ$ rotation?
\end{question}


\begin{question}
Are compositions of rotations commutative?  Are they
associative?
\end{question}


\end{document}
