\documentclass[handout,nooutcomes,noauthor,12pt]{ximera}
%% page layout
\usepackage[in,headings]{fullpage}
\raggedright
\setlength\headheight{13.6pt}


%% fonts
\usepackage{euler}

\usepackage{FiraMono}
\renewcommand\familydefault{\ttdefault} 
\usepackage{mathastext}
\usepackage[htt]{hyphenat}

\usepackage[T1]{fontenc}
\usepackage[scaled=1]{FiraSans}

\usepackage{wedn}
\usepackage[T1]{fontenc}

%% wrap text around scripts
\usepackage{wrapfig}

\tikzset{>=stealth}
%% snap! scripts
\usepackage{scratch3}

\usepackage{adjustbox}

%% journal style
\makeatletter
\newcommand\journalstyle{%
  \def\activitystyle{activity-chapter}
  \def\maketitle{%
    \addtocounter{titlenumber}{1}%
                {\flushleft\small\sffamily\bfseries\@pretitle\par\vspace{-1.5em}}%
                {\flushleft\LARGE\sffamily\bfseries\thetitlenumber\hspace{1em}\@title \par }%
                {\vskip .6em\noindent\textit\theabstract\setcounter{question}{0}\setcounter{sectiontitlenumber}{0}}%
                    \par\vspace{2em}
                    \phantomsection\addcontentsline{toc}{section}{\thetitlenumber\hspace{1em}\textbf{\@title}}%
                     }}
\makeatother



%% thm like environments
\let\question\relax
\let\endquestion\relax

\newtheoremstyle{QuestionStyle}{\topsep}{\topsep}%%% space between body and thm
		{}                      %%% Thm body font
		{}                              %%% Indent amount (empty = no indent)
		{\bfseries}            %%% Thm head font
		{)}                              %%% Punctuation after thm head
		{ }                           %%% Space after thm head
		{\thmnumber{#2}\thmnote{ \bfseries(#3)}}%%% Thm head spec
\theoremstyle{QuestionStyle}
\newtheorem{question}{}



\let\freeResponse\relax
\let\endfreeResponse\relax

%% \newtheoremstyle{ResponseStyle}{\topsep}{\topsep}%%% space between body and thm
%% 		{\wedn\bfseries}                      %%% Thm body font
%% 		{}                              %%% Indent amount (empty = no indent)
%% 		{\wedn\bfseries}            %%% Thm head font
%% 		{}                              %%% Punctuation after thm head
%% 		{3ex}                           %%% Space after thm head
%% 		{\underline{\underline{\thmname{#1}}}}%%% Thm head spec
%% \theoremstyle{ResponseStyle}

\usepackage[tikz]{mdframed}
\mdfdefinestyle{ResponseStyle}{leftmargin=1cm,linecolor=black,roundcorner=5pt,frametitlefont=\wedn\bfseries,%frametitle={\underline{\underline{Response}}:}
, font=\wedn\bfseries,}%\begin{mdframed}[style=mystyle]foo\end{mdframed}


\ifhandout
\NewEnviron{freeResponse}{}
\else
%\newtheorem{freeResponse}{Response:}
\newenvironment{freeResponse}{\begin{mdframed}[style=ResponseStyle]}{\end{mdframed}}
\fi



%% attempting to automate outcomes.

\newwrite\outcomefile
  \immediate\openout\outcomefile=\jobname.oc
\renewcommand{\outcome}[1]{\edef\theoutcomes{\theoutcomes #1~}%
\immediate\write\outcomefile{\unexpanded{\outcome}{#1}}}

%% \newcommand{\outcomelist}{\begin{itemize}\theoutcomes\end{itemize}}



%% my commands

\newcommand{\snap}{{\bfseries\itshape\textsf{Snap!}}}
\newcommand{\flavor}{\link[\snap]{https://snap.berkeley.edu/}}


\usepackage{tkz-euclide}
\tikzstyle geometryDiagrams=[rounded corners=.5pt,ultra thick,color=black]
\colorlet{penColor}{black} % Color of a curve in a plot

\title{Introduction}

\author{Bart Snapp}

\begin{document}
\begin{abstract}
  Let's get started.
\end{abstract}
\maketitle

Welcome to this journal of your journey in mathematics---with a focus in
geometry. This is a course designed not just to teach but to actively engage
you in the learning process. This course is structured through contextualized
lessons that move simple, yet profound, ideas from  one context to another,
each time revealing a new aspect. This Geometry Journal has been enriched by the
contributions of many educators over the years. Most importantly, it will now
include a \textbf{new} contributor---YOU!

\subsection*{Our goals and your role}

In this course, we aim to:
\begin{enumerate}
  \item Reinforce foundational mathematics and support the mathematics needed
        for
        advanced concepts.
  \item Help students make fewer mistakes while learning to discover
        mathematics
        independently.
  \item Translate classroom mathematics into real-world mathematics.
  \item Encourage you to question, critique, and dismantle reasonable
        hypotheses in geometry and arithmetic.
  \item Make the learning process enjoyable and engaging.
\end{enumerate}

We believe in the maxim: ``It is best to learn by doing.'' Hence, you must
assume the  role  of an active learner, someone who does things to learn.

Your job is to \textbf{demonstrate understanding} through your
answers. You will do this by explaining your work and your ideas.
Your target audience is your peers—fellow students in this class. Your
explanations should be clear, specific, and insightful, demonstrating
your own understanding without overwhelming your readers. Your goal is
to enlighten, not to obscure.

\subsection*{The art of explanation}

To excel in this course, it's vital to adopt a reflective approach to
learning:
\begin{description}
  \item[Explain with Clarity:] When you explain a mathematical concept, imagine
    you're teaching a fellow student. Your explanation should be
    straightforward,
    using language and examples that resonate with your peers.
  \item[Be Convincing:] Show your understanding through detailed explanations.
    Your task is to make complex ideas accessible, not to transfer the burden
    of
    understanding to your reader.
  \item[Reflect and Review:] After completing a problem, take a step back.
    Return
    to it later with fresh eyes and critically assess your solution. Ask
    yourself:
    \begin{itemize}
      \item Have I addressed all aspects of the problem?
      \item Do the numbers and results logically make sense?
      \item Can I validate my results through a quick approximation
            or an alternative method?
    \end{itemize}
\end{description}

\subsection*{Seeking help and collaborative learning}

It's natural to encounter challenges or to question your solutions. In
such moments, don't hesitate to seek help. Articulate your process and
understanding, and openly discuss where you feel uncertain. Often, the
errors you encounter are common and discussing them can lead to
valuable insights for the entire class. Remember, \textbf{every
  mistake is a learning opportunity for everyone.}

\subsection*{Authors of this journal}

This ``geometry journal'' was created by Dr.\ Claire Merriman,
Dr.\ Jenny Sheldon, and Dr.\ Bart Snapp at The Ohio State
University. Additional content was developed/inspired by the work of
Dr.\ Betsy McNeal, Dr.\ Vic Ferdinand, Dr.\ Bradford Findell,
Dr.\ Herbert Clemens, and you.

% And of course, there is \textbf{one more
%   content creator} in this Journal, \textbf{YOU!}

\end{document}
