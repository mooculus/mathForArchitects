\documentclass{ximera}

%% page layout
\usepackage[in,headings]{fullpage}
\raggedright
\setlength\headheight{13.6pt}


%% fonts
\usepackage{euler}

\usepackage{FiraMono}
\renewcommand\familydefault{\ttdefault} 
\usepackage{mathastext}
\usepackage[htt]{hyphenat}

\usepackage[T1]{fontenc}
\usepackage[scaled=1]{FiraSans}

\usepackage{wedn}
\usepackage[T1]{fontenc}

%% wrap text around scripts
\usepackage{wrapfig}

\tikzset{>=stealth}
%% snap! scripts
\usepackage{scratch3}

\usepackage{adjustbox}

%% journal style
\makeatletter
\newcommand\journalstyle{%
  \def\activitystyle{activity-chapter}
  \def\maketitle{%
    \addtocounter{titlenumber}{1}%
                {\flushleft\small\sffamily\bfseries\@pretitle\par\vspace{-1.5em}}%
                {\flushleft\LARGE\sffamily\bfseries\thetitlenumber\hspace{1em}\@title \par }%
                {\vskip .6em\noindent\textit\theabstract\setcounter{question}{0}\setcounter{sectiontitlenumber}{0}}%
                    \par\vspace{2em}
                    \phantomsection\addcontentsline{toc}{section}{\thetitlenumber\hspace{1em}\textbf{\@title}}%
                     }}
\makeatother



%% thm like environments
\let\question\relax
\let\endquestion\relax

\newtheoremstyle{QuestionStyle}{\topsep}{\topsep}%%% space between body and thm
		{}                      %%% Thm body font
		{}                              %%% Indent amount (empty = no indent)
		{\bfseries}            %%% Thm head font
		{)}                              %%% Punctuation after thm head
		{ }                           %%% Space after thm head
		{\thmnumber{#2}\thmnote{ \bfseries(#3)}}%%% Thm head spec
\theoremstyle{QuestionStyle}
\newtheorem{question}{}



\let\freeResponse\relax
\let\endfreeResponse\relax

%% \newtheoremstyle{ResponseStyle}{\topsep}{\topsep}%%% space between body and thm
%% 		{\wedn\bfseries}                      %%% Thm body font
%% 		{}                              %%% Indent amount (empty = no indent)
%% 		{\wedn\bfseries}            %%% Thm head font
%% 		{}                              %%% Punctuation after thm head
%% 		{3ex}                           %%% Space after thm head
%% 		{\underline{\underline{\thmname{#1}}}}%%% Thm head spec
%% \theoremstyle{ResponseStyle}

\usepackage[tikz]{mdframed}
\mdfdefinestyle{ResponseStyle}{leftmargin=1cm,linecolor=black,roundcorner=5pt,frametitlefont=\wedn\bfseries,%frametitle={\underline{\underline{Response}}:}
, font=\wedn\bfseries,}%\begin{mdframed}[style=mystyle]foo\end{mdframed}


\ifhandout
\NewEnviron{freeResponse}{}
\else
%\newtheorem{freeResponse}{Response:}
\newenvironment{freeResponse}{\begin{mdframed}[style=ResponseStyle]}{\end{mdframed}}
\fi



%% attempting to automate outcomes.

\newwrite\outcomefile
  \immediate\openout\outcomefile=\jobname.oc
\renewcommand{\outcome}[1]{\edef\theoutcomes{\theoutcomes #1~}%
\immediate\write\outcomefile{\unexpanded{\outcome}{#1}}}

%% \newcommand{\outcomelist}{\begin{itemize}\theoutcomes\end{itemize}}



%% my commands

\newcommand{\snap}{{\bfseries\itshape\textsf{Snap!}}}
\newcommand{\flavor}{\link[\snap]{https://snap.berkeley.edu/}}


\usepackage{tkz-euclide}
\tikzstyle geometryDiagrams=[rounded corners=.5pt,ultra thick,color=black]
\colorlet{penColor}{black} % Color of a curve in a plot



\title{Tessellations}

\author{Bart Snapp}



\begin{document}
\begin{abstract}
  We introduce tessellations.
\end{abstract}
\maketitle

Go to the internet and look up M.C.\ Escher.\index{Escher, M.C.} He
was an artist. Look at some of his work. When you do your search be
sure to include the word ``tessellation''. OK? Back already? Very
good. Sometimes Escher worked with tessellations. What's a
tessellation? I'm glad you asked:

\begin{definition}\index{tessellation} A \textbf{tessellation} is a pattern of 
polygons fitted together to cover the entire plane without
overlapping.  
\end{definition}
While it is impossible to actually cover the entire plane with shapes,
if we give you enough of a tessellation, you should be able to continue
its pattern indefinitely.  Here are pieces of tessellations:
\[
\includegraphics{semiRegTess.pdf}
\]
On the left we have a tessellation of a square and an octagon. On the
right we have a ``brick-like'' tessellation.

\begin{definition}\index{tessellation!regular}\index{regular!tessellation}
A tessellation is called a \textbf{regular tessellation} if it is
composed of copies of a single regular polygon and these polygons meet
vertex to vertex.\index{regular!polygon}
\end{definition}


\begin{example} Here are some examples of regular tessellations:
\[
\includegraphics{regtess.pdf}
\]
\end{example}

Johannes Kepler\index{Kepler, Johannes}, who lived from 1571--1630,
was one of the first people to study tessellations. He certainly knew
the next theorem:

\begin{theorem} There are only three regular tessellations.
\end{theorem}

\begin{question} Why is the theorem above true?
\end{question}


Since one can prove that there are only three regular tessellations,
and we have shown three above, then that is all of them. On the other
hand there are lots of nonregular tessellations. Here are two
different ways to tessellate the plane with a
triangle:\index{tessellation!triangles}
\[
\includegraphics{triangletess.pdf}
\]
Here is a way that you can tessellate the plane with any old
quadrilateral:
\[\index{tessellation!any quadrilateral}\index{quadrilateral!tessellation of}
\includegraphics{quadtess.pdf}
\]

\section{Tessellations and Art}

How does one make art with tessellations? To start, a little
decoration goes a long way. Check this out: Decorate two squares as
such:
\[
\includegraphics{lightningsquares.pdf}
\]
Tessellate them randomly in the plane to get this lightning-like picture:
\[
\includegraphics{lightningtess.pdf}
\]
\begin{question} 
What sort of picture do you get if you tessellate these decorated
squares randomly in a plane?
\[
\includegraphics{watersquares.pdf}
\]
\end{question}


Another way to go is to start with your favorite tessellation:
\[
\includegraphics{nonconvextess.pdf}
\]
Then you modify it a bunch to get something different:
\[
\includegraphics{birdstess.pdf}
\]

\begin{question} What kind of art can you make with tessellations?
\end{question}



I'm not a very good artist, but I am a mathematician. So let's use a
tessellation to give a proof! Let me ask you something:

\begin{question} What is the most famous theorem in mathematics? 
\end{question}
Probably the Pythagorean Theorem comes to mind. Let's recall the statement of the Pythagorean Theorem:

\begin{theorem}[Pythagorean Theorem]\index{Pythagorean Theorem} Given a right triangle, the sum of the squares of the 
lengths of the two legs is equal to the square of the length of 
the hypotenuse.  Symbolically, if $a$ and $b$ represent the 
lengths of the legs and $c$ is the length of the hypotenuse, 
\[
\includegraphics{pbppyth.pdf}
\]
then 
\[
a^2 + b^2 = c^2.
\]
\end{theorem}


Let's give a proof! Check out this tessellation involving $2$ squares:
\[
\includegraphics{pbppyth2.pdf}
\]
\begin{question} How does the picture above ``prove'' the Pythagorean Theorem?
\end{question}
\begin{proof}[Solution]  
The white triangle is our right triangle. The area of the middle
overlaid square is $c^2$, the area of the small dark squares is $a^2$,
and the area of the medium lighter square is $b^2$. Now label all the
``parts'' of the large overlaid square:
\[
\includegraphics{pbppyth2a.pdf}
\]
From the picture we see that
\begin{align*}
a^2 &= \{\text{3 and 4}\}\\
b^2 &= \{\text{1, 2, and 5}\}\\
c^2 &= \{\text{1, 2, 3, 4, and 5}\}
\end{align*}
Hence
\[
c^2 = a^2 + b^2
\]
Since we can always put two squares together in this pattern, this
proof will work for any right triangle.
\end{proof}

\begin{question} Can you use the above tessellation to give a dissection proof of the Pythagorean Theorem?
\end{question}




\end{document}
