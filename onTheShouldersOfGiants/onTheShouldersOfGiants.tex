\documentclass[handout,nooutcomes,noauthor,hints,12pt]{ximera}

\graphicspath{  
{./}
{./whoAreYou/}
{./drawingWithTheTurtle/}
{./bisectionMethod/}
{./circles/}
{./anglesAndRightTriangles/}
{./lawOfSines/}
{./lawOfCosines/}
{./plotter/}
{./staircases/}
{./pitch/}
{./qualityControl/}
{./symmetry/}
{./nGonBlock/}
}


%% page layout
\usepackage[cm,headings]{fullpage}
\raggedright
\setlength\headheight{13.6pt}


%% fonts
\usepackage{euler}

\usepackage{FiraMono}
\renewcommand\familydefault{\ttdefault} 
\usepackage[defaultmathsizes]{mathastext}
\usepackage[htt]{hyphenat}

\usepackage[T1]{fontenc}
\usepackage[scaled=1]{FiraSans}

%\usepackage{wedn}
\usepackage{pbsi} %% Answer font


\usepackage{cancel} %% strike through in pitch/pitch.tex


%% \usepackage{ulem} %% 
%% \renewcommand{\ULthickness}{2pt}% changes underline thickness

\tikzset{>=stealth}

\usepackage{adjustbox}

\setcounter{titlenumber}{-1}

%% journal style
\makeatletter
\newcommand\journalstyle{%
  \def\activitystyle{activity-chapter}
  \def\maketitle{%
    \addtocounter{titlenumber}{1}%
                {\flushleft\small\sffamily\bfseries\@pretitle\par\vspace{-1.5em}}%
                {\flushleft\LARGE\sffamily\bfseries\thetitlenumber\hspace{1em}\@title \par }%
                {\vskip .6em\noindent\textit\theabstract\setcounter{question}{0}\setcounter{sectiontitlenumber}{0}}%
                    \par\vspace{2em}
                    \phantomsection\addcontentsline{toc}{section}{\thetitlenumber\hspace{1em}\textbf{\@title}}%
                     }}
\makeatother



%% thm like environments
\let\question\relax
\let\endquestion\relax

\newtheoremstyle{QuestionStyle}{\topsep}{\topsep}%%% space between body and thm
		{}                      %%% Thm body font
		{}                              %%% Indent amount (empty = no indent)
		{\bfseries}            %%% Thm head font
		{)}                              %%% Punctuation after thm head
		{ }                           %%% Space after thm head
		{\thmnumber{#2}\thmnote{ \bfseries(#3)}}%%% Thm head spec
\theoremstyle{QuestionStyle}
\newtheorem{question}{}



\let\freeResponse\relax
\let\endfreeResponse\relax

%% \newtheoremstyle{ResponseStyle}{\topsep}{\topsep}%%% space between body and thm
%% 		{\wedn\bfseries}                      %%% Thm body font
%% 		{}                              %%% Indent amount (empty = no indent)
%% 		{\wedn\bfseries}            %%% Thm head font
%% 		{}                              %%% Punctuation after thm head
%% 		{3ex}                           %%% Space after thm head
%% 		{\underline{\underline{\thmname{#1}}}}%%% Thm head spec
%% \theoremstyle{ResponseStyle}

\usepackage[tikz]{mdframed}
\mdfdefinestyle{ResponseStyle}{leftmargin=1cm,linecolor=black,roundcorner=5pt,
, font=\bsifamily,}%font=\wedn\bfseries\upshape,}


\ifhandout
\NewEnviron{freeResponse}{}
\else
%\newtheorem{freeResponse}{Response:}
\newenvironment{freeResponse}{\begin{mdframed}[style=ResponseStyle]}{\end{mdframed}}
\fi



%% attempting to automate outcomes.

%% \newwrite\outcomefile
%%   \immediate\openout\outcomefile=\jobname.oc
%% \renewcommand{\outcome}[1]{\edef\theoutcomes{\theoutcomes #1~}%
%% \immediate\write\outcomefile{\unexpanded{\outcome}{#1}}}

%% \newcommand{\outcomelist}{\begin{itemize}\theoutcomes\end{itemize}}

%% \NewEnviron{listOutcomes}{\small\sffamily
%% After answering the following questions, students should be able to:
%% \begin{itemize}
%% \BODY
%% \end{itemize}
%% }
\usepackage[tikz]{mdframed}
\mdfdefinestyle{OutcomeStyle}{leftmargin=2cm,rightmargin=2cm,linecolor=black,roundcorner=5pt,
, font=\small\sffamily,}%font=\wedn\bfseries\upshape,}
\newenvironment{listOutcomes}{\begin{mdframed}[style=OutcomeStyle]After answering the following questions, students should be able to:\begin{itemize}}{\end{itemize}\end{mdframed}}



%% my commands

\newcommand{\snap}{{\bfseries\itshape\textsf{Snap!}}}
\newcommand{\flavor}{\link[\snap]{https://snap.berkeley.edu/}}
\newcommand{\mooculus}{\textsf{\textbf{MOOC}\textnormal{\textsf{ULUS}}}}


\usepackage{tkz-euclide}
\tikzstyle geometryDiagrams=[rounded corners=.5pt,ultra thick,color=black]
\colorlet{penColor}{black} % Color of a curve in a plot



\ifhandout\newcommand{\mynewpage}{\newpage}\else\newcommand{\mynewpage}{}\fi

\title{On the shoulders of giants}

\author{Bart Snapp}

\begin{document}
\begin{abstract}
  We think about scaling and its basic implications.
\end{abstract}
\maketitle

\begin{listOutcomes}
\item Understand how scaling affects length, area, and volume.
\item Connect length, area, and volume to real-world properties of objects.
\item Make deductions from ones understanding of scaling.
\item View unit conversion as a type of scaling.
\end{listOutcomes}

\mynewpage
\begin{question}
  We as humans have wonderful muscles that move our bodies
  around. A muscle's strength is roughly proportional to
  its cross-sectional area. See:
  \begin{center}
    \url{https://en.wikipedia.org/wiki/Physiological_cross-sectional_area}
  \end{center}
  On the other hand, the weight of our bodies is proportional to our
  volume. Explain why there can never be a $100'$ tall giant with human proportions.
  \begin{freeResponse}
    So let's say we scaled a $6'$ tall person to $100'$.
    \[
    \frac{100}{6}\approx 16.
    \]
    So the surface area (strength) of the muscles would increase by
    around $16^2$ times. But the weight would increase like the volume, so $16^3$ times.
    This means the giant would not be able to move, as it would be
    \[
    \frac{16^2}{16^3} = \frac{1}{16}
    \]
    as strong!
  \end{freeResponse}
\end{question}
\mynewpage


\begin{question}
  A Boeing $747$ weighs around $800,000$ pounds and is around $200$
  feet long.
  \begin{enumerate}
    \item If you made an exact scale model of a $747$, using all of
      the same materials, that was $1$ foot long, how much would you
      expect it to weigh? Explain your reasoning.
    \item Find an example of an everyday object that weighs about as
      much as your answer to the previous part.
  \end{enumerate}
      \begin{freeResponse}
        \begin{enumerate}
          \item Well, since the linear scale factor is $1/200$, the volume scale
          factor is $1/200^3$. Hence it would weigh
          \[
          \frac{8\cdot10^5}{200^3} =\frac{8\cdot10^5}{8\cdot 10^6}=1/10~\text{of a pound.}
          \]
        \item Well a tenth of a pound is about $45$ grams. And from
          \begin{center}
          \url{https://weightofstuff.com/things-that-weight-around-45-grams/}
          \end{center}
          we see: 10 sheets of paper, 3 empty soda cans, or 2 AA
          batteries. Wow, the plane is light!
        \end{enumerate}
        \end{freeResponse}
\end{question}
\mynewpage



\begin{question}
  Again, let's think about a Boeing $747$.
  \begin{enumerate}
  \item How \textbf{long} is Boeing $747$ in yards? What's the \textbf{scale factor} between feet and yards?
  \item A Boeing $747$ has a wing area of $5650$ square feet. What's the
    \textbf{wing area in square yards}? What's the \textbf{scale factor} between square
    feet and square yards?
  \item What's the \textbf{scale factor} between cubic feet and cubic yards?
  \item What's the connection between ``scale factors'' and ``unit conversions?''
  \end{enumerate}
\end{question}



\end{document}
	

    
