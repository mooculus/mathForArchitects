\documentclass[handout,nooutcomes,noauthor]{ximera}

%% page layout
\usepackage[in,headings]{fullpage}
\raggedright
\setlength\headheight{13.6pt}


%% fonts
\usepackage{euler}

\usepackage{FiraMono}
\renewcommand\familydefault{\ttdefault} 
\usepackage{mathastext}
\usepackage[htt]{hyphenat}

\usepackage[T1]{fontenc}
\usepackage[scaled=1]{FiraSans}

\usepackage{wedn}
\usepackage[T1]{fontenc}

%% wrap text around scripts
\usepackage{wrapfig}

\tikzset{>=stealth}
%% snap! scripts
\usepackage{scratch3}

\usepackage{adjustbox}

%% journal style
\makeatletter
\newcommand\journalstyle{%
  \def\activitystyle{activity-chapter}
  \def\maketitle{%
    \addtocounter{titlenumber}{1}%
                {\flushleft\small\sffamily\bfseries\@pretitle\par\vspace{-1.5em}}%
                {\flushleft\LARGE\sffamily\bfseries\thetitlenumber\hspace{1em}\@title \par }%
                {\vskip .6em\noindent\textit\theabstract\setcounter{question}{0}\setcounter{sectiontitlenumber}{0}}%
                    \par\vspace{2em}
                    \phantomsection\addcontentsline{toc}{section}{\thetitlenumber\hspace{1em}\textbf{\@title}}%
                     }}
\makeatother



%% thm like environments
\let\question\relax
\let\endquestion\relax

\newtheoremstyle{QuestionStyle}{\topsep}{\topsep}%%% space between body and thm
		{}                      %%% Thm body font
		{}                              %%% Indent amount (empty = no indent)
		{\bfseries}            %%% Thm head font
		{)}                              %%% Punctuation after thm head
		{ }                           %%% Space after thm head
		{\thmnumber{#2}\thmnote{ \bfseries(#3)}}%%% Thm head spec
\theoremstyle{QuestionStyle}
\newtheorem{question}{}



\let\freeResponse\relax
\let\endfreeResponse\relax

%% \newtheoremstyle{ResponseStyle}{\topsep}{\topsep}%%% space between body and thm
%% 		{\wedn\bfseries}                      %%% Thm body font
%% 		{}                              %%% Indent amount (empty = no indent)
%% 		{\wedn\bfseries}            %%% Thm head font
%% 		{}                              %%% Punctuation after thm head
%% 		{3ex}                           %%% Space after thm head
%% 		{\underline{\underline{\thmname{#1}}}}%%% Thm head spec
%% \theoremstyle{ResponseStyle}

\usepackage[tikz]{mdframed}
\mdfdefinestyle{ResponseStyle}{leftmargin=1cm,linecolor=black,roundcorner=5pt,frametitlefont=\wedn\bfseries,%frametitle={\underline{\underline{Response}}:}
, font=\wedn\bfseries,}%\begin{mdframed}[style=mystyle]foo\end{mdframed}


\ifhandout
\NewEnviron{freeResponse}{}
\else
%\newtheorem{freeResponse}{Response:}
\newenvironment{freeResponse}{\begin{mdframed}[style=ResponseStyle]}{\end{mdframed}}
\fi



%% attempting to automate outcomes.

\newwrite\outcomefile
  \immediate\openout\outcomefile=\jobname.oc
\renewcommand{\outcome}[1]{\edef\theoutcomes{\theoutcomes #1~}%
\immediate\write\outcomefile{\unexpanded{\outcome}{#1}}}

%% \newcommand{\outcomelist}{\begin{itemize}\theoutcomes\end{itemize}}



%% my commands

\newcommand{\snap}{{\bfseries\itshape\textsf{Snap!}}}
\newcommand{\flavor}{\link[\snap]{https://snap.berkeley.edu/}}


\usepackage{tkz-euclide}
\tikzstyle geometryDiagrams=[rounded corners=.5pt,ultra thick,color=black]
\colorlet{penColor}{black} % Color of a curve in a plot

\usepackage{fullpage}
\makeatletter
%% no number for activity
\newcommand\logostyle{%
  \def\activitystyle{activity-chapter}
  \def\maketitle{%
                {\flushleft\small\sffamily\bfseries\@pretitle\par\vspace{-1.5em}}%
                {\flushleft\LARGE\sffamily\bfseries\@title \par }%
                {\vskip .6em\noindent\textit\theabstract\setcounter{problem}{0}\setcounter{sectiontitlenumber}{0}}%
                    \par\vspace{2em}
                    \phantomsection\addcontentsline{toc}{section}{\textbf{\@title}}%
                     \setcounter{titlenumber}{0}}}
\makeatother
\newcommand{\nameblankgen}{\noindent\textbf{Name(s) (please print):}\ \hrulefill \\

\hrulefill}
\logostyle



%% %%% Graph paper background
%% \def\mygraphpaper{%
%%   \begin{tikzpicture}
%%     %\draw[line width=.4pt,draw=black!30] (0,0) grid[step=1mm] (\paperwidth,\paperheight);
%%     \draw[line width=.4pt,draw=blue!50] (0,0) grid[step=.2in] (\paperwidth,\paperheight);
%%   \end{tikzpicture}%
%% }
%% \usepackage{background}
%% \backgroundsetup{
%%   angle=0,
%%   contents=\mygraphpaper,
%%   color=black,
%%   scale=1,
%% }





\pagenumbering{gobble}


\title{A bigger apartment please}

\author{Bart Snapp}

\begin{document}
\begin{abstract}
  We study dilations.
\end{abstract}
\maketitle

\noindent\textbf{Group members (please print):}\ \hrulefill \\

\hrulefill


Marcy Matrix designs apartments. Here is her latest design:
\[
\includegraphics{apt.pdf}
\]
In the picture above, one square has a side-length of $5'$.  Marcy's
client, Large Linus, is unhappy with the design. He insists that it
must be three times bigger:  he wants every dimension of his
apartment to be bigger! For Marcy this is no problem at all because she
knows about the following \textit{dilation} matrix:
\[
\mat{D}_s = \begin{bmatrix}
s & 0 & 0 \\
0 & s & 0 \\
0 & 0 & 1
\end{bmatrix}
\]
This handy matrix will dilate (think scale) pictures by a factor of
$s$. However, there is one minor hiccup: Does Linus want
the \textit{perimeter} or the \textit{area} of his apartment to be
three times bigger?

\begin{problem}
What is the current perimeter of Linus' apartment? What is the current
area of Linus' apartment?
\end{problem}

                         
\begin{problem}
Apply $\mat{D}_3$ to every point of the apartment. What is the new
perimeter? What is the new area?
\end{problem}

\begin{problem}
What dilation matrix should Marcy use to make the area of Linus' apartment
three times bigger?  Use this matrix to actually scale the
apartment. What is the perimeter in this case?
\end{problem}

\begin{problem}
Once someone told me that ``five square feet'' is the same as ``five
feet square.'' Do you agree or disagree with this statement?  Explain
how you arrived at your conclusion.
\end{problem}


%% Maverick Metric likes Linus' apartment, but he wants the measurements
%% to be metric. This is not a problem for Marcy, as she knows the
%% following \textit{conversion} matrices (note, we rounded here!):
%% \[
%% \mat{C}_{\text{ft} \to \text{m}} = \begin{bmatrix} 
%% 0.305 & 0 & 0 \\
%% 0 & 0.305 & 0 \\
%% 0 & 0 & 1
%% \end{bmatrix}
%% \qquad\text{and}\qquad
%% \mat{C}_{\text{m} \to \text{ft}} = \begin{bmatrix} 
%% 3.28 & 0 & 0 \\
%% 0 & 3.28 & 0 \\
%% 0 & 0 & 1
%% \end{bmatrix}
%% \]

%% \begin{problem}
%% Compute:
%% \[
%% \mat{C}_{\text{ft} \to \text{m}}\mat{C}_{\text{m} \to \text{ft}} \qquad\text{and}\qquad \mat{C}_{\text{m} \to \text{ft}}\mat{C}_{\text{ft} \to \text{m}}
%% \]
%% What do you get in each case? Explain why this makes sense.
%% \end{problem}

%% \begin{problem}
%% What's the perimeter of the apartment in meters? What is the area in
%% square-meters?
%% \end{problem}

%% \begin{problem}
%% Compare/contrast the scaling matrix to the conversion matrix. What do
%% you notice?
%% \end{problem}


\begin{problem}
Maverick Metric likes Linus' apartment, but he wants the measurements
to be metric. He asks you for the measurement for the area of Linus'
apartment in feet. Since there are around $.3$ feet per meter,
Maverick Metric claims that the area of the apartment in meters is:
\[
\text{area of apartment in feet} \times .3 
\]
Is this correct? Hint: No. Explain why not and solve the problem
correctly.
\end{problem}




\end{document}
