\documentclass[nooutcomes,noauthor,hints,handout,12pt]{ximera}

\graphicspath{  
{./}
{./whoAreYou/}
{./drawingWithTheTurtle/}
{./bisectionMethod/}
{./circles/}
{./anglesAndRightTriangles/}
{./lawOfSines/}
{./lawOfCosines/}
{./plotter/}
{./staircases/}
{./pitch/}
{./qualityControl/}
{./symmetry/}
{./nGonBlock/}
}


%% page layout
\usepackage[cm,headings]{fullpage}
\raggedright
\setlength\headheight{13.6pt}


%% fonts
\usepackage{euler}

\usepackage{FiraMono}
\renewcommand\familydefault{\ttdefault} 
\usepackage[defaultmathsizes]{mathastext}
\usepackage[htt]{hyphenat}

\usepackage[T1]{fontenc}
\usepackage[scaled=1]{FiraSans}

%\usepackage{wedn}
\usepackage{pbsi} %% Answer font


\usepackage{cancel} %% strike through in pitch/pitch.tex


%% \usepackage{ulem} %% 
%% \renewcommand{\ULthickness}{2pt}% changes underline thickness

\tikzset{>=stealth}

\usepackage{adjustbox}

\setcounter{titlenumber}{-1}

%% journal style
\makeatletter
\newcommand\journalstyle{%
  \def\activitystyle{activity-chapter}
  \def\maketitle{%
    \addtocounter{titlenumber}{1}%
                {\flushleft\small\sffamily\bfseries\@pretitle\par\vspace{-1.5em}}%
                {\flushleft\LARGE\sffamily\bfseries\thetitlenumber\hspace{1em}\@title \par }%
                {\vskip .6em\noindent\textit\theabstract\setcounter{question}{0}\setcounter{sectiontitlenumber}{0}}%
                    \par\vspace{2em}
                    \phantomsection\addcontentsline{toc}{section}{\thetitlenumber\hspace{1em}\textbf{\@title}}%
                     }}
\makeatother



%% thm like environments
\let\question\relax
\let\endquestion\relax

\newtheoremstyle{QuestionStyle}{\topsep}{\topsep}%%% space between body and thm
		{}                      %%% Thm body font
		{}                              %%% Indent amount (empty = no indent)
		{\bfseries}            %%% Thm head font
		{)}                              %%% Punctuation after thm head
		{ }                           %%% Space after thm head
		{\thmnumber{#2}\thmnote{ \bfseries(#3)}}%%% Thm head spec
\theoremstyle{QuestionStyle}
\newtheorem{question}{}



\let\freeResponse\relax
\let\endfreeResponse\relax

%% \newtheoremstyle{ResponseStyle}{\topsep}{\topsep}%%% space between body and thm
%% 		{\wedn\bfseries}                      %%% Thm body font
%% 		{}                              %%% Indent amount (empty = no indent)
%% 		{\wedn\bfseries}            %%% Thm head font
%% 		{}                              %%% Punctuation after thm head
%% 		{3ex}                           %%% Space after thm head
%% 		{\underline{\underline{\thmname{#1}}}}%%% Thm head spec
%% \theoremstyle{ResponseStyle}

\usepackage[tikz]{mdframed}
\mdfdefinestyle{ResponseStyle}{leftmargin=1cm,linecolor=black,roundcorner=5pt,
, font=\bsifamily,}%font=\wedn\bfseries\upshape,}


\ifhandout
\NewEnviron{freeResponse}{}
\else
%\newtheorem{freeResponse}{Response:}
\newenvironment{freeResponse}{\begin{mdframed}[style=ResponseStyle]}{\end{mdframed}}
\fi



%% attempting to automate outcomes.

%% \newwrite\outcomefile
%%   \immediate\openout\outcomefile=\jobname.oc
%% \renewcommand{\outcome}[1]{\edef\theoutcomes{\theoutcomes #1~}%
%% \immediate\write\outcomefile{\unexpanded{\outcome}{#1}}}

%% \newcommand{\outcomelist}{\begin{itemize}\theoutcomes\end{itemize}}

%% \NewEnviron{listOutcomes}{\small\sffamily
%% After answering the following questions, students should be able to:
%% \begin{itemize}
%% \BODY
%% \end{itemize}
%% }
\usepackage[tikz]{mdframed}
\mdfdefinestyle{OutcomeStyle}{leftmargin=2cm,rightmargin=2cm,linecolor=black,roundcorner=5pt,
, font=\small\sffamily,}%font=\wedn\bfseries\upshape,}
\newenvironment{listOutcomes}{\begin{mdframed}[style=OutcomeStyle]After answering the following questions, students should be able to:\begin{itemize}}{\end{itemize}\end{mdframed}}



%% my commands

\newcommand{\snap}{{\bfseries\itshape\textsf{Snap!}}}
\newcommand{\flavor}{\link[\snap]{https://snap.berkeley.edu/}}
\newcommand{\mooculus}{\textsf{\textbf{MOOC}\textnormal{\textsf{ULUS}}}}


\usepackage{tkz-euclide}
\tikzstyle geometryDiagrams=[rounded corners=.5pt,ultra thick,color=black]
\colorlet{penColor}{black} % Color of a curve in a plot



\ifhandout\newcommand{\mynewpage}{\newpage}\else\newcommand{\mynewpage}{}\fi

\title{Rain, rain\dots}

\author{Bart Snapp}

\begin{document}
\begin{abstract}
  We think about proportions in a real-world setting.
\end{abstract}
\maketitle

\begin{listOutcomes}
\item Think about proportions.
\item Critique and dismantle reasonable hypotheses in regard to
  geometry and arithmetic.
\end{listOutcomes}
Check out this dialogue between two students (based on a true story):

\begin{mdframed}[style=OutcomeStyle]
  \begin{quote}
\begin{dialogue}
\item[Devyn] Riley, there was a lot of rain last night.
\item[Riley] April showers make May flowers. 
\item[Devyn] I left my $1'$ diameter soup-pot out in the rain, it's
  filled with water---it was empty before it started raining.
\item[Riley] Hmm. I left my $6'$ diameter kiddie-pool out in the rain.
  It was also emptied before the rain, and now it is full of water.
\item[Devyn] I'm sure the water level in my soup-pot is higher,
  because there is less volume to fill.
\item[Riley] No way! Since my kiddie-pool collected more rain (it's
  bigger yo!) the water level is higher in my pool.
\end{dialogue}
  \end{quote}
\end{mdframed}
We'll think about this conversation. 




\mynewpage


\begin{question}
  Both the soup-pot and the kiddie-pool are approximately cylinders.
  \begin{enumerate}
  \item Compute the VOLUME of rain-water of the soup-pot and
    kiddie-pool respectively with respect to the HEIGHT of the
    water-level in inches. Show all work and explain your reasoning.
  \item Use your previous answer to explain why:
    \begin{quote}
      the \textbf{volume} of water in a cylinder filled to a height $h$ \\ \quad is
      \textbf{proportional} to \\
      \quad\quad the \textbf{area of the base} of the cylinder.
    \end{quote}
    As a gesture of friendship, I'll remind you that a value $x$ is
    \textbf{proportional} to a value $y$ if
  \[
  x = k y
  \]
  where $k$ is a constant and does not change when $x$ and $y$ change.
  \end{enumerate}
  \begin{freeResponse}
    \begin{enumerate}
      \item For a given height of rain water in either container, say
        $h$ inches, find the volume of rain water in each container.
        \[
        V_{soup-pot} = 3 \pi 6^2
        \]
        and
        \[
        V_{kiddie-pool} = 3 \pi 36^2 
        \]
      \item The VOLUME of water in each cylinder is proportional to
        the AREA of the base. We know this because the formula for the
        VOLUME of a cylinder is
        \begin{align*}
          \text{volume} &= h \cdot \pi r^2\\
          &= h \cdot A.
        \end{align*}
        Thus $V = h\cdot A$.
    \end{enumerate}
  \end{freeResponse}
\end{question}
\mynewpage



\begin{question}
  Now let's think about rain. When it rains, assume that the amount of
  water is distributed across the ground evenly.
  \begin{enumerate}
  \item Explain why the VOLUME of rain collected in the cylinders is
    proportional to the area of the base.
  \item What's the constant of this proportion? Think about its UNITS.
  \item Use your answers to all questions above to RESOLVE Devyn and
    Riley's question of which water level is higher.
  \end{enumerate}
  \begin{freeResponse}
    \begin{enumerate}
    \item Since the rain is coming down the same way in every
      location, the only variable of how much rain is collected is the
      area of the base. So
      \[
      V= k A.
      \]
    \item The constant of proportionality has units of inches, because
      the left-hand side of the equation has units of inches cubed,
      and area has units of inches squared. Hence it must be the height
      of the water.
    \item The height in either cylinder will be the SAME. 
    \end{enumerate}
  \end{freeResponse}
\end{question}
\mynewpage



\begin{question}
  Devyn and Riley go to move their cylinders without dumping the
  water. Devyn's weighs around $10$ pounds. At this point their mutual
  friend \textit{Geometry Giorgio} shows up and claims:
  \begin{quote}
    Since Devyn's soup-pot and water weigh $10$ pounds, and Riley's
    kiddie pool has $6$ times the diameter, then Riley's kiddie-pool
    and water must weigh $10\cdot 6^3 = 2160$ pounds.
  \end{quote}
  \textbf{Is Geometry Giorgio’s claim correct?  If so, EXPLAIN why. If
    not, explain why not and give a correct solution.}
  \begin{freeResponse}
    Geometry Giorgio’s claim is not correct. Since the height of the
    water in the soup-pot is the same as the height of the water in
    the kiddie-pool, the the scale factor is $6^2$.
    \[
    10\cdot 6^2=360~\text{pounds}.
    \]
  \end{freeResponse}
\end{question}



\end{document}
