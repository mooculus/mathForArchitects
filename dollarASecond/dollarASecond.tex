\documentclass[nooutcomes,noauthor,hints]{ximera}

\graphicspath{  
{./}
{./whoAreYou/}
{./drawingWithTheTurtle/}
{./bisectionMethod/}
{./circles/}
{./anglesAndRightTriangles/}
{./lawOfSines/}
{./lawOfCosines/}
{./plotter/}
{./staircases/}
{./pitch/}
{./qualityControl/}
{./symmetry/}
{./nGonBlock/}
}


%% page layout
\usepackage[cm,headings]{fullpage}
\raggedright
\setlength\headheight{13.6pt}


%% fonts
\usepackage{euler}

\usepackage{FiraMono}
\renewcommand\familydefault{\ttdefault} 
\usepackage[defaultmathsizes]{mathastext}
\usepackage[htt]{hyphenat}

\usepackage[T1]{fontenc}
\usepackage[scaled=1]{FiraSans}

%\usepackage{wedn}
\usepackage{pbsi} %% Answer font


\usepackage{cancel} %% strike through in pitch/pitch.tex


%% \usepackage{ulem} %% 
%% \renewcommand{\ULthickness}{2pt}% changes underline thickness

\tikzset{>=stealth}

\usepackage{adjustbox}

\setcounter{titlenumber}{-1}

%% journal style
\makeatletter
\newcommand\journalstyle{%
  \def\activitystyle{activity-chapter}
  \def\maketitle{%
    \addtocounter{titlenumber}{1}%
                {\flushleft\small\sffamily\bfseries\@pretitle\par\vspace{-1.5em}}%
                {\flushleft\LARGE\sffamily\bfseries\thetitlenumber\hspace{1em}\@title \par }%
                {\vskip .6em\noindent\textit\theabstract\setcounter{question}{0}\setcounter{sectiontitlenumber}{0}}%
                    \par\vspace{2em}
                    \phantomsection\addcontentsline{toc}{section}{\thetitlenumber\hspace{1em}\textbf{\@title}}%
                     }}
\makeatother



%% thm like environments
\let\question\relax
\let\endquestion\relax

\newtheoremstyle{QuestionStyle}{\topsep}{\topsep}%%% space between body and thm
		{}                      %%% Thm body font
		{}                              %%% Indent amount (empty = no indent)
		{\bfseries}            %%% Thm head font
		{)}                              %%% Punctuation after thm head
		{ }                           %%% Space after thm head
		{\thmnumber{#2}\thmnote{ \bfseries(#3)}}%%% Thm head spec
\theoremstyle{QuestionStyle}
\newtheorem{question}{}



\let\freeResponse\relax
\let\endfreeResponse\relax

%% \newtheoremstyle{ResponseStyle}{\topsep}{\topsep}%%% space between body and thm
%% 		{\wedn\bfseries}                      %%% Thm body font
%% 		{}                              %%% Indent amount (empty = no indent)
%% 		{\wedn\bfseries}            %%% Thm head font
%% 		{}                              %%% Punctuation after thm head
%% 		{3ex}                           %%% Space after thm head
%% 		{\underline{\underline{\thmname{#1}}}}%%% Thm head spec
%% \theoremstyle{ResponseStyle}

\usepackage[tikz]{mdframed}
\mdfdefinestyle{ResponseStyle}{leftmargin=1cm,linecolor=black,roundcorner=5pt,
, font=\bsifamily,}%font=\wedn\bfseries\upshape,}


\ifhandout
\NewEnviron{freeResponse}{}
\else
%\newtheorem{freeResponse}{Response:}
\newenvironment{freeResponse}{\begin{mdframed}[style=ResponseStyle]}{\end{mdframed}}
\fi



%% attempting to automate outcomes.

%% \newwrite\outcomefile
%%   \immediate\openout\outcomefile=\jobname.oc
%% \renewcommand{\outcome}[1]{\edef\theoutcomes{\theoutcomes #1~}%
%% \immediate\write\outcomefile{\unexpanded{\outcome}{#1}}}

%% \newcommand{\outcomelist}{\begin{itemize}\theoutcomes\end{itemize}}

%% \NewEnviron{listOutcomes}{\small\sffamily
%% After answering the following questions, students should be able to:
%% \begin{itemize}
%% \BODY
%% \end{itemize}
%% }
\usepackage[tikz]{mdframed}
\mdfdefinestyle{OutcomeStyle}{leftmargin=2cm,rightmargin=2cm,linecolor=black,roundcorner=5pt,
, font=\small\sffamily,}%font=\wedn\bfseries\upshape,}
\newenvironment{listOutcomes}{\begin{mdframed}[style=OutcomeStyle]After answering the following questions, students should be able to:\begin{itemize}}{\end{itemize}\end{mdframed}}



%% my commands

\newcommand{\snap}{{\bfseries\itshape\textsf{Snap!}}}
\newcommand{\flavor}{\link[\snap]{https://snap.berkeley.edu/}}
\newcommand{\mooculus}{\textsf{\textbf{MOOC}\textnormal{\textsf{ULUS}}}}


\usepackage{tkz-euclide}
\tikzstyle geometryDiagrams=[rounded corners=.5pt,ultra thick,color=black]
\colorlet{penColor}{black} % Color of a curve in a plot



\ifhandout\newcommand{\mynewpage}{\newpage}\else\newcommand{\mynewpage}{}\fi

\title{A dollar a second}

\author{Bart Snapp}

\begin{document}
\begin{abstract}
  Let's start to think about the size of numbers.
\end{abstract}
\maketitle

\begin{listOutcomes}
\item SIGNIFICANT DIGITS?
\end{listOutcomes}


It is difficult to grasp the size of large numbers without a context
that seems real to us. We'll use the context of ``cold, hard, cash''
and the ``cruel curse of time'' to help us understand the magnitude of
numbers.  Dream this daydream with me:
\begin{mdframed}[style=OutcomeStyle]
  \begin{quote}
    \textbf{Dollar-a-second Daydream:}\\
  Suppose you were paid the secondly wage of $\$1$, meaning you are
  paid a dollar \text{every second} of every minute, of every hour, of
  every day, and so on.
\end{quote}
\end{mdframed}
Below we will learn about numbers by exploring this daydream.



\mynewpage

\begin{question}
  How long will it take you to collect $\$100,000$? EXPLAIN using
  words, equations, and/or pictures (as helpful/needed) how you arrive
  at your conclusion and give your answer to the nearest DAY.
  \begin{freeResponse}
    Well, all we need are $100,000$ seconds to make $\$100,000$. So
    we'll divide $100,000$ (the number of seconds we need) by $60$
    (number of seconds in a minute) times $60$ (number of minutes in
    an hour) times $24$ (number of hours in a day.  Since
    \[
    \frac{100000}{60\cdot 60\cdot 24} = 1.1574\dots
    \]
    it would take us $1$ day to make around $100,000$.
  \end{freeResponse}
\end{question}
\mynewpage


%% I REALLY WANT SIGNIFICAT DIGITS.


\begin{question}
  Use your answer from the previous question to reason how much money
  you could collect in:
  \begin{enumerate}
  \item A week.
  \item A month.
  \item A year.
  \item Twenty years.
  \item A lifetime.
  \end{enumerate}
  \textbf{Round your answer from each part to $\boldsymbol{1}$ significant digit.}
  At each step, you may/should use the previous step to help. Use
  ROUGH estimates and EXPLAIN how these can all be computed WITHOUT a
  calculator.
  \begin{freeResponse}
    \begin{enumerate}
      \item So if we collect around $\$100,000$ in one day, we'll
        collect around $\$700,000$ in a week as there are $7$ days in
        a week.
      \item Since there are around $4$ weeks in a month, we'll collect
        $4\cdot 700000 = 2800000$ or nearly $3$ million dollars.
      \item In a year, there are around $10$ months, and we collect
        $3$ million per month. So $30$ million dollars.
      \item In twenty years, we'll collect $30\cdot 20 = 600$ million.
      \item In a lifetime, say $100$ years, we'll collect $(600\cdot 5)$
        million, also known as $3$ billion.
    \end{enumerate}
  \end{freeResponse}
\end{question}
\mynewpage

\begin{question}
  Let's turn the last question on it's head. Use your work from the
  PREVIOUS PARTS to tell me about how long it would take to collect:
  \begin{enumerate}
  \item $1$ million dollars?
  \item $1$ billion dollars?
  \item $1$ trillion dollars?
  \item Use the INTERNET to look up the wealthiest person in the
    world. However much wealth they have.
  \end{enumerate}
  In each case, \textbf{round your answer from each part to $\boldsymbol{1}$
    significant digit} and EXPLAIN your reasoning.
  \begin{freeResponse}
    \begin{itemize}
      \item Since we collect around $\$7k$ in about a week, we'll
        collect a million in less than two weeks.
      \item Since we collect around $\$3$ billion in 100 years, it
        will take around $30$ years to collect $1$ billion dollars.
      \item A trillion is a thousand billion, so it would take around
        $30,000$ years to collect that much.
      \item Answers will vary.
    \end{itemize}
  \end{freeResponse}
\end{question}




\end{document}
