\documentclass[noauthor,nooutcomes,hints,handout,12pt]{ximera}

\graphicspath{  
{./}
{./whoAreYou/}
{./drawingWithTheTurtle/}
{./bisectionMethod/}
{./circles/}
{./anglesAndRightTriangles/}
{./lawOfSines/}
{./lawOfCosines/}
{./plotter/}
{./staircases/}
{./pitch/}
{./qualityControl/}
{./symmetry/}
{./nGonBlock/}
}


%% page layout
\usepackage[cm,headings]{fullpage}
\raggedright
\setlength\headheight{13.6pt}


%% fonts
\usepackage{euler}

\usepackage{FiraMono}
\renewcommand\familydefault{\ttdefault} 
\usepackage[defaultmathsizes]{mathastext}
\usepackage[htt]{hyphenat}

\usepackage[T1]{fontenc}
\usepackage[scaled=1]{FiraSans}

%\usepackage{wedn}
\usepackage{pbsi} %% Answer font


\usepackage{cancel} %% strike through in pitch/pitch.tex


%% \usepackage{ulem} %% 
%% \renewcommand{\ULthickness}{2pt}% changes underline thickness

\tikzset{>=stealth}

\usepackage{adjustbox}

\setcounter{titlenumber}{-1}

%% journal style
\makeatletter
\newcommand\journalstyle{%
  \def\activitystyle{activity-chapter}
  \def\maketitle{%
    \addtocounter{titlenumber}{1}%
                {\flushleft\small\sffamily\bfseries\@pretitle\par\vspace{-1.5em}}%
                {\flushleft\LARGE\sffamily\bfseries\thetitlenumber\hspace{1em}\@title \par }%
                {\vskip .6em\noindent\textit\theabstract\setcounter{question}{0}\setcounter{sectiontitlenumber}{0}}%
                    \par\vspace{2em}
                    \phantomsection\addcontentsline{toc}{section}{\thetitlenumber\hspace{1em}\textbf{\@title}}%
                     }}
\makeatother



%% thm like environments
\let\question\relax
\let\endquestion\relax

\newtheoremstyle{QuestionStyle}{\topsep}{\topsep}%%% space between body and thm
		{}                      %%% Thm body font
		{}                              %%% Indent amount (empty = no indent)
		{\bfseries}            %%% Thm head font
		{)}                              %%% Punctuation after thm head
		{ }                           %%% Space after thm head
		{\thmnumber{#2}\thmnote{ \bfseries(#3)}}%%% Thm head spec
\theoremstyle{QuestionStyle}
\newtheorem{question}{}



\let\freeResponse\relax
\let\endfreeResponse\relax

%% \newtheoremstyle{ResponseStyle}{\topsep}{\topsep}%%% space between body and thm
%% 		{\wedn\bfseries}                      %%% Thm body font
%% 		{}                              %%% Indent amount (empty = no indent)
%% 		{\wedn\bfseries}            %%% Thm head font
%% 		{}                              %%% Punctuation after thm head
%% 		{3ex}                           %%% Space after thm head
%% 		{\underline{\underline{\thmname{#1}}}}%%% Thm head spec
%% \theoremstyle{ResponseStyle}

\usepackage[tikz]{mdframed}
\mdfdefinestyle{ResponseStyle}{leftmargin=1cm,linecolor=black,roundcorner=5pt,
, font=\bsifamily,}%font=\wedn\bfseries\upshape,}


\ifhandout
\NewEnviron{freeResponse}{}
\else
%\newtheorem{freeResponse}{Response:}
\newenvironment{freeResponse}{\begin{mdframed}[style=ResponseStyle]}{\end{mdframed}}
\fi



%% attempting to automate outcomes.

%% \newwrite\outcomefile
%%   \immediate\openout\outcomefile=\jobname.oc
%% \renewcommand{\outcome}[1]{\edef\theoutcomes{\theoutcomes #1~}%
%% \immediate\write\outcomefile{\unexpanded{\outcome}{#1}}}

%% \newcommand{\outcomelist}{\begin{itemize}\theoutcomes\end{itemize}}

%% \NewEnviron{listOutcomes}{\small\sffamily
%% After answering the following questions, students should be able to:
%% \begin{itemize}
%% \BODY
%% \end{itemize}
%% }
\usepackage[tikz]{mdframed}
\mdfdefinestyle{OutcomeStyle}{leftmargin=2cm,rightmargin=2cm,linecolor=black,roundcorner=5pt,
, font=\small\sffamily,}%font=\wedn\bfseries\upshape,}
\newenvironment{listOutcomes}{\begin{mdframed}[style=OutcomeStyle]After answering the following questions, students should be able to:\begin{itemize}}{\end{itemize}\end{mdframed}}



%% my commands

\newcommand{\snap}{{\bfseries\itshape\textsf{Snap!}}}
\newcommand{\flavor}{\link[\snap]{https://snap.berkeley.edu/}}
\newcommand{\mooculus}{\textsf{\textbf{MOOC}\textnormal{\textsf{ULUS}}}}


\usepackage{tkz-euclide}
\tikzstyle geometryDiagrams=[rounded corners=.5pt,ultra thick,color=black]
\colorlet{penColor}{black} % Color of a curve in a plot



\ifhandout\newcommand{\mynewpage}{\newpage}\else\newcommand{\mynewpage}{}\fi


\title{Nothing but nets}
\author{Bart Snapp}

\begin{document}
\begin{abstract}
  We use nets to study paths on boxes.
\end{abstract}
\maketitle

\begin{listOutcomes}
\item Draw nets of cubes and boxes.
\item Use nets to find shortest paths on cubes and boxes.
\end{listOutcomes}



We are going to think about cubes, how a paper cube could be unfolded, and about paths on these cubes. An unfolded cube is called a \textit{net}.


\begin{definition}
  A \textbf{net}\index{net of a polyhedron} of a polyhedron is a
  single piece arrangement of polygons that are connected along their
  edges so that they can be folded into the polyhedron.
\end{definition}


\mynewpage


\begin{question}
  There are $11$ distinct nets for cubes. Use the INTERNET (or your
  brain) to find all $11$ nets for a cube. Carefully draw them here.
  
\end{question}
\mynewpage

\begin{question}
  Here's an old puzzler: Suppose you are an ant standing at the corner
  of a cube whose side length is $10'$. Find and describe the shortest
  path the ant could walk to get to the opposite corner of the cube. What's the
  length of this shortest path? Show all work and explain your reasoning.
 %%  Time to think about nets.
%%   \begin{enumerate}
%%   \item Draw all possible nets of regular tetrahedra.
%%   \item Think about building these nets using poster-board and
%%     tape. Which nets would be ``best'' to use and why? Think about
%%     ease of construction, strength of final solid, number of folds,
%%     number of cuts, amount of tape, and \textbf{at least one more}
%%     consideration.
%%   \end{enumerate}
\end{question}
\mynewpage

\begin{question}
  Here's another old puzzler: Suppose you have a room that is the
  shape of a $10'\times 10' \times 30'$ box. Suppose you are an ant
  standing $1'$ directly below the middle of the top edge of one of
  the $10'\times 10'$ faces.  Find and describe the shortest path the
  ant could walk to get $1'$ directly above the middle of the bottom
  edge of the opposite $10'\times10'$ face.  What's the length of this
  shortest path? Show all work and explain your reasoning.

  %% Most of the mathematics you learn in school is very old. The concept
  %% of Platonic solids is thousands of years old. However, there is
  %% always new knowledge to be created. As far as I know, this question
  %% \begin{quote}
  %%   ``Consider a Platonic solid. Given a side length, what's the
  %%   \textbf{average} width of the solid?''
  %% \end{quote}
  %% wasn't completely addressed until the late Twentieth Century.
  %% %% https://mathoverflow.net/questions/306318/average-caliper-diameter-mean-width-of-a-polyhedron
  %% %% [1] J.W. Cahn, The significance of average mean curvature and its determination by quantitative metallography, Trans. Met. Soc. AIME 239 (1967) 610-616. behind a paywall
  %% %% [2] R.E. Miles, Poisson flats in Euclidean spaces. Part I: A finite number of random uniform flats, Adv. App. Prob. 1 (1969) 211-237.
  %% %% [3] R.E. Miles, Direct Derivations of Certain Surface Integral Formulae for the Mean Projections of a Convex Set, Adv. Applied Prob. 7 (1975), 818-829.
  %% The answer is given by the formula
  %% \[
  %% n \cdot l \cdot (180-\theta)/720
  %% \]
  %% where $n$ is the number of edges, $l$ is the edge-length, and
  %% $\theta$ is the \textbf{dihedral angle} of the regular $n$-gon for the given
  %% Platonic solid. Suppose you wish to make a model of each of the
  %% Platonic solids so that their average width is $10$
  %% centimeters. \textbf{What should the edge length be in each case?
  %%   Show your work.}
\end{question}

\end{document}
