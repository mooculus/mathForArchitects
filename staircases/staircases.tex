\documentclass[noauthor,nooutcomes,handout,12pt]{ximera}

\graphicspath{  
{./}
{./whoAreYou/}
{./drawingWithTheTurtle/}
{./bisectionMethod/}
{./circles/}
{./anglesAndRightTriangles/}
{./lawOfSines/}
{./lawOfCosines/}
{./plotter/}
{./staircases/}
{./pitch/}
{./qualityControl/}
{./symmetry/}
{./nGonBlock/}
}


%% page layout
\usepackage[cm,headings]{fullpage}
\raggedright
\setlength\headheight{13.6pt}


%% fonts
\usepackage{euler}

\usepackage{FiraMono}
\renewcommand\familydefault{\ttdefault} 
\usepackage[defaultmathsizes]{mathastext}
\usepackage[htt]{hyphenat}

\usepackage[T1]{fontenc}
\usepackage[scaled=1]{FiraSans}

%\usepackage{wedn}
\usepackage{pbsi} %% Answer font


\usepackage{cancel} %% strike through in pitch/pitch.tex


%% \usepackage{ulem} %% 
%% \renewcommand{\ULthickness}{2pt}% changes underline thickness

\tikzset{>=stealth}

\usepackage{adjustbox}

\setcounter{titlenumber}{-1}

%% journal style
\makeatletter
\newcommand\journalstyle{%
  \def\activitystyle{activity-chapter}
  \def\maketitle{%
    \addtocounter{titlenumber}{1}%
                {\flushleft\small\sffamily\bfseries\@pretitle\par\vspace{-1.5em}}%
                {\flushleft\LARGE\sffamily\bfseries\thetitlenumber\hspace{1em}\@title \par }%
                {\vskip .6em\noindent\textit\theabstract\setcounter{question}{0}\setcounter{sectiontitlenumber}{0}}%
                    \par\vspace{2em}
                    \phantomsection\addcontentsline{toc}{section}{\thetitlenumber\hspace{1em}\textbf{\@title}}%
                     }}
\makeatother



%% thm like environments
\let\question\relax
\let\endquestion\relax

\newtheoremstyle{QuestionStyle}{\topsep}{\topsep}%%% space between body and thm
		{}                      %%% Thm body font
		{}                              %%% Indent amount (empty = no indent)
		{\bfseries}            %%% Thm head font
		{)}                              %%% Punctuation after thm head
		{ }                           %%% Space after thm head
		{\thmnumber{#2}\thmnote{ \bfseries(#3)}}%%% Thm head spec
\theoremstyle{QuestionStyle}
\newtheorem{question}{}



\let\freeResponse\relax
\let\endfreeResponse\relax

%% \newtheoremstyle{ResponseStyle}{\topsep}{\topsep}%%% space between body and thm
%% 		{\wedn\bfseries}                      %%% Thm body font
%% 		{}                              %%% Indent amount (empty = no indent)
%% 		{\wedn\bfseries}            %%% Thm head font
%% 		{}                              %%% Punctuation after thm head
%% 		{3ex}                           %%% Space after thm head
%% 		{\underline{\underline{\thmname{#1}}}}%%% Thm head spec
%% \theoremstyle{ResponseStyle}

\usepackage[tikz]{mdframed}
\mdfdefinestyle{ResponseStyle}{leftmargin=1cm,linecolor=black,roundcorner=5pt,
, font=\bsifamily,}%font=\wedn\bfseries\upshape,}


\ifhandout
\NewEnviron{freeResponse}{}
\else
%\newtheorem{freeResponse}{Response:}
\newenvironment{freeResponse}{\begin{mdframed}[style=ResponseStyle]}{\end{mdframed}}
\fi



%% attempting to automate outcomes.

%% \newwrite\outcomefile
%%   \immediate\openout\outcomefile=\jobname.oc
%% \renewcommand{\outcome}[1]{\edef\theoutcomes{\theoutcomes #1~}%
%% \immediate\write\outcomefile{\unexpanded{\outcome}{#1}}}

%% \newcommand{\outcomelist}{\begin{itemize}\theoutcomes\end{itemize}}

%% \NewEnviron{listOutcomes}{\small\sffamily
%% After answering the following questions, students should be able to:
%% \begin{itemize}
%% \BODY
%% \end{itemize}
%% }
\usepackage[tikz]{mdframed}
\mdfdefinestyle{OutcomeStyle}{leftmargin=2cm,rightmargin=2cm,linecolor=black,roundcorner=5pt,
, font=\small\sffamily,}%font=\wedn\bfseries\upshape,}
\newenvironment{listOutcomes}{\begin{mdframed}[style=OutcomeStyle]After answering the following questions, students should be able to:\begin{itemize}}{\end{itemize}\end{mdframed}}



%% my commands

\newcommand{\snap}{{\bfseries\itshape\textsf{Snap!}}}
\newcommand{\flavor}{\link[\snap]{https://snap.berkeley.edu/}}
\newcommand{\mooculus}{\textsf{\textbf{MOOC}\textnormal{\textsf{ULUS}}}}


\usepackage{tkz-euclide}
\tikzstyle geometryDiagrams=[rounded corners=.5pt,ultra thick,color=black]
\colorlet{penColor}{black} % Color of a curve in a plot



\ifhandout\newcommand{\mynewpage}{\newpage}\else\newcommand{\mynewpage}{}\fi


\title{Staircases}
\author{Bart Snapp}

\begin{document}
\begin{abstract}
  We solve the staircase problem.
\end{abstract}
\maketitle

\begin{listOutcomes}  
\item{Understand the geometric issues with designing a staircase.}
\item{Organize and accommodate data.}
\item{Interpret numerical data concerning the geometric issues of designing a staircase.}
\item{Critique and dismantle reasonable hypotheses in regard to geometry and arithmetic.}
\item{Recognize the limits of simple scaling.}
\end{listOutcomes}

We learn by seeing \textbf{contrast}. Let me show you a REAL WORLD
example where ``scaling'' \textbf{does not give an easy solution.}


Most building codes have specific restrictions on how a staircase can
be designed. Suppose the \mooculus~\textit{Prefab Company} make
prefabricated staircases. They designed a staircase with $n$ stairs
that looks like this
\begin{center}
  \begin{tikzpicture}[scale = .3]
    \draw[thick]
    (10,0) -- (9,0) -- (9,1) -- (8,1) --
    ( 8,2) -- (7,2) -- (7,3) -- (6,3) --
    ( 6,4) -- (5,4) -- (5,5) -- (4,5) --
    ( 4,6) -- (3,6) -- (3,7) -- (2,7) --
    ( 2,8) -- (1,8) -- (1,9) -- (0,9) --
    (0,10) -- (0,0) -- (10,0);
    \draw [decorate,decoration={brace,amplitude=10pt,mirror},xshift=-4pt,yshift=0pt,]
    (0,10) -- (0,0) node [black,midway,xshift=-1cm] 
          {\footnotesize $n=10$};
    \draw [decorate,decoration={brace,amplitude=10pt},xshift=0pt,yshift=-4pt,]
    (10,0) -- (0,0) node [black,midway,yshift=-.8cm] 
          {\footnotesize $n=10$};     
\end{tikzpicture}
\end{center}
in the case where $n=10$. For safety and comfort, typical restrictions
on stairs are:
\begin{itemize}
\item The stair-step height can be at most $7\frac{3}{4}''$. 
\item The stair-step depth must be at least $10''$.
\item From step-to-step, these dimensions cannot change (in real-life there is
an allowed variance for these dimensions).
\end{itemize}

So suppose you want a \mooculus~Prefab staircase in a space that is
$w=15$ feet wide and you need to connect to a floor $h=10$ feet above the
ground. In this case you must have \emph{at least}
\[
\frac{h}{\left(31/48\right)}=\frac{10}{\left(31/48\right)} \ \text{stairs}
\]
and can have \emph{at most}
\[
\frac{w}{\left(5/6\right)}=\frac{15}{\left(5/6\right)} \ \text{stairs}.
\]


Of course, you must have a \emph{whole} number of stairs. We have two special functions that help out, \emph{floor} and \emph{ceiling}:
\begin{description}
\item[\emph{floor}:] Denoted $\lfloor x\rfloor$, this function, \emph{rounds-down}.
\item[\emph{ceiling}:] Denoted $\lceil x\rceil$ this function, \emph{rounds-up}.
\end{description}
So we see
\[
\text{min number of stairs} = \left\lceil \frac{h}{\left(31/48\right)} \right\rceil = \left\lceil \frac{10}{\left(31/48\right)} \right\rceil
\]
and
\[
\text{max number of stairs} =  \left\lfloor \frac{w}{\left(5/6\right)} \right\rfloor=\left\lfloor \frac{15}{\left(5/6\right)} \right\rfloor.
\]


\mynewpage


\begin{question}
  Find all valid \mooculus~prefab staircases that will fit in a total
  width of $15$ feet and total height of $10$ feet.

  \textbf{List all valid staircases in a table}, where the first
  column is the \textbf{number of stairs}, the second column is the
  \textbf{tread width}, the third column is the \textbf{riser height},
  the fourth is the $\boldsymbol{(\textbf{number of stairs}) \cdot (\textbf{tread width})}$, and the final column is $\boldsymbol{(\textbf{number of stairs}) \cdot (\textbf{riser height})}$.

  \begin{freeResponse}
  When $w=15$ and $h=10$:
  
  This means we can make three different staircases:
  \begin{enumerate}
  \item $16$ steps, with width $0.938$ feet, and height of $0.625$ feet.
  \item $17$ steps, with width $0.883$ feet, and height of $0.588$ feet.
  \item $18$ steps, with width $0.833$ feet, and height of $0.556$ feet.
  \end{enumerate}
  \end{freeResponse}
\end{question}


\mynewpage

\begin{question}
  Use an \link[online stair
    calculator]{https://www.calculator.net/stair-calculator.html} to
  find \textbf{all} valid \mooculus~prefab staircases that will fit in
  a total width of $20$ feet and total height of $13$ feet.

  \textbf{List all valid staircases in a table}, where the first
  column is the \textbf{number of stairs}, the second column is the
  \textbf{tread width}, the third column is the \textbf{riser height},
  the fourth is the $\boldsymbol{(\textbf{number of stairs}) \cdot (\textbf{tread width})}$, and the final column is $\boldsymbol{(\textbf{number of stairs}) \cdot (\textbf{riser height})}$.

 Finally, \textbf{choose the most optimal one, according to Blondel's
   formula.}

  \begin{freeResponse}
    From the calculator, I see we can make four different staircases:
    \begin{itemize}
    \item $21$ steps, with width $11.43$ inches, and height of $7.43$ inches.
    \item $22$ steps, with width $10.91$ inches, and height of $7.09$ inches.
    \item $23$ steps, with width $10.43$ inches, and height of $6.78$ inches.
    \item $24$ steps, with width $10$ inches, and height of $6.5$ inches.
    \end{itemize}
    Blondel's formula (in inches) is:
    \[
    2\cdot h + w \approx 25
    \]
    Now, we have
    \begin{align*}
      2\cdot 7.43 + 11.43 &\approx 26.3,\\
      2\cdot 7.09 + 10.91 &\approx 25,\\
      2\cdot 6.78 + 10.43 &\approx 24,\\
      2\cdot 6.5 + 10 &= 23.
    \end{align*}
    Hence a staircase of $22$ steps, with width $10.91$ inches, and
    height of $7.09$ inches is optimal by Blondel's formula.
  \end{freeResponse}
\end{question}
\mynewpage




\begin{question}
  Your friend \textit{Geometry Giorgio} suggests:
  \begin{quote}
    The staircase problem isn't hard, you can build a staircase that
    will fit in a space of $w\times h$, if and only if
    \[
    \frac{w}{h} \ge 1.29.
    \]
  \end{quote}
  \begin{enumerate}
    \item How did \textit{Geometry Giorgio} come up with this ``rule-of-thumb?''
    \item Can you build a staircase where  $w=130'$ $h=100'$ what about $w=13'$ and $h=10'$?
    \item Is \textit{Geometry Giorgio} correct?
       
  \end{enumerate}
  In each case, explain your reasoning with words, pictures, examples, and so on.
  \begin{freeResponse}
    \begin{enumerate}
    \item This person took minimum depth of $5/6'$ and divided it by
      the maximum height of $31/48'$ to get $1.29$. They then reason that if
      \[
      \frac{w}{h} <1.29.
      \]
      then a stair case CANNOT be built, because the stairs will be
      too narrow and tall. Moreover, they assert that if
      \[
      \frac{w}{h} \ge 1.29.
      \]
    \item You CAN make a staircase that will fit when $w=130'$ and
      $h=100'$. Use $155$ steps with a tread of $0.84$ feet ($10.8$
      inches) and a riser of $0.65$ feet ($7.8$ inches).


      You cannot build a staircase when $w=13'$ and $h=10'$ as
      \[
      \text{min number of stairs} = \left\lceil
      \frac{10}{\left(31/48\right)} \right\rceil=16
      \]
      and
      \[
      \text{max number of stairs} = \left\lfloor
      \frac{13}{\left(5/6\right)} \right\rfloor=15.
      \]
      But the max number of stairs cannot be less than the min number
      of stairs!

    \item They are not correct! For example you cannot build a
      staircase when $w=13'$ and $h=10'$, but:
      \[
      \frac{13}{10}\ge 1.29
      \]
    \end{enumerate}
  \end{freeResponse}
\end{question}


\end{document}
