\documentclass[noauthor,nooutcomes,handout,hints,12pt]{ximera}

\graphicspath{  
{./}
{./whoAreYou/}
{./drawingWithTheTurtle/}
{./bisectionMethod/}
{./circles/}
{./anglesAndRightTriangles/}
{./lawOfSines/}
{./lawOfCosines/}
{./plotter/}
{./staircases/}
{./pitch/}
{./qualityControl/}
{./symmetry/}
{./nGonBlock/}
}


%% page layout
\usepackage[cm,headings]{fullpage}
\raggedright
\setlength\headheight{13.6pt}


%% fonts
\usepackage{euler}

\usepackage{FiraMono}
\renewcommand\familydefault{\ttdefault} 
\usepackage[defaultmathsizes]{mathastext}
\usepackage[htt]{hyphenat}

\usepackage[T1]{fontenc}
\usepackage[scaled=1]{FiraSans}

%\usepackage{wedn}
\usepackage{pbsi} %% Answer font


\usepackage{cancel} %% strike through in pitch/pitch.tex


%% \usepackage{ulem} %% 
%% \renewcommand{\ULthickness}{2pt}% changes underline thickness

\tikzset{>=stealth}

\usepackage{adjustbox}

\setcounter{titlenumber}{-1}

%% journal style
\makeatletter
\newcommand\journalstyle{%
  \def\activitystyle{activity-chapter}
  \def\maketitle{%
    \addtocounter{titlenumber}{1}%
                {\flushleft\small\sffamily\bfseries\@pretitle\par\vspace{-1.5em}}%
                {\flushleft\LARGE\sffamily\bfseries\thetitlenumber\hspace{1em}\@title \par }%
                {\vskip .6em\noindent\textit\theabstract\setcounter{question}{0}\setcounter{sectiontitlenumber}{0}}%
                    \par\vspace{2em}
                    \phantomsection\addcontentsline{toc}{section}{\thetitlenumber\hspace{1em}\textbf{\@title}}%
                     }}
\makeatother



%% thm like environments
\let\question\relax
\let\endquestion\relax

\newtheoremstyle{QuestionStyle}{\topsep}{\topsep}%%% space between body and thm
		{}                      %%% Thm body font
		{}                              %%% Indent amount (empty = no indent)
		{\bfseries}            %%% Thm head font
		{)}                              %%% Punctuation after thm head
		{ }                           %%% Space after thm head
		{\thmnumber{#2}\thmnote{ \bfseries(#3)}}%%% Thm head spec
\theoremstyle{QuestionStyle}
\newtheorem{question}{}



\let\freeResponse\relax
\let\endfreeResponse\relax

%% \newtheoremstyle{ResponseStyle}{\topsep}{\topsep}%%% space between body and thm
%% 		{\wedn\bfseries}                      %%% Thm body font
%% 		{}                              %%% Indent amount (empty = no indent)
%% 		{\wedn\bfseries}            %%% Thm head font
%% 		{}                              %%% Punctuation after thm head
%% 		{3ex}                           %%% Space after thm head
%% 		{\underline{\underline{\thmname{#1}}}}%%% Thm head spec
%% \theoremstyle{ResponseStyle}

\usepackage[tikz]{mdframed}
\mdfdefinestyle{ResponseStyle}{leftmargin=1cm,linecolor=black,roundcorner=5pt,
, font=\bsifamily,}%font=\wedn\bfseries\upshape,}


\ifhandout
\NewEnviron{freeResponse}{}
\else
%\newtheorem{freeResponse}{Response:}
\newenvironment{freeResponse}{\begin{mdframed}[style=ResponseStyle]}{\end{mdframed}}
\fi



%% attempting to automate outcomes.

%% \newwrite\outcomefile
%%   \immediate\openout\outcomefile=\jobname.oc
%% \renewcommand{\outcome}[1]{\edef\theoutcomes{\theoutcomes #1~}%
%% \immediate\write\outcomefile{\unexpanded{\outcome}{#1}}}

%% \newcommand{\outcomelist}{\begin{itemize}\theoutcomes\end{itemize}}

%% \NewEnviron{listOutcomes}{\small\sffamily
%% After answering the following questions, students should be able to:
%% \begin{itemize}
%% \BODY
%% \end{itemize}
%% }
\usepackage[tikz]{mdframed}
\mdfdefinestyle{OutcomeStyle}{leftmargin=2cm,rightmargin=2cm,linecolor=black,roundcorner=5pt,
, font=\small\sffamily,}%font=\wedn\bfseries\upshape,}
\newenvironment{listOutcomes}{\begin{mdframed}[style=OutcomeStyle]After answering the following questions, students should be able to:\begin{itemize}}{\end{itemize}\end{mdframed}}



%% my commands

\newcommand{\snap}{{\bfseries\itshape\textsf{Snap!}}}
\newcommand{\flavor}{\link[\snap]{https://snap.berkeley.edu/}}
\newcommand{\mooculus}{\textsf{\textbf{MOOC}\textnormal{\textsf{ULUS}}}}


\usepackage{tkz-euclide}
\tikzstyle geometryDiagrams=[rounded corners=.5pt,ultra thick,color=black]
\colorlet{penColor}{black} % Color of a curve in a plot



\ifhandout\newcommand{\mynewpage}{\newpage}\else\newcommand{\mynewpage}{}\fi


\author{Claire Merriman}

\title{Sun and shadows}

\begin{document}
\begin{abstract}
Tangent allows us to calculate the ratio of the legs of a right triangle.
\end{abstract}
\maketitle

\begin{listOutcomes}
\item  Recall the basic definitions of tangent.
\item Explore the differences in Sunlight throughout the year.
\item Use tangent to determine the length of shades.
\item Translate classroom mathematics into real world mathematics.
\end{listOutcomes}

%% \begin{listObjectives}
%%  \item Learn and apply basic geometric formulas,
%% \item Use trigonometry to solve common problems.
%% \end{listObjectives}

\begin{definition}
 In a right triangle, \textbf{tangent} of an angle is the ratio of the length of the side opposite the angle to the length of the side adjacent to the angle. 
\end{definition}
We will use this to explore some ideas around shade and shadows.  The
path of the Sun in the sky changes throughout the year. The Sun
reaches its highest point for any given day at \textbf{solar noon;} in
the Northern Hemisphere this occurs when the Sun is due South. The Sun
reaches its highest point for the year on the \textbf{summer solstice.}

\begin{center}
 
\begin{tikzpicture}[x=.75cm,y=.75cm,scale=.8]
 \draw[thick] (0.2,0)--(3.2,0)--(3.2,-1.6);
 \draw[dashed] (-2.5,1.5)--(3.2,-1.6)--(3.2,-6.6)--(-2,4.97);
 \draw[thick] (3.2,-6.6)--(3.2,-9.6);
 \filldraw[color=yellow, fill=yellow, very thick](-2,4.97) circle (.5);
 \filldraw[color=yellow, fill=yellow, very thick](-2.5,1.5) circle (.5);
 \node at (1.8,.5) {\small overhang};
 \node[anchor=west] at (3.5, -.8) {\small gap between window and ceiling};
 \node[anchor=west] at (3.5, -4) {\small window};
 \node[anchor=west] at (3.5, -8) {\small height of the window from the floor};
\end{tikzpicture}
\end{center}

\mynewpage


\begin{question}
 One practice for designing South facing windows is to build an
 overhang so that
 \begin{itemize}
 \item The Sun fills the window at noon on the winter solstice.
 \item The window is completely in the shade at noon on the summer
   solstice.
 \end{itemize}
\begin{enumerate}
\item Draw a triangle that relates the angle of the Sun, the length of
  the horizontal overhang, and the length of the vertical shadow.
  \textbf{Express the overhang as a function of the solar zenith angle and length of the shadow.}
\item The maximum possible overhang length in this scenario is the one
  where the bottom of the shadow at \textbf{noon} on the \textbf{winter solstice} is at
  the top of the window. Look up the solar zenith angle for Columbus,
  cite your source, and then use this information to find a formula relating the shadow and
  length of the overhang. 
\item The minimum possible overhang length in this scenario is the one
  where the bottom of the shadow at \textbf{noon} on the \textbf{summer solstice} is at
  the bottom of the window. Look up the solar zenith angle for Columbus, cite
  your source, and then use this information to find a formula relating the shadow and length
  of the overhang. 
\end{enumerate}
\end{question}
\mynewpage


\begin{question}
 A common ceiling height is $8$ feet, with a $16$ inch gap between the
 top of window and the ceiling.
 
\begin{enumerate}
 \item Find the maximum overhang length, assuming the overhang is at
   the top of the ceiling.
 \item Find two standard window sizes, then find the minimum overhang
   length and height of the window from the floor in both cases.
\end{enumerate}
In both cases, explain your reasoning and show work.
\end{question}
\mynewpage

\begin{question}
 We can also use tangent to find the length of shadows from a
 building. For a $52\ foot$ tall building in Columbus, find:
 
\begin{enumerate}
 \item The length of the area North of the building that is in the
   shade year round.
 \item The difference in the length of this shadow on the summer
   solstice and the winter solstice.
\end{enumerate}
In both cases, explain your reasoning and show work.
\end{question}
\end{document}
