\usepackage{gillius}
\usepackage{concmath}
\usepackage[T1]{fontenc}

%% \usepackage[sfmath]{kpfonts} %% sfmath option only to make math in sans serif. Probablye only for use when base font is sans serif.
%% \renewcommand*\familydefault{\sfdefault} %% Only if the base font of the document is to be sans serif
%% \usepackage[T1]{fontenc}



\usepackage{multicol} %% for logo
\setlength{\columnseprule}{1pt} %%  for gutter rule

\graphicspath{  
{./}
{basicDirections/}
{repeatingSteps/}
{curvesAndHeadings/}
{repcountAndRainbows/}
{ximeraTutorial/}
{matricesAsFunctions/}
{matricesAsFunctions/exercises/}
{symmetry/}
{stringArt/}
{activities/whatIsAnIsometry/}
{activities/aBiggerApartmentPlease/}
{activities/allYouNeedAreReflections/}
{activities/aMovingImage/}
{activities/moreTransformations/}
{activities/superKaleidoscopes/}
{activities/symmetries/}
{activities/whatIsAnIsometry/}
{activities/whoMappedTheWhatWhere/}
{activities/envelopes/}
{frieze/exercises/}
{frieze/}
{activities/onlySeven/}
{activities/louieLlamaAndTheTriangle/}
{tessellations/}
{scaling/}
{activities/areYouGettingStairs/}
{activities/rep-tiles/}
{activities/snowflakes/}
}





\usepackage{amssymb,amsmath,amsfonts,amsthm,fancyhdr,graphicx}
%% \usepackage{tkz-euclide}
%% \usetkzobj{all}
\tikzset{%% partial ellipse
    partial ellipse/.style args={#1:#2:#3}{
        insert path={+ (#1:#3) arc (#1:#2:#3)}
    }
}

\tikzstyle geometryDiagrams=[rounded corners=.5pt,ultra thick,color=black]

\usetikzlibrary{calc}


\usetikzlibrary{turtle} %for LOGO

\DefineVerbatimEnvironment{logo}{Verbatim}{framerule=1pt,frame=lines,label=\LOGO,labelposition=topline}

\NewEnviron{logoout}{
\begin{center}
\sep~\LOGO~OUTPUT~\sep \\[.5cm]
\BODY\\[-.2cm]
\sep
\end{center}
}

%%% This set of code is all of our user defined commands
\usepackage{wasysym}
\newcommand{\turtle}{\text{\UParrow}}
\newcommand{\turtler}{\rotatebox[origin=c]{-90}{\text{\UParrow}}}
\newcommand{\turtlel}{\rotatebox[origin=c]{90}{\text{\UParrow}}}
\newcommand{\turtled}{\rotatebox[origin=c]{180}{\text{\UParrow}}}
\newcommand{\sep}{\hrulefill}
\makeatletter
\def\hrulefill{\leavevmode\leaders\hrule height 1pt\hfill\kern\z@}
\makeatother
\newcommand{\lc}[1]{$\pmb{\texttt{#1}}$}

\newcommand{\bysame}{\mbox{\rule{3em}{.4pt}}\,}
\newcommand{\N}{\mathbb N}
\newcommand{\Z}{\mathbb Z}
\newcommand{\Zt}{\mathcal{Z}_\mat{T}}
\newcommand{\Zg}{\mathcal{Z}_\mat{G}}
\newcommand{\Zgf}{\mathcal{Z}_{\mat{G},\mat{F}}}
\newcommand{\Ztf}{\mathcal{Z}_{\mat{T},\mat{F}}}
\newcommand{\Ztr}{\mathcal{Z}_{\mat{T},\mat{R}}}
\newcommand{\Zgr}{\mathcal{Z}_{\mat{G},\mat{R}}}
\newcommand{\Ztfr}{\mathcal{Z}_{\mat{T},\mat{F},\mat{R}}}
\newcommand{\R}{\mathcal R}
\newcommand{\D}{\mathcal D}
\newcommand{\F}{\mathcal F}
\newcommand{\C}{\mathbb C}
\newcommand{\ph}{\varphi}
\newcommand{\ep}{\varepsilon}
\newcommand{\aph}{\alpha}
\newcommand{\QM}{\begin{center}{\huge\textbf{?}}\end{center}}
\renewcommand{\le}{\leqslant}
\renewcommand{\ge}{\geqslant}
\renewcommand{\a}{\wedge}
\renewcommand{\v}{\vee}
\renewcommand{\l}{\ell}
\renewcommand{\subset}{\subseteq}
\renewcommand{\supset}{\supseteq}
\renewcommand{\emptyset}{\varnothing}
\newcommand{\xto}{\xrightarrow}
\renewcommand{\qedsymbol}{$\blacksquare$}
%% \renewcommand{\bibname}{References and Further Reading}
%% \renewcommand{\abstractname}{Distributing this Document}
\newcommand{\LOGO}{\textsl{\textsf{LOGO}}}
\newcommand{\logoFlavor}{\link[Turtle Academy]{https://turtleacademy.com/}}

\renewcommand{\bar}{\protect\overline}
\renewcommand{\hat}{\protect\widehat}
\renewcommand{\tilde}{\widetilde}
\renewcommand{\star}[2]{\left\{\frac{#1}{#2}\right\}}
\newcommand{\iso}{\simeq}
\newcommand{\problems}{\subsection*{Problems for Section~\thesection}\hrule\vspace{1ex}}
\newcommand{\tri}{\triangle}


%%% thicker hline
\makeatletter
\usepackage{tabularx}
\def\hlinewd#1{%
\noalign{\ifnum0=`}\fi\hrule \@height #1 %
\futurelet\reserved@a\@xhline} 
\makeatother

\newcommand{\mat}{\mathsf}
\renewcommand{\vec}{\mathbf}

\newcommand{\dfn}[1]{\textbf{#1}\index{#1}}


\usepackage{currfile}
\makeatletter
\ifxake
% The code below the \else is executed in a sagecell on the Ximera
% server, so \makerandom doesn't have to do anything when run under
% xake.
\newcommand{\makerandom}{}
\else
\newcommand{\makerandom}{%
  \ST@wsf{jobname="\currfilebase"}%
  \ST@wsf{import hashlib}%
  \ST@wsf{set_random_seed(int(hashlib.sha256(jobname.encode('utf-8')).hexdigest(), 16))}%
}
\fi
\makeatother



%%% This next bit of code defines all our theorem environments
\makeatletter
\let\c@theorem\relax
\let\c@corollary\relax
%\let\c@example\relax

\let\problem\relax
\let\endproblem\relax
\let\c@problem\relax 

\makeatother

\let\definition\relax
\let\enddefinition\relax

\let\theorem\relax
\let\endtheorem\relax

\let\proposition\relax
\let\endproposition\relax

\let\exercise\relax
\let\endexercise\relax

\let\question\relax
\let\endquestion\relax

\let\remark\relax
\let\endremark\relax

\let\corollary\relax
\let\endcorollary\relax


\let\example\relax
\let\endexample\relax

\let\warning\relax
\let\endwarning\relax

\let\lemma\relax
\let\endlemma\relax


\let\algorithm\relax
\let\endalgorithm\relax
\usepackage{algpseudocode}

\newtheoremstyle{SlantTheorem}{\topsep}{\topsep}%%% space between body and thm
		{\slshape}                      %%% Thm body font
		{}                              %%% Indent amount (empty = no indent)
		{\bfseries\sffamily}            %%% Thm head font
		{}                              %%% Punctuation after thm head
		{3ex}                           %%% Space after thm head
		{\thmname{#1}\thmnumber{ #2}\thmnote{ \bfseries(#3)}}%%% Thm head spec
\theoremstyle{SlantTheorem}
\newtheorem{theorem}{Theorem}
%\newtheorem{definition}[theorem]{Definition}
%\newtheorem{proposition}[theorem]{Proposition}
%% \newtheorem*{dfnn}{Definition}
%% \newtheorem{ques}{Question}[theorem]
%% \newtheorem*{war}{WARNING}
%% \newtheorem*{cor}{Corollary}
%% \newtheorem*{eg}{Example}
\newtheorem*{remark}{Remark}
\newtheorem*{touchstone}{Touchstone}
\newtheorem{corollary}{Corollary}[theorem]
\newtheorem*{warning}{WARNING}
\newtheorem{example}[corollary]{Example}
\newtheorem{lemma}[theorem]{Lemma}
\newtheorem*{question}{Question}



\newtheoremstyle{Definition}{\topsep}{\topsep}%%% space between body and thm
		{}                              %%% Thm body font
		{}                              %%% Indent amount (empty = no indent)
		{\bfseries\sffamily}            %%% Thm head font
		{}                              %%% Punctuation after thm head
		{3ex}                           %%% Space after thm head
		{\thmname{#1}\thmnumber{ #2}\thmnote{ \bfseries(#3)}}%%% Thm head spec
\theoremstyle{Definition}
\newtheorem{definition}[theorem]{Definition}



\let\algorithm\relax
\let\endalgorithm\relax
\newtheoremstyle{Alg}{\topsep}{\topsep}%%% space between body and thm
		{}                              %%% Thm body font
		{}                              %%% Indent amount (empty = no indent)
		{\bfseries\sffamily}            %%% Thm head font
		{}                              %%% Punctuation after thm head
		{3ex}                           %%% Space after thm head
		{\thmname{#1}\thmnumber{ #2}\thmnote{ \bfseries(#3)}}%%% Thm head spec
\theoremstyle{Alg}
\newtheorem*{algorithm}{Algorithm}
\newtheorem*{construction}{Construction}




\newtheoremstyle{Exercise}{\topsep}{\topsep} %%% space between body and thm
		{}                           %%% Thm body font
		{}                           %%% Indent amount (empty = no indent)
		{\bfseries\sffamily}         %%% Thm head font
		{)}                          %%% Punctuation after thm head
		{ }                          %%% Space after thm head
		{\thmnumber{#2}\thmnote{ \bfseries(#3)}}%%% Thm head spec
\theoremstyle{Exercise}
\newtheorem{exercise}[corollary]{}%[theorem]


\newtheoremstyle{problem}
{\topsep}{\topsep}{}{}{\bfseries\sffamily}{)}{ }{}
\theoremstyle{problem}
\newtheorem{problem}{}


\makeatletter
\renewcommand\section{\@startsection{paragraph}{10}{\z@}%
                                     {-3.25ex\@plus -1ex \@minus -.2ex}%
                                     {1.5ex \@plus .2ex}%
                                     {\normalfont\large\sffamily\bfseries}}
\renewcommand\subsection{\@startsection{subparagraph}{10}{\z@}%
                                    {3.25ex \@plus1ex \@minus.2ex}%
                                    {-1em}%
                                    {\normalfont\normalsize\sffamily\bfseries}}
\makeatother

%% Fix weird index/bib issue.
\makeatletter
\gdef\ttl@savemark{\sectionmark{}}
\makeatother






\usetikzlibrary{math} %% for assigning variables

%% N-GON code
\tikzset{
    pics/tikzngon/.style={
        code={
        \tikzmath{\xx = #1;\rr=1.7;}
        \draw[ultra thick,rounded corners=.05mm] ({\rr*sin(0*360/\xx)},{\rr*cos(0*360/\xx)})
        \foreach \x in {-1,0,...,\xx}
        {
        -- ({\rr*sin(\x*360/\xx)},{\rr*cos(\x*360/\xx)})
        }
           -- cycle;
  }}}

%% N-GON code (even)
\tikzset{
    pics/tikzegon/.style={
        code={
        \tikzmath{\xx = #1;\rr=1.7;}
        \draw[ultra thick,rounded corners=.05mm] ({\rr*sin(0*360/\xx+180/\xx)},{\rr*cos(0*360/\xx+180/\xx)})
        \foreach \x in {-1,0,...,\xx}
           {
           -- ({\rr*sin(\x*360/\xx+180/\xx)},{\rr*cos(\x*360/\xx+180/\xx)}) 
           }
           -- cycle;
  }}}
