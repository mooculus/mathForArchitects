
\graphicspath{  
{./}
{ximeraTutorial/}
{matricesAsFunctions/}
{matricesAsFunctions/exercises}  
} 

\usepackage{microtype,amssymb,amsmath,amsfonts,amsthm,fancyhdr,graphicx}
%% \usepackage{tkz-euclide}
%% \usetkzobj{all}
\tikzset{%% partial ellipse
    partial ellipse/.style args={#1:#2:#3}{
        insert path={+ (#1:#3) arc (#1:#2:#3)}
    }
}

\tikzstyle geometryDiagrams=[rounded corners=.5pt,ultra thick,color=black]

\usetikzlibrary{calc}
%%% This set of code is all of our user defined commands
\newcommand{\bysame}{\mbox{\rule{3em}{.4pt}}\,}
\newcommand{\N}{\mathbb N}
\newcommand{\Z}{\mathbb Z}
\newcommand{\Zt}{\mathcal{Z}_\mat{T}}
\newcommand{\Zg}{\mathcal{Z}_\mat{G}}
\newcommand{\Zgf}{\mathcal{Z}_{\mat{G},\mat{F}}}
\newcommand{\Ztf}{\mathcal{Z}_{\mat{T},\mat{F}}}
\newcommand{\Ztr}{\mathcal{Z}_{\mat{T},\mat{R}}}
\newcommand{\Zgr}{\mathcal{Z}_{\mat{G},\mat{R}}}
\newcommand{\Ztfr}{\mathcal{Z}_{\mat{T},\mat{F},\mat{R}}}
\newcommand{\R}{\mathcal R}
\newcommand{\D}{\mathcal D}
\newcommand{\F}{\mathcal F}
\newcommand{\C}{\mathbb C}
\newcommand{\ph}{\varphi}
\newcommand{\ep}{\varepsilon}
\newcommand{\aph}{\alpha}
\newcommand{\QM}{\begin{center}{\huge\textbf{?}}\end{center}}
\renewcommand{\le}{\leqslant}
\renewcommand{\ge}{\geqslant}
\renewcommand{\a}{\wedge}
\renewcommand{\v}{\vee}
\renewcommand{\l}{\ell}
\renewcommand{\subset}{\subseteq}
\renewcommand{\supset}{\supseteq}
\renewcommand{\emptyset}{\varnothing}
\newcommand{\xto}{\xrightarrow}
\renewcommand{\qedsymbol}{$\blacksquare$}
%% \renewcommand{\bibname}{References and Further Reading}
%% \renewcommand{\abstractname}{Distributing this Document}
\renewcommand{\bar}{\protect\overline}
\renewcommand{\hat}{\protect\widehat}
\renewcommand{\tilde}{\widetilde}
\renewcommand{\star}[2]{\left\{\frac{#1}{#2}\right\}}
\newcommand{\iso}{\simeq}
\newcommand{\problems}{\subsection*{Problems for Section~\thesection}\hrule\vspace{1ex}}
\newcommand{\tri}{\triangle}


\newcommand{\mat}{\mathsf}
\renewcommand{\vec}{\mathbf}

\newcommand{\dfn}[1]{\textbf{#1}\index{#1}}


\usepackage{currfile}
\makeatletter
\ifxake
% The code below the \else is executed in a sagecell on the Ximera
% server, so \makerandom doesn't have to do anything when run under
% xake.
\newcommand{\makerandom}{}
\else
\newcommand{\makerandom}{%
  \ST@wsf{jobname="\currfilebase"}%
  \ST@wsf{import hashlib}%
  \ST@wsf{set_random_seed(int(hashlib.sha256(jobname.encode('utf-8')).hexdigest(), 16))}%
}
\fi
\makeatother
