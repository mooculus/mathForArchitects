\documentclass[noauthor,nooutcomes,hints,handout,12pt]{ximera}


%% Note,

\graphicspath{  
{./}
{./whoAreYou/}
{./drawingWithTheTurtle/}
{./bisectionMethod/}
{./circles/}
{./anglesAndRightTriangles/}
{./lawOfSines/}
{./lawOfCosines/}
{./plotter/}
{./staircases/}
{./pitch/}
{./qualityControl/}
{./symmetry/}
{./nGonBlock/}
}


%% page layout
\usepackage[cm,headings]{fullpage}
\raggedright
\setlength\headheight{13.6pt}


%% fonts
\usepackage{euler}

\usepackage{FiraMono}
\renewcommand\familydefault{\ttdefault} 
\usepackage[defaultmathsizes]{mathastext}
\usepackage[htt]{hyphenat}

\usepackage[T1]{fontenc}
\usepackage[scaled=1]{FiraSans}

%\usepackage{wedn}
\usepackage{pbsi} %% Answer font


\usepackage{cancel} %% strike through in pitch/pitch.tex


%% \usepackage{ulem} %% 
%% \renewcommand{\ULthickness}{2pt}% changes underline thickness

\tikzset{>=stealth}

\usepackage{adjustbox}

\setcounter{titlenumber}{-1}

%% journal style
\makeatletter
\newcommand\journalstyle{%
  \def\activitystyle{activity-chapter}
  \def\maketitle{%
    \addtocounter{titlenumber}{1}%
                {\flushleft\small\sffamily\bfseries\@pretitle\par\vspace{-1.5em}}%
                {\flushleft\LARGE\sffamily\bfseries\thetitlenumber\hspace{1em}\@title \par }%
                {\vskip .6em\noindent\textit\theabstract\setcounter{question}{0}\setcounter{sectiontitlenumber}{0}}%
                    \par\vspace{2em}
                    \phantomsection\addcontentsline{toc}{section}{\thetitlenumber\hspace{1em}\textbf{\@title}}%
                     }}
\makeatother



%% thm like environments
\let\question\relax
\let\endquestion\relax

\newtheoremstyle{QuestionStyle}{\topsep}{\topsep}%%% space between body and thm
		{}                      %%% Thm body font
		{}                              %%% Indent amount (empty = no indent)
		{\bfseries}            %%% Thm head font
		{)}                              %%% Punctuation after thm head
		{ }                           %%% Space after thm head
		{\thmnumber{#2}\thmnote{ \bfseries(#3)}}%%% Thm head spec
\theoremstyle{QuestionStyle}
\newtheorem{question}{}



\let\freeResponse\relax
\let\endfreeResponse\relax

%% \newtheoremstyle{ResponseStyle}{\topsep}{\topsep}%%% space between body and thm
%% 		{\wedn\bfseries}                      %%% Thm body font
%% 		{}                              %%% Indent amount (empty = no indent)
%% 		{\wedn\bfseries}            %%% Thm head font
%% 		{}                              %%% Punctuation after thm head
%% 		{3ex}                           %%% Space after thm head
%% 		{\underline{\underline{\thmname{#1}}}}%%% Thm head spec
%% \theoremstyle{ResponseStyle}

\usepackage[tikz]{mdframed}
\mdfdefinestyle{ResponseStyle}{leftmargin=1cm,linecolor=black,roundcorner=5pt,
, font=\bsifamily,}%font=\wedn\bfseries\upshape,}


\ifhandout
\NewEnviron{freeResponse}{}
\else
%\newtheorem{freeResponse}{Response:}
\newenvironment{freeResponse}{\begin{mdframed}[style=ResponseStyle]}{\end{mdframed}}
\fi



%% attempting to automate outcomes.

%% \newwrite\outcomefile
%%   \immediate\openout\outcomefile=\jobname.oc
%% \renewcommand{\outcome}[1]{\edef\theoutcomes{\theoutcomes #1~}%
%% \immediate\write\outcomefile{\unexpanded{\outcome}{#1}}}

%% \newcommand{\outcomelist}{\begin{itemize}\theoutcomes\end{itemize}}

%% \NewEnviron{listOutcomes}{\small\sffamily
%% After answering the following questions, students should be able to:
%% \begin{itemize}
%% \BODY
%% \end{itemize}
%% }
\usepackage[tikz]{mdframed}
\mdfdefinestyle{OutcomeStyle}{leftmargin=2cm,rightmargin=2cm,linecolor=black,roundcorner=5pt,
, font=\small\sffamily,}%font=\wedn\bfseries\upshape,}
\newenvironment{listOutcomes}{\begin{mdframed}[style=OutcomeStyle]After answering the following questions, students should be able to:\begin{itemize}}{\end{itemize}\end{mdframed}}



%% my commands

\newcommand{\snap}{{\bfseries\itshape\textsf{Snap!}}}
\newcommand{\flavor}{\link[\snap]{https://snap.berkeley.edu/}}
\newcommand{\mooculus}{\textsf{\textbf{MOOC}\textnormal{\textsf{ULUS}}}}


\usepackage{tkz-euclide}
\tikzstyle geometryDiagrams=[rounded corners=.5pt,ultra thick,color=black]
\colorlet{penColor}{black} % Color of a curve in a plot



\ifhandout\newcommand{\mynewpage}{\newpage}\else\newcommand{\mynewpage}{}\fi


\title{Fair division}
\author{Bart Snapp}

\begin{document}
\begin{abstract}
  We think about fair division and gerrymandering.
\end{abstract}
\maketitle

\begin{listOutcomes}
\item Describe what gerrymandering is.
\item Understand what fair division is.
\item Apply greedy methods that result in fair division.
\item Understand what ``efficiency gap'' is in reference to
  districting.
\end{listOutcomes}





\mynewpage






\begin{question}
 %% I'm old, and my dad, who is now deceased, grew up a long time ago. He
 %% told me that when he was young, sharing a single bottle of
 %% \textit{Coca-Cola} with his older brother was a \textbf{very special treat} (in
 %% extreme contrast to today).

 %% However, each brother wanted their own glass and there are no measuring
 %% cups in the house and all cups are different sizes!

 %% How can the brothers divide the \textit{Coca-Cola} fairly it so that
 %% each brother is satisfied?  Note, neither brother can be trusted to
  %% be fair!  Explain your reasoning.

  Here is a moist, delicious, piece of cake:
            
\begin{verbatim}
                                              ,:/+/-
                        /M/              .,-=;//;-
                   .:/= ;MH/,    ,=/+%$XH@MM#@:
                  -$##@+$###@H@MMM#######H:.    -/H#
             .,H@H@ X######@ -H#####@+-     -+H###@X
              .,@##H;      +XM##M/,     =%@###@X;-
            X%-  :M##########$.    .:%M###@%:
            M##H,   +H@@@$/-.  ,;$M###@%,          -
            M####M=,,---,.-%%H####M$:          ,+@##
            @##################@/.         :%H##@$-
            M###############H,         ;HM##M$=
            #################.    .=$M##M$=
            ################H..;XM##M$=          .:+
            M###################@%=           =+@MH%
            @################M/.          =+H#X%=
            =+M##############M,       -/X#X+;.
              .;XM##########H=    ,/X#H+:,
                 .=+HM######M+/+HM@+=.
                     ,:/%XM####H/.
                          ,.:=-.                    
\end{verbatim}

Suppose you wish to split this cake with your friend \textit{Geometry
  Giorgio.} However, you both want as much cake as possible. Also you
do not trust each other to be fair. How do you agree to fairly split
the cake?
\end{question}


\mynewpage


\begin{question}
Gerrymander the array below into $8$ districts ($6$ in each district)
so that the \textbf{circles} win.
\[
\renewcommand{\arraystretch}{1}
\setlength\tabcolsep{5mm}
\huge
\begin{array}{|c|c|c|c|c|c|}\hline
\hspace{1em}\blacksquare \hspace*{1em} &\hspace{1em} \bigcirc\hspace*{1em} &\hspace{1em} \bigcirc\hspace*{1em} &\hspace{1em} \blacksquare\hspace*{1em} &\hspace{1em} \blacksquare \hspace*{1em} & \hspace*{1em}\bigcirc\hspace*{1em} \\\hline
\bigcirc & \bigcirc & \blacksquare & \bigcirc & \blacksquare & \blacksquare \\\hline
\bigcirc & \bigcirc & \bigcirc & \blacksquare & \bigcirc & \blacksquare \\\hline
\bigcirc & \bigcirc & \blacksquare & \bigcirc & \bigcirc & \blacksquare \\\hline
\blacksquare & \blacksquare & \bigcirc & \bigcirc & \bigcirc & \bigcirc \\\hline
\bigcirc & \blacksquare & \blacksquare & \blacksquare & \blacksquare & \blacksquare \\\hline
\blacksquare & \bigcirc & \blacksquare & \blacksquare & \bigcirc & \bigcirc \\\hline
\blacksquare & \blacksquare & \blacksquare & \bigcirc & \blacksquare & \bigcirc \\\hline
\end{array}
\]
Additionally, compute and interpret the \textbf{efficiency gap}. Show work and
explain your reasoning.
\end{question}

\mynewpage


\begin{question}
  A recent paper, \link[\textit{A Partisan Solution to Partisan
    Gerrymandering: The Define–Combine Procedure}, authors give a
  method for districting based on fair division.]{https://www.cambridge.org/core/journals/political-analysis/article/partisan-solution-to-partisan-gerrymandering-the-definecombine-procedure/B0792DD0A49332944F2AF5FF6828E275} Here's the idea:
 \begin{mdframed}[style=OutcomeStyle]
    \begin{quote}
      \textbf{To fairly district a total of $\boldsymbol N$ districts:}
      \begin{itemize}
      \item One party draws a map with $2N$ districts of equal population. 
      \item The other party combines adjacent districts into $N$ districts of equal population. 
      \end{itemize}
    \end{quote}
 \end{mdframed}
 \begin{enumerate}
 \item Gerrymander the array below into $16$ districts ($3$ in each district)
   so that the \textbf{circles} win. 
   \[
   \renewcommand{\arraystretch}{1}
   \setlength\tabcolsep{5mm}
   \huge
   \begin{array}{|c|c|c|c|c|c|}\hline
     \hspace{1em}\blacksquare \hspace*{1em} &\hspace{1em} \bigcirc\hspace*{1em} &\hspace{1em} \bigcirc\hspace*{1em} &\hspace{1em} \blacksquare\hspace*{1em} &\hspace{1em} \blacksquare \hspace*{1em} & \hspace*{1em}\bigcirc\hspace*{1em} \\\hline
     \bigcirc & \bigcirc & \blacksquare & \bigcirc & \blacksquare & \blacksquare \\\hline
     \bigcirc & \bigcirc & \bigcirc & \blacksquare & \bigcirc & \blacksquare \\\hline
     \bigcirc & \bigcirc & \blacksquare & \bigcirc & \bigcirc & \blacksquare \\\hline
     \blacksquare & \blacksquare & \bigcirc & \bigcirc & \bigcirc & \bigcirc \\\hline
     \bigcirc & \blacksquare & \blacksquare & \blacksquare & \blacksquare & \blacksquare \\\hline
     \blacksquare & \bigcirc & \blacksquare & \blacksquare & \bigcirc & \bigcirc \\\hline
     \blacksquare & \blacksquare & \blacksquare & \bigcirc & \blacksquare & \bigcirc \\\hline
   \end{array}
   \]
   Additionally, compute and interpret the \textbf{efficiency gap}. Show work and
   explain your reasoning.
 \item Now combine adjacent districts, favoring \textbf{squares} as
   much as possible.
    \[
   \renewcommand{\arraystretch}{1}
   \setlength\tabcolsep{5mm}
   \huge
   \begin{array}{|c|c|c|c|c|c|}\hline
     \hspace{1em}\blacksquare \hspace*{1em} &\hspace{1em} \bigcirc\hspace*{1em} &\hspace{1em} \bigcirc\hspace*{1em} &\hspace{1em} \blacksquare\hspace*{1em} &\hspace{1em} \blacksquare \hspace*{1em} & \hspace*{1em}\bigcirc\hspace*{1em} \\\hline
     \bigcirc & \bigcirc & \blacksquare & \bigcirc & \blacksquare & \blacksquare \\\hline
     \bigcirc & \bigcirc & \bigcirc & \blacksquare & \bigcirc & \blacksquare \\\hline
     \bigcirc & \bigcirc & \blacksquare & \bigcirc & \bigcirc & \blacksquare \\\hline
     \blacksquare & \blacksquare & \bigcirc & \bigcirc & \bigcirc & \bigcirc \\\hline
     \bigcirc & \blacksquare & \blacksquare & \blacksquare & \blacksquare & \blacksquare \\\hline
     \blacksquare & \bigcirc & \blacksquare & \blacksquare & \bigcirc & \bigcirc \\\hline
     \blacksquare & \blacksquare & \blacksquare & \bigcirc & \blacksquare & \bigcirc \\\hline
   \end{array}
   \]
   At this point you should have $8$ districts ($6$
   in each district). Additionally, compute and interpret the
   \textbf{efficiency gap}. Show work and explain your reasoning.
 \item Discuss the effectiveness of this so-called \textit{Partisan
   Solution to Partisan Gerrymandering.}
 \end{enumerate}
 \end{question}




\end{document}
