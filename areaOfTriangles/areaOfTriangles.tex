\documentclass[nooutcomes,noauthor,handout,hints,12pt]{ximera}

\graphicspath{  
{./}
{./whoAreYou/}
{./drawingWithTheTurtle/}
{./bisectionMethod/}
{./circles/}
{./anglesAndRightTriangles/}
{./lawOfSines/}
{./lawOfCosines/}
{./plotter/}
{./staircases/}
{./pitch/}
{./qualityControl/}
{./symmetry/}
{./nGonBlock/}
}


%% page layout
\usepackage[cm,headings]{fullpage}
\raggedright
\setlength\headheight{13.6pt}


%% fonts
\usepackage{euler}

\usepackage{FiraMono}
\renewcommand\familydefault{\ttdefault} 
\usepackage[defaultmathsizes]{mathastext}
\usepackage[htt]{hyphenat}

\usepackage[T1]{fontenc}
\usepackage[scaled=1]{FiraSans}

%\usepackage{wedn}
\usepackage{pbsi} %% Answer font


\usepackage{cancel} %% strike through in pitch/pitch.tex


%% \usepackage{ulem} %% 
%% \renewcommand{\ULthickness}{2pt}% changes underline thickness

\tikzset{>=stealth}

\usepackage{adjustbox}

\setcounter{titlenumber}{-1}

%% journal style
\makeatletter
\newcommand\journalstyle{%
  \def\activitystyle{activity-chapter}
  \def\maketitle{%
    \addtocounter{titlenumber}{1}%
                {\flushleft\small\sffamily\bfseries\@pretitle\par\vspace{-1.5em}}%
                {\flushleft\LARGE\sffamily\bfseries\thetitlenumber\hspace{1em}\@title \par }%
                {\vskip .6em\noindent\textit\theabstract\setcounter{question}{0}\setcounter{sectiontitlenumber}{0}}%
                    \par\vspace{2em}
                    \phantomsection\addcontentsline{toc}{section}{\thetitlenumber\hspace{1em}\textbf{\@title}}%
                     }}
\makeatother



%% thm like environments
\let\question\relax
\let\endquestion\relax

\newtheoremstyle{QuestionStyle}{\topsep}{\topsep}%%% space between body and thm
		{}                      %%% Thm body font
		{}                              %%% Indent amount (empty = no indent)
		{\bfseries}            %%% Thm head font
		{)}                              %%% Punctuation after thm head
		{ }                           %%% Space after thm head
		{\thmnumber{#2}\thmnote{ \bfseries(#3)}}%%% Thm head spec
\theoremstyle{QuestionStyle}
\newtheorem{question}{}



\let\freeResponse\relax
\let\endfreeResponse\relax

%% \newtheoremstyle{ResponseStyle}{\topsep}{\topsep}%%% space between body and thm
%% 		{\wedn\bfseries}                      %%% Thm body font
%% 		{}                              %%% Indent amount (empty = no indent)
%% 		{\wedn\bfseries}            %%% Thm head font
%% 		{}                              %%% Punctuation after thm head
%% 		{3ex}                           %%% Space after thm head
%% 		{\underline{\underline{\thmname{#1}}}}%%% Thm head spec
%% \theoremstyle{ResponseStyle}

\usepackage[tikz]{mdframed}
\mdfdefinestyle{ResponseStyle}{leftmargin=1cm,linecolor=black,roundcorner=5pt,
, font=\bsifamily,}%font=\wedn\bfseries\upshape,}


\ifhandout
\NewEnviron{freeResponse}{}
\else
%\newtheorem{freeResponse}{Response:}
\newenvironment{freeResponse}{\begin{mdframed}[style=ResponseStyle]}{\end{mdframed}}
\fi



%% attempting to automate outcomes.

%% \newwrite\outcomefile
%%   \immediate\openout\outcomefile=\jobname.oc
%% \renewcommand{\outcome}[1]{\edef\theoutcomes{\theoutcomes #1~}%
%% \immediate\write\outcomefile{\unexpanded{\outcome}{#1}}}

%% \newcommand{\outcomelist}{\begin{itemize}\theoutcomes\end{itemize}}

%% \NewEnviron{listOutcomes}{\small\sffamily
%% After answering the following questions, students should be able to:
%% \begin{itemize}
%% \BODY
%% \end{itemize}
%% }
\usepackage[tikz]{mdframed}
\mdfdefinestyle{OutcomeStyle}{leftmargin=2cm,rightmargin=2cm,linecolor=black,roundcorner=5pt,
, font=\small\sffamily,}%font=\wedn\bfseries\upshape,}
\newenvironment{listOutcomes}{\begin{mdframed}[style=OutcomeStyle]After answering the following questions, students should be able to:\begin{itemize}}{\end{itemize}\end{mdframed}}



%% my commands

\newcommand{\snap}{{\bfseries\itshape\textsf{Snap!}}}
\newcommand{\flavor}{\link[\snap]{https://snap.berkeley.edu/}}
\newcommand{\mooculus}{\textsf{\textbf{MOOC}\textnormal{\textsf{ULUS}}}}


\usepackage{tkz-euclide}
\tikzstyle geometryDiagrams=[rounded corners=.5pt,ultra thick,color=black]
\colorlet{penColor}{black} % Color of a curve in a plot



\ifhandout\newcommand{\mynewpage}{\newpage}\else\newcommand{\mynewpage}{}\fi

\title{Area of triangles}

\author{Jenny Sheldon \and Bart Snapp}

\begin{document}
\begin{abstract}
  We build a formula for the area of a triangle based on the area of a
  rectangle and observe relationships between perimeter and area.
\end{abstract}
\maketitle


\begin{listOutcomes}
\item Give the standard formula for the area of a triangle.
\item Explain Cavalieri's principle.
\item Explain why the formula for the area of a triangle is true.
\item Explain how shapes of the same area can have drastically
  different perimeters.
\end{listOutcomes}

%% \begin{listObjectives}
%%  \item Explain why presented concepts and formulas are true,
%%  \item Increase student confidence in their ability to solve difficult math problems by using previous results, trying different methods, asking questions, and working with others,
%%  \item Improve student’s mathematical communication skills.

%% \end{listObjectives}

\mynewpage



\begin{question} 
  Explain how the following picture ``proves'' that the area of a right
  triangle is one half of the base times the height.
  \begin{center}
    \begin{tikzpicture}[geometryDiagrams]
      \coordinate (A) at (0,0);
      \coordinate (B) at (2,0);
      \coordinate (C) at (2,3);
      \coordinate (D) at (0,3);
      \tkzDrawSegment (A,B)
      \tkzDrawSegment (D,B)
      \tkzDrawSegment[dashed](B,C)
      \tkzDrawSegment[dashed](C,D)
      \tkzDrawSegment (D,A)
            
      \tkzMarkRightAngle[thin](B,A,D)
    \end{tikzpicture}
  \end{center}
  \begin{freeResponse}
    We know that the area of a rectangle of height $h$ and width $b$
    has area $h\cdot b$. However, we can also see that every right
    triangle can be placed in such a rectangle, while occupying
    exactly half the area.
    \begin{center}
      \begin{tikzpicture}[geometryDiagrams]
        \coordinate (A) at (0,0);
        \coordinate (B) at (2,0);
        \coordinate (C) at (2,3);
        \coordinate (D) at (0,3);
        \tkzDrawSegment (A,B)
        \tkzDrawSegment (D,B)
        \tkzDrawSegment[dashed](B,C)
        \tkzDrawSegment[dashed](C,D)
        \tkzDrawSegment (D,A)
        \tkzLabelSegment[left](A,D){$h$}
        \tkzLabelSegment[below](A,B){$b$}    
        \tkzMarkRightAngle[thin](B,A,D)
      \end{tikzpicture}
    \end{center}
    Hence the area of a RIGHT triangle is:
    \[
    \text{Area of right triangle} = \frac{bh}{2}
    \]
  \end{freeResponse}
\end{question}
\mynewpage


\begin{question}
  \emph{Shearing} is a process where you take a shape, cut it into thin strips, 
  then push the strips around in one direction to make a new shape.  
  Cavalieri's principle, or ``the shearing principle,'' states:\index{Cavalieri's principle}
  \begin{quote}
    Shearing parallel to a fixed direction does not change the
    $n$-dimensional measure of an object.
  \end{quote}
 \begin{enumerate}
  \item Explain how the first problem, combined with the following
  picture ``proves'' (via Cavalieri's principle) that the area of \emph{any}
  triangle is one half of the base times the height.
  \begin{center}
    \begin{tikzpicture}[geometryDiagrams]
      \coordinate (A) at (0,0);
      \coordinate (B) at (2,0);
      \coordinate (C) at (.5,3);
      \draw[pattern=horizontal lines] (A)--(B)--(C)--cycle;
      
      \draw[line width=5pt, ->] (2,1.5)--(4,1.5);
      
      \coordinate (AA) at (4,0);
      \coordinate (BB) at (6,0);
      \coordinate (CC) at (6,3);
      \draw[pattern=horizontal lines] (AA)--(BB)--(CC)--cycle;
      
      \end{tikzpicture}
  \end{center}
  \vfill 
  \item Explain how to use a picture to prove that a triangle of a
    given (fixed) area could have an arbitrarily large perimeter.
    \vfill \vfill
  \end{enumerate}
  
  \begin{freeResponse}
    If we want to know the area of any triangle, simply shear it to be a right triangle:
      \begin{center}
        \begin{tikzpicture}[geometryDiagrams]
          \coordinate (A) at (0,0);
          \coordinate (B) at (2,0);
          \coordinate (C) at (.5,3);
          \draw[pattern=horizontal lines] (A)--(B)--(C)--cycle;
          
          \draw[line width=5pt, ->] (2,1.5)--(4,1.5);
          
          \coordinate (AA) at (4,0);
          \coordinate (BB) at (6,0);
          \coordinate (CC) at (6,3);
          \draw[pattern=horizontal lines] (AA)--(BB)--(CC)--cycle;
          \tkzLabelSegment[right](BB,CC){$h$}
          \tkzLabelSegment[below](BB,AA){$b$}
          \tkzLabelSegment[below](B,A){$b$} 
        \end{tikzpicture}
      \end{center}
    Since shearing doesn't change the area or the height of the
    triangle $h$, by Cavalieri's principle, we see this right triangle
    has the same area as our original triangle. Thus the area of the
    original triangle is:
    \[
    \text{Area of any triangle} = \frac{bh}{2}
    \]
  \end{freeResponse}
\end{question}
\mynewpage


\begin{question} Finally:
   Give an intuitive argument explaining why the shearing
    principle is true.
  
  \begin{freeResponse}
    \begin{enumerate}
    \item Imagine our triangle as being a cross section of stacked
      papers. Moving these papers around doesn't change the area of
      each of the slices, hence, shearing doesn't change the area.
    \item Consider this picture:
      \begin{center}
        \begin{tikzpicture}[geometryDiagrams]
          \coordinate (A) at (0,0);
          \coordinate (B) at (2,0);
          \coordinate (C) at (.5,3);
          \draw[pattern=horizontal lines] (A)--(B)--(C)--cycle;
          
          \draw[line width=5pt, ->] (2,1.5)--(4,1.5);
          
          \coordinate (AA) at (4,0);
          \coordinate (BB) at (6,0);
          \coordinate (CC) at (11,3);
          \draw[pattern=horizontal lines] (AA)--(BB)--(CC)--cycle;
          %% \tkzLabelSegment[right](BB,CC){$h$}
          %% \tkzLabelSegment[below](BB,AA){$b$}
          %% \tkzLabelSegment[below](B,A){$b$}


          \draw[thin,dashed] (-1,0)--(12,0);
          \draw[thin,dashed] (-1,3)--(12,3);
        \end{tikzpicture}
      \end{center}
           The triangle on the right has the same area as the triangle
           on the left. However, the perimeter of the triangle on the
           right is MUCH greater.  To obtain an arbitrarily large
           perimeter, simply shear the triangle more---by Cavalieri's
           principle the area will be unchanged.
    \end{enumerate}
  \end{freeResponse}
\end{question}

\end{document}
